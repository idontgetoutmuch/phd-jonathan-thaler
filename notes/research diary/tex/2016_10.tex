\section*{2016 October 6th}

\subsection*{First official supervision-meeting}
Had a nice and relaxed first official supervision meeting with Peer-Olaf today. We talked about two things: organisational and research topic related stuff.

\subsubsection*{Organisational stuff}

\begin{itemize}
\item I should contact Nadine Holmes \url{https://www.nottingham.ac.uk/computerscience/people/nadine.holmes} for both my office key and for a website presence of me on the IMA page.
\item Peer-Olaf does not care much about which kind of courses I attend for my Ph.D. but he suggests that I should go to "Thesiswriting" not in the last year.
\item Regarding publishing papers: the computer-science Ph.D. here at Nottingham is organized in the way that officially no published papers are required and only the final thesis is sufficient to become a Ph.D. (if everything is OK). BUT it would be nice to have at least 1-2 journal papers (2 would be good and are suitable for my topic) because that is a very strong backup of the thesis against the neutral reviewer and also because one should have the aim to publish something in a Ph.D. (I really want to publish something).
\item Conference-Papers are mandatory: the first should conference (with paper of course) should be at the end of the first year, with more conferences and papers to come.
\item Peer-Olaf suggests that I may attend courses/lecture in economics in the 2nd year when required (e.g. Market Microstructure or Equilibrium Theory).
\item I should visit Simon Gächter for a quick chat and to build relationship to the economic guys.
\item Sometimes there are interesting Summer-Schools and Workshops and I may attend some if the topics are interesting for me.
\item I am now member of the IMA and will thus attend seminars held by IMA over the next 3 years.
\item I will have 2 supervision-meetings a month with Peer-Olaf.
\end{itemize}

\subsubsection*{Research topic related stuff}
We discussed intensely what I really want to do and what Peer-Olaf has in mind.

\begin{itemize}
\item His students go in the direction of developing some method and then showing that it can be applied to various fields e.g. agent-based computational economics (ACE), social simulation, epidemiology. This is also the way I will work: develop a framework which allows implementing and reasoning in one of the various fields - thus the framework should be domain-agnostic as much as possible.
\item Peer-Olaf mentioned Akka to me, that I should take a look at it.
\item I clearly stated (and Peer-Olaf told me to write that down) that: I don't want to force OO-concepts to pure functional programming but want to approach the agent-based modelling/simulation methodology from a pure functional way (which I guess will be then category theory).
\end{itemize}

\subsection*{First work-package}
Peer-Olaf suggested to start with something very small but highly abstract and then extend it more and more. Thus my starting points will be

\begin{itemize}
\item How can an Agent be represented in a functional way?
\item How can an Agent be implemented in a pure functional language?
\end{itemize}

I need to start with a general agent-model of interaction, concurrency and pro-activity and then map these to functional concepts and then translate these concepts to a pure functional language. Thus first I have to look at various agent-models like the Actor-Model, select the one which fits best (what are the criteria?). Then I have to find a pure functional representation and finally implement this in Haskell. Note that it is highly desirable that the mapping from functional specification to pure functional implementation should be straight forward without loosing much expressiveness thus category theory would be the first way to go for the functional specification. To test the ideas and implementations I should apply it to a simple epidemic model e.g. the SIR model as it is very well known and researched and the results can easily be compared. After all has been implemented then I should use Agda to implement the whole application in a pure functional language with dependent types and then do reasoning about this model e.g. termination checking.

\begin{enumerate}
\item Find agent-model of interaction, concurrency and pro-activity: \textit{Actor-Model, Process Calculi}.
\item Create functional representation/specification of agent-model: \textit{Category Theory, still open to research}.
\item Create functional representation/specification of application: SIR-model.
\item Implementation in pure functional language: \textit{Haskell}
\item Implement in dependently typed pure functional language: \textit{Agda}
\item Do formal verification and reasoning about the model.
\end{enumerate}


\section*{2016 October 7th}

\subsection*{Start of implementation}
Maybe I still went too formal yesterday but should be more practical thus I should go right into implementation of the SIR model using agents in Haskell. As the underlying model for agents I will use the actor-model as it is very well defined, researched and proven to be useful (Erlang) and I am already familiar with it (Gul Aghas Book) and have experimented with it in Erlang. So the starting point will be to bring the actor-model to Haskell. Fortunately there are exist already a number of libraries which either directly implement the actor-model or provide mechanisms to implement them directly. Here is an overview, loosely based on \url{https://wiki.haskell.org/Applications_and_libraries/Concurrency_and_parallelism} and on searching for 'actor' on hackage - note that the ordering reflects the order of interest in the library:

\begin{enumerate}
\item \textbf{Hackage-Package 'hactor'. Conclusion: WINNER, it seems to be mature (version 1.2), minimalistic, not as outdated as the others (last commit 2 years ago), installing using cabal works, interface looks 'good'.}
\item Hackage-Package 'hactors. Conclusion: last commit 5 years ago, could install it using cabal.
\item Cloud Haskell is a full blown Erlang-style concurrent and distributed programming framework for Haskell. Conclusion: not simple enough, this is too much for now because I don't need distributed computation but Cloud Haskell would incur doing up a lot of infrastructure stuff.
\item Communicating Haskell Processes (CHP): is a Haskell library following the CSP (Communicating Sequential Processes). It may be an alternative to the Actor-Model but for now, follow the Actor-Model.
\item Hackage-Package 'simple-actors': last commit 4 years ago, couldn't install it using cabal, conflicts
\item Hackage-Package 'actor: Actors with multi-headed receive clauses by Martin Sulzmann et al. \url{http://sulzmann.blogspot.co.uk/2008/10/actors-with-multi-headed-receive.html}. Conclusion: couldn't install it using cabal, seems to have conflicting/missing dependencies (project is from 2008!)
\item Hackage-Package 'gore-and-ash-actor': is actually a game-engine extension of the (gore\&Ash engine) to implement actor style of programming. Conclusion: too focused on game-engine, not minimal enough, didn't try it out.
\end{enumerate}

Note that all of these libraries build upon the very low-level parallel \& concurrency frameworks available in Haskell in the Concurrency package. Thus I also have to understand the basics of parallel (non-interacting, parallel computation) and concurrent (indeterministic-interacting, parallel computation) programming in Haskell. Using this low level actor-model an agent-model must be implemented. Then using the agent-model the actual simulation-model can be implemented: SIR.

\section*{2016 October 10th}
I have reflected on my first steps implementing the agent-model in Haskell and I've come to the conclusion that I won't use an existing actor-model implementation as it is too tedious to get into their usage and I need to build an agent-protocol on top of it anyway. Thus I will implement it on my own - the question is how? There needs to be given quite a few thoughts:

look at formal/functional definition of agents in wooldridge: online library. 2 approaches: 1. formal and 2. implementation

3 approaches: 1. no parallelism/concurrency, 2. deterministic parallelism, 3. indeterministic concurrency

but in all cases: use IVar, MVar or STM to "communicate" among agents: need some way of sharing data with each other. question: what would be the alternative? 

more fundamental question: how can agents send messages in pure functional languages?

is it possible to use deterministic parallelism instead of classic concurrency e.g. transparent parallel execution and communication of data? would resemble more a pure functional style and allows to reason about programs

what about ignoring parallelism and concurrency for now and just iterate round robin through agents?

experiment with basic forkIO, MVar and STM: two threads communicating. install threadscope and look at execution

instead of function to forkIO: use lambda which accesses data from outside forkio

\item It is yet not real functional, need to approach it from a different perspective
\item Can we keep determinism? Real concurrency \& random-variables kill determinism. Are there workarounds? Wooldrige defines the environment in which an agent is situated as inherently non-deterministic, thus an agent-based simulation will be always indeterministic. But can we reduce it?
\item How to do communication? Look at Wooldriges fundamental book chapter communication