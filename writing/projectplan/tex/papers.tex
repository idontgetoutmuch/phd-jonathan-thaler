\section{Papers}
This is the list of papers I have in my mind and includes both work mandatory for the PhD and optional ones. The latter ones are just fun/philosophical-papers, not directly related to my PhD but somehow tangential with the very basic direction - they are intended to be worked on in my free-time and to free my head when wrestling too hard with my PhDs main work.

\subsection{The Art of Iterating: Update-Strategies in Agent-Based Simulations}
\textbf{Type:} Groundwork \\
\textbf{Target:} Conference/Journal \\
\textbf{Requirement:} Mandatory \\

When developing a model for an Agent-Based Simulation (ABS) it is of very importance to select the right update-strategy for the agents to produce the desired results. In this paper we develop a systematic treatment of all general properties, derive the possible update-strategies in ABS and discuss their interpretation and semantics something which is still lacking in the literature on ABS. Further we investigate the suitability of the three very different programming languages Java, Haskell and  Scala with Actors to implement each of the update-strategies. Thus this papers contribution is the development of a new, general terminology of update-strategies and their implementation comparison in various kinds of programming languages.

\subsection{Specification equals Code: An EDSL for pure functional Agent-Based Modelling \& Simulation}
\textbf{Type:} Groundwork \\
\textbf{Target:} Conference/Journal \\
\textbf{Requirement:} Mandatory \\

Building upon our previous work on update-strategies in Agent-Based Modelling \& Simulation (ABM/S) where we showed that Haskell is a very attractive alternative to existing object-oriented approaches we ask in this paper if the declarative power of the pure functional language can be utilized to write specifications for simple ABM/S models which can be directly translated to our Haskell implementation.

\subsection{Reactive Agents:\\Functional Reactive Programming and ABM/S}
\textbf{Type:} Groundwork \\
\textbf{Target:} Conference/Journal \\
\textbf{Requirement:} Mandatory \\

In our previous work on update-strategies in Agent-Based Modelling \& Simulation (ABM/S) we showed that Haskell is a very attractive alternative to existing object-oriented approaches but our presented approach was too limited and we hypothesized that embedding it within a functional reactive framework like Yampa would leverage it to be able to build much more complex models. In this paper we investigate whether this hypothesis is true by testing if our approach can easily be transferred to Yampa, what we really gain from it and if more complex models can become reality. As a proof-of-concept we build a large, complex model from Agent-Based Computational Economics (ACE) which simulates a whole economy.





\subsection{Pure Functional ACE (Catchy title yet to be defined)}
\textbf{Type:} Main PhD Work \\
\textbf{Target:} Journal \\
\textbf{Requirement:} Mandatory \\

Is the main work of the PhD and targeted at publication in a Journal. The exact topic and content will be clarified at the beginning of the 2nd year. Mainly it will describe how to implement Ionescus Framework of Gintis trading model and extend it to a more general Market-Model. It will also give an outlook on implementing it using dependent types.

\subsection{Pure by Nature: A Library for pure Agent-Based Simulation \& Modelling in Haskell}
\textbf{Type:} Extension \\
\textbf{Target:} Conference \\
\textbf{Requirement:} Optional \\

This paper describes the ideas and theory behind the implementation of my ABM/S library "PureAgents" in Haskell.

\subsection{Time in Games: a Tron Light-Cycle Game in Dunai}
\textbf{Type:} Fun \\
\textbf{Target:} Conference \\
\textbf{Requirement:} Optional \\

This paper describes the 2D light-cycle game inspired by the movie Tron implemented in Dunai. It allows to turn back time.

\subsection{Pure Functional Islamic Design}
\textbf{Type:} Fun \\
\textbf{Target:} Conference \\
\textbf{Requirement:} Optional \\

Inspired by the paper "Functional Geometry" by Peter Henderson I had the idea to come up with a  EDSL for declaratively describing pictures of islamic design which are then rendered using the gloss-library. From its focus totally unrelated to the PhD topic but still a great opportunity to learn Haskell, to learn to think functional, to learn to design my own EDSL - thus it may be a great paper to pursue even if I won't finish or produce something publishable.

\subsection{The Genesis According to Computer-Science: Reality as Simulation of Free Will}
\textbf{Type:} Philosophy \\
\textbf{Target:} ? \\
\textbf{Requirement:} Optional \\

I've always been interested in a deeper meaning behind things so I want to look into the philosophy and future of simulation: why do we simulate, what can we derive from simulations, what does it say that we humans simulate, what will the future of simulation be? \\
I claim that our ability to "simulate" in our mind separates our intelligence from those of the animals and that this is a unique property of humans. Also i think the future of simulation will be that humankind will do its own creation/live (artifical life, conciousness) which allows to accurately simulate a given setting - this of course could have ethical implications.