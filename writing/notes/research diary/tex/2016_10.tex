\section*{2016 October 6th}

\subsection*{First official supervision-meeting}
Had a nice and relaxed first official supervision meeting with Peer-Olaf today. We talked about two things: organisational and research topic related stuff.

\subsubsection*{Organisational stuff}

\begin{itemize}
\item I should contact Nadine Holmes \url{https://www.nottingham.ac.uk/computerscience/people/nadine.holmes} for both my office key and for a website presence of me on the IMA page.
\item Peer-Olaf does not care much about which kind of courses I attend for my Ph.D. but he suggests that I should go to "Thesiswriting" not in the last year.
\item Regarding publishing papers: the computer-science Ph.D. here at Nottingham is organized in the way that officially no published papers are required and only the final thesis is sufficient to become a Ph.D. (if everything is OK). BUT it would be nice to have at least 1-2 journal papers (2 would be good and are suitable for my topic) because that is a very strong backup of the thesis against the neutral reviewer and also because one should have the aim to publish something in a Ph.D. (I really want to publish something).
\item Conference-Papers are mandatory: the first should conference (with paper of course) should be at the end of the first year, with more conferences and papers to come.
\item Peer-Olaf suggests that I may attend courses/lecture in economics in the 2nd year when required (e.g. Market Microstructure or Equilibrium Theory).
\item I should visit Simon Gächter for a quick chat and to build relationship to the economic guys.
\item Sometimes there are interesting Summer-Schools and Workshops and I may attend some if the topics are interesting for me.
\item I am now member of the IMA and will thus attend seminars held by IMA over the next 3 years.
\item I will have 2 supervision-meetings a month with Peer-Olaf.
\end{itemize}

\subsubsection*{Research topic related stuff}
We discussed intensely what I really want to do and what Peer-Olaf has in mind.

\begin{itemize}
\item His students go in the direction of developing some method and then showing that it can be applied to various fields e.g. agent-based computational economics (ACE), social simulation, epidemiology. This is also the way I will work: develop a framework which allows implementing and reasoning in one of the various fields - thus the framework should be domain-agnostic as much as possible.
\item Peer-Olaf mentioned Akka to me, that I should take a look at it.
\item I clearly stated (and Peer-Olaf told me to write that down) that: I don't want to force OO-concepts to pure functional programming but want to approach the agent-based modelling/simulation methodology from a pure functional way (which I guess will be then category theory).
\end{itemize}

\subsection*{First work-package}
Peer-Olaf suggested to start with something very small but highly abstract and then extend it more and more. Thus my starting points will be

\begin{itemize}
\item How can an Agent be represented in a functional way?
\item How can an Agent be implemented in a pure functional language?
\end{itemize}

I need to start with a general agent-model of interaction, concurrency and pro-activity and then map these to functional concepts and then translate these concepts to a pure functional language. Thus first I have to look at various agent-models like the Actor-Model, select the one which fits best (what are the criteria?). Then I have to find a pure functional representation and finally implement this in Haskell. Note that it is highly desirable that the mapping from functional specification to pure functional implementation should be straight forward without loosing much expressiveness thus category theory would be the first way to go for the functional specification. To test the ideas and implementations I should apply it to a simple epidemic model e.g. the SIR model as it is very well known and researched and the results can easily be compared. After all has been implemented then I should use Agda to implement the whole application in a pure functional language with dependent types and then do reasoning about this model e.g. termination checking.

\begin{enumerate}
\item Find agent-model of interaction, concurrency and pro-activity: \textit{Actor-Model, Process Calculi}.
\item Create functional representation/specification of agent-model: \textit{Category Theory, still open to research}.
\item Create functional representation/specification of application: SIR-model.
\item Implementation in pure functional language: \textit{Haskell}
\item Implement in dependently typed pure functional language: \textit{Agda}
\item Do formal verification and reasoning about the model.
\end{enumerate}


\section*{2016 October 7th}

\subsection*{Start of implementation}
Maybe I still went too formal yesterday but should be more practical thus I should go right into implementation of the SIR model using agents in Haskell. As the underlying model for agents I will use the actor-model as it is very well defined, researched and proven to be useful (Erlang) and I am already familiar with it (Gul Aghas Book) and have experimented with it in Erlang. So the starting point will be to bring the actor-model to Haskell. Fortunately there are exist already a number of libraries which either directly implement the actor-model or provide mechanisms to implement them directly. Here is an overview, loosely based on \url{https://wiki.haskell.org/Applications_and_libraries/Concurrency_and_parallelism} and on searching for 'actor' on hackage - note that the ordering reflects the order of interest in the library:

\begin{enumerate}
\item \textbf{Hackage-Package 'hactor'. Conclusion: WINNER, it seems to be mature (version 1.2), minimalistic, not as outdated as the others (last commit 2 years ago), installing using cabal works, interface looks 'good'.}
\item Hackage-Package 'hactors. Conclusion: last commit 5 years ago, could install it using cabal.
\item Cloud Haskell is a full blown Erlang-style concurrent and distributed programming framework for Haskell. Conclusion: not simple enough, this is too much for now because I don't need distributed computation but Cloud Haskell would incur doing up a lot of infrastructure stuff.
\item Communicating Haskell Processes (CHP): is a Haskell library following the CSP (Communicating Sequential Processes). It may be an alternative to the Actor-Model but for now, follow the Actor-Model.
\item Hackage-Package 'simple-actors': last commit 4 years ago, couldn't install it using cabal, conflicts
\item Hackage-Package 'actor: Actors with multi-headed receive clauses by Martin Sulzmann et al. \url{http://sulzmann.blogspot.co.uk/2008/10/actors-with-multi-headed-receive.html}. Conclusion: couldn't install it using cabal, seems to have conflicting/missing dependencies (project is from 2008!)
\item Hackage-Package 'gore-and-ash-actor': is actually a game-engine extension of the (gore\&Ash engine) to implement actor style of programming. Conclusion: too focused on game-engine, not minimal enough, didn't try it out.
\end{enumerate}

Note that all of these libraries build upon the very low-level parallel \& concurrency frameworks available in Haskell in the Concurrency package. Thus I also have to understand the basics of parallel (non-interacting, parallel computation) and concurrent (indeterministic-interacting, parallel computation) programming in Haskell. Using this low level actor-model an agent-model must be implemented. Then using the agent-model the actual simulation-model can be implemented: SIR.

\section*{2016 October 12th}

\subsection*{Implementation so far}
I rejected all the Actor-Model libraries as they either couldn't be installed using cabal/manual install or were tiresome to get into. The point is that I would have had to build on top of one of them my agent implementation which would have complicated things. So instead I chose to start from nothing and build up experience. The following points became clear

\begin{itemize}
\item My implementation technique so far is pretty straight-forward: all state is carried around, everything works deterministic.
\item I tried experimenting with Concurrency-Mechanisms built into Haskell2010: forkIO, TChan, MVar but the problem is that they are inherently INDETERMINISTIC and thus one looses the ability to reason.
\item Haskell also supports deterministic parallelism, which could be the solution to the dilemma of wanting to have parallel execution paths but still keeping the determinism.
\item In the end the questions will be: 1. how can agents send messages in pure functional languages? 2. is deterministic parallelism possible?
\end{itemize}

\subsection*{Meeting with Paolo}
In the morning I had an interesting meeting with Paolo, a post-doc, who is very experienced in Haskell \& Agda. I showed him my current implementation and we discussed a bit about it, especially the problem of how state can be represented and handed around in the simulation using Haskell. He proposed \textit{Coalgebras} and also explained it a bit to me but it was too complicated to understand it with the quick tutorial - I should contact Venanzio for this. Also he suggests to get into Category Theory, which will be a big task which will take months but will pay of in the long run. He also mentioned that a guy called Ivan is working on Functional Reactive Programming, a paradigm which can be used to program games in Haskell - this struck me as games can be understood as a continuous agent-based simulation, so maybe this is one way to go.

\subsection*{Sum-up}
\begin{itemize}
\item Start learning \textit{Category Theory}
\item Start looking at \textit{Coalgebras}
\item Look at \textit{Functional Reactive Programming (FRP)}
\item Look at formal/functional definition of agent architecture in Wooldridges Book Chatper 2.6 Abstract Architectures for Intelligent Agents. 
\item How to do communication? Look at Wooldriges fundamental book chapter 7 communication.
\end{itemize}

\subsection*{Presentation of aim \& impact}
I had a 10 min. presentation of my aims \& impact of my thesis to my PhD colleagues. It went very nice with good discussions and also the other presentations of my colleagues were very interesting - I hope to do that more often. \\
One thing I noticed was not very well structured and transported in my presentation. What the others communicated very clearly was

\begin{enumerate}
\item Which kind of problem there is.
\item What the aim is e.g. solving the problem.
\item How the aim is achieved by enumerating VERY CLEAR objectives.
\item What the impact one expects (hypothesis) and what it is (after results).
\end{enumerate}

Of course I am just at the start of my research so I couldn't provide all the given details. Formulating these things will be the major topic until summer.

\section*{2016 October 17th}
On Friday I participated in the FP (functional programming) lunch where everyone buys sandwiches and then gathers in a presentation-room and one is holding a presentation for about 45 mins + 15 mins discussion. I met Nenrik Nilsson there and he said we should talk to each other as he is the main force behind YAMPA a FRP framework. This are good news as I thinkg FRP may be a very good approach to functional ABM/S. 

\subsection*{Feedback on my presentation}
Today Peer-Olaf gave me feedback on my presentation. He said that speaking was very good and confident but he mentioned a few things which I should focus on

\begin{itemize}
\item Use page-numbers, so that the audience will know where the presentation currently is (beginning, middle, end,...)
\item Mention your name in the beginning, maybe not everyone knows you already.
\item \textit{the way to go} means that it is the future. This is not what I wanted to say when I said that OO is the way to go in ABM/S - I wanted to express, that it is the current state-of-the art, the current way how it is done.
\item Know your audience! Always assume that you have a mixed audience who have very basic knowledge. 5 points is the most people can take from a presentation.
\item The aim is something general, needs to be testable, so to be proofed that I have achieved it. 
\item Discussing my contribution: what is a framework?
\item The research-question(s) should come before the aim. The aim is the statement, maybe just a statement of the research questions. Only phrase one of the two as both mean basically the same.
\end{itemize}

Peer-Olaf also said that I have to really proof that there is a gap in my field I am researching e.g. that there is a gap when using Akka, Erlang,... as one can't reason in these languages. He also told me to mention DEVS which I should look at and mentioned that there is a book by Wainer. \\
We also discussed the structure of how to build up a presentation: 

\begin{enumerate}
\item Start with motivation \& interest.
\item Give a background what is out there.
\item Develop this towards a gap and show what is missing.
\item This then leads to research question(s).
\item This in the end leads to an aim which is condensed research questions.
\item Refine the aim with objectives to be solved for the aim to be reached.
\end{enumerate}

\section*{2016 October 18th}
\subsection*{Meeting with Henrik Nilsson}
I had a meeting with Henrik Nilsson, the inventor/maintainer of Yampa, a FRP framework for Haskell. We talked about FRP being suitable to be used for ABM/S and he was pretty confident that Yampa could handle the things I described (Agent-Model, agent-based economics). The main point when following this road will be developing a \textbf{EDSL for ACE} on top of Haskell and Yampa. This sound awesome as this would give a domain-expert a tool for developing his models and then reason about them. Henrik said I should play around with Yampa a few days and see if it is suitable for what I want to do - thats what I'm going to do: implement SIR in Yampa!

 
\section*{2016 October 20-21th}
\subsection*{Experiences with \textit{classic} SIR/SIRS model-implementation}
I finally got the SIR implementation in classic Haskell running - with classic I refer to the kind of programming: no Monads, no FRP, no other fancy stuff just pure Haskell with maybe IO for text-output/files writing. I am pretty proud of it: it performs replications, detects when a replication has finished (no more infected people) and exports the dynamics as JSON to a file which can be then visualized using a Matlab/Octave script.

\begin{itemize}
\item After having implemented replications only the first replication actually did calculate, the remaining ones were just copies of the first one. My guess was that because the compiler / runtime-system infers due to referential transparency that the call to calculating one replication is the same for all. Initially I thought that to run multiple replications properly I would have to introduce pure use of RNG - which I did - but that was not the origin of the problem. The problem was that replications had \textit{no input}, which of course led Haskell to believe that it is constant after having been calculated. So after having added an input - the index of the replication - all replications were calculated properly. This is also known as constant applicative form (CAF): \url{https://wiki.haskell.org/Constant_applicative_form} \\
It is important that one is careful detecting CAF throughout all the program in which RNG are used: they will lead to the same results when repeatedly called - this also applies to the main-function!!! This was a major point of confusion but works now: population-count and replication-count are now passed in from outside (command-line args/function arguments).
\item It was very easy to extend the model to SIRS, which is more interesting.
\item It was not so easy to debug due to confusing call-hierarchies. Maybe unit tests are a remedy for this.
\item To a non-programmer / unfamiliar with Haskell this implementation is probably a horror to work on due to scattering of functions all over the place, unclear call-hierarchy, implementation details showing up.
\item In my opinion this solution is too near to the OO-thinking: agents are a structure of data, have message-handler and in all functions the agent-instance is passed in. This is clearly oo-technique disguised as functional. Just because all is implemented in (pure) functions where all the state is threaded in and out through parameters/return values doesn't make it functional yet. The next approach should be one using Monads where the goal will be to encapsulate the whole thing and provide a small EDSL for this SIRS model.
\end{itemize}

\section*{2016 October 24th}
I noticed that I have lot of hidden beliefs about science/computers. I need to detect them and write them down, so they don't take me by surprise

I removed all the todos, as they are terrible repetitive and I've put a todo/goals list in research proposal for 1st year.

\subsection*{STM with lambdas and data-closures}
I tested STM with lambdas and data-closures. It is possible to send a lambda with data-closure using STM. This would allow to send a computation which encloses e.g. Agent a to another Agent b which then executes this computation.

\subsection*{What is my contribution?}
In the PGR seminar today the lecturer made it VERY clear that when we defend our research and present it to others we have to show very clearly what our contribution is. This is what many PhD students fail to show, especially in the beginning. He made it clear that original work is not enough, there must be new knowledge in the research. Also I have still difficulties and to be honest at the moment I don't know it yet but my hope is that I can develop a \textit{pure functional model of ABM/S} - this is what I am currently very deeply thinking about, how such a pure functional model could look like and what it actually is. The problem is that I fall into the thinking about implementation details too fast, I must refrain from that and keep it purely in the abstract domain without going into Haskell details.

WHY FUNCTIONAL? "because its the ultimate approach to scientific computing": fewer bugs due to mutable state (why? is thos shown obkectively by someone?), shorter (again as above, productivity), more expressive and closer to math, EDSL, EDSL=model=simulation, better parallelising due to referental transparency, reasoning

scientific results need to be reproduced, especially when they have high impact. a more formal approach of specifying the model and the simulation (model=simulation) could lead to easier sharing and easier reporduction without ambigouites

pure functional agent-model \& theory, EDSL framework in Haskell for ACE

\subsection*{A paper on Akka: using the actor-model in ABM}

\subsection*{Unit-Tests}
Updated Todos: Unit-Tests are written (very few), ignore data-parallelism and stuff for now - will be an implementation detail later, removed AKKA for now. What I want to look into is how to test functions which use RNG: the RNG must be made deterministic.
unit tests: also test invariants: specific data of an agent hasnt changend

\section*{2016 October 26th}

\subsection*{Supversision-Meeting with Peer-Olaf}
Peer-Olaf told me what he thought about should the direction of my PhD be: \\
I should think about 4 scenarios from economics (e.g. Auctions, Supply-Chain, Public-Goods Games,...) in which 1 is supposed to be VERY suitable for functional approaches, one is supposed to be absolutely not and the other 2 are somewhere in between. My contribution would be to show "What are the upsides and the limits using functional approach in ABM". Also I should apply 3-4 different technologies for implementing them: AnyLogic/Repast/Netlogo (very high level), Scala with Actors: Akka (concurrent object-oriented functional), Haskell (pure functional), Java (object-oriented, non-functional) and show what the differences are between them implementing the 4 scenarios. In the end I need to come up with criteria for deciding which is "better" suited than the other. Mainly it is all about robustness: making less errors. This is influenced by:

\begin{itemize}
\item Testability: how and how easily can it be tested, Unit-Tests?
\item Size of Code: how many lines of code have to be written?
\item How is state treated?
\item Is it suited for scientific computing?
\item Parallelism: how difficult is it?
\end{itemize}