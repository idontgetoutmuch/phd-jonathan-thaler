\section*{2016 December 1st}
Heroes and Cowards is done in AnyLogic, ReLogo and also in Haskell! \\
Today I had a 3-minute presentation of my topic in a seminar of the IMA group. Afterwards I had a short discussion with Peer about my Haskell-Implementation of Heroes \& Cowards and what I should do next. I argued that this simulation is very trivial and has nothing to do with the ACE stuff I want to tackle. Peer told me that I am (again) too fast and that I should keep in mind that I am doing basic research here thus should start small and simple and look in-depth into these results. Thus he told me to play a little bit more (e.g. obstacles, avoiding themselves, wrap-around world, no limit world, clipping world) with this Haskell-implementation and then go with explanations to in-depth and explain what the differences are and what we have gained doing so (my point: reasoning using the types e.g. I can show that there is no randomness in the simulation \textit{after} the random initialization just by looking at the types!). Also I should meet with Thorsten or present at the FP lunch at some time to get input from the pure functional and theoretical side.

\subsection*{Finally: a direction for a conference paper}
Peer also told me to write all my observations and thoughts about the Haskell-Implementation down (20pages or so) which could then be the basis for a future conference paper next year in spring. Thus I created finally a paper template in LaTex in which I will write down all the stuff and ideas and thoughts and observations of the Haskell-Implementation. This will then be the basis for my ACE research.

\subsection*{Make a project-plan}
Peer told me that I should come up with a project-plan in the next meeting. I should organize it into Workpackages (e.g. prototyping, literature-research, theory,...) which should be subdivided into 75\% Peer-Olafs field and 25\% of Thorsten. I will put the project-plan then in the research-proposal which I also had to create fresh as I condensed it already into a basic structure of my thesis.

\section*{2016 December 13th}
Had Supervision Meeting with Peer today. I think I could show that I did a lot of good work: prototyping and taking notes of the experiments I conducted with the simulations. I will start now writing the first paper. I am very happy about that because I love to write and can do so finally. I came up with the idea to write also a second paper about Actors (Akka) but Peer said I should focus on the one paper and write the results of the 2nd paper down in the 1st year report. Also Peer said the annual review will be around July thus I have not as much time as I expected (September).