\section{Project-Plan}

\subsection{Years}
The whole PhD lasts for 3 years and thus I will structure it according to 3 years where each year will be a major milestone - which is also intended by the Computer School.

\subsubsection{1st Year: Basics}
In this year I will learn basics and develop and research the methodology I will use for the main work in the 2nd year. Also I want to write a paper of how to apply functional programming to ABM/S. The time-frame will be set by the date of the 1st year annual oral report which will happen beginning of July thus there are about 6 Months time (counting including January '17 and leaving out July '17). These are the things I want / need to achieve this year:

\begin{itemize}
\item Prototyping in Haskell, Scala and Java
\item Study Actor-Model theory: Hewitt, Greif, Clinger, Agha
\item Get into reasoning about programs
\item Basics of Economics \cite{bowles_understanding_2005}, \cite{kirman_complex_2010}
\item Basics of ACE: Tesfatsion
\item Implement PureAgents Library
\item Write the paper
\item Write 
\item Prototype Auction/Fishmarket using Akka and Haskell and describe results in 1st year report.
\item Write 1st year report
\end{itemize}

Important and mile-stone dates:

\begin{center}
\begin{tabular}{ c|c } 
	April (End) & Finished Paper: functional ABM \\ 
	\hline
	June (Mid) & Finished writing 1st year report  \\ 
	\hline
	July & Oral annual report \\
\end{tabular}
\end{center}

TODO-List as of \today

\begin{itemize}
\item performance unacceptable: 1000 in haskell vs 100.000 in java is a shame on haskell, more should be possible. investigate using profiling both of CPU and memory: \url{http://keera.co.uk/blog/2014/10/15/from-60-fps-to-500/}. strictness & tail-recursion!
	Par-Version seems to be working ok compared to Seq. Seq consumes hundreds of MB of memory!


\item implement environment for parallel: global read, local write

\item embed PureAgentsPar and Seq in Yampa: PureAgentsYampa so we can leverage the power of the EDSL, SFs, continuations,... of Yampa/Dunai.
\item implement agent monad: PureAgentsMonadic. but what is an Agent-Monad? build a monad to chain actions of the agent and always run inside an agent-monad
\item embed PureAgentsMonadic in Dunai

\item implement wait blocking for a message so far. utilize yampas event mechanism?

\item implement conways game of life: parallel discrete: read global environment, write local
\item implement schelling segregation
\item implement sugarscape ?


\item look into QuickCheck and HPC
\item problem: so far only agents with same static messagetypes, environment and states, can communicate: the agents are homogenous. how can we implement hetereogenous agents in this library?
\item implement a general-purpose rendering-frontend with gloss but let the simulation be driven by Yampa/SimulationBackend instead of frontend (not real-time)


\item investigate speed of divergence of simulation runs using the terminology of gleichzeitige ungleichzeitigkeiten
\item 2 complex system can lead to a total different outcome when initial settings differ only by fractions. thus it may also be the case for specific ABM/S?
\item todo: render paths of the agents in heroes and cowards
\item todo: abm/s of a go game
\item IDEA: what about an ABM/S of karma and rebirth? add to genesis paper
\item IDEA: what about ABM/S generating sound? could be a perfect example for Yampa due to its signal functions. the sound is the result of interactions of agents which try to generate harmonies and agents trying to create dissonance
\item IDEA: what about abm/s creating drawings/art? 2d continuous and each agents path is drawn
\item haskell: dont have objects with methods which can call between each other but we need some way of representing agents. this is done using a struct type with a behaviour function and messaging mechanisms. important: agents are not carried arround but messages are sent to a receiver identified by an id. This is also a big difference to java where don't really need this kind of abstraction due to the use of objects and their 'messaging'. 
\item messaging mechanisms have up- and downsides, elaborate on it.
\item reason for patterns: heroes try to stay 50 \% in between and have selected 2 cowards which themselves are at the borders opposite. need much more cowards than heroes 75/25. TODO: investigate pair-wise movement.
\item hypothesis: individual runs diverge from the first step on but the global behaviour stays the same.
\item question: do emergent patterns break down / global dynamics change completely in some ABM/S when changing sim-semantic? which kind of ABM could show this behaviour? which properties are responsible for it?
\item parallel more natural in haskell
\item sequential more natural in java
\item concurrent difficult in both, using stm in haskell it becomes very natural, STM available in java too. actors are an even better approach but is having problems with time in simulation
\item where does 1st and 2nd year research meet: reasoning about emergent properties of the simulation. in case of HAC the crosspattern, in case of ACE the convergence to stable prices

\item Continue investigations regarding the heroes and cowards game
\item Explore the literature on complex systems to find an explanation for the emergence of the macro level patterns in the Heroes \& Cowards game
\item Consider the next step: Implementing an economics example at different levels of complexity 
\item Continue writing conference paper and looking out for an appropriate conference
\end{itemize}

\paragraph{January}
Haskell
	investigate and increase low performance of implementations
	
	finish Seq & Par: environment access
	implement Concurrent semantic
	
Paper
	Rework abstract
	30min presentation of content
	
Reading
	Gleichzeitige Ungleichzeitigkeiten
	Complex Systems Stuff
	TODO: genauer alle papers aufzählen
	
\paragraph{February}
Haskell
	embed Seq & Par in Yampa
	implement monadic version of Seq & Par
	embed monadic Version in Dunai

Paper
	first very basic draft
	
Reading
	Semantics of Actors (Greif & Clinger)
	TODO: genauer alle papers aufzählen
	
\paragraph{March}
Java
	implement version with messaging
	implement concurrent semantic
	
Scala \& Actors
	implement SimTime Actors: have global simulation-time instead of local real-time

Paper
	second refined draft
	
Reading
	Functional programming and simulations
	TODO: genauer alle papers aufzählen
	
\paragraph{April}
No programming: code-freeze

Paper
	refinement
	
Reading
	TODO: genauer alle papers aufzählen
	
\paragraph{May}
Paper
	submitting

Reading
	Functional programming and simulations
	TODO: genauer alle papers aufzählen
	
\paragraph{June}
1st Year Report
	Writing

1st Year oral Exam

Reading
	final stuff for 1st year report
	
\paragraph{July to September}
Prototyping ACE using my Haskell-Library
	bilateral trading
	EDSL
	qualitative modelling
	reasoning 
	
Reading
	ACE models for trading/bartering
	ACE papers
	TODO: genauer alle papers aufzählen
	
\subsubsection{2nd Year: Main Work}
Applying 1st year results, methods and experiences to develop and write main paper to be published in a journal in 3rd year thus in 2nd year the main  work and implementation will be done. The idea is to start from Ionescus Framework \cite{Botta20114025} and build on his paper.

\begin{itemize}
\item Implement Ionescous framework using the methodology developed in 1st year
\item Generalize implementation to market models
\item Learn Agda and dependent types
\item Dig deeper into equilibrium theory 
\item Get into market-microstructure: \cite{LehalleLaruelle2013}, \cite{baker_market_2013}
\item Dig into \textit{emergent properties} of systems. Can they be formalized?
\item Get into basics of Category-Theory \cite{Pierce1991} \cite{spivak_category_2014}
\item Get into basics of Type-Theory (found good lectures on Youtube)

\end{itemize}

\subsubsection{3rd Year: Finalizing, Publishing \& Writing}
I plan to be finished - or nearly finished - at the end of the 3rd year. In this year I will finalize the work of the 2nd year, publish the my main journal paper (and optional fun-papers if possible) and will write down the thesis. \\ To have a bit of distraction and to prevent myself to become too locked in in writing on the thesis I will also work on my optional fun-papers (see below) and hope to at least finish them and maybe publish them - at least I want to present them to 2-3 audiences (e.g. FP Lunch) to test the reaction (especially the Genesis-Paper).

\begin{itemize}
\item Finalize research of 2nd year
\item Publish journal paper
\item Write thesis
\item Work on fun-papers
\end{itemize}

\subsection{Papers}
This is the list of papers I want to work on. The goals of publishing I set for myself are given beside the paper.

\begin{enumerate}
\item \textit{Dynamics of Agent-Based Simulation \& Modelling under different Simulation-Semantics.} - Conference paper
\item \textit{Pure Functional ACE (Catchy title yet to be defined)} - Journal paper
\item \textit{Pure by Nature: A Library for pure Agent-Based Simulation \& Modelling in Haskell} - Journal paper ?
\item \textit{Time in Games: a Tron Light-Cycle Game in Dunai} - Journal/Conference paper
\item \textit{The Genesis According to Computer-Science: Reality as Simulation of Free Will} - Optional publishing
\item \textit{Pure Functional Islamic Design} - Optional publishing
\end{enumerate}

\subsubsection{Dynamics of Agent-Based Simulation \& Modelling under different Simulation-Semantics.}
The first paper which describes how one can implement ABM/S in Haskell and compares the implementation and results to Java and Akka. A major focus are update-strategies, parallelism, reproducibility, reasoning and comparability between the various implementations. 

Actors: The Future in Agent-Based Simulation \& Modelling?
Although the actor-model is quite old (beginning of the 70s) it seems to have a revival both in Erlang in the 90s and now in the Framework Akka (based on Scala). It is one way of organizing highly parallel (and optionally distributed) applications. Also the actor-model is very close to the agent-metaphor where the latter one was strongly inspired by the former one. Thus It would be very interesting to look closer into how the Actor-Model can be utilized to ABM/S as it seems that this has not been properly done yet.

This paper will establish my methodology in using Haskell / pure functional programming in the 2nd year main work.

\subsubsection{Pure Functional ACE (Catchy title yet to be defined)}
Is the main work of the PhD and targeted at publication in a Journal. The exact topic and content will be clarified at the beginning of the 2nd year. Mainly it will describe how to implement Ionescus Framework of Gintis trading model and extend it to a more general Market-Model. It will also give an outlook on implementing it using dependent types.

\subsubsection{Pure by Nature: A Library for pure Agent-Based Simulation \& Modelling in Haskell}
This paper describes the ideas and theory behind the implementation of my ABM/S library "PureAgents" in Haskell.

\subsubsection{Time in Games: a Tron Light-Cycle Game in Dunai}
This paper describes the 2D light-cycle game inspired by the movie Tron implemented in Dunai. It allows to turn back time.

\subsubsection{The Genesis According to Computer-Science: Reality as Simulation of Free Will}
I've always been interested in a deeper meaning behind things so I want to look into the philosophy and future of simulation: why do we simulate, what can we derive from simulations, what does it say that we humans simulate, what will the future of simulation be? \\
I claim that our ability to "simulate" in our mind separates our intelligence from those of the animals and that this is a unique property of humans. Also i think the future of simulation will be that humankind will do its own creation/live (artifical life, conciousness) which allows to accurately simulate a given setting - this of course could have ethical implications. \\
This is fun paper 1.

\subsubsection{Pure Functional Islamic Design}
Inspired by the paper "Functional Geometry" by Peter Henderson I had the idea to come up with a  EDSL for declaratively describing pictures of islamic design which are then rendered using the gloss-library. \\
This is fun paper 2 - from its focus totally unrelated to the PhD topic but still a great opportunity to learn Haskell, to learn to think functional, to learn to design my own EDSL - thus it may be a great paper to pursue even if I won't finish or produce something publishable.

