\section{TODO-List as of \today}

\begin{enumerate}

\item Paper
	\begin{itemize}
    \item 24th Feb: revise and incorporate all issues as noted by Peer and Thorsten (and Thomas Schwarz)
    \item 10th March: implement seq, par \& con in Scala and add to Case-Studies
    \item 10th March: read 2 or 3 paper published on this conference: (essa member-list / namelist / programme -> check google scholar)
    \item 17th March: final version
    \item 24th March: submit 
    \item 31th Match: deadline
    \item If time: maybe look into SIR/SIRS over 1000 steps and report differences
    \end{itemize}
    
\item 5 general research questions
	\begin{itemize}
    \item 2 related to FP
    \item 1 related to integration of FP to ABM/S
    \item 2 related to ABM/S
    \end{itemize}
    
\item ACE
	\begin{itemize}
    \item look into ACE in general
    \item look into agent-based market models
	\item READ NASDAQ MARKET SIMULATION: seems extremely interesting as it builds an ABM for simulating the stock market.
	\end{itemize}
	 
\item Contribution
	\begin{itemize}    
    \item NOT in economics but a METHOD IN COMPUTER SCIENCE
    \item THUS: no research in combining complicated models (e.g. applying budish paper of batch-auctions to ABS model of markets): my direction agent-based market models IN GENERAL and what we can earn from doing that in FP
	\end{itemize}
	
\item Work on 2nd Paper: Is it possible to define a minimal functional specification-language for ABM which is at the same time equals to Haskell code / very close to it? 

\item Start 3rd Paper: does Yampa help us to leverage our initial implementation towards bigger, more complex models with heterogeneous agents?

\item Phd Thesis Title: Foundations of functional ABM/S
	\begin{itemize}
		\item functional specification language
		\item theory of computation in ABM/S: type theoretic and category theoreric foundations of ABM/S
		\item implementing pure functional ABS/M: fusion with FRP
		\item reasoning in functional abm/s
		\item a case study: pure functional ABM/S in ACE
	\end{itemize}

\item Bring PureAgents to Yampa
	\begin{itemize}
		\item embed PureAgentsPar and Seq in Yampa: PureAgentsYampa so we can leverage the power of the EDSL, SFs, continuations,... of Yampa/Dunai.
		\item implement agent monad: PureAgentsMonadic. but what is an Agent-Monad? build a monad to chain actions of the agent and always run inside an agent-monad
		\item embed PureAgentsMonadic in Dunai
		\item problem: so far only agents with same static messagetypes, environment and states, can communicate: the agents are homogenous. how can we implement hetereogenous agents in this library?
		\item implement wait blocking for a message so far. utilize yampas event mechanism?
		\item look into statistics of retries in STM \url{https://hackage.haskell.org/package/stm-stats-0.2.0.0/docs/Control-Concurrent-STM-Stats.html}: nothing revealing, all is running in parallel (although very inefficiently)
	\end{itemize}

\item Thorsten Meeting preparation
	\begin{itemize}
		\item Meinen derzeitigen Stand (Paper und Code) aus Haskell Sicht
		\item Idee für ein Paper: eine Spezifikationssprache für Agent-Based Modelling \& Simulation, die quasi Haskell-Code ist - die Idee ist, der Frage nachgehen inwieweit der Spezifikation-Implementations Gap (mehr oder weniger) geschlossen werden kann, wenn wir das ganze in Haskell mit meinem Ansatz machen. Auch sind reasoning-Techniken hier sicher auch von Interesse.
		\item Idee für ein weiteres Paper: meine derzeitige ABM/S Implementierung in Haskell mit FRP (Yampa) zu verbinden, was dem ganzen sicherlich ordentlich leverage gibt. Die Hypothese: die Richtung die ich eingeschlagen habe, würde sich sowieso nach FRP entwickeln, ausserdem erlaubt es wesentlich komplexere und größere Modelle zu implementieren.
		\item FP Lunch Präsentation: steht immer noch aus, ist aber noch zu früh und das Thema war bisher nicht entsprechend, könnte aber mit dem ersten Paper eine passende Richtung vorgeben.
	\end{itemize}

\item got hint for publishing paper in: ppsm parallel problem solving in nature OR natural computation
	
\item my change my focus from ACE to a different focus - why? because i am not sure if this community is open to the ideas of mine
\item find 1-3 interesting and large model/simulation to implement using the methods developed in Iteration- \& Yampa Paper. don't focus only on ACE but try different fields
	\begin{itemize}
		\item ACE: simulate an economy or markets (e.g. Ionescus implementation of Gintis, simulating the stock-market,...)
		\item Social Simulation e.g. Axelrod Sugarscape? But not very interesting, only 2D Discrete 
		\item Religious studies: so far only altruistic behaviour and population behavioural change but I would be interested in things like: "God as an emergent property of the population, with Gods attributes (e.g. loving/hating, punishing/forgiving) reflecting the attitudes of the population. I think this was the case of Yahweh created by the Israelitic tribes from local influences."
	\end{itemize}

\end{enumerate} 

\section{Future-List as of \today}
\begin{enumerate}
\item Topics \& Issues of Haskell Implementation
	\begin{itemize}
		\item performance unacceptable: 1000 in haskell vs 100.000 in java is a shame on haskell, more should be possible. investigate using profiling both of CPU and memory: \url{http://keera.co.uk/blog/2014/10/15/from-60-fps-to-500/}. strictness \& tail-recursion!	
		\item look into QuickCheck and HPC
		\item implement a general-purpose rendering-frontend with gloss but let the simulation be driven by Yampa/SimulationBackend instead of frontend (not real-time)
	\end{itemize}
	
\item Religious studies in ABS
	\begin{itemize}
		\item God From The Machine
		\item God as an emergent Property of a believer-Crowd: god is a reflection of their beliefs e.g. it is a demiurge or a loving entity,...
		\item Dynamics of Karma and Rebirth
		\item Emergence of a Religion: God experiences and missionaries
	\end{itemize}
	
	\item Religious studies in ABS
	\begin{itemize}
			\item IDEA: what about ABM/S generating sound? could be a perfect example for Yampa due to its signal functions. the sound is the result of interactions of agents which try to generate harmonies and agents trying to create dissonance
		\item IDEA: what about abm/s creating drawings/art? 2d continuous and each agents path is drawn
	\end{itemize}
\end{enumerate} 
