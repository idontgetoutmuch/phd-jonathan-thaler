\section{Executive Summary}
In this paper we look at the very simple social-simulation of \textit{Heroes \& Cowards} invented by \cite{wilensky_introduction_2015} to study the impact of different simulation-semantics on the dynamics of the simulation. By simulation-semantics we understand the different approaches of how to iterate a simulation and we ask whether the dynamics are somewhat stable, change or completely break down under different semantics. We draw parallels to complex dynamic systems for which it is well known that slightly different starting-conditions can lead to completely different dynamics after a few steps. We develop a classification of all potential simulation-semantics and discuss general implementation-considerations independent of the programming language. We then give implementations of each simulation-semantic in varying programming languages and compare the resulting dynamics. Depending on the given semantics we choose out of three languages: Java, Haskell and Scala with Actors where each of them has their strengths and limitations implementing simulation-semantics. We implement only the semantics for which the given language is suited for without abusing it. Thus in Java we focus on object-oriented programming, side-effects and global data, where in Haskell we focus on pure functional programming with local-only immutable data, explicit dataflow, higher order functions and recursion, and in Scala with Actors we put emphasis on a mixed-paradigm approach and the usage of Actors as defined by \cite{agha_actors:_1986}. \\
It is important to note that the implementations in Haskell present a novel, pure functional approach to ABM/S which strengths are the declarative style of programming and the strong static type-system. This allows to reason about a program and implement embedded domain-specific languages (EDSL) where ideally the distinction between a formal specification and the implementation disappears. We present the EDSL implemented for this Model and investigate what and how we can reason about properties and dynamics of our simulation and of ABM/S in general having the EDSL, the types and our program at hand. \\
Thus the main contributions of this paper are fourfold. First it establishes a terminology for speaking about and classifying simulation-semantics, second it gives a general framework to discuss dynamics of an ABM/S under different simulation-semantics, third it discusses the suitability of three very different programming-languages to implement the various simulation-semantics with a special emphasis on pure functional approach with comparison to state-of-the-art OO and fourth it looks into the possibility and power of reasoning about properties and dynamics of an ABM/S with specific simulation-semantics implemented in the pure functional language Haskell.

