\section{SIRS Specification}
This section provides the specification of the SIR model.

\subsection{SIRS Model}
TODO: explanation of it in natural language 
 
\subsection{Formal Specification}
TODO: first the formal 'mathematical' specification, then the translation to haskell-code

First we declare the 5 tuples of our SirAgent. It has a domain-specific state, message-protocol and transformer-function as can be seen in the types
SirAgent = <Aid, sirState, sirMsg, mbox, sirTf>

We define the domain-specific state to be the following. Each Agent is in one of the 3 SIR states and knows its duration it has been in this state.
SIR = { Susceptible , Infected, Recovered }
sirState = (state element of SIR, durInState :: Double )

We define the domain-specific message-protocol to cover the possibilities of making contact.
sirMsg = { ContactSusceptible, ContactInfected, ContactRecovered }

Next we declare our transformer-function and handle all types of messages here. Note that in the case of domain-specific messages we only handle ContactInfected as in the other cases the model-specification in natural language does not cover these cases: the agent is not affected or changed in these contacts.
sirTf :: SirAgent -> (Aid, Msg sirMsg ) -> SirAgent
sirTf a (_, Dt d) 				= sirTimeStep a 
sirTf a (_, ContactInfected) 	= infectedContact a
sirTf a _ 						= a 

In each discrete time-step which happens in the sir-model as specified in the natural-language description an agent first recovers and then makes contact with neighbours
sirTimeStep :: SirAgent -> SirAgent
sirTimeStep = makeContact . recover

Making contact with other neighbours is straight-forward, using send 
makeContact :: SirAgent -> SirAgent
makeContact a 
	| is a Susceptible 	= send a (receiver, ContactSusceptible)
	| is a Infected 	= send a (receiver, ContactInfected)
	| is a Recovered 	= send a (receiver, ContactRecovered)
		where
			receiver = randomNeighbour
			
We introduced the is function which returns true if the SirAgent is in the given state
is :: SirAgent -> SIR -> Boolean
is a s = (sirState a) == s 

Now we define how an Agent recovers. An Agent can only recover if it is actually infected, otherwise there is no need to recover
recover :: SirAgent -> SirAgent
recover a 
	| is a Infected = recoverInfected a
	| otherwise 	= a
	
If an agent is indeed infected, it has recovered when it has spent the given time-steps in the infection-state
recoverInfected :: SirAgent -> SirAgent
recoverInfected a = if (durInState a >= I) then
						a { state = Recovered, durInState = 0.0 }
						else
							a { durInState = durInState' }
	where
		durInState' = (durInState a) + 1.0
		
	
Finally we need to describe how an agent is infected when having contact with an other agent which is infected
infect :: SirAgent -> SirAgent
infect a
	| yes = a {state = Infected, durInState = 0.0}
	| otherwise = a
		where
			yes = p >= uniformRandom [0.0, 1.0]