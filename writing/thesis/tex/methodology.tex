\chapter{Methodology}
This chapter introduces the methodology, used in the following chapters:

1. time-driven vs. event-driven ABS
2. time-driven: start with update-strategies
3. present pure functional approach
4. how can we implement the 4 update-strategies in our approach
5. additional research: interaction between agents
6. additional research: event-driven ABS

\begin{itemize}
	\item Defining and introducing Agent-Based Simulation (ABS) (History, ABS vs. MAS, examples, event- vs. time-driven).
	\item Introduce established implementation approaches to ABS (Frameworks: NetLogo, Anylogic, Libraries: RePast, DesmoJ, Programming: Java, Python, Correctness: ad-hoc, manual testing, test-driven development)
	\item Introduction Verification \& Validation (V \& V in the context of ABS).
	\item Introduction to functional Programming in Haskell (functions, types, recursion, algebraic data-types, higher-order functions, continuations, Define and explain side-effects and purity: monads, different types of effects, explain IO and that it is of fundamental importance to avoid it in our research).
	\item Introduction to dependent types (Example, Equality as Type, Philosophical Foundations: Constructive mathematics)
\end{itemize}
50\%

TODO: need carefully establish why Haskell:
\begin{itemize}
	\item Rich Feature-Set - it has all fundamental concepts of the pure functional programming paradigm of which we explain the most important below.
	\item Real-World applications - the strength of Haskell has been proven through a vast amount of highly diverse real-world applications \cite{hudak_history_2007}, is applicable to a number of real-world problems \cite{osullivan_real_2008} and has a large number of libraries available \footnote{\url{https://wiki.haskell.org/Applications_and_libraries}}.
	\item Modern - Haskell is constantly evolving through its community and adapting to keep up with the fast changing field of computer science. Further, the community is the main source of high-quality libraries.
	\item It is as closest to pure functional programming, as in the lambda-calculus, as we want to get. Other languages are often a mix of paradigms and soften some criteria / are not strictly functional and have different purposes. Also Haskell is very strong rooted in Academia and lots of knowledge is available, especially at Nottingham, 
		Lisp / Scheme was considered because it was the very first functional programming language but deemed to be not modern enough with lack of sufficient libraries. Also it would have given the 
		Erlang was considered in prototyping and allows to map the messaging concept of ABS nicely to a concurrent language but was ultimately rejected due to its main focus on concurrency and not being purely functional.
		Scala was considered as well and has been used in the research on the Art Of Iterating paper but is not purely functional and can be also impure.
\end{itemize}

TODO: need to present the established approach to ABS: object-oriented, java/python, unit-tests, test-driven, RePast, AnyLogic, NetLogo
