\section{Introduction}

Important: The strength of simulations is to put hypotheses to tests: The hypothesis-simulation-refinement cycle: 

All implementations of ABM/S models must solve two problems: 1. the model/data/ part: how agents can be represented and 2. the dynamic part: how the simulation is stepped

\section{Agent-Based Modelling and Simulation (ABM/S)}
ABM/S is a method of modelling and simulating a system where the global behaviour may be unknown but the behaviour and interactions of the parts making up the system is of knowledge (Wooldrige, M. (2009). An Introduction to MultiAgent Systems. John Wiley \& Sons). Those parts, called agents, are modelled and simulated out of which then the aggregate global behaviour of the whole system emerges. Thus the central aspect of ABM/S is the concept of an Agent which can be understood as a metaphor for a pro-active unit, able to spawn new Agents, and interacting with other Agents in a network of neighbours by exchange of messages. The implementation of Agents can vary and strongly depends on the programming language and the kind of domain the simulation and model is situated in.

\section{Functional programming}
TODO: read \cite{backus_can_1978}

The state-of-the-art approach to implementing Agents are object-oriented methods and programming as the metaphor of an Agent as presented above lends itself very naturally to object-orientation (OO). The author of this thesis claims that OO in the hands of inexperienced or ignorant programmers is dangerous, leading to bugs and hardly maintainable and extensible code. The reason for this is that OO provides very powerful techniques of organising and structuring programs through Classes, Type Hierarchies and Objects, which, when misused, lead to the above mentioned problems. Also major problems, which experts face as well as beginners are 1. state is highly scattered across the program which disguises the flow of data in complex simulations and 2. objects don’t compose as well as functions. The reason for this is that objects always carry around some internal state which makes it obviously much more complicated as complex dependencies can be introduced according to the internal state.
All this is tackled by (pure) functional programming which abandons the concept of global state, Objects and Classes and makes data-flow explicit. This then allows to reason about correctness, termination and other properties of the program e.g. if a given function exhibits side-effects or not. Other benefits are fewer lines of code, easier maintainability and ultimately fewer bugs thus making functional programming the ideal choice for scientific computing and simulation and thus also for ACE. A very powerful feature of functional programming is Lazy evaluation. It allows to describe infinite data-structures and functions producing an infinite stream of output but which are only computed as currently needed. Thus the decision of how many is decoupled from how to (Hughes, J. (1989). Why functional programming matters. Comput. J., 32(2):98–107.).
The most powerful aspect using pure functional programming however is that it allows the design of embedded domain specific languages (EDSL). In this case one develops and programs primitives e.g. types and functions in a host language (embed) in a way that they can be combined. The combination of these primitives then looks like a language specific to a given domain, in the case of this thesis ACE. The ease of development of EDSLs in pure functional programming is also a proof of the superior extensibility and composability of pure functional languages over OO (Henderson P. (1982). Functional Geometry. Proceedings of the 1982 ACM Symposium on LISP and Functional Programming.).
One of the most compelling example to utilize pure functional programming is the reporting of Hudak (Hudak P., Jones M. (1994). Haskell vs. Ada vs. C++ vs. Awk vs. ... An Experiment in Software Prototyping Productivity. Department of Computer Science, Yale University.)  where in a prototyping contest of DARPA the Haskell prototype was by far the shortest with 85 lines of code. Also the Jury mistook the code as specification because the prototype did actually implement a small EDSL which is a perfect proof how close EDSL can get to and look like a specification.

Functional languages can best be characterized by their way computation works: instead of \textit{how} something is computed, \textit{what} is computed is described. Thus functional programming follows a declarative instead of an imperative style of programming. The key points are:
\begin{itemize}
\item No assignment statements - variables values can never change once given a value.
\item Function calls have no side-effect and will only compute the results - this makes order of execution irrelevant, as due to the lack of side-effects the logical point in \textit{time} when the function is calculated within the program-execution does not matter.
\item higher-order functions
\item lazy evaluation
\item Looping is achieved using recursion, mostly through the use of the general fold or the more specific map.
\item Pattern-matching
\end{itemize}

This alone does not really explain the \textit{real} advantages of functional programming and one must look for better motivations using functional programming languages. One motivation is given in \cite{hughes_why_1989} which is a great paper explaining to non-functional programmers what the significance of functional programming is and helping functional programmers putting functional languages to maximum use by showing the real power and advantages of functional languages. The main conclusion is that \textit{modularity}, which is the key to successful programming, can be achieved best using higher-order functions and lazy evaluation provided in functional languages like Haskell. \cite{hughes_why_1989} argues that the ability to divide problems into sub-problems depends on the ability to glue the sub-problems together which depends strongly on the programming-language and \cite{hughes_why_1989} argues that in this ability functional languages are superior to structured programming.

TODO: comparison of functional and object-oriented programming. My points are:
\begin{itemize}
\item The way state can be changed and treated - distributed over multiple objects - is often very difficult to understand.
\item Inheritance is a dangerous thing if not used with care because inheritance introduces very strong dependencies which cannot be changed during runtime anymore.
\item Objects don't compose very well: \url{http://zeroturnaround.com/rebellabs/why-the-debate-on-object-oriented-vs-functional-programming-is-all-about-composition/}
\item (Nearly) impossible to reason about programs
\end{itemize}

In conclusion the upsides of functional programming as opposed to OO are:
\begin{itemize}
\item Much more explicit flow of data \& control
\item Much better compose-able
\item Much better parallelism
\end{itemize}


\subsection{Agent-definition}
An Agent is a metaphor for a pro-active unit, able to spawn new Agents, and interacting with other Agents in a network of neighbours by exchange of messages. The implementation of Agents can vary and strongly depends on the programming language and the kind of domain the simulation and model is situated in. This paper looks at how Agents can be implemented in various languages using the simple Heroes \& Cowards model (see below). The following implementations are discussed:

\begin{enumerate}
\item Java, Object-Oriented
\item Haskell, Pure Functional
\item Akka, Mixed Functional with Actors
\end{enumerate}

\subsection{The Model: Heroes and Cowards}
One starts with a crowd of Agents where each Agents is positioned \textit{randomly} in a continuous 2D-space. Each of the Agents then selects \textit{randomly} one friend and one enemy (except itself in both cases) and decided with a given probability whether the Agent acts in the role of a "Hero" or a "Coward" - friend, enemy and role don't change after the initial set-up. Now the simulation can start: in each step the Agent will move a given distance towards a given point. If the Agent is in the role of a "Hero" this point will be the half-way distance between the Agents friend and enemy - the Agent tries to protect the friend from the enemy. If the Agent is acting like a "Coward" it will try to hide behind the friend also the half-way distance between the Agents friend and enemy, just in the opposite direction. \\
Note that this simulation is determined by the random starting positions, random friend \& enemy selection, random role selection and number of agents. Note also that during the simulation-stepping no randomness is mentioned in the model and given the initial random set-up, the simulation-\textit{model} is completely deterministic - whether this is the case for the implementations is another question, not relevant to the model.

TODO: add random-noise with a configurable strength to direction: is closer to reality and also agents will never stop completely - may result in completely different pattern

The most obvious shortcomings of this model are its simplicity but that was chosen intentionally to prevent the research of the methods to be cluttered with too many subtle details of the model - also young children can easily understand how this model works and thus one does not need to constantly explain subtle implementation details because everything is pretty obvious. 

\subsubsection{Extension 1: World}

The coordinates calculated by the agents are \textit{virtual} ones ranging between 0.0 and 1.0. This prevents us from knowing the rendering-resolution and polluting code which has nothing to do with rendering with these implementation-details. Also this simulation could run without rendering-output or any rendering-frontend thus sticking to virtual coordinates is also very useful regarding this (but then again: what is the use of this simulation without any visual output=

\begin{itemize}
\item Infinite: movement is unrestricted.
\item Clipping: coordinates are clipped at 0.0 and 1.0 
\item Wraparound: coordinates are wrapped around to 0.0 when reaching 1.0. Will lead pursuing friends to change direction apruptly when wrapping around.
\end{itemize}

\subsubsection{Extension 2: Random-Noise}


\section{Visualization}
Render all
Render cowards only
Render heroes only

\section{Implementations}
\subsection{ABM/S Frameworks}
NetLogo
AnyLogic
ReLogo

\subsection{Object-Oriented in Java}
Mutable / Immutable
Side-Effects all over the place?

TODO: show how one can program more in a functional style in Java.

\subsection{Pure Functional in Haskell}
No framework
Yampa
Gloss
Aivika

\subsection{Multi-Paradigm Functional in Scala}
