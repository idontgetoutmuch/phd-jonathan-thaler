\section{Introduction}
Today simulations are at the very heart of many sciences. They allow to put hypotheses to test by building a model which abstracts from reality, keeping only the important and relevant details, and then bringing this model to life in simulation. Based on the results shown by the dynamics, previously formulated hypotheses can be verified or falsified resulting in a formulate-simulate-refine cycle. \\
The meaning of simulating a model can be understood as calculating the dynamics of a (model of a) system over time thus the state of the system at time t depends on the state of the system at time t - epsilon. Here we only consider simulations in a computer-system (TODO: are there simulations NOT in a computer?), which is an inherently discrete system which poses us with the question of how to represent time which seems linear and continuously flowing to us in reality (NOTE: this may not be physically the case but for our considerations this should be a good approximation). Being in a discrete system, of course implies that time has to be discretised as well and there are two ways of doing it: discrete and continuous where in discrete case time advances in steps of the natural numbers and where in the continuous case time advances in steps of real-numbers. Note that in both cases the system is iterated in steps where only the \textit{numerical type} of the input to the time-dependent functions differs. Thus a simulation in a computer can be understood as an iteration over a model for a given number of steps where each step advances time by dt (either discrete or continuous) and, based on the previous model-state, producing an updated model-state which again becomes the input for the next step. Thus in each simulation we have three inputs: 1. the model, 2. number of steps, 3. time dt. There are of course different models and types of simulations and in this paper we will focus on one particular one: agent-based, which will be described next.

\subsection{Agent-Based Modelling and Simulation (ABM/S)}
ABM/S is a method of modelling and simulating a system where the global behaviour may be unknown but the behaviour and interactions of the parts making up the system is of knowledge \cite{wooldridge_introduction_2009}. Those parts, called Agents, are modelled and simulated out of which then the aggregate global behaviour of the whole system emerges. Thus the central aspect of ABM/S is the concept of an Agent which can be understood as a metaphor for a pro-active unit, able to spawn new Agents, and interacting with other Agents in a network of neighbours by exchange of messages. The implementation of Agents can vary and strongly depends on the programming language and the kind of domain the simulation and model is situated in. Whereas the majority of ABM/S are implemented in object-oriented (OO) languages e.g. Java, C++, this paper focuses on functional ones.

\subsection{Why Functional programming?}
Object-oriented (OO) programming is the current state-of-the-art method used in implementing ABM/S due to the natural way of mapping concepts and models of ABM/S to an OO-language. Although this dominance in the field we claim that OO has also its serious drawbacks:

\begin{itemize}
\item Mutable State is distributed over multiple objects which is often very difficult to understand, track and control.
\item Inheritance is a dangerous thing if not used properly and with care because it introduces very strong dependencies which cannot be changed during runtime any-more.
\item Objects don't compose very well due to their internal (mutable) state (note that we are aware that there is the concept of immutable objects which are becoming more and more popular but that does not solve the fundamental problem.
\item It is (nearly) impossible to reason about programs.
\end{itemize}

We claim that these drawbacks are non-existent in pure functional programming like Haskell due to the nature of the functional approach. To give an introduction into functional programming is out of scope of this paper but we refer to the classical paper of \cite{hughes_why_1989} which is a great paper explaining to non-functional programmers what the significance of functional programming is and helping functional programmers putting functional languages to maximum use by showing the real power and advantages of functional languages. The main conclusion of this classical paper is that \textit{modularity}, which is the key to successful programming, can be achieved best using higher-order functions and lazy evaluation provided in functional languages like Haskell. \cite{hughes_why_1989} argues that the ability to divide problems into sub-problems depends on the ability to glue the sub-problems together which depends strongly on the programming-language and \cite{hughes_why_1989} argues that in this ability functional languages are superior to structured programming.

\subsection{The Model: Heroes \& Cowards}
To study various properties of implementations of ABM/S we select the very simple model \textit{Heroes \& Cowards} from social-simulation invented by \cite{wilensky_introduction_2015}. Although it is very simple, it will prevent the research of the methods to be cluttered with too many subtle details of the model thus focusing on the methods and implementation than rather on the model itself. \\
One starts with a crowd of Agents where each Agent is positioned \textit{randomly} in a continuous 2D-space. Each of the Agents then selects \textit{randomly} one friend and one enemy (except itself in both cases) and decides with a given probability whether the Agent acts in the role of a "Hero" or a "Coward" - friend, enemy and role don't change after the initial set-up. Now the simulation can start: in each step the Agent will move a given distance towards a given point. If the Agent is in the role of a "Hero" this point will be the half-way distance between the Agents friend and enemy - the Agent tries to protect the friend from the enemy. If the Agent is acting like a "Coward" it will try to hide behind the friend also the half-way distance between the Agents friend and enemy, just in the opposite direction. \\
The world this model is situated in is restricted by borders in the form of a rectangle: the agents cannot move out of it and will be clipped against the border if the calculation would end them up outside. \\
Note that this simulation is determined by the random starting positions, random friend \& enemy selection, random role selection and number of agents. Note also that during the simulation-stepping no randomness is mentioned in the model and given the initial random set-up, the simulation-\textit{model} is completely deterministic - whether this is the case for the implementations is another question, not relevant to the model. 

\subsubsection{Extension 1: World-Types}
We extend the model by introducing 2 additional world-types: Infinite and Wrapping thus ending up with 3 World-Types:

\begin{enumerate}
\item Border - Agents cannot move out of the restricted rectangle and are clipped at the border. This is the world-type of the original model.
\item Infinite - Agents can move unrestricted.
\item Wrapping - Same as Border but when crossing border the Agents "teleport" to the opposite side of the world thus the world folds back unto itself in 2D.
\end{enumerate}

\subsubsection{Extension 2: Random-Noise}
The original model is completely deterministic after the initial set-up. We add the possibility for optional random noise at two points: the distance / position when hiding/protecting can be made subject to random noise in a given range as well as the width of the step the agent makes when moving towards the hiding/protecting position.

\subsection{Other Models: SIRS, Wildfire and Schelling Segregation}
We will also look into other classic Models of ABM/S but not with the same focus as \textit{Heroes \& Cowards} and just to show that it is very easy to implement them using our ABM/S library in Haskell.

\subsubsection{SIRS}
Simulates the spreading of a disease throughout a population.

\subsubsection{Wildfire}
Simulates a Wildfire on a discrete 2D-Grid where each cell corresponds to a part of a forest with varying amount of burnable wood. The wildfire is then ignited at some random cell and spreads over the whole 2D-Grid by igniting neighbouring cells which burn for a duration proportional to the amount of burnable woods in this cell.

\subsubsection{Schelling Segregation}
TODO: describe

\subsection{Contributions}
When using pure functional instead of the state-of-the-art OO programming we need to find other ways of representing Agents and implement an ABS. This is the main contributions of this paper in form of an implementation of a library for ABM/S implemented in pure Haskell as so far no proper library which meets the requirements for ABM/S exists in Haskell (see Related Research). Also the paper compares the library to implementations with state-of-the-art ABM/S Frameworks: NetLogo, AnyLogic and ReLogo. Also comparisons are drawn with an OO solution in Java. Of special interest is the comparison to the multi-paradigm functional solution in Scala \& Actors as this is the method closest to the Haskell-Library implemented. \\
We claim that our library is well suited for ABM/S in the field of scientific computing as it provides an array of benefits over the methods compared with: explicit data-flow, EDSL, higher order functions, avoidance of true randomness, reasoning, pattern matching,... TODO: mention ABM/S Frameworks, TODO: mention OO solutions. Although the Scala \& Actors version might be a competitor in this field, it is not as its strength lies in the development of highly distributed high-availability applications e.g. Twitter and not in simulation where we need fine-control over (global) time and interactions.
The other main contribution is to systematically look at the Actor-Model in implementing ABM/S.