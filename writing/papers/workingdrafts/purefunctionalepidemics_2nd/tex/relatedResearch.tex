\section{Related Work}
\label{sec:related_work}
The amount of research on using pure functional programming with Haskell in the field of ABS has been moderate so far. Most of the papers are more related to the field of Multi Agent Systems and look into how agents can be specified using the belief-desire-intention paradigm \cite{de_jong_suitability_2014, sulzmann_specifying_2007, jankovic_functional_2007}.

A library for Discrete Event Simulation (DES) and System Dynamics (SD) in Haskell called \textit{Aivika 3} is described in the technical report \cite{sorokin_aivika_2015}. It is not pure, as it uses the IO Monad under the hood and comes only with very basic features for event-driven ABS, which allows to specify simple state-based agents with timed transitions.

Using functional programming for DES was discussed in \cite{jankovic_functional_2007} where the authors explicitly mention the paradigm of FRP to be very suitable to DES.

A domain-specific language for developing functional reactive agent-based simulations was presented in \cite{vendrov_frabjous:_2014}. This language called FRABJOUS is human readable and easily understandable by domain-experts. It is not directly implemented in FRP/Haskell but is compiled to Yampa code which they claim is also readable. This supports that FRP is a suitable approach to implement ABS in Haskell. Unfortunately, the authors do not discuss their mapping of ABS to FRP on a technical level, which would be of most interest to functional programmers.

Object-oriented programming (OOP) and simulation have a long history together as OOP emereged out of Simula 67 \cite{dahl_birth_2002} which was created for simulation purposes. Simula 67 already supported discrete event simulation (DES) and was highly influential for todays object-oriented languages TODO: more details. Although the language was important and influential, in our research we look into different approaches, orthogonal to the existing object-oriented concepts.

Lustre is a formally defined, declarative and synchronous dataflow programming language for programming reactive systems \cite{halbwachs_synchronous_1991}. TODO: more details. It seems that this language has solved already a big part of what we tried to achieve. Still our contribution and research goes into a very different direction and can be understood as a first stepping stone towards a more formal way of expressing ABS towards a dependently typed approach.