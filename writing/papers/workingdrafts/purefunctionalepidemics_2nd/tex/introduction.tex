\section{Introduction}
The traditional approach to Agent-Based Simulation (ABS) has so far always been object-oriented techniques, due to the influence of the seminal work of Epstein et al \cite{epstein_growing_1996} in which the authors claim "[..] object-oriented programming to be a particularly natural development environment for Sugarscape specifically and artificial societies generally [..]" (p. 179). This work established the metaphor in the ABS community, that \textit{agents map naturally to objects} \cite{north_managing_2007} which still holds up today.

In this paper we challenge this metaphor and explore ways of approaching ABS in a pure (lack of implicit side-effects) functional way using Haskell. By doing this we expect to leverage the benefits of pure functional programming \cite{hudak_history_2007}: higher expressivity through declarative code, being polymorph and explicit about side-effects through monads, more robust and less susceptible to bugs due to explicit data flow and lack of implicit side-effects.

As use case we introduce the SIR model of epidemiology with which one can simulate epidemics, that is the spreading of an infectious disease through a population, in a realistic way.

Over the course of four steps, we derive all necessary concepts required for a full agent-based implementation. We start from a very simple solution running in the Random Monad which has all general concepts already there and then refine it in various ways, making the transition to Functional Reactive Programming (FRP) \cite{wan_functional_2000} and to Monadic Stream Functions (MSF) \cite{perez_functional_2016}.

The aim of this paper is to show how ABS can be implemented in \textit{pure} Haskell and what the benefits and drawbacks are. By doing this we give the reader a good understanding of what ABS is, what the challenges are when implementing it and how we solve these in our approach.

The contributions of this paper are:
\begin{itemize}
	\item We present an approach to agent-based simulation using \textit{declarative} analysis with FRP in which we systematically introduce the concepts of ABS to \textit{pure} functional programming in a step-by-step approach. Also this work presents a new field of application to FRP as to the best of our knowledge the application of FRP to ABS (on a technical level) has not been addressed before. The result of using FRP allows expressing continuous time-semantics in a very clear, compositional and declarative way, abstracting away the low-level details of time-stepping and progress of time within an agent.
	
	\item Our approach can guarantee reproducibility already at compile time, which means that repeated runs of the simulation with the same initial conditions will always result in the same dynamics, something highly desirable in simulation in general. This can only be achieved through purity, which guarantees the absence of implicit side-effects which allows to rule out non-deterministic influences at compile time through the strong static type system. This only becomes possible in pure functional programming where we can control the side-effects and can program side-effect polymorph, something not possible with traditional object-oriented approaches. Further, through purity and the strong static type system we can rule out important classes of run-time bugs e.g. related to dynamic typing, and the lack of implicit data-dependencies which are common in traditional imperative object-oriented approaches.
\end{itemize}

In Section \ref{sec:background} we define agent-based simulation, introduce functional reactive programming, arrowized programming and monadic stream functions, because our approach builds heavily on these concepts. In Section \ref{sec:sir_model} we introduce the SIR model of epidemiology as an example model to explain the concepts of ABS. The heart of the paper is Section \ref{sec:functional_approach} in which we derive the concepts of a pure functional approach to ABS in four steps, using the SIR model. Section \ref{sec:related_work} discusses related work. Finally, we draw conclusions and discuss issues in Section \ref{sec:conclusions} and point to further research in Section \ref{sec:further_research}.

