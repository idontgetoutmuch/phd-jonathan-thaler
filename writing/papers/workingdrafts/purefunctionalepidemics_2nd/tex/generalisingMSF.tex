\subsection{Generalising to Monadic Stream Functions}
\label{sec:generalising_msfs}
A part of the library Dunai is BearRiver, a wrapper which re-implements Yampa on top of Dunai, which should allow us to easily replace Yampa with MSFs. This will enable us to run arbitrary monadic computations in a signal function, which we will need in the next step when adding an environment.

\subsubsection{Identity Monad}
We start by making the transition to BearRiver by simply replacing Yampas signal function by BearRivers' which is the same but takes an additional type parameter \textit{m} indicating the monadic context. If we replace this type-parameter with the Identity Monad we should be able to keep the code exactly the same, except from a few type-declarations, because BearRiver re-implements all necessary functions we are using from Yampa. We simply re-define our agent signal function, introducing the monad stack our SIR implementation runs in:

\begin{HaskellCode}
type SIRMonad    = Identity
type SIRAgent    = SF SIRMonad [SIRState] SIRState
\end{HaskellCode}

\subsubsection{Random Monad}
Using the Identity Monad does not gain us anything but it is a first step towards a more general solution. Our next step is to replace the Identity Monad by the Random Monad which will allow us to get rid of the RandomGen arguments to our functions and run the whole simulation within the Random Monad \textit{again} just as we started but now with the full features functional reactive programming.
We start by re-defining the SIRMonad and SIRAgent:

\begin{HaskellCode}
type SIRMonad g = Rand g
type SIRAgent g = SF (SIRMonad g) [SIRState] SIRState
\end{HaskellCode}

The question is now how to access this Random Monad functionality within the MSF context. For the function \textit{occasionally}, there exists a monadic pendant \textit{occasionallyM} which requires a MonadRandom type-class. Because we are now running within a MonadRandom instance we simply replace \textit{occasionally} with \textit{occasionallyM}. 

\subsubsection{Discussion}
So far making the transition to MSFs does not seem as compelling as making the move from the Random Monad to FRP in the beginning. Running in the Random Monad within FRP is convenient but we could achieve the same with passing RandomGen around as we already demonstrated. In the next step we introduce the concept of a read/write environment which we realise using a StateT monad. This will show the real benefit of the transition to MSFs as without it, implementing a general environment access would be quite cumbersome.