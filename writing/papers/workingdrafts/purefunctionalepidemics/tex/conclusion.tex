%\section{Conclusions and Further Research}
\section{Conclusions}
\label{sec:conclusions}

Our approach is radically different from traditional approaches in the ABS community. First it builds on the already quite powerful FRP paradigm. Second, due to our hybrid approach, it forces one to think properly of time-semantics of the model, how small $\Delta t$ should be, whether one needs super-sampling or not and how many samples one should take. Third it requires to think about agent-interactions in a new way instead of being just method-calls. % Also due to the underlying nature and motivation of FRP and Yampa, agents can be seen as signals which are generated and consumed by a signal function. If an agent does not change, the output signal should stay constant and if the agent changes e.g. by sending a message or changing its state, the output signal should change as well. A dead agent then should have no signal at all. In all but the very first step of our approach, the agents are completely time-dependent which means they will not act when time does not advance. Thus when running our simulation with $\Delta t = 0$ the dynamics stay constant and won't change.

Because no part of the simulation runs in the IO Monad and we do not use unsafePerformIO we can rule out a serious class of bugs caused by implicit data-dependencies and side-effects which can occur in traditional imperative implementations. Also we can statically guarantee the reproducibility of the simulation. Within the agents there are no side effects possible which could result in differences between same runs. Every agent has access to its own random-number generator or the Random Monad, allowing randomness to occur in the simulation but the random-generator seed is fixed in the beginning and can never be changed within an agent. This means that after initialising the agents, which \textit{could} run in the IO Monad, the simulation itself runs completely deterministicly. This is also ensured through fixing the $\Delta t$ and not making it dependent on the performance of e.g. a rendering-loop or other system-dependent sources of non-determinism as described by \cite{perez_testing_2017}. Also by using FRP we gain all the benefits from it and can use research on testing, debugging and exploring FRP systems \cite{perez_testing_2017}, \cite{perez_back_2017}.

\subsection*{Issues}
Unfortunately, the hybrid approach of SD/ABS amplifies the performance issues of agent-based approaches, which requires much more processing power compared to SD, because each agent is modelled individually in contrast to aggregates in SD \cite{macal_agent-based_2010}. With the need to sample the system with high frequency, this issue gets worse. We haven't investigated how to optimize the performance by using efficient functional data structures, hence in the moment our program performs much worse than an imperative implementation that exploits in-place updates.

Despite the strengths and benefits we get by leveraging on FRP, there are errors that are not raised at compile-time, e.g. we can still have infinite loops and run-time errors. This was for example investigated by \cite{sculthorpe_safe_2009} who use dependent types to avoid some runt-time errors in FRP. We suggest that one could go further and develop a domain specific type system for FRP that makes the FRP based ABS more predictable and that would support further mathematical analysis of its properties.

We can conclude that the main difficulty of a pure functional approach evolves around the communication and interaction between agents. This is straight-forward in object-oriented programming where it is achieved using method-calls mutating the internal state of the agent but which comes at the cost of a new class of bugs due to implicit data flow. In pure functional programming these data flows are explicit but our current approach of feeding back the states of all agents as inputs is not very general and we have added further mechanisms of agent interaction which we had to omit due to lack of space. We leave the search for a better and more flexible mechanism of agent communication in pure functional programming for further research.

We started with high hopes for the pure functional approach and hypothesized that it will be truly superior to existing traditional object-oriented approaches but we come to the conclusion that this is not so. The single real benefit is the lack of implicit side-effects and reproducibility guaranteed at compile time. Still, our research was not in vain as we see it as an intermediary step towards using dependent types. Moving to dependent types would pose a unique benefit over the object-oriented approach and should allow us to express and guarantee properties at compile time which is not possible with imperative approaches. We leave this for further research.