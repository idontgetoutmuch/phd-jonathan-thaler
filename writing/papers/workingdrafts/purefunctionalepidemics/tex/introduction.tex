\section{Introduction}
The traditional approach to Agent-Based Simulation (ABS) has so far always been object-oriented techniques due to the influence of the seminal work \cite{epstein_growing_1996} in which the authors claim that "[..] object-oriented programming to be a particularly natural development environment for Sugarscape specifically and artificial societies generally [..]" (p. 179). This work established the metaphor in the ABS community, that \textit{agents map naturally to objects} \cite{north_managing_2007} which still holds up today.
In this paper we fundamentally challenge this metaphor and explore ways of approaching ABS in a pure functional way using Haskell. By doing this we expect to leverage the benefits of pure functional programming \cite{hudak_history_2007}: higher expressivity through declarative code, being polymorph and explicit about side-effects through monads, more robust and less susceptible for bugs due to explicit data flow and lack of implicit side-effects.
As use-case we introduce the simple SIR model of epidemiology with which one can simulate epidemics in a realistic way and over the course of five steps we derive all necessary concepts required for a full agent-based implementation. We start from a very simple solution running in the Random Monad which has all general concepts already there but then refine it in various ways, making the transition to Functional Reactive Programming (FRP) \cite{wan_functional_2000} and to Monadic Stream Functions (MSF) \cite{perez_functional_2016}.
The aim of this paper is to show how ABS can be done in Haskell and what the benefits and drawbacks are. By doing this we give the reader a good understanding of what ABS is, what the challenges are when implementing it and how we solve these in our approach. We then discuss details which must be paid attention to in our approach and its benefits and drawbacks. For the first time concepts of ABS are explored step-by-step in a pure functional way in Haskell where the contribution is a novel, hybrid approach to agent-based simulation, emphasising robustness and continuous time-semantics which is quite different from traditional object-oriented approaches.