\section{Defining Agent-Based Simulation}
\label{sec:defining_abs}

Agent-Based Simulation (ABS) is a methodology to model and simulate a system where the global behaviour may be unknown but the behaviour and interactions of the parts making up the system is of knowledge. Those parts, called agents, are modelled and simulated out of which then the aggregate global behaviour of the whole system emerges. So the central aspect of ABS is the concept of an agent which can be understood as a metaphor for a pro-active unit, situated in an environment, able to spawn new agents and interacting with other agents in some neighbourhood by exchange of messages. 
We informally assume the following about our agents \cite{siebers_introduction_2008}, \cite{wooldridge_introduction_2009}, \cite{macal_everything_2016}:

%\cite{dawson_opening_2014}, \cite{siebers_discrete-event_2010}

\begin{itemize}
	\item They are uniquely addressable entities with some internal state over which they have full, exclusive control.
	\item They are pro-active which means they can initiate actions on their own e.g. change their internal state, send messages, create new agents, terminate themselves.
	\item They are situated in an environment and can interact with it.
	\item They can interact with other agents which are situated in the same environment by means of messaging.
\end{itemize} 