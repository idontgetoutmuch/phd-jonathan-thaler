\subsection{Generalising to Monadic Stream Functions}
TODO: write a bit introductory words for this subsection

\subsubsection{Identity Monad}
We start by making the transition to BearRiver by simply replacing Yampas signal-function by BearRivers which is the same but takes an additional type-parameter \textit{m} which indicates the monad. If we replace this type-parameter with the identity monad we should be able to keep the code exactly the same, except from a few type-declarations, because BearRiver re-implements all necessary functions we are using from Yampa \footnote{This was not quite true at the time we wrote this paper, where \textit{occasionally}, \textit{noiseR} and \textit{dpSwitch} were missing. We simply implemented these functions and created a pull request using git.}.
We start by re-defining our general agent signal-function, introducing the monad (stack) our SIR implementation runs in and the agents signal-function:

\begin{minted}[fontsize=\footnotesize]{haskell}
type Agent m o d = SF m (AgentIn d) (AgentOut o d)
type SIRMonad    = Identity
type SIRAgent    = Agent SIRMonad SIRState SIRMsg
\end{minted}

We also have to add the \textit{SIRMonad} to the existing \textit{stepSimulation} type-declarations and we are nearly done. The function \textit{embed} for running the simulation is not provided by BearRiver but by Dunai which has important implications. Dunai does not know and care about time in MSFs, which is exactly what BearRiver builds on top of MSFs. It does so by adding a \textit{ReaderT Double} which carries the $\Delta t$. This means that \textit{embed} returns a computation in the \textit{ReaderT Double} Monad which we need to run explicitly using \textit{runReaderT}. This then results in an identity computation which we simply peel away using \textit{runIdentity}. Here is the complete code of \textit{runSimulation}:

\begin{minted}[fontsize=\footnotesize]{haskell}
runSimulation :: RandomGen g => g -> Time -> DTime -> [(AgentId, SIRState)] -> [[SIRState]]
runSimulation g t dt as = map (map aoObservable) aoss
  where
    steps        = floor (t / dt)
    dts          = replicate steps ()
    n            = length as
    (rngs, _)    = rngSplits g n []
    ais          = map fst as
    sfs          = map (\ (g', (_, s)) -> sirAgent g' ais s) (zip rngs as)
    ains         = map (\ (aid, _) -> agentIn aid) as
    aossReader   = embed (stepSimulation sfs ains) dts
    aossIdentity = runReaderT aossReader dt
    aoss         = runIdentity aossIdentity
\end{minted}

Note that embed does not take a list of $\Delta t$ any more but simply a list of inputs for each step to the top level signal-function.

\subsubsection{Random Monad}
Using the Identity Monad does not gain us anything but it was a first step towards a more general solution. Our next step is to replace the Identity Monad by the Random Monad which will allow us to get rid of the RandomGen arguments to our functions and run the whole simulation within the RandomMonad \textit{again} just as we started but now with the full features functional reactive programming!
We start by re-defining the SIRMonad and SIRAgent:

\begin{minted}[fontsize=\footnotesize]{haskell}
type SIRMonad g = Rand g
type SIRAgent g = Agent (SIRMonad g) SIRState SIRMsg
\end{minted}

Note that we parametrise the Random Monad with a RandomGen g thus this requires to add the RandomGen type-class to all functions where it was not yet added. We also simply remove all RandomGen arguments to all functions except \textit{runSimulation}. The question is now how to access this random monad functionality within the MSF context.
For the function \textit{occasionally}, there exists a monadic pendant \textit{occasionallyM} which requires a MonadRandom type-class. Because we are now running within a MonadRandom instance we simply replace \textit{occasionally} with \textit{occasionallyM}.
Running \textit{gotInfected} is now much easier. Using the function \textit{arrM} of Dunai allows us to run a monadic action in the stack as an arrow. We then directly run gotInfected by lifting it into the random-monad.
This can be seen in the susceptible agent running in the random monad SF:
\begin{minted}[fontsize=\footnotesize]{haskell}
susceptibleAgent :: RandomGen g => [AgentId] -> SIRAgent g
susceptibleAgent ais = switch susceptible(const (infectedAgent))
  where
    susceptible :: RandomGen g => SF (SIRMonad g) SIRAgentIn (SIRAgentOut, Event ())
    susceptible = proc ain -> do
      infected <- arrM (lift . gotInfected infectivity) -< ain
      if infected 
        then returnA -< (agentOut Infected, Event ())
        else (do
          makeContact <- occasionallyM (1 / contactRate) () -< ()
          contactId   <- drawRandomElemSF                   -< ais
          let ao = agentOut Susceptible
          if isEvent makeContact
            then returnA -< (dataFlow (contactId, Contact Susceptible) ao, NoEvent)
            else returnA -< (ao, NoEvent))
\end{minted}

Note also that \textit{drawRandomElemSF} doesn't take a random number generator as well as it has been reimplemented to make full use of the MonadRandom in the stack:

\begin{minted}[fontsize=\footnotesize]{haskell}
drawRandomElemS :: MonadRandom m => SF m [a] a
drawRandomElemS = proc as -> do
  r <- getRandomRS ((0, 1) :: (Double, Double)) -< ()
  let len = length as
  let idx = fromIntegral len * r
  let a =  as !! floor idx
  returnA -< a
\end{minted}

Instead of \textit{noiseR} which requires a RandomGen, it makes use of Dunai \textit{getRandomRS} stream function which simply runs \textit{getRandomR} in the MonadRandom.

Finally because our innermost monad is now the Random Monad instead of the Identity we run it by \textit{evalRand}:

\begin{minted}[fontsize=\footnotesize]{haskell}
aossReader = embed (stepSimulation sfs ains) dts
  aossRand = runReaderT aossReader dt
      aoss = evalRand aossRand g
\end{minted}

\subsubsection{Discussion}
By making the transition to MSFs we can now stack arbitrary number of Monads. As an example we could add a StateT monad on the type of AgentOut which would allow to conveniently manipulate the AgentOut e.g. in case where one sends more than one message or the construction of the final AgentOut is spread across multiple functions which allows easy composition. When implementing this one needs to replace the dpSwitch with an individual implementation in which one runs the state monad isolated for each agent.
We could even add the IO monad if our agents require arbitrary IO e.g. reading/writing from files or communicating over TCP/IP. Although one could run in the IO monad, one should not do so as we would loose all static guarantees about the reproducibility of our simulation. In ABS we need deterministic behaviour under all circumstances where repeated runs with the same initial conditions, including the random-number generator, should result in the same dynamics. If we allow the use of the IO Monad we loose the ability to guarantee the reproducibility at compile-time even if the agents never use IO facilities and just run in the IO for printing debug messages.
So far making the transition to MSFs does not seem as compelling as making the move from the RandomMonad to FRP in the beginning. Running in the RandomMonad within FRP is convenient but we could achieve the same with passing RandomGen around as we already demonstrated. In the next step we introduce the concept of a read/write environment which we realise using a StateT monad. This will show the real benefit of the transition to MSFs as without it, implementing a general environment access would be quite cumbersome.