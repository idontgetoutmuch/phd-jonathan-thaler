\section{Metaphysics}
The term \textit{Metaphysics} was first coined by Martin Heidegger in "Sein und Zeit" and "Metaphysik". Using our system of reference we can now discuss metaphysics from a computer science perspective: what is the meaning of \textit{to be} (sein), of existence \textit{Das Seiende} and of nothing \textit{das Nichts}. Unfortunately the english language cannot translate these words properly but the basic concepts should be intuitively clear.

interpret mystical traditions in the context of the simulation argument

some mystics talk about e.g. 12 spheres: 12 simulation levels

akashik record: just the memory of all actions of all agents in the past.

The following questions will be addressed and explained in this new context

\begin{itemize}
\item Who or what is God?
\item Who or what is Christ, Buddah, Vishnu, Mohammed,...?
\item What is free will? 
\item What is meditation?
\item What is conciousness?
\item What is life after death?
\item Where do we come from?
\item What is the omega point?
\item What is spiritual enlightenment?
\end{itemize}

karma: cause and effect. is a controlling factor in the simulation to prevebt the dynamics to get out of hand

The nothing was something: it was indeed nothingness, a

\subsection{Beyond}
It is reasonable to assume that the sphere of existence which initiated the simulation is itself a kind of simulation of a further level up. It is also reasonable to assume an infinite regress. The question is what the culimnation-point of this infinite regress is. Probably we can say it is God. This is by no means a novel idea an inspired by Tiplers Omega Point and other scholastic ideas.