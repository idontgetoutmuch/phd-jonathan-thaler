\section{Introduction}


We build on the discussion of the \textit{simulation argument} proposed by \cite{bostrom_are_2003} which states that we \textit{may} live in a simulation, created by transhumans (TODO: explain). Although we neither can proof or disproof the \textit{simulation argument}, we find the idea of existence being a simulation highly intriguing as it finally allows us to \textit{investigate how existence can be understood from the perspective of computer science in a scientific way}. This was so far not possible due to the lack of a scientific context, which is now given through the \textit{simulation argument}. The obvious question which raises, is then \textit{why are we simulated?}. Bostrom does not give any reason for it in his paper as this obviously touches on ideological and religious ground. We try to develop hypothesis for why we are being simulated and approach it scientifically by drawing parallels to agent-based social simulation.
\cite{steinhart_theological_2010} discusses theological implications of the simulation argument and looks into reasons why we are simulated. He mentions 'evolution of complexity' and gaining of knowledge for the transhumans as the reasons why theses simulations are enacted. We agree with Steinhart but our central point is that we hypothesize that these reasons are not first-order and that there is a first-order reason of this simulation.

main questions: 
	- what is free will in computer science?
	- what is conciousness in computer science?
	
Our reasoning is as follows:

Free Will leads to ideologies.

Ideologies result in separation, inequality, suffering and ultimately in death.

To prevent destruction of all life Free Will needs to be withdrawn from life-forms.

Learning can only happen through Free Will.

To enable life-forms to learn, they are granted Free Will but in a protected environment.

This existence is a simulation which allows conciousness to express Free Will. The emergent property of this simulation are ideologies.

NOTE:  I don't believe we are in a simulation but that we are separated from something which is beyond our existence. The simulation argument only serves as a metaphor to be able to address metaphysical issues with scientific tools and language from computer science, physics and ultimately mathematics.

\cite{irtem_simulation_1978} defines artificial free will of a machine to be:
TODO: We follow the concept of irreducibility of computation which, due to undecidability - requires to run (=simulate) a program to actually know its output.
TODO: 'the quantum system develops according to a wave-function with superposed states, and something, that isn't micro-physics, causes the wave to collapse into one of the superposed states. This something might very well be mental states'. Thus we draw the parallels to our ABS: the description, of the simulation is its unrealized state of superposition. when we run the simulation we realize superposed states. 

We focus on free will which we define as having the ability to anticipate results from one actions thus creating a feedback on the actions. Thus Free Will can be regarded as the very basic form of learning and IS learning and can be identified with learning. Here we follow the interpretation that anticipation means 'to simulate' in the context of ABS. We follow the concept of irreducibility of computation which, due to undecidability - requires to run (=simulate) a program to actually know its output. The idea is in each step of the simulation let agents simulate the same simulation they are situated in from their point of view for a given number of steps before they take their next step. Thus the simulation is defined in terms of recursion, a novelty and something we will describe in depth in this paper. We claim that functional programming is especially well suited to implement MetaABS due to its lack of implicit side-effects, natural parallelism and declarative way of describing WHAT systems are instead of HOW they compute. Obviously this new approach has a problem: an agent would need to have complete information about the whole simulation, an assumption which is completely unrealistic. We introduce the term of 'reduced-recursive' which assumes that the next recursive level is a reduced version of the original, very local to the agent, thus solving the problem of complete information.
Our method may look like another optimization method but we will show that this is not the case. Also our motivation and intention is completely different and our intended area of application are the social sciences, artificial life, philosophy and maybe religious studies. Thus we must be very careful to not confuse it with e.g. game-theory where one makes assumptions about the other players - this may seem to be the same but we are not interested in this approach and we argue it is very different: we simulate instead of rational calculation. We test our method using the famous \textit{Sugarscape} model.

- We ask how existence can be understood from the perspective of computer science and follow in our ideas the simulation hypothesis as of \cite{bostrom_are_2003} which states that existence is a simulation (TODO: bostrom does not state this!)

- We differ from the simulation hypothesis which does not state any deeper meaning or reason for existence by hypothesizing that existence is a simulation which allows conciousness to express free will.

- We will show that granting free will outside of a simulation is too dangerous as we claim that it will always leads to ideologies which result in separation, inequality, suffering and ultimately in death, something this simulation clearly exhibits.

- We propose computational models of free will and conciousness and of the simulation itself.
- Further we touch on the questions who is behind this simulation, what is outside of it and what may be beyond of it.
- Our final hypothesis is that the ultimate purpose of this simulation of free will is to lead the entities which populating this existence to an understanding of the problem of free will and ideology, transcending both and giving them up for love.
- We claim that love in this simulation can be understood as a total and complete understanding of what is - a complete understanding and grasping of truth: to experience objective existence. This love will enable the entities to abandon the feeling of separation, devoiding them of their primal fear, resulting in true inner peace, enabling them to step up the simulation level after 'death'.

- One implication is that there is ultimately no death but only a transformation of data.
- Another rather cruel implication is that the suffering serves a purpose and is not real anyway.
- We hypothesize that there are further levels of simulations up to 12 layers where the final layer ultimately is what religion calls god: the reasoning of \cite{steinhart_theological_2010} would support this
- we claim that free will is able to 'simulate' its actions thus anticipating them which allows to behave in various non-deterministic ways.
- we give a new model of an agent-based simulation, called meta abs in which agents run the simulation locally enabling them to anticipate in a limited way what is going to happen. we show that pure functional programming is especially well suited in implementing this new technique and explore it in a model of free will
- question: how could abs be capable of simulating conciousness and free will? hypothesis: we need a meta-level in our simulation which allows us to simulate the simulation
- question: can we develop a model and a simulation of the simulation hypothesis?\\