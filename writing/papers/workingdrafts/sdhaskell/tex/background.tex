\section{Background}
TODO: describe FP

\subsection{Functional Reactive Programming}
Functional Reactive Programming is a way to implement systems with continuous and discrete time-semantics in pure functional languages. There are many different approaches and implementations but in our approach we use \textit{Arrowized} FRP \cite{hughes_generalising_2000, hughes_programming_2005} as implemented in the library Yampa \cite{hudak_arrows_2003, courtney_yampa_2003, nilsson_functional_2002}.

The central concept in Arrowized FRP is the Signal Function (SF) which can be understood as a \textit{process over time} which maps an input- to an output-signal. A signal can be understood as a value which varies over time. Thus, signal functions have an awareness of the passing of time by having access to $\Delta t$ which are positive time-steps with which the system is sampled. 

\begin{flalign*}
Signal \, \alpha \approx Time \rightarrow \alpha \\
SF \, \alpha \, \beta \approx Signal \, \alpha \rightarrow Signal \, \beta 
\end{flalign*}

Yampa provides a number of combinators for expressing time-semantics, events and state-changes of the system. They allow to change system behaviour in case of events, run signal functions and generate stochastic events and random-number streams. We shortly discuss the relevant combinators and concepts we use throughout the paper. For a more in-depth discussion we refer to \cite{hudak_arrows_2003, courtney_yampa_2003, nilsson_functional_2002}.