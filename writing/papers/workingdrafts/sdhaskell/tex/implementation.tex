\section{A correct-by-construction implementation}
\label{sec:impl}

In this section we develop a correct-by-construction implementation step-by-step. The complete code is attached in Appendix \ref{app:code}. Note that the constant parameters \textit{populationSize, infectedCount, contactRate, infectivity, illnessDuration} are defined globally and omitted for clarity.

Computing the dynamics of a SD model happens by integrating the time over the equations. So conceptually we treat our SD model as an continuous function which is defined over time = 0 -> infinity and at each point in time outputs the values of each stock. In the case of the SIR model we have 3 stocks: Susceptible, Infected and Recovered. Thus we start our implementation by defining the output of our SD function: for each time-step we have the values of the 3 stocks:

\begin{HaskellCode}
type SIRStep = (Time, Double, Double, Double)
\end{HaskellCode}

Next we define our continuous SD function which we obviously make a signal function. It has no input, because a SD system is only defined in its own terms and parameters without external input and has as output the \textit{SIRStep}. Thus we define the following function type:

\begin{HaskellCode}
sir :: SF () SIRStep
\end{HaskellCode}

An SD model is fundamentally built on feedback: the values at time t depend on the history. Thus we introduce feedback in which we feed the last step into the next step. Yampa provides the \textit{loopPre :: c $\to$ SF (a, c) (b, c) $\to$ SF a b} function for that. It takes an initial value and a feedback signal function which receives the input \textit{a} and the previous (or initial) value of the feedback and has to return the output \textit{b} and the new feedback value \textit{c}. \textit{loopPre} then returns simply a signal function from \textit{a} to \textit{b} with the feedback happening transparent in the feedback signal function. Our initial feedback value is the initial state of the SD model at $t = 0$. Further we define the type of the feedback signal function:

\begin{HaskellCode}
sir = loopPre (0, initSus, initInf, initRec) sirFeedback
  where
    initSus = populationSize - infectedCount
    initInf = infectedCount
    initRec = 0
  
    sirFeedback :: SF ((), SIRStep) (SIRStep, SIRStep)
\end{HaskellCode}

The next step is to implement the feedback signal function. As input we get \textit{(a, c)} where \textit{a} is the empty tuple () because a SD simulation has no input, and \textit{c} is the fed back \textit{SIRStep} from the previous (initial) step. With this we have all relevant data so we can implement the feedback function. We first match on the tuple inputs and construct a signal function using \textit{proc}:

\begin{HaskellCode}
    sirFeedback = proc (_, (_, s, i, _)) -> do
\end{HaskellCode}

Now we define our flows which are \textit{infection rate} and \textit{recovery rate}. The formulas for both of them can be seen in equations TODO (refer to the differential equations). This directly translates into Haskell code:

\begin{HaskellCode}
      let infectionRate = (i * contactRate * s * infectivity) / populationSize
          recoveryRate  = i / illnessDuration
\end{HaskellCode}

Next we need to compute the values of the three stocks, following the formulas of TODO (refer to the Integral formulas). For this we need the \textit{integral} function of Yampa which integrates over a numerical input using the rectangle rule. Adding initial values can be achieved with the (\^ <<) operator of arrowized programming. This directly translates into Haskell code:

\begin{HaskellCode}
      s' <- (initSus+) ^<< integral -< (-infectionRate)
      i' <- (initInf+) ^<< integral -< (infectionRate - recoveryRate)
      r' <- (initRec+) ^<< integral -< recoveryRate
\end{HaskellCode}

We also need the current time of the simulation. For this we use Yampas \textit{time} function:

\begin{HaskellCode}
      t <- time -< ()
\end{HaskellCode}

Now we only need to return the output and the feedback value. Both types are the same thus we simply duplicate the tuple:

\begin{HaskellCode}
      returnA -< dupe (t, s', i', r')

    dupe :: a -> (a, a)
    dupe a = (a, a)
\end{HaskellCode}

We want to run the SD model for a given time with a given $\Delta t$ by running the \textit{sir} signal function. To \textit{purely} run a signal function Yampa provides the function \textit{embed :: SF a b -> (a, [(DTime, Maybe a)]) -> [b]} which allows to run an SF for a given number of steps where in each step one provides the $\Delta t$ and an input \textit{a}. The function then returns the output of the signal function for each step. Note that the input is optional, indicated by \textit{Maybe}. In the first step at $t = 0$, the initial \textit{a} is applied and whenever the input is \textit{Nothing} in subsequent steps, the last \textit{a} which was not \textit{Nothing} is re-used.

\begin{HaskellCode}
runSD :: Time -> DTime -> [SIRStep]
runSD t dt = embed sir ((), steps)
  where
    steps = replicate (floor (t / dt)) (dt, Nothing)
\end{HaskellCode}
