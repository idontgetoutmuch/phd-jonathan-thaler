\section{Related Work}
In his masterthesis \cite{bezirgiannis_improving_2013} the author investigated Haskells parallel and concurrency features to implement (amongst others) \textit{HLogo}, a Haskell clone of the NetLogo simulation package, focusing on using Software Transactional Memory for a limited form of agent-interactions. \textit{HLogo} is basically a re-implementation of NetLogos API in Haskell where agents run within IO and thus can also make use of STM functionality. The benchmarks show that this approach does indeed result in a speed-up especially under larger agent-populations. The authors thesis can be seen as one of the first works on ABS using Haskell. Despite the concurrency and parallel aspect our work share, our approach is rather different: we avoid IO within the agents under all costs, build on FRP and explore on a more conceptual level the use of STM rather than implementing a ABS library.

There exists some research \cite{di_stefano_using_2005, varela_modelling_2004, based_2013} of using the functional programming language Erlang \cite{armstrong_erlang_2010} to implement ABS. The language is inspired by the actor model \cite{agha_actors:_1986} and was created in the 1980’s by Joe Armstrong for Eriksson for developing distributed high reliability software in telecommunications. The actor model can be seen as quite influential to the development of the concept of agents in ABS which borrowed it from Multi Agent Systems \cite{wooldridge_introduction_2009}. It emphasises message-passing concurrency with share-nothing semantics (no shared state between agents) which maps nicely to functional programming concepts. Erlang implements light-weight processes which allows to spawn thousands of them without heavy memory overhead. The mentioned papers investigate how the actor model can be used to close the conceptual gap between agent-specifications which focus on message-passing and their implementation. Further they also showed that using this kind of concurrency allows to overcome some problems of low level concurrent programming as well.
Also \cite{bezirgiannis_improving_2013} ported NetLogos API to Erlang mapping agents to concurrently running processes which interact with each other by message-passing. With some restrictions on the agent-interactions this model worked, which shows at using concurrent message-passing for parallel ABS is at least \textit{conceptually} feasible.