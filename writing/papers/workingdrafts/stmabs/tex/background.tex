\section{Background}
\label{sec:background}

\subsection{Software Transactional Memory}
Software Transactional Memory (STM) was introduced in \cite{shavit_software_1995} in 1995 as an alternative to lock-based synchronisation in concurrent programming which, in general, is notoriously difficult to get right because reasoning about the interactions of multiple concurrently running threads and low level operational details of synchronisation primitives and locks is \textit{very hard}. The main problems are:

\begin{itemize}
	\item Race conditions due to forgotten locks.
	\item Deadlocks resulting from inconsistent lock ordering.
	\item Corruption caused by uncaught exceptions.
	\item Lost wake-ups induced by omitted notifications.
\end{itemize}

Worse, concurrency does not compose. It is utterly difficult to write two functions (or methods in an object) acting on concurrent data which can be composed into a larger concurrent behaviour. The reason for it is that one has to know about internal details of locking, which breaks encapsulation and makes composition depend on knowledge about their implementation. Also it is impossible to compose two functions e.g. where one withdraws some amount of money from an account and the other deposits this amount of money into a different account: one ends up with a temporary state where the money is in none of either accounts, creating an inconsistency - a potential source for errors because threads can be rescheduled at any time.

STM promises to solve all these problems for a low cost by executing actions atomically, where modifications made in such an action are invisible to other threads and changes by other threads are invisible as well until actions are committed - STM actions are atomic and isolated. When an STM action exits, either one of two outcomes happen: if no other thread has modified the same data as the thread running the STM action, then the modifications performed by the action will be committed and become visible to the other threads. If other threads have modified the data then the modifications will be discarded, the action block rolled-back and automatically restarted.

STM in Haskell is implemented using optimistic synchronisation, which means that instead of locking access to shared data, each thread keeps a transaction log for each read and write to shared data it makes. When the transaction exits, this log is checked whether other threads have written to memory it has read - it checks whether it has a consistent view to the shared data or not. This might look like a serious overhead but the implementations are very mature by now, being very performant and the benefits outweigh its costs by far. In the paper \cite{heindl_modeling_2009} the authors used a model of STM to simulate optimistic and pessimistic STM behaviour under various scenarios using the AnyLogic simulation package. They concluded that optimistic STM may lead to 25\% less retries of transactions. The authors of \cite{perfumo_limits_2008} analyse several Haskell STM programs with respect to their transactional behaviour. They identified the roll-back rate as one of the key metric which determines the scalability of an application. Although STM might promise better performance, they also warn of the overhead it introduces which could be quite substantial in particular for programs which do not perform much work inside transactions as their commit overhead appears to be high.

\subsection{Parallelism, Concurrency \& STM in Haskell}
In our case-study implementations we are using the functional programming language Haskell. The paper of \citep{hudak_history_2007} gives a comprehensive overview over the history of the language, how it developed and its features and is very interesting to read and get accustomed to the background of the language. Note that Haskell is a \textit{lazy} language which means that expressions are only evaluated when they are actually needed.

\subsubsection{Side-Effects}
One of the fundamental strengths of Haskell is its way of dealing with side-effects in functions. A function with side-effects has observable interactions with some state outside of its explicit scope. This means that the behaviour depends on history and that it loses its referential transparency character, which makes understanding and debugging much harder. Examples for side-effects are (amongst others): modifying a global variable, await an input from the keyboard, read or write to a file, open a connection to a server, drawing random-numbers,...

The unique feature of Haskell is that it allows to indicate in the \textit{type} of a function that it does \textit{not} or does have side-effects and what kind of effects they are. There are a broad range of different effect types available, to restrict the possible effects a function can have e.g. drawing random-numbers, sharing read/write state between functions,... Depending on the type, only specific operations are available, which is then checked by the compiler. This means that a program in which one tries to e.g. read a file in a function which only allows drawing random-numbers will fail to compile.

Here we are only concerned with two effect types: The \textit{IO} effect context can be seen as completely unrestricted as the main entry point of each Haskell program runs in the \textit{IO} context which means that this is the most general and powerful one. It allows all kind of I/O related side-effects: reading/writing a file, creating threads, write to the standard output, read from the keyboard, opening network-connections, mutable references,... Also the \textit{IO} context provides functionality for concurrent locks and global shared references. The other effect context we are concerned with is \textit{STM} and indicates the STM context of a function - we discuss it below. 

A function without any side-effect type is called \textit{pure}. A function with a given effect-type needs to be executed with a given effect-runner which takes all necessary parameters depending on the effect and runs a given effectful function returning its return value and depending on the effect also an effect-related result. Note that we cannot call functions of different effect-types from a function with another effect-type, which would violate the guarantees. Calling a \textit{pure} function though is always allowed because it has by definition no side-effects. 

Although it might seem very restrictive at first, we get a number of benefits from making the type of effects we can use explicit. First we can restrict the side-effects a function can have to a very specific type which is guaranteed at compile time. This means we can have much stronger guarantees about our program and the absence of potential errors already at compile-time, which implies that we don't need test them with e.g. Unit-Tests. Second, by the use of effect-runners, we can execute effectful functions in a very controlled way, by making the effect-context explicit in the parameters to the effect-runner.

\subsubsection{Parallelism \& Concurrency}
Haskell makes a very clear distinction between parallelism and concurrency. Parallelism is always deterministic and thus pure without side-effects because although parallel code runs concurrently, it does by definition not interact with data of other threads. This can be indicated through types: we can run pure functions in parallel because for them it doesn't matter in which order they are executed, the result will always be the same due to the concept of referential transparency.

Concurrency is potentially non-deterministic because of non-deterministic interactions of concurrently running threads through shared data. Although data in functional programming is immutable, Haskell provides primitives which allow to share immutable data between threads. Accessing these primitives is but only possible from within an \textit{IO} or \textit{STM} context (see below) which means that when we are using concurrency in our program, the types of our functions change from pure to either \textit{IO} or \textit{STM} effect context.

Note that spawning thousands of threads in Haskell is no problem and has very low memory footprint because they are lightweight user-space threads, managed by the Haskell Runtime System which maps them to physical operating-system threads. 

\subsubsection{STM}
The work of \cite{harris_composable_2005, harris_transactional_2006} added STM to Haskell which was one of the first programming languages to incorporate STM into its main core and added the ability to composable operations. There exist various implementations of STM in other languages as well (Python, Java, C\#, C/C++,...) but it is in Haskell with its type-system and the way how side-effects are treated where it truly shines.

In the Haskell implementation, STM actions run within the \textit{STM} context. This restricts the operations to only STM primitives as shown below, which allows to enforce that STM actions are always repeatable without persistent side-effects because such persistent side-effects (writing to a file, launching a missle) are not possible in an \textit{STM} context. This is also the fundamental difference to  \textit{IO}, where all bets are off because \textit{everything} is possible as there are basically no restrictions because \textit{IO} can run everything.

Thus the ability to \textit{restart} a block of actions without any visible effects is only possible due to the nature of Haskells type-system: by restricting the effects to STM only, ensures that no uncontrolled effects, which cannot be rolled-back, occur.

STM comes with a number of primitives to share transactional data. Amongst others the most important ones are:

\begin{itemize}
	\item TVar - A transactional variable which can be read and written arbitrarily. 
	\item TArray - A transactional array where each cell is an individual shared data, allowing much finer-grained transactions instead of e.g. having the whole array in a TVar.
	\item TChan - A transactional channel, representing an unbounded FIFO channel.
	\item TMVar - A transactional \textit{synchronising} variable which is either empty of full. To read from an empty or write to a full TMVar will cause the current thread to retry its transaction.
\end{itemize}

To run an \textit{STM} action the function \textit{atomically :: STM a $\to$ IO a} is provided, which can be seen as the STM effect-runner as it performs a series of \textit{STM} actions atomically within an \textit{IO} context. It takes the STM action which returns a value of type \textit{a} and returns an \textit{IO} action which returns a value of type \textit{a}. This \textit{IO} action can only be executed within an \textit{IO} context.