\section{Further Research} % (Lived happily ever after...)
\label{sec:further}

So far we only implemented a tiny bit of the Sugarscape model and left out the later chapters which are more involved as they incorporate direct synchronous communication between agents. Such mechanisms are very difficult to approach in GPU based approaches \cite{lysenko_framework_2008} but should be quite straightforward in STM using \textit{TChan} and retries.

We have not focused on implementing an approach like \textit{Sense-Think-Act} cycle as mentioned in \cite{xiao_survey_2018}. This could offer lot of potential for parallelisation due to sense and think happening isolated from each agent, without interfering with global shared data. We expect additional speed-up from such an approach.

So far we only looked at time-driven models (note that the Sugarscape is basically a time-driven model where each agent acts in each step) where all agents run concurrently in lock-step. Such models are very well suited to concurrent execution because  normally in such models there is no restriction on the order of agents imposed and if then it is almost always to be random, making it perfect for concurrent execution. It would be of interest whether we can apply STM and concurrency to an event-driven approach as well. Generally one could run agents concurrently and undo actions when there are inconsistencies - something which pure functional programming in Haskell together with STM supports out of the box. Such an approach could be theoretically implemented using Parallel Discrete Event Simulation (PDES) \cite{fujimoto_parallel_1990} and it would be interesting to see how an event-driven ABS approach based on an underlying PDES implementation would perform.

Our implementations focus only on local parallelisation and concurrency and avoided distributed computing as it was out of the scope of this paper. It would be of interest to see how we can map functional ABS to distributed computing using a message-based approach found in the Cloud Haskell framework.