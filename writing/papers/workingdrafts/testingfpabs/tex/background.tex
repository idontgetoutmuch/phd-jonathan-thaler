\section{Background}
\subsection{Verification \& Validation}
The book \cite{robinson_simulation:_2014} discusses various aspects of testing and making sure a simulation is correct and distinguishes in this process between \textit{validation} and \textit{verification}. It defines validation to be the process of ensuring that a model or specification is sufficiently accurate for the purpose at hand whereas verification to be the process of ensuring that the model design has been transformed into a computer model with sufficient accuracy. In other words, \cite{balci_verification_1998} define validation as \textit{"are we building the right model?"} and verification as \textit{"are we building the model right?"}.

In this paper we will only focus on verification, because it is there where one ensures that the model is programmed correctly, the algorithms have been implemented properly, and the model does not contain errors, oversights, or bugs. Note that verification has a narrow definition and can be seen as a subset of the wider issue of validation. One distinguishes between:

\begin{itemize}
	\item White-box Verification: compares the content of the model to the \textit{conceptual} model by detailed, micro check if each part of the implementation represent the conceptual model with sufficient accuracy. 
	\item Black-box Verification: treating the functionality to test as a black-box with inputs and outputs and comparing controlled inputs to expected outputs.
\end{itemize}

So in general one can see verification as a test of the fidelity with which the conceptual model is converted into the computer model. Verification is a continuous process and if it is already there in the programming language / supported by then this is much easier to do. A fundamental hypothesis of this paper is that by choosing a programming language which supports this continuous verification and validation process, then the result is an implementation of a model which is more likely to be correct.

Unfortunately, there is no such thing as general validity: a model should be built for one purpose as simple as possible and not be too general, otherwise it becomes too bloated and too difficult or impossible to analyse. Also, there may be no real world to compare against: simulations are developed for proposed systems, new production or facilities which don't exist yet. Further, it is questionable which real world one is speaking of: the real world can be interpreted in different ways, therefore a model valid to one person might not be valid to another. Sometimes validation struggles because the real world data are inaccurate or there is not enough time to verify and validate everything.

In general this implies that we can only \textit{raise the confidence} in the correctness of the simulation: it is not possible to prove that a model is valid, instead one should think of confidence in its validity. Therefore, the process of verification and validation is not the proof that a model is correct but trying to prove that the model is incorrect! The more tests/checks one carries out which show that it is not incorrect, the more confidence we can place on the models validity.

In our research we focus primarily on the \textit{Verification} aspect of agent-based simulation: ensuring that the implementation reflects the specifications of the \textit{conceptual} model - have we built the model right? Thus we are not interested in our research into making connections to the real world and always see the model specifications as our "last resort", our ground truth beyond nothing else exists. When there are hypotheses formulated, we always treat and interpret them in respect of the conceptual model.

The authors \cite{ormerod_validation_2006} make the important point that because the current process of building ABS is a discovery process, often models of an ABS lack an analytical solution (in general) which makes verification much harder if there is no such solution.

So the baseline is that either one has an analytical model as the foundation of an agent-based model or one does not. In the former case, e.g. the SIR model, one can very easily validate the dynamics generated by the simulation to the one generated by the analytical solution through System Dynamics. In the latter case one has basically no idea or description of the emergent behaviour of the system prior to its execution e.g. SugarScape. In this case it is important to have some hypothesis about the emergent property / dynamics. The question is how verification / validation works in this setting as there is no formal description of the expected behaviour: we don't have a ground-truth against which we can compare our simulation dynamics.

The paper \cite{polhill_ghost_2005} makes the very important point that ABS should require more rigorous programming standards than other computer simulations. Because researchers in ABS look for an emergent behaviour in the dynamics of the simulation, they are always tempted to look for some surprising behaviour and expect something unexpected from their simulation. Thus it is of utmost importance to ensure that the implementation is matching the specification as closely as possible and nothing is left to chance. The paper \cite{galan_errors_2009} supports this as well and emphasises that the fundamental problem of ABS is that due to its mostly exploratory nature, there exists some amount of uncertainty about the dynamics the simulation will produce before running it. Thus it is often very difficult to judge whether an unexpected outcome can be attributed to the model or has in fact its roots in a subtle programming error.

The work of \cite{kleijnen_verification_1995} suggests good programming practice which is extremely important for high quality code and reduces bugs but real world practice and experience show that this alone is not enough, even the best programmers make mistakes which often can be prevented through a strong static or a dependent type system already at compile-time. What we can guarantee already at compile-time, doesn't need to be checked at run-time which saves substantial amount of time as at run-time there may be a large number of execution paths through the simulation which is almost always simply not feasible to check (note that we also need to check all combinations). This paper also cites modularity as very important for verification: divide and conquer and test all modules separately. We claim that this is especially easy in functional programming as code composes better than in traditional object-oriented programming due to the lack of interdependence between data and code as in objects and the lack of global mutable state (e.g. class variables or global variables) - this makes code extremely convenient to test. The paper also discusses statistical tests (the t test) to check if the outcome of a simulation is sufficiently close to real-world dynamics - we explicitly omit this as it part of validation and not the focus of this research.

\subsection{Test-Driven Development}
While there exists quite some work on validating an ABS, there doesn't exist much work on verification which discusses the problem from an implementation perspective with program code. The only work we could find is \cite{collier_test-driven_2013} which briefly discusses how to apply the test-driven development approach to ABS, using unit-testing to check the correctness of the implementation.

\subsection{Property-Based Testing}
Property-based testing allows to formulate \textit{functional specifications} in code which then a property-based testing library tries to falsify by \textit{automatically} generating test-data with some user-defined coverage. When a case is found for which the property fails, the library then reduces it to the most simple one. It is clear to see that this kind of testing is especially suited to ABS, because we can formulate specifications, meaning we describe \textit{what} to test instead of \textit{how} to test. Also the deductive nature of falsification in property-based testing suits very well the constructive and exploratory nature of ABS. Further, the automatic test-generation can make testing of large scenarios in ABS feasible as it does not require the programmer to specify all test-cases as required in unit-tests.

Property-based testing was invented the authors of \cite{claessen_quickcheck:_2000, claessen_testing_2002} in which they present the QuickCheck library, which tries to falsify the specifications by \textit{randomly} sampling the space. We argue, that the stochastic sampling nature of this approach is particularly well suited to ABS, because it is itself almost always driven by stochastic events and randomness in the agents behaviour thus this correlation makes it straight-forward to map ABS to property-testing. The main challenge when using QuickCheck, as will be shown later, is to write test-data generators for agents which cover the space sufficiently enough to not miss out on important test-cases.

There exists also a deterministic property-testing library called SmallCheck \cite{runciman_smallcheck_2008} which instead of randomly sampling the test-space, it enumerates test-cases exhaustively up to some depth. It is based on the observation two observations, derived from model-checking, that (1) \textit{"If a program fails to meet its specification in some cases, it almost always fails in some simple case"} and (2) \textit{"If a program does not fail in any simple case, it hardly ever fails in any case} \cite{runciman_smallcheck_2008}. This non-stochastic approach to property-based testing might be a complementary addition in some cases where the tests are of non-stochastic nature with a search-space which is too large to implement manually by unit-tests but might be easy to enumerate exhaustively.

Although it is now available in a wide range of programming languages and paradigms, property-based testing has its origins in Haskell and indeed both QuickCheck and SmallCheck are Haskell libraries and we argue that for that reason property-based testing really shines in pure functional programming. Thus we conduct all implementation and research of this paper using Haskell and therefore we also give a brief introduction into the concepts of pure functional programming in Haskell without going into too much technical detail.

\subsection{Pure Functional Programming}
Functional programming makes functions the main concept of programming, promoting them to first-class citizens. Its roots lie in the Lambda Calculus which was first described by Alonzo Church \citep{church_unsolvable_1936}. This is a fundamentally different approach to computation than imperative and object-oriented programming which roots lie in the Turing Machine \citep{turing_computable_1937}. Rather than describing \textit{how} something is computed as in the more operational approach of the Turing Machine, due to the more declarative nature of the Lambda Calculus, code in functional programming describes \textit{what} is computed.

In this paper we are using the functional programming language Haskell. The paper of \citep{hudak_history_2007} gives a comprehensive overview over the history of the language, how it developed and its features and is very interesting to read and get accustomed to the background of the language. The main points why we decided to go for Haskell are:

\begin{itemize}
	\item Rich Feature-Set - it has all fundamental concepts of the pure functional programming paradigm of which we explain the most important below.
	\item Real-World applications - the strength of Haskell has been proven through a vast amount of highly diverse real-world applications \cite{hudak_history_2007}, is applicable to a number of real-world problems \cite{osullivan_real_2008} and has a large number of libraries available \footnote{\url{https://wiki.haskell.org/Applications_and_libraries}}.
	\item Modern - Haskell is constantly evolving through its community and adapting to keep up with the fast changing field of computer science. Further, the community is the main source of high-quality libraries.
\end{itemize}

As a short example we give an implementation of the factorial function in Haskell:
\begin{HaskellCode}
factorial :: Integer -> Integer
factorial 0 = 1
factorial n = n * factorial (n-1)
\end{HaskellCode}

When looking at this function we can already see the central concepts of functional programming: 
\begin{enumerate}
	\item Declarative - we describe \textit{what} the factorial function is rather than how to compute it. This is supported by \textit{pattern matching} which allows to give multiple equations for the same function, matching on its input. 
	\item Immutable data - in functional programming we don't have mutable variables - after a variable is assigned, it cannot change its contents. This also means that there is no destructive assignment operator which can re-assign values to a variable. To change values, we employ recursion.
	\item Recursion - the function calls itself with a smaller argument and will eventually reach the case of 0. Recursion is the very meat of functional programming because they are the only way to implement loops in this paradigm due to immutable data.
	\item Static Types - the first line indicates the name and the types of the function. In this case the function takes one Integer as input and returns an Integer as output. Types are static in Haskell which means that there can be no type-errors at run-time e.g. when one tries to cast one type into another because this is not supported by this kind of type-system.
	\item Explicit input and output - all data which are required and produced by the function have to be explicitly passed in and out of it. There exists no global mutable data whatsoever and data-flow is always explicit.
	\item Referential transparency - calling this function with the same argument will \textit{always} lead to the same result, meaning one can replace this function by its value. This means that when implementing this function one can not read from a file or open a connection to a server. This is also known as \textit{purity} and is indicated in Haskell in the types which means that it is also guaranteed by the compiler.
\end{enumerate}

It may seem that one runs into efficiency-problems in Haskell when using algorithms which are implemented in imperative languages through mutable data which allows in-place update of memory. The seminal work of \cite{okasaki_purely_1999} showed that when approaching this problem from a functional mind-set this does not necessarily be the case. The author presents functional data structures which are asymptotically as efficient as the best imperative implementations and discusses the estimation of the complexity of lazy programs.

For an excellent and widely used introduction to programming in Haskell we refer to \cite{hutton_programming_2016}. Other, more exhaustive books on learning Haskell are \cite{lipovaca_learn_2011, allen_haskell_2016}. For an introduction to programming with the Lambda-Calculus we refer to \cite{michaelson_introduction_2011}. For more general discussion of functional programming we refer to \cite{hughes_why_1989, maclennan_functional_1990, hudak_history_2007}.

\subsubsection{Side-Effects}
One of the fundamental strengths of functional programming and Haskell is their way of dealing with side-effects in functions. A function with side-effects has observable interactions with some state outside of its explicit scope. This means that the behaviour depends on history and that it loses its referential transparency character, which makes understanding and debugging much harder. Examples for side-effects are (amongst others): modifying a global variable, modifying a variable through a reference, await an input from the keyboard, read or write to a file, open a connection to a server, drawing random-numbers,...

Obviously, to write real-world programs which interact with the outside-world we need side-effects. Haskell allows to indicate in the \textit{type} of a function that it does or does \textit{not} have side-effects. Further there are a broad range of different effect types available, to restrict the possible effects a function can have to only the required type. This is then ensured by the compiler which means that a program in which one tries to e.g. read a file in a function which only allows drawing random-numbers will fail to compile. Haskell also provides mechanisms to combine multiple effects e.g. one can define a function which can draw random-numbers and modify some global data. The most common side-effect types are: \textit{IO} allows all kind of I/O related side-effects: reading/writing a file, creating threads, write to the standard output, read from the keyboard, opening network-connections, mutable references,...; \textit{Rand} allows to draw random-numbers from a random-number stream, \textit{Reader} allows to read from a read-only environment, \textit{Writer} allows to write to a write-only environment, \textit{State} allows to read and write a read/write environment.

A function with side-effects has to indicate this in their type e.g. if we want to give our factorial function for debugging purposes the ability to write to the standard output, we add IO to its type: factorial :: Integer -> IO Integer. A function without any side-effect type is called \textit{pure}. A function with a given effect-type needs to be executed with a given effect-runner which takes all necessary parameters depending on the effect and runs a given effectful function returning its return value and depending on the effect also an effect-related result. For example when running a function with a State-effect one needs to specify the initial environment which can be read and written. After running such a function with a State-effect the effect-runner returns the changed environment in addition with the return value of the function itself. Note that we cannot call functions of different effect-types from a function with another effect-type, which would violate the guarantees. Calling a \textit{pure} function though is always allowed because it has by definition no side-effects. An effect-runner itself is a \textit{pure} function. The exception to this is the IO effect type which does not have a runner but originates from the \textit{main} function which is always of type IO.

Although it might seem very restrictive at first, we get a number of benefits from making the type of effects we can use explicit. First we can restrict the side-effects a function can have to a very specific type which is guaranteed at compile time. This means we can have much stronger guarantees about our program and the absence of potential errors already at compile-time which implies that we don't need test them with e.g. unit-tests. Second, because effect-runners are themselves \textit{pure}, we can execute effectful functions in a very controlled way by making the effect-context explicit in the parameters to the effect-runner. This allows a much easier approach to isolated testing because the history of the system is made explicit.

For a technical, in-depth discussion of the concept of side-effects and how they are implemented in Haskell using Monads, we refer to the following papers: \cite{moggi_computational_1989, wadler_essence_1992, wadler_monads_1995, wadler_how_1997, jones_tackling_2002}.