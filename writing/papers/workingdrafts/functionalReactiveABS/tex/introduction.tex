\section{Introduction}
% Drop Cap Letter needed in case of IEEE Journal
%\IEEEPARstart{W}{ith}

In Agent-Based Simulation (ABS) one models and simulates a system by modeling and implementing the constituting parts of the system, called \textit{Agents} and their local interactions. From these local interactions then the emergent property of the system emerges. ABS is still a young field, having emerged in the early to mid 90s primarily in the fields of social simulation and computational economics. 

The authors of the seminal Sugarscape model \cite{epstein_growing_1996} explicitly advocate object-oriented programming as "a particularly natural development environment for Sugarscape specifically and artificial societies generally." and report about 20.000 lines of code which includes GUI, graphs and plotting. They implemented their simulation software in Object Pascal and C where they used the former for programming the agents and the latter for low-level graphics \cite{axtell_aligning_1996}. Axelrod \cite{axelrod_advancing_1997} recommends Java for experienced programmers and Visual Basic for beginners. Up until now most of ABS seems to have followed this suggestion and are implemented using programming languages which follow the object-oriented imperative paradigm.

A serious problem of object-oriented implementations is the blurring of the fundamental difference between agent and object - an agent is first of all a metaphor and \textit{not} an object. In object-oriented programming this distinction is obviously lost as in such languages agents are implemented as objects which leads to the inherent problem that one automatically reasons about agents in a way as they were objects - agents have indeed become objects in this case. The most notable difference between an agent and an object is that the latter one do not encapsulate behaviour activation \cite{jennings_agent-based_2000} - it is passive. Also it is remarkable that \cite{jennings_agent-based_2000} a paper from 1999 claims that object-orientation is not well suited for modelling complex systems because objects behaviour is too fine granular and method invocation a too primitive mechanism.

As ABS is almost always used for scientific research, producing often break-through scientific results as pointed out in \cite{axelrod_chapter_2006}, these ABS need to be \textit{free of bugs}, \textit{verified against their specification}, \textit{validated against hypotheses} and ultimately be \textit{reproducible}. One of the biggest challenges in ABS is the one of validation. In this process one needs to connect the results and dynamics of the simulation to initial hypotheses e.g. \textit{are the emergent properties the ones anticipated? if it is completely different why?}. It is important to understand that we always \textit{must have} a hypothesis regarding the outcome of the simulation, otherwise we leave the path of scientific discovery. We must admit that sometimes it is extremely hard to anticipate \textit{emergent patterns} but still there must be \textit{some} hypothesis regarding the dynamics of the simulation otherwise we drift off into guesswork.

In this paper we ask how ABS can be done using the pure functional programming paradigm using Haskell, what the benefits are and if it could overcome the critique of using object-orientation in this field.