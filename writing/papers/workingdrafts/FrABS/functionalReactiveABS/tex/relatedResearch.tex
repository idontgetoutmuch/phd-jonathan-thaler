\section{Related Research}
The amount of research on using the pure functional paradigm using Haskell in the field of ABS has been moderate so far. Most of the papers look into how agents can be specified using the belief-desire-intention paradigm \cite{de_jong_suitability_2014}, \cite{sulzmann_specifying_2007}, \cite{jankovic_functional_2007}. A library for Discrete Event Simulation (DES) and System Dynamics (SD) in Haskell called \textit{Aivika 3} is described in \cite{sorokin_aivika_2015}. It comes with very basic features for ABS but only allows to specify simple state-based agents with timed transitions.
The authors of \cite{jankovic_functional_2007} discuss using functional programming for DES and explicitly mention the paradigm of FRP to be very suitable to DES. 
In his talk \cite{sweeney_next_2006}, Tim Sweeney CTO of Epic Games discussed programming languages in the development of game-engines and scripting of game-logic. Although the fields of games and ABS seem to be very different, in the end they have also very important similarities: both are simulations which perform numerical computations and update objects in a loop either concurrently or sequential \footnote{Gregory \cite{gregory_game_2018} defines computer-games as "soft real-time interactive agent-based computer simulations".}. In games these objects are called \textit{game-objects} and in ABS they are called \textit{agents} but they are conceptually the same thing. The two main points Sweeney made were that dependent types could solve most of the run-time failures and that parallelism is the future for performance improvement in games. He distinguishes between pure functional algorithms which can be parallelized easily in a pure functional language and updating game-objects concurrently using software transactional memory (STM).

The thesis of \cite{bezirgiannis_improving_2013} constructs two frameworks: an agent-modelling framework and a DES framework, both written in Haskell. They put special emphasis on parallel and concurrency in their work. The author develops two programs with strong emphasis on parallelism: HLogo which is a clone of the NetLogo agent-modelling framework and HDES, a framework for discrete event simulation.

The authors of \cite{schneider_towards_2012} and \cite{vendrov_frabjous:_2014} present a domain-specific language for developing functional reactive agent-based simulations. This language called FRABJOUS is human readable and easily understandable by domain-experts. It is not directly implemented in FRP/Haskell but is compiled to Yampa code which they claim is also readable. This supports our initial claim that FRP is a promising approach to implement ABS in Haskell.