\section{Introduction}
% Drop Cap Letter needed in case of IEEE Journal
%\IEEEPARstart{W}{ith}

In Agent-Based Simulation (ABS) one models and simulates a system by modeling and implementing the pro-active constituting parts of the system, called \textit{Agents} and their local interactions. From these local interactions then the emergent property of the system emerges. ABS is still a young field, having emerged in the early-to-mid 90s primarily in the fields of social simulation and computational economics. 

The aim of this paper is to show how ABS can be done in Haskell and what the benefits and drawbacks are. We do this by introducing the SIR model of epidemiology and derive an agent-based implementation for it based on Functional Reactive Programming. By doing this we give the reader a good understanding of what ABS is, what the challenges are when implementing it and how we solved these in our approach. We then discuss details which must be paid attention to in our approach and its benefits and drawbacks. The contribution is a novel approach to implementing ABS with powerful time-semantics and more emphasis on specification and possibilities to reason about the correctness of the simulation.