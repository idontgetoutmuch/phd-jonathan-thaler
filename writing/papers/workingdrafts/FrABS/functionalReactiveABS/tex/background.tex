\section{Background}

\subsection{Agent Based Simulation}
ABS is a methodology to model and simulate a system where the global behaviour may be unknown but the behaviour and interactions of the parts making up the system is of knowledge. Those parts, called agents, are modelled and simulated out of which then the aggregate global behaviour of the whole system emerges. So the central aspect of ABS is the concept of an agent which can be understood as a metaphor for a pro-active unit, situated in an environment, able to spawn new agents and interacting with other agents in some neighbourhood by exchange of messages \cite{wooldridge_introduction_2009}. We informally assume the following about our agents:

\begin{itemize}
	\item They are uniquely addressable entities with some internal state over which they have full, exclusive control.
	\item They are pro-active which means they can initiate actions on their own e.g. change their internal state, send messages, create new agents, terminate themselves.
	\item They are situated in an environment and can interact with it.
	\item They can interact with other agents which are situated in the same environment by means of messaging.
\end{itemize} 

Epstein \cite{epstein_generative_2012} identifies ABS to be especially applicable for analysing \textit{"spatially distributed systems of heterogeneous autonomous actors with bounded information and computing capacity"}. Thus in the line of the simulation types \textit{Statistic} $^\dag$, \textit{Markov} $^\ddag$, \textit{System Dynamics} $^\S$, \textit{Discrete Event} $^\mp$, ABS is the most powerful one as it allows to model  the following:

\begin{itemize}
	\item Linearity \& Non-Linearity $^{\dag \ddag \S \mp}$ - the dynamics of the simulation can exhibit both linear and non-linear behaviour. 
	\item Time $^{\dag \ddag \S \mp}$ - agents act over time, time is also the source of pro-activity.
	\item States $^{\ddag \S \mp}$ - agents encapsulate some state which can be accessed and changed during the simulation.
	\item Feedback-Loops $^{\S \mp}$ - because agents act continuously and their actions influence each other and themselves, feedback-loops are the norm in ABS. 
	\item Heterogeneity $^{\mp}$ - although agents can have same properties like height, sex,... the actual values can vary arbitrarily between agents.
	\item Interactions - agents can be modelled after interactions with an environment or other agents, making this a unique feature of ABS, not possible in the other simulation models.
	\item Spatiality \& Networks - agents can be situated within e.g. a spatial (discrete 2d, continuous 3d,...) or network environment, making this also a unique feature of ABS, not possible in the other simulation models.
\end{itemize}

\subsection{Functional Reactive Programming}
Functional Reactive Programming (FRP) is a way to implement systems with continuous and discrete time-semantics in pure functional languages. There are many different approaches and implementations but in our paper we use \textit{arrowized} FRP as implemented in the library Yampa which we will introduce shortly. For an in-depth introduction into FRP using Yampa we refer to the papers of \cite{hudak_arrows_2003}, \cite{courtney_yampa_2003} and \cite{nilsson_functional_2002}.
The central concept in arrowized FRP is the Signal Function (SF) which can be understood as a mapping from an input- to an output-signal. A signal can be understood as a value which varies over time.

\begin{flalign*}
Signal \, \alpha \approx Time \rightarrow a \\
SF \, \alpha \, \beta \approx Signal \, \alpha \rightarrow Signal \, \beta 
\end{flalign*}

Signal functions are implemented as continuations which don't take a $\Delta t$ at $t = 0$ but then change their signature into one which takes a $\Delta t$ for $t > 0$. This allows to hide $\Delta t$ completely from the types which makes them much more suitable for declarative programming. We also make use of Patersons do-notation for arrows \cite{paterson_new_2001} which makes the code much more readable.
TODO: this section needs to be more refined but we cannot spend too much space on FRP as it is out of scope of the paper and we only need a rough understanding of the signal function concept because agents are implemented as signal functions.