\subsection{Step VI: Adding agent transactions}
Imagine two agents A and B want to engage in a bartering process where agent A, is the seller who wants to sell an asset to agent B who is the buyer. Agent A sends Agent B a sell offer depending on how much agent A values this asset. Agent B receives this sell offer, checks if the price satisfies its utility, if it has enough wealth to buy the asset and replies with either a refusal or its own price offer. Agent A then considers agent Bs offer and if it is happy it replies to agent B with an acceptance of the offer, removes the asset from its inventory and increases its wealth. Agent B receives this acceptance offer, puts the asset in its inventory and decreases its wealth (note that this process could involve a potentially arbitrary number of steps without loss of generality).
We can see this behaviour as a kind of multi-step transactional behaviour because agents have to respect their budget constraints which means that they cannot spend more wealth or assets than they have. This implies that they have to 'lock' the asset and the amount of cash they are bartering about during the bartering process. If both come to an agreement they will swap the asset and the cash and if they refuse their offers they have to 'unlock' them.
In classic OO implementations it is quite easy to implement this as normally only one agent is active at a time due to sequential (discrete event scheduling approach) scheduling of the simulation. This allows then agent A which is active, to directly interact with agent B through method calls. The sequential updating ensures that no other agent will touch the asset or cash and the direct method calls ensure a synchronous updating of the mutable state of both objects with no time passing between these updates.
TODO: this is a novel concept as it is implicitly there already in OO implementations due to global mutual state.

TX: running agents and with 0 dt is easy with bearriver.

\subsubsection{Implementation}

\subsubsection{Reflection}