\section{Conclusions and Further Research}
Criticism of the FRP approach:
- can lead to infinite loops and needs much care of the programmer who has to consider operational details. This is already visible in the SIR model where we need to be careful to break a potential infinity loop in our simulation stepping. Also the state-machine of the SIR model is quite simple and thus the implementation quite straightforward, for more complex models with more complex e.g. nested state-machines, things will get pretty rough.
- having a two layer (arrows and pure functions) language in Yampa \cite{jeffrey_causality_2013} and three a layer (arrows, monadic and pure functions) language in Dunai / BearRiver adds expressivity and power but can makes things quite complex already in the simple SIR example. Fortunately with a more complex model the complexity in this context does not increase - in the end it is the price we need to pay for this high expressivity like occasionally.

We barely scratched the surface when discussing the use of dependent types in ABS. In contrast to many dependently typed programs in our approach there is actually an interest and need in running them as only then the dynamics of the simulation unfold over time. So far we looked only at how we can ensure the correctness of our mechanisms when using dependent types. It would be of immense interest whether we could apply dependent types also to the model meta-level or not - this boils down to the question if we can encode our model specification in a dependent type way? This would allow the ABS community for the first time to reason about a proper formalisation of a model, something the ABS community hasn't been very fond of so far.