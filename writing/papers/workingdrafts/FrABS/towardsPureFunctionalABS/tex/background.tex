\section{Background}

\subsection{Agent-Based Simulation}
We understand ABS as a method of modelling and simulating a system where the global behaviour may be unknown but the behaviour and interactions of the parts making up the system is of knowledge. Those parts, called agents, are modelled and simulated out of which then the aggregate global behaviour of the whole system emerges. So the central aspect of ABS is the concept of an agent which can be understood as a metaphor for a pro-active unit, situated in an environment, able to spawn new agents and interacting with other agents in a network of neighbours by exchange of messages \cite{wooldridge_introduction_2009}. It is important to note that we focus our understanding of ABS on a very specific kind of agents where the focus is on communicating entities with individual, localized behaviour from out of which the global behaviour of the system emerges. We informally assume the following about our agents:

\begin{itemize}
	\item They are uniquely addressable entities with some internal state over which they have full, exclusive control.
	\item They are pro-active which means they can initiate actions on their own e.g. change their internal state, send messages, create new agents, terminate themselves.
	\item They are situated in an environment and can interact with it.
	\item They can interact with other agents which are situated in the same environment by means of message-passing.
\end{itemize} 

\subsection{SIR Model}
- do not introduce SIR model in that length, also don't discuss SD and ABS, only minimal definition of what we understand as ABS, ignore definition of SD completely
- good introduction to pure functional programming in Haskell: this is VERY difficult as it is a VAST topic where one can get lost quickly. focus on the central concepts: no assignment, recursion, pattern matching, static type-system with higher-kinded polymorphism
- focus on the benefits of the pure functional approach
	-> program looks very much like a specification
	-> can rule out bugs at compile time
	-> can guarantee reproducibility at compile time
	-> 2 update-strategies without the need of different
	-> testing using quickcheck, testing = writing program spec
	-> reasoning: TODO

\subsection{Functional Reactive Programming}
FRP is a paradigm for programming hybrid systems which combine continuous and discrete components. Time is explicitly modelled: there is a continuous and synchronous time flow. There have been many attempts to implement FRP in libraries which each has its benefits and deficits. The very first functional reactive language was Fran, a domain specific language for graphics and animation. At Yale FAL, Frob, Fvision and Fruit were developed. The ideas of them all have then culminated in Yampa, the most recent FRP library \cite{nilsson_functional_2002}. The essence of FRP with Yampa is that one describes the system in terms of signal functions in a declarative manner using the EDSL of Yampa. During execution the top level signal functions will then be evaluated and return new signal functions which act as continuations. A major design goal for FRP is to free the programmer from 'presentation' details by providing the ability to think in terms of 'modeling'. It is common that an FRP program is concise enough to also serve as a specification for the problem it solves \cite{wan_functional_2000}.

Yampa has been used in multiple agent-based applications: \cite{hudak_arrows_2003} uses Yampa for implementing a robot-simulation, \cite{courtney_yampa_2003} implement the classical Space Invaders game using Yampa, \cite{nilsson_declarative_2014} implements a Pong-clone, the thesis of \cite{meisinger_game-engine-architektur_2010} shows how Yampa can be used for implementing a Game-Engine, \cite{mun_hon_functional_2005} implemented a 3D first-person shooter game with the style of Quake 3 in Yampa. Note that although all these applications don't focus explicitly on agents all of them inherently deal with kinds of agents which share properties of classical agents: game-entities, robots,... Other fields in which Yampa was successfully used were programming of synthesizers, network routers, computer music development and has been successfully combined with monads \cite{perez_functional_2016}.

This leads to the conclusion that Yampa is mature, stable and suitable to be used in functional ABS. This and the reason that we have the in-house knowledge lets us focus on Yampa. Also it is out-of-scope to do a in-depth comparison of the many existing FRP libraries.

\subsection{NetLogo}
One can look at NetLogo as a functional approach to ABMS which comes with its own EDSL. Our approach differs fundamentally in the following way
	- untyped
	- no side-effects possible
	- no direct access to other agents, communication happens through asynchronous messages or synchronized conversations
	- powerful time-semantics which NetLogo completely lacks 
