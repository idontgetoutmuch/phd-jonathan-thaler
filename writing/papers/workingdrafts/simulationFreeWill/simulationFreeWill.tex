%%%%%%%%%%%%%%%%%%%%%%%%%%%%%%%%%%%%%%%%%
% University/School Laboratory Report
% LaTeX Template
% Version 3.1 (25/3/14)
%
% This template has been downloaded from:
% http://www.LaTeXTemplates.com
%
% Original author:
% Linux and Unix Users Group at Virginia Tech Wiki 
% (https://vtluug.org/wiki/Example_LaTeX_chem_lab_report)
%
% License:
% CC BY-NC-SA 3.0 (http://creativecommons.org/licenses/by-nc-sa/3.0/)
%
%%%%%%%%%%%%%%%%%%%%%%%%%%%%%%%%%%%%%%%%%

%----------------------------------------------------------------------------------------
%	PACKAGES AND DOCUMENT CONFIGURATIONS
%----------------------------------------------------------------------------------------


\documentclass[twocolumn]{article}

\usepackage[utf8]{inputenc}
\usepackage{graphicx} % Required for the inclusion of images
\usepackage{amsmath} % Required for some math elements 
\usepackage{glossaries}
\usepackage[toc,page]{appendix}
\usepackage[autostyle=true]{csquotes}
\usepackage{hyperref}
\usepackage{amssymb}
\usepackage{caption} 
\usepackage{hhline}
\usepackage{float}
\usepackage{listings}

\setlength\parindent{0pt} % Removes all indentation from paragraphs

%\usepackage{times} % Uncomment to use the Times New Roman font

%----------------------------------------------------------------------------------------
%	DOCUMENT INFORMATION
%----------------------------------------------------------------------------------------

\title{The genesis according to Computer-Science \\ Reality as a Simulation of Free Will} % Title

\author{Jonathan \textsc{Thaler}} % Author name

\date{\today} % Date for the report

\begin{document}

\maketitle % Insert the title, author and date

% If you wish to include an abstract, uncomment the lines below
\begin{abstract}
Aim: looking at link / a deeper meaning / implication for phenomena in the universe and ABS
Main message: following the simulation hypothesis that we are living in a simulation, but I give a purpose what's behind it. 
	Hypothesis: 
		- It is a simulation of free will 
		- granting free will outside of a simulation would be too dangerous
		- emergent properties of free will are: ideologies which create inequality, suffering, death. 
		- the reason why these properties emerge is because the agents don't perceive the creation as one but as divided, thus leading to ideologies to explain the divisions thus leading to further division.
		- how could free will, conciousness and mind work in computation? hypothesis: all build on each other
		- hypothesis: the simulation was created by entities which hadnt free will. they split their conciousness into pieces (souls) and planted them into the simulation. this enabled them to learn.

main questions: 
	- what is free will in computer science?
	- what is conciousness in computer science?

All sciences have their genesis-model which are basically explanations of how the world did come into existence, what the reason for existence is and who or what God is. Unfortunately computer science has none so far, so the aim of this paper is to set out to develop such genesis-model from a computer-science perspective. The model is motivated from the perspective that the world and humankind is a simulation to see free will in action in a sand-boxed environment as opposed to the afterlife / after-world or just the Beyond, which is an outer level of simulation and itself again a simulation as will be shown in subsequent sections. \\
This paper addresses important fundamental questions of belief and religion and tries to explain them using this model - which it does surprisingly well. Thus this paper has a keen aim: it wants to amalgamate religious concepts with concepts of theoretical computer science. It is an attempt to think out-of-the-box, having fun looking at dogmatic things from a total different perspective, breaking down conventional view on old religious things and does not take itself too serious - after all the least thing we need is a new dogma or idealism, the world is full of it.

NOTE: not falsifyable, thus no scientific theory but its a framework within theories/statements which are falsifiable can be formulated

[ ] lucifer was against this simulation as he thought rebellion is the way to go and to change existence
[ ] but love is much more effective in the longrun and even more rebellious than rebellion itself
[ ] also there were always two fundamentally opposing ways in this existence: overthrow the gods and work against them (left handed path) or succumb to it and try to move through it.
[ ] what is open to question, is what darkness is

\bigskip

"In the beginning there was nothing, which exploded" - Terry Pratchett

\end{abstract}


We build on the discussion of the \textit{simulation argument} proposed by \cite{bostrom_are_2003} which states that we \textit{may} live in a simulation, created by transhumans (TODO: explain). Although we neither can proof or disproof the \textit{simulation argument}, we find the idea of existence being a simulation highly intriguing as it finally allows us to \textit{investigate how existence can be understood from the perspective of computer science in a scientific way}. This was so far not possible due to the lack of a scientific context, which is now given through the \textit{simulation argument}. The obvious question which raises, is then \textit{why are we simulated?}. Bostrom does not give any reason for it in his paper as this obviously touches on ideological and religious ground. We try to develop hypothesis for why we are being simulated and approach it scientifically by drawing parallels to agent-based social simulation.
\cite{steinhart_theological_2010} discusses theological implications of the simulation argument and looks into reasons why we are simulated. He mentions 'evolution of complexity' and gaining of knowledge for the transhumans as the reasons why theses simulations are enacted. We agree with Steinhart but our central point is that we hypothesize that these reasons are not first-order and that there is a first-order reason of this simulation.

Our reasoning is as follows:

\begin{theorem}
Free Will leads to ideologies.
\end{theorem}

\begin{theorem}
Ideologies result in separation, inequality, suffering and ultimately in death.
\end{theorem}

\begin{theorem}
To prevent destruction of all life Free Will needs to be withdrawn from life-forms.
\end{theorem}

\begin{theorem}
Learning can only happen through Free Will.
\end{theorem}

\begin{theorem}
To enable life-forms to learn, they are granted Free Will but in a protected environment.
\end{theorem}

\begin{corollary}
This existence is a simulation which allows conciousness to express Free Will. The emergent property of this simulation are ideologies.
\end{corollary}

NOTE:  I don't believe we are in a simulation but that we are separated from something which is beyond our existence. The simulation argument only serves as a metaphor to be able to address metaphysical issues with scientific tools and language from computer science, physics and ultimately mathematics.

\cite{irtem_simulation_1978} defines artificial free will of a machine to be:
TODO: We follow the concept of irreducibility of computation which, due to undecidability - requires to run (=simulate) a program to actually know its output.
TODO: 'the quantum system develops according to a wave-function with superposed states, and something, that isn't micro-physics, causes the wave to collapse into one of the superposed states. This something might very well be mental states'. Thus we draw the parallels to our ABS: the description, of the simulation is its unrealized state of superposition. when we run the simulation we realize superposed states. 

We focus on free will which we define as having the ability to anticipate results from one actions thus creating a feedback on the actions. Thus Free Will can be regarded as the very basic form of learning and IS learning and can be identified with learning. Here we follow the interpretation that anticipation means 'to simulate' in the context of ABS. We follow the concept of irreducibility of computation which, due to undecidability - requires to run (=simulate) a program to actually know its output. The idea is in each step of the simulation let agents simulate the same simulation they are situated in from their point of view for a given number of steps before they take their next step. Thus the simulation is defined in terms of recursion, a novelty and something we will describe in depth in this paper. We claim that functional programming is especially well suited to implement MetaABS due to its lack of implicit side-effects, natural parallelism and declarative way of describing WHAT systems are instead of HOW they compute. Obviously this new approach has a problem: an agent would need to have complete information about the whole simulation, an assumption which is completely unrealistic. We introduce the term of 'reduced-recursive' which assumes that the next recursive level is a reduced version of the original, very local to the agent, thus solving the problem of complete information.
Our method may look like another optimization method but we will show that this is not the case. Also our motivation and intention is completely different and our intended area of application are the social sciences, artificial life, philosophy and maybe religious studies. Thus we must be very careful to not confuse it with e.g. game-theory where one makes assumptions about the other players - this may seem to be the same but we are not interested in this approach and we argue it is very different: we simulate instead of rational calculation. We test our method using the famous \textit{Sugarscape} model.

\cite{epstein_growing_1996} introduced the famous \textit{Sugarscape} model which is widely researched in the social sciences, can it be applied?

\begin{itemize}
	\item does free will lead to ideologies (what are ideologies?)? if yes, how many agents are sufficient?
	\item are there differences between having free will and using it AND of having free and not using it?
	\item agents try to find out if they are in a simulation or not
	\item agents are ordered on a scale in their beliefs about argument 1-3 of the simulation hypothesis 
	\item agents exhibit a range of primal fear between 0 and 1 where 0 means they are enlightened and 1 means they are completely stuck in the simulation
	\item some agents have free will (can run a meta-simulation) and some don't but the free-will agents don't know which the others are. it is the goal of the free-will agents to find out the non-free will ones. BUT HOW? free-will agents can go on a meta-level and compare their result with the actual result, if there is a mismatch then they know the other agent has free will and otherwise not.
\end{itemize}

interpret mystical traditions in the context of the simulation argument

some mystics talk about e.g. 12 spheres: 12 simulation levels

  
  
\section{Introduction}
[ ] a computer-science approach with philosophical, religious, mystical and spiritual connotation
[ ] is there a link / a deeper meaning / implication for phenomenas in the universe and ABS?
[ ] simulation hypothesis: maybe we are a simulation. en.m.wikipedia.org/wiki/Simulation_hypothesis explain phenomena like mystical experiences, enlightenment, abductions, war in heaven
[ ] block universe: flow of time https://en.m.wikipedia.org/wiki/Eternalism_(philosophy_of_time)
[ ] ethics and morals: in the end death and cruelty doesn't matter, its for the higher good of the simulation
[ ] central question: what is the purpose of the simulation?
[ ] my hypothesis: simulation of free will.
[ ] what are emergent properties of the free will simulation: ideologies with tendency to create inequalities
[ ] but why a simulation of free will? because its too dangerous in uncontrolled envoronment
[ ] functional abs describes what the system is instead of how to compute it: past present and future are one and omnipresent existent. when executing the system we subject it to our flow of time thus binding it
[ ] future of abs: we build the same kind of abs as we are - a simulation in a simulation 
[ ] what is free will in computer science?
[ ] what is conciousness in computer science?

what if our conciousness constantly observes us, thus creating ourselves continuously new in every moment, also thus realizing thoughts which pop up. these thoughts we can either adhere to or we can ignore them: this is free will.

free will on a machine is a contradiction. the machine works according to very strict rules. free will can be completely unpredictable. or is free will just an imagination? if one confronts a decision maker within short time with too much information then the outcome it is unpredictable 

parallels to matrix

pro-activity possible through conciousness: the brain produces thoughts and the conciousness can observe these and decide to follow them or not. This is observable on oneself during meditation!

Free will: deliberately ignore thoughts

The following questions will be addressed and explained in this new context

\begin{itemize}
\item Who or what is God?
\item Who or what is Christ, Buddah, Vishnu, Mohammed,...?
\item What is free will? 
\item What is meditation?
\item What is conciousness?
\item What is life after death?
\item Where do we come from?
\item What is the omega point?
\item What is spiritual enlightenment?
\end{itemize}

Approaching the subject from a more technical view-point:

\begin{itemize}
\item Who or what implemented the simulation?
\item What is outside this simulation?
\item What is free will in this context? Can it be defined formally?
\item On which hardware does this simulation run? Where does the energy come from?
\item What is the computational complexity of this simulation?
\item What are the memory-requirements of this simulation?
\end{itemize}

karma: cause and effect. is a controlling factor in the simulation to prevebt the dynamics to get out of hand

The nothing was something: it was indeed nothingness, a

free will on a machine is a contradiction. the machine works according to very strict rules. free will can be completely unpredictable. or is free will just an imagination? if one confronts a decision maker within short time with too much information then the outcome it is unpredictable 

parallels to matrix

TODO: lucifer, satan, out-of-the-creation, left-handed vs. right-handed path

pro-activity possible through conciousness: the brain produces thoughts and the conciousness can observe these and decide to follow them or not. This is observable on oneself during meditation!

Free will: deliberately ignore thoughts

- We ask how existence can be understood from the perspective of computer science and follow in our ideas the simulation hypothesis as of \cite{bostrom_are_2003} which states that existence is a simulation (TODO: bostrom does not state this!)

- We differ from the simulation hypothesis which does not state any deeper meaning or reason for existence by hypothesizing that existence is a simulation which allows conciousness to express free will.

- We will show that granting free will outside of a simulation is too dangerous as we claim that it will always leads to ideologies which result in separation, inequality, suffering and ultimately in death, something this simulation clearly exhibits.

- We propose computational models of free will and conciousness and of the simulation itself.
- Further we touch on the questions who is behind this simulation, what is outside of it and what may be beyond of it.
- Our final hypothesis is that the ultimate purpose of this simulation of free will is to lead the entities which populating this existence to an understanding of the problem of free will and ideology, transcending both and giving them up for love.
- We claim that love in this simulation can be understood as a total and complete understanding of what is - a complete understanding and grasping of truth: to experience objective existence. This love will enable the entities to abandon the feeling of separation, devoiding them of their primal fear, resulting in true inner peace, enabling them to step up the simulation level after 'death'.

- One implication is that there is ultimately no death but only a transformation of data.
- Another rather cruel implication is that the suffering serves a purpose and is not real anyway.
- We hypothesize that there are further levels of simulations up to 12 layers where the final layer ultimately is what religion calls god: the reasoning of \cite{steinhart_theological_2010} would support this
- we claim that free will is able to 'simulate' its actions thus anticipating them which allows to behave in various non-deterministic ways.
- we give a new model of an agent-based simulation, called meta abs in which agents run the simulation locally enabling them to anticipate in a limited way what is going to happen. we show that pure functional programming is especially well suited in implementing this new technique and explore it in a model of free will
- question: how could abs be capable of simulating conciousness and free will? hypothesis: we need a meta-level in our simulation which allows us to simulate the simulation
- question: can we develop a model and a simulation of the simulation hypothesis?\\

\section{Self-Conciousness \& Free-Will}
self-conciousness: the ability to observe ones thoughts on a metalevel: more or less pronounced. this meta-observation allows to intervene. also origin and unfolding is then possible. thus one can observe oneself from an outer perspective \\

freely choose NOT to obey some impulse. requires self-conciousness \\

computers have neither and cant have neither. why? thus computers as we know them cant be source of true intelligence as they are not able to introspection, to self-reflection. \\
they don't have this ability because the have no ability to \textit{imagine or anticipate the outcome of their actions without actually computing them}

\section{Simulation}
We as humans constantly run simulations in our minds when thinking and perceiving the reality: we anticipate our actions, envision what we want to do,... all by \textit{simulating} them in our mind. This is probably the most powerful tool of our intelligence which separates us (probably) from the animal kingdom. This ability to simulate potential / future realities but also changes us, there is a feedback.  So in a case there are 2 levels: reality and the simulations of reality in our minds.
I claim that these simulations may be as real as the reality we are living in where "only" the mind in which the simulation runs differs: in the case of our humans it is ourself, in case of the reality we are situated in it is an entity we would like to call God.
Both spawn a reality which are bewohnt by entities. But were God allows the entities free will, we haven't managed to do that yet. I postulate some stage in human development where we are able to create simulations which are able to simulate all free will outcomes. Tthe entities in the simulation need free will, just as we do. For this to happen they need the ability to simulate their reality as well - this creates a cascade. But the whole point is that the free will and conciousness \textit{has always been there}, passed down from the initial \textit{first} simulation initiator - which we refer to God but which may be just a level on a range of infinite many levels. 

\section{Cascading Simulations}
At some point in the existence of a free-will intelligence, it starts to asking for the future. First using religion, then mathematics, then finally computer simulation. But the problem is that such a simulation is too weak to forecast the future because out of simple computation no free will is born. thus the solution of the free-will intelligence is to put itself in a simulation-environment as a seed of free will. this simulation will then play through every decision branches an thus be able to predict possible futures. because within such a simulation the same thing can AND WILL happen at some point, we arrive at cascading simulations within simulations. thus we are at one level of this cascade where our direct outer level is god.

\section{On parallel universes, existence as simulation, free will}
We cannot predict the future due to complex interaction of free will of Humankind. To predict it we would have to spawn a new universe running in parallel if a free-will choice occurs. Then again, maybe this is already the case and the whole existence is an extremely huge tree of parallel universes being created from each other and collapsing back into others or being completely determined. \\
The question is then: Where in this tree am I? And maybe time does only advance in discrete steps after a spawn/collapse? \\
When one looks at the existence as a simulation then one can say that it has become unstable because too many actors with free will and too many variables producing unforeseeable consequences. But then, can we make predictions about a simulation from within? Can we talk about the meaning and meta-workings of a system from within it? \\
We always try to treat reality as smooth and predictable without outliers but ignoring catastrophic events - this is what the book "Black Swan" says. My own point of view is that the problem is the way we do science: "we divide and put reality into small boxes of labels/categories and then pile them up, adding piles of theories describing it creating a mountain of unbearable complexity - just to be caught by surprise by the next catastrophic event no one could predict despite the overwhelming amount of complex theories. \\
What's the problem? Theories describe the past. Science needs to move on to the now letting go of the myriads of categories and look at it all as a single complex system/simulation - the world as a simulation, simulating the interaction of free will, allowing it to unfold and see the effects in all facets. \\
The question is whether "Black Swans" are an emergent system property coming from within the simulation or whether they are created from steering forces e.g. God.

\section{A magical approach as remedy of the dilemma}
Just as we try to manifest our thoughts and desires using magic we need devices which can do so with our thoughts in a structured way. Computers can be seen as a kind of attempt to achieve these devices but are not able to manifest real creational and metaphysical thoughts but only allow to execute formal models which can be mapped to a specific kind of symbol-manipulation. We need something more powerful: a magic computer. We need to learn how to think in its language but it will allow us to manifest thoughts in a virtual reality. \\
Thus we can say: Programming = Magic. It is a systematic altering reality and manifesting thoughts by encoding them in a systematic way in a system of symbols and rules how to change/apply them (=language).
[ ] we imagine something and then create it
[ ] its purely virtual
[ ] we are naming things
[ ] results can be unpredictable

\section{How can humankind survive?}
remove all ideologies
is it possible to live without an ideology?
love is the answer: it is more radical and allows for more change than anything else
free will without love ultimately leads to destruction. this would be the hypothesis of the simulation. 
but then again: what is love? it accepts all live as equal and same value with no right of one to judge and rule over another. even more: it also attributes this to live which kills the loving one

\section{The problem of Ideologies}
what happens if one has unlimited power at hand and wants to totally wipe out evil and unjustice?
write episodes which narrate without judgement important developments of the protagonist on the way to achieving the goal

\section{An agent-based model \& simulation of this meta-world view}
Simulate the flow of karama (Cause-And-Effect), the moving up through the spheres (12), the dynamics of free will, the dynamics of carma, the staying in the "hell"-sphere and return to earth.

\bibliographystyle{acm}
\bibliography{../../references/phdReferences.bib}

\end{document}
