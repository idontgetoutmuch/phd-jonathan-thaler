\section{Introduction}
The traditional approach to Agent-Based Simulation (ABS) has so far always been object-oriented techniques, due to the influence of the seminal work of Epstein et al \cite{epstein_growing_1996} in which the authors claim "[..] object-oriented programming to be a particularly natural development environment for Sugarscape specifically and artificial societies generally [..]" (p. 179). This work established the metaphor in the ABS community, that \textit{agents map naturally to objects} \citep{north_managing_2007} which still holds up today.

In this paper we challenge this metaphor and explore ways of approaching ABS using the functional programming paradigm with the language Haskell. We present fundamental concepts and advanced features of functional programming and we show how to leverage the benefits of it \citep{hudak_history_2007} to become available when implementing ABS functionally.

We show that by using functional programming, it is harder to make mistakes, the resulting simulations are easier to test and verify, guaranteed to be reproducible, have less potential sources of bugs as in \citep{vipindeep_list_2005} and are ultimately more likely to be correct. Further we explore testing and scaling up of simulations in the functional programming context. All of this is of paramount importance in scientific computing, in which results need to be reproducible and correct while simulations should be able to scale up to massive numbers of agents as well. 

We present case-studies in which we employ the well known SugarScape \citep{epstein_growing_1996} and agent-based SIR \citep{macal_agent-based_2010} model to test our hypothesis. The former model can be seen as one of the most influential exploratory models in ABS which laid the foundations of object-oriented implementation of agent-based models. The latter one is an easy-to-understand explanatory model which has the advantage that it has an analytical theory behind it which can be used for verification and validation.

The aim of this paper is to conceptually show \textit{how} to implement ABS in functional programming using Haskell and \textit{why} it is of benefit of doing so. Further, we give the reader a good understanding of what functional programming is, what the challenges are in applying it to ABS and how we solve these in our approach. Although functional programming is a highly technical subject, we avoid technical discussions and follow a very high-level approach, focusing on the concepts instead of their implementation. For readers which are interested in technical details we refer to relevant literature in the respective parts of the paper.

The paper makes the following contributions:

\begin{itemize*}
	\item To the best of our knowledge, we are the first to introduce the functional programming paradigm using Haskell to ABS on a \textit{conceptual} level, identifying benefits, difficulties and drawbacks.
	\item We show how functional programming concepts can be used to create simulation software which has less sources of bugs, is more likely to be correct and guaranteed to be reproducible already at compile-time.
	\item We show how functional programming concepts can be used to scale ABS up to massively large-scale without the problems of low level concurrent programming.
	\item We show how functional programming concepts can be used to test ABS implementations using property-based testing, which allows a very expressive way of testing, shifting from unit- towards specification-based testing.
\end{itemize*}
