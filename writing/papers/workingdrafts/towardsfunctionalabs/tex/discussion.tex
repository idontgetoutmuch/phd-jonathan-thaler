\section{Discussion}
TODO: show why the claim of the introduction holds up

The restrictions functional programming imposes, directly removes serious sources of bugs which leads to simulation which is more likely to be correct. These restrictions force us to solve the fundamental concepts in ABS implementation differently. Note that we could fall back to using IO throughout all the simulation in which case we have access to mutable references but then we lose important compile-time guarantees and introduce those serious sources of bugs we want to get rid of - also testing becomes more complicated and not as strong any more because we cannot guarantee at compile time that no random IO stuff is happening within the agents. Also note that obviously no one would do random IO stuff in an agent (e.g. read from a file, open connection to server...) but one must not understimate the value of guaranteeing its absence at compile-time.

space leaks biggest issue in fp approach, but also big strength as it allows to separate consumption from production

emphasis that aivika is multi-method, gis, distributed. 