\section{Conclusions}

- being explicit and polymorph about side-effects: can have 'pure' (no side-effects except state), 'random' (can draw random-numbers), 'IO' (all bets are off), STM (concurrency) agents
- hybrid between SD and ABS due to continuous time AND parallel dataFlow (parallel update-strategy)

Our approach is radically different from traditional approaches in the ABS community. First it builds on the already quite powerful FRP paradigm. Second, due to our continuous time approach, it forces one to think properly of time-semantics of the model and how small $\Delta t$ should be. Third it requires one to think about agent interactions in a new way instead of being just method-calls.

Because no part of the simulation runs in the IO Monad and we do not use unsafePerformIO we can rule out a serious class of bugs caused by implicit data-dependencies and side-effects which can occur in traditional imperative implementations.

Also we can statically guarantee the reproducibility of the simulation, which means that repeated runs with the same initial conditions are guaranteed to result in the same dynamics. Although we allow side-effects within agents, we restrict them to only the Random and State Monad in a controlled, deterministic way and never use the IO Monad which guarantees the absence of non-deterministic side effects within the agents and other parts of the simulation.

Determinism is also ensured by fixing the $\Delta t$ and not making it dependent on the performance of e.g. a rendering-loop or other system-dependent sources of non-determinism as described by \cite{perez_testing_2017}. Also by using FRP we gain all the benefits from it and can use research on testing, debugging and exploring FRP systems \cite{perez_testing_2017, perez_back_2017}.
