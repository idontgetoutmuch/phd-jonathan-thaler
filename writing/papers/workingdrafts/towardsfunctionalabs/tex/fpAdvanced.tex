\section{Advanced Concepts}
In this section we give a brief overview over advanced concepts found in functional programming. 

\subsection{Parallelism and Concurrency}
TODO: write this section

Also Haskell makes a very clear distinction between parallelism and concurrency. Parallelism is always deterministic and thus pure without side-effects because although parallel code runs concurrently, it does by definition not interact with data of other threads. This can be indicated through types: we can run pure functions in parallel because for them it doesn't matter in which order they are executed, the result will always be the same due to the concept of referential transparency. Concurrency is potentially non-deterministic because of non-deterministic interactions of concurrently running threads through shared data. Although data in functional programming is immutable, Haskell provides primitives which allow to share immutable data between threads. Accessing these primitives is but only possible from within an IO or STM context which means that when we are using concurrency in our program, the types of our functions change from pure to either IO or STM effect context.

Spawning thousands of threads in Haskell is no problem and has very low memory footprint because they are lightweight threads, managed by the Haskell Runtime System which maps them to physical threads. 

TODO: in haskell we can distinguish between parallelism and concurrency in the types: parallelism is pure, concurrency is impure
TODO: parallelism for free because all isolated e.g. running multiple replications or parameter-variations

TODO: explain STM, Problem: live locks, For a technical, in-depth discussion on Software Transactional Memory in Haskell we refer to the following papers: \citep{harris_composable_2005, osullivan_real_2008}. TODO: need much more papers on STM, parallelism and concurrency

\subsection{Functional Reactive Programming}
\label{sec:frp}
Functional Reactive Programming (FRP) is a way to implement systems with continuous and discrete time-semantics in functional programming. The central concept in FRP is the Signal Function which can be understood as a process over time which maps an input- to an output-signal. A signal in turn, can be understood as a value
which varies over time. Thus, signal functions have an awareness of the passing of time by having access to a $\Delta t$ which are positive time-steps with which the system is sampled. In general, signal functions can be understood to be computations that represent processes, which have an input of a specific type, process it and output a new type. Note that this is an important building block to represent agents in functional programming: by implementing agents as signal functions allows us to implement them as processes which act continuously over time, which implies a time-driven approach to ABS. We have also applied the concept of FRP to event-driven ABS \citep{meyer_event-driven_2014}.

FRP provides a number of functions for expressing time-semantics, generating events and making state-changes of the system. They allow to change system behaviour in case of events, run signal functions, generate deterministic (after fixed time) and stochastic (exponential arrival rate) events and provide random-number streams. 

TODO: libraries Yampa and Dunai

For a technical, in-depth discussion on FRP in Haskell we refer to the following papers: \citep{wan_functional_2000, hughes_generalising_2000, hughes_programming_2005, nilsson_functional_2002, hudak_arrows_2003, courtney_yampa_2003, perez_functional_2016, perez_extensible_2017}

\subsection{Property-Based Testing}
TODO: write this section

Although property-based testing has been brought to non-functional languages like Java and Python as well, it has its origins in Haskell and it is here where it truly shines.

We found property-based testing particularly well suited for ABS. Although it is now available in a wide range of programming languages and paradigms, propert-based testing has its origins in Haskell \citep{claessen_quickcheck:_2000, claessen_testing_2002} and we argue that for that reason it really shines in pure functional programming. Property-based testing allows to formulate \textit{functional specifications} in code which then the property-testing library (e.g. QuickCheck \citep{claessen_quickcheck:_2000}) tries to falsify by automatically generating random test-data covering as much cases as possible. When an input is found for which the property fails, the library then reduces it to the most simple one. It is clear to see that this kind of testing is especially suited to ABS, because we can formulate specifications, meaning we describe \textit{what} to test instead of \textit{how} to test (again the declarative nature of functional programming shines through). Also the deductive nature of falsification in property-based testing suits very well the constructive nature of ABS.

For a technical, in-depth discussion on property-based testing in Haskell we refer to the following papers: \citep{claessen_quickcheck:_2000, claessen_testing_2002}.

%\subsection{Software Transactional Memory}
%Although there exist STM implementations in non-functional languages like Java and Python, due to the nature of Haskells type-system, the use of STM has unique benefits in this setting.
%
%Concurrent programming is notoriously difficult to get right because reasoning about the interactions of multiple concurrently running threads and low level operational details of synchronisation primitives and locks is \textit{very hard}. The main problems are:
%
%\begin{itemize}
%	\item Race conditions due to forgotten locks.
%	\item Deadlocks resulting from inconsistent lock ordering.
%	\item Corruption caused by uncaught exceptions.
%	\item Lost wakeups induced by omitted notifications.
%\end{itemize}
%
%Worse, concurrency does not compose. It is utterly difficult to write two functions (or methods in an object) acting on concurrent data which can be composed into a larger concurrent behaviour. The reason for it is that one has to know about internal details of locking, which breaks encapsulation and makes composition depend on knowledge about their implementation. Also it is impossible to compose two functions e.g. where one withdraws some amount of money from an account and the other deposits this amount of money into a different account: one ends up with a temporary state where the money is in none of either accounts, creating an inconsistency - a potential source for errors because threads can be rescheduled at any time.
%
%STM promises to solve all these problems for a very low cost. In STM one executes actions atomically where modifications made in such an action are invisible to other threads until the action is performed. Also the thread in which this action is run, doesn't see changes made by other threads - thus execution of STM actions are isolated. When a transaction exists one of the following things will occur:
%
%\begin{enumerate}
%	\item If no other thread concurrently modified the same data as us, all of our modifications will simultaneously become visible to other threads.
%	\item Otherwise, our modifications are discarded without being performed, and our block of actions is automatically restarted.
%\end{enumerate}
%
%Note that the ability to \textit{restart} a block of actions without any visible effects is only possible due to the nature of Haskells type-system which allows being explicit about side-effects: by restricting the effects to STM only ensures that no uncontrolled effects, which cannot be rolled-back, occur.
%
%STM is implemented using optimistic synchronisation. This means that instead of locking access to shared data, each thread keeps a transaction log for each read and write to shared data it makes. When the transaction exists, this log is checked whether other threads have written to memory it has read - it checks whether it has a consistent view to the shared data or not. This might look like a serious overhead but the implementations are very mature by now, being very performant and the benefits outweigh its costs by far.
%
%Applying this to our agents is very simple: because we already use Dunai / BearRiver as our FRP library, we can run in arbitrary Monadic contexts. This allows us to simply run agents within an STM Monad and execute each agent in their own thread. This allows then the agents to communicate concurrently with each other using the STM primitives without problems of explicit concurrency, making the concurrent nature of an implementation very transparent. Further through optimistic synchronisation we should arrive at a much better performance than with low level locking.
%
%Problem: live locks
%
%For a technical, in-depth discussion on Software Transactional Memory in Haskell we refer to the following papers: \citep{harris_composable_2005, osullivan_real_2008}.