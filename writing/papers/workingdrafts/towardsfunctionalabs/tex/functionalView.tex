\section{A functional approach}
Due to the fundamentally different approaches of functional programming (FP) an ABS needs to be implemented fundamentally different as well compared to established object-oriented (OO) approaches. We face the following challenges:

\begin{enumerate}
	\item How can we represent an Agent, its internal state and its interface? \\
	In OO the obvious approach is to map an agent directly onto an object which encapsulates data and provides methods which implement the agents actions. Obviously we don't have objects in FP thus we need to find a different approach to represent the agents actions and to encapsulate its state.
	In the classic OO approach one represents the state of an Agent explicitly in mutable member variables of the object which implements the Agent. As already mentioned we don't have objects in FP and state is immutable which leaves us with the very tricky question how to represent state of an Agent which can be actually updated.
	In the established OO approach, agents have a well-defined interface through their public methods through which one can interact with the agent and query information about it. Again we don't have this in FP as we don't have objects and globally mutable state.
	
	\item How can we implement pro-activity of an Agent? \\
	In the classic OO approach one would either expose the current time-delta in a mutable variable and implement time-dependent functions or ignore it at all and assume agents act on every step. At first this seems to be not a big deal in FP but when considering that it is yet unclear how to represent Agents and their state, which is directly related to time-dependent and reactive behaviour it raises the question how we can implement time-varying and reactive behaviour in a purely functional way.
	
	\item How can we represent an environment and its various types? \\
	In the classic OO approach an environment is almost always a mutable object which can be easily made dynamic by implementing a method which changes its state and then calling it every step as well. In FP we struggle with this for the same reasons we face when deciding how to represent an Agent, its state and proactivity.
	
	\item How can we implement agent-agent and agent-environment interactions? \\
	In the classic OO approach Agents can directly invoke other Agents methods which makes direct Agent interaction straight forward. Again this is obviously not possible in FP as we don't have objects with methods and mutable state inside.
		In the classic OO approach agents simply have access to the environment either through global mechanisms (e.g. Singleton or simply global variable) or passed as parameter to a method and call methods which change the environment. Again we don't have this in FP as we don't have objects and globally mutable state.
	
	\item How can we step the simulation? \\
	In the classic OO approach agents are run one after another (with being optionally shuffled before to uniformly distribute the ordering) which ensures mutual exclusive access in the agent-agent and agent-environment interactions. Obviously in FP we cannot iteratively mutate a global state.
\end{enumerate}

The fundamental building blocks to solve these problems are \textit{recursion} and \textit{continuations}. In recursion a function is defined in terms of itself: in the process of computing the output it \textit{might} call itself with changed input data. Continuations in turn allow to encapsulate the execution state of a program including local variables and pick up computation from that point later on. We present an example for continuations and recursions:

\begin{HaskellCode}
newtype Cont a = Cont (a -> (a, Cont a))

adder :: Int -> Cont Int
adder x = Cont (\x' -> (x + x', adder (x + x')))

runCont :: Int
        -> Cont Int
        -> IO ()
runCont 0 _ = return ()
runCont n (Cont cont) = do
  let (x, cont') = cont 1
  print x
  runCont (n-1) cont'

test :: IO ()
test = runCont 100 (adder 0)
\end{HaskellCode}

Fortunately FRP (see Section \ref{sec:frp}) provides us already with a suitable abstraction, the signal function, which are built on \textit{recursion} and \textit{continuations}. Using signal functions and FRP allows us to solve the presented problems.

\subsection{Agent representation, internal state and interface}
Continuation: It is apparent that functional programming supports very strong encapsulation of local state which is not accessible and mutable from outside. This makes testing in some respect easier, in some harder. 

\subsection{Agent pro-activity}
TODO: time-driven vs. event-driven 
time is represented using the FRP concept: Signal-Functions which are sampled at (fixed) time-deltas, the dt is never visible directly but only reflected in the code and read-only.

\subsection{Agent-Agent interactions}
Agent-transactions are necessary when an arbitrary number of interactions between two agents need to happen instantaneously without time-lag. The use-case for this are price negotiations between multiple agents where each pair of agents needs to come to an agreement in the same time-step \cite{epstein_growing_1996}. In object-oriented programming, the concept of synchronous communication between agents is trivially implemented directly with method calls.

\subsection{Environment representation and Agent-Environment interactions}
TODO: basically we follow the same approach as in agent representation, depending on what kind of environments we need

Generally, there exist four different types of environments in agent-based simulation:
\begin{enumerate}
	\item Passive read-only - implemented in the previous steps, where the environment never changes and is passed as static information, e.g. list of neighbours, to each agent.
	\item Passive read/write - implemented in this step. The environment itself is not modelled as an active process but just as shared data which can be accessed and manipulated by the agents.
	\item Active  read-only - can be implemented by adding an environment agent which broadcasts changes in the environment to all agents using the data-flow mechanism.
	\item Active read/write - can be implemented as in this step plus adding an environment agent which reads/writes the environment e.g. regrowing some resources.
\end{enumerate}

\subsection{Stepping the simulation}
TODO: time-driven vs. event-driven 
This allows to implement feedback by recursively stepping a simulation and feeding in the output of the last step into the next step.

TODO: depending on the time-semantics of the model we approach it different. 
TODO: SIR maps nicely to continuous time-semantics and state-transitions provided by FRP, property-testing can be used to directly express parts of the SD specification
TODO: Sugarscape: no need for continuous time-semantics as agents act in all time-step, main difficulty: synchronous agent-interactions, property-based testing: can we express hypotheses?