\section{Concepts of Functional Programming}
In our research we are using the functional programming language Haskell. The paper of \citep{hudak_history_2007} gives a comprehensive overview over the history of the language, how it developed and its features and is very interesting to read and get accustomed to the background of the language. A widely used introduction to programming in Haskell is \citep{hutton_programming_2016}, a more conceptual introduction to functional programming can be found in \citep{maclennan_functional_1990}. The main points why we decided to go for Haskell are:

\begin{itemize*}
	\item Rich Feature-Set - it has all fundamental concepts of the pure functional programming paradigm of which we explain the most important below.
	\item Real-World applications - the strength of Haskell has been proven through a vast amount of highly diverse real-world applications \footnote{\url{https://wiki.haskell.org/Applications_and_libraries}} \cite{hudak_history_2007}, is applicable to a number of real-world problems \cite{osullivan_real_2008} and has a large number of libraries available.
	\item Modern - Haskell is constantly evolving through its community and adapting to keep up with the fast changing field of computer science. Further, the community is the main source of high-quality libraries.
\end{itemize*}

The roots of functional programming lie in the Lambda Calculus which was first described by Alonzo Church \citep{church_unsolvable_1936}. This is a fundamentally different approach to computation than imperative and object-oriented programming which roots lie in the Turing Machine \citep{turing_computable_1937}. Rather than describing \textit{how} something is computed as in the more operational approach of the Turing Machine, due to the more declarative nature of the Lambda Calculus, code in functional programming describes \textit{what} is computed.

As a short example we give an implementation of the factorial function in Haskell:
factorial :: Integer -> Integer
factorial 0 = 1
factorial n = n * factorial (n-1)

When looking at this function we can already see the central concepts of functional programming: 
\begin{enumerate*}
	\item Declarative - we describe \textit{what} the factorial function is rather than how to compute it. This is supported by \textit{pattern matching} which allows to give multiple equations for the same function, matching on its input. 
	\item Immutable data - in functional programming we don't have mutable variables - after a variable is assigned, it cannot change its contents. This also means that there is no destructive assignment operator which can re-assign values to a variable. To change values, we employ recursion.
	\item Recursion - the function calls itself with a smaller argument and will eventually reach the case of 0. Recursion is the very meat of functional programming because they are the only way to implement loops in this paradigm due to immutable data.
	\item Static Types - the first line indicates the name and the types of the function. In this case the function takes one Integer as input and returns an Integer as output. Types are static in Haskell which means that there can be no type-errors at run-time e.g. when one tries to cast one type into another because this is not supported by this kind of type-system.
	\item Explicit input and output - all data which are required and produced by the function have to be explicitly passed in and out of it. There exists no global mutable data whatsoever and data-flow is always explicit.
	\item Referential transparency - calling this function with the same argument will \textit{always} lead to the same result, meaning one can replace this function by its value. This means that when implementing this function one can not read from a file or open a connection to a server. This is also known as \textit{purity} and is indicated in Haskell in the types which means that it is also guaranteed by the compiler.
\end{enumerate*}

It may seem that one runs into efficiency-problems in Haskell when using algorithms which are implemented in imperative languages through mutable data which allows in-place update of memory. The seminal work of \citep{okasaki_purely_1999} showed that when approaching this problem from a functional mind-set this does not necessarily be the case. The author presents functional data structures which are asymptotically as efficient as the best imperative implementations and discusses the estimation of the complexity of lazy programs.

We refer to the following works for a more in-depth introduction to functional programming: \citep{hughes_why_1989}.

\subsection{Side-Effects}
One of the fundamental strengths of functional programming and Haskell is their way of dealing with side-effects in functions. A function with side-effects has observable interactions with some state outside of its explicit scope. This means that the behaviour it depends on history and that it loses its referential transparency character, which makes understanding and debugging much harder. Examples for side-effects are (amongst others): modifying a global variable, await an input from the keyboard, read or write to a file, open a connection to a server, drawing random-numbers,...

Obviously, to write real-world programs which interact with the outside-world we need side-effects. Haskell allows to indicate in the \textit{type} of a function that it does or does \textit{not} have side-effects. Further there are a broad range of different effect types available, to restrict the possible effects a function can have to only the required type. This is then ensured by the compiler which means that a program in which one tries to e.g. read a file in a function which only allows drawing random-numbers will fail to compile. Haskell also provides mechanisms to combine multiple effects e.g. one can define a function which can draw random-numbers and modify some global data. The most common side-effect types are:
\begin{itemize*}
	\item IO - Allows all kind of I/O related side-effects: reading/writing a file, creating threads, write to the standard output, read from the keyboard, opening network-connections,... 
	\item Rand - Allows to draw random-numbers.
	\item Reader - Allows to read from an environment.
	\item Writer - Allows to write to an environment.
	\item State - Allows to read and write an environment.
\end{itemize*}

A function with side-effects has to indicate this in their type e.g. if we want to give our factorial function for debugging purposes the ability to write to the standard output, we add IO to its type: factorial :: Integer -> IO Integer. A function without any side-effect type is called \textit{pure}. A function with a given effect-type needs to be executed with a given effect-runner which takes all necessary parameters depending on the effect and runs a given effectful function returning its return value and depending on the effect also an effect-related result. For example when running a function with a State-effect one needs to specify the initial environment which can be read and written. After running such a function with a State-effect the effect-runner returns the changed environment in addition with the return value of the function itself. Note that we cannot call functions of different effect-types from a function with another effect-type, which would violate the guarantees. Calling a \textit{pure} function though is always allowed because it has by definition no side-effects. An effect-runner itself is a \textit{pure} function. The exception to this is the IO effect type which does not have a runner but originates from the \textit{main} function which is always of type IO.

Although it might seem very restrictive at first, we get a number of benefits from making the type of effects we can use explicit. First we can restrict the side-effects a function can have to a very specific type which is guaranteed at compile time. This means we can have much stronger guarantees about our program and the absence of potential errors already at compile-time which implies that we don't need test them with e.g. unit-tests. Second, because effect-runners are themselves \textit{pure}, we can execute effectful functions in a very controlled way by making the effect-context explicit in the parameters to the effect-runner. This allows a much easier approach to isolated testing because the history of the system is made explicit.

We refer to the following papers for a much more technical in-depth discussion of the concept of side-effects and how they are implemented in Haskell: \citep{moggi_computational_1989, wadler_essence_1992, wadler_monads_1995, wadler_how_1997, jones_tackling_2002}.

\subsection{Functional Reactive Programming}
Functional Reactive Programming (FRP) is a way to implement systems with continuous and discrete time-semantics in functional programming. The central concept in FRP is the Signal Function which can be understood as a process over time which maps an input- to an output-signal. A signal in turn, can be understood as a value
which varies over time. Thus, signal functions have an awareness of the passing of time by having access to a $\Delta t$ which are positive time-steps with which the system is sampled. In general, signal functions can be understood to be computations that represent processes, which have an input of a specific type, process it and output a new type. Note that this is an important building block to represent agents in functional programming: by implementing agents as signal functions allows us to implement them as processes which act continuously over time, which implies a time-driven approach to ABS. We have also applied the concept of FRP to event-driven ABS \citep{meyer_event-driven_2014}.

TODO: explain that there are events and dynamical switching of behaviour

We refer to the following papers for a technical, in-depth discussion on FRP in Haskell: \citep{wan_functional_2000, hughes_generalising_2000, hughes_programming_2005, hudak_arrows_2003, courtney_yampa_2003, nilsson_functional_2002, perez_functional_2016, perez_extensible_2017}

\subsection{Related Research}
TODO: paper by James Odell "Objects and Agents Compared"
