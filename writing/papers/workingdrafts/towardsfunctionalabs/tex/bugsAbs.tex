\section{Bugs and Errors in Agent-Based Simulation}
TODO: general introduction %https://en.wikipedia.org/wiki/Software_bug

The problem of correctness in agent-based simulations became more apparent in the work of Ionescu et al \cite{ionescu_dependently-typed_2012} which tried to replicate the work of Gintis \cite{gintis_emergence_2006}. In his work Gintis claimed to have found a mechanism in bilateral decentralized exchange which resulted in walrasian general equilibrium without the neo-classical approach of a tatonement process through a central auctioneer. This was a major break-through for economics as the theory of walrasian general equilibrium is non-constructive as it only postulates the properties of the equilibrium \cite{colell_microeconomic_1995} but does not explain the process and dynamics through which this equilibrium can be reached or constructed - Gintis seemed to have found just this process. Ionescu et al. \cite{ionescu_dependently-typed_2012} failed and were only able to solve the problem by directly contacting Gintis which provided the code - the definitive formal reference. It was found that there was a bug in the code which led to the "revolutionary" results which were seriously damaged through this error. They also reported ambiguity between the informal model description in Gintis paper and the actual implementation. TODO: it is still not clear what this bug was, find out! look at the master thesis 

This is supported by a talk \cite{sweeney_next_2006}, in which Tim Sweeney, CEO of Epic Games, discusses the use of main-stream imperative object-oriented programming languages (C++) in the context of Game Programming. Although the fields of games and ABS seem to be very different, in the end they have also very important similarities: both are simulations which perform numerical computations and update objects in a loop either concurrently or sequential \cite{gregory_game_2018}. Sweeney reports that reliability suffers from dynamic failure in such languages e.g. random memory overwrites, memory leaks, accessing arrays out-of-bounds, dereferencing null pointers, integer overflow, accessing uninitialized variables. He reports that 50\% of all bugs in the Game Engine Middleware Unreal can be traced back to such problems and presents dependent types as a potential rescue to those problems.

TODO: list common bugs in object-oriented / imperative programming
TODO: java solved many problems 
TODO: still object-oriented / imperative ultimately struggle when it comes to concurrency / parallelism due to their mutable nature.

TODO: \cite{vipindeep_list_2005}

TODO: software errors can be costly %https://raygun.com/blog/costly-software-errors-history/
TODO: bugs per loc %https://www.mayerdan.com/ruby/2012/11/11/bugs-per-line-of-code-ratio