%%%%%%%%%%%%%%%%%%%%%%%%%%%%%%%%%%%%%%%%%
% University/School Laboratory Report
% LaTeX Template
% Version 3.1 (25/3/14)
%
% This template has been downloaded from:
% http://www.LaTeXTemplates.com
%
% Original author:
% Linux and Unix Users Group at Virginia Tech Wiki 
% (https://vtluug.org/wiki/Example_LaTeX_chem_lab_report)
%
% License:
% CC BY-NC-SA 3.0 (http://creativecommons.org/licenses/by-nc-sa/3.0/)
%
%%%%%%%%%%%%%%%%%%%%%%%%%%%%%%%%%%%%%%%%%

%----------------------------------------------------------------------------------------
%	PACKAGES AND DOCUMENT CONFIGURATIONS
%----------------------------------------------------------------------------------------


\documentclass{article}

\usepackage[utf8]{inputenc}
\usepackage{graphicx} % Required for the inclusion of images
\usepackage{amsmath} % Required for some math elements 


\setlength\parindent{0pt} % Removes all indentation from paragraphs

%\usepackage{times} % Uncomment to use the Times New Roman font

%----------------------------------------------------------------------------------------
%	DOCUMENT INFORMATION
%----------------------------------------------------------------------------------------

\title{Meta ABS: \\ An approach to Free Will in Agent-Based Simulation} % Title

\author{Jonathan \textsc{Thaler}} % Author name

\date{\today} % Date for the report

\begin{document}

\maketitle % Insert the title, author and date

% If you wish to include an abstract, uncomment the lines below
\begin{abstract}
Meta Agent-Based Simulation (MetaABS) is an attempt of modelling free will on a computational basis. We focus on free will which we define as having the ability to anticipate results from one actions thus creating a feedback on the actions. Here we follow the interpretation that anticipation means 'to simulate' in the context of ABS. We follow the concept of irreducibility of computation which, due to undecidability - requires to run (=simulate) a program to actually know its output. The idea is in each step of the simulation let agents simulate the same simulation they are situated in from their point of view for a given number of steps before they take their next step. Thus the simulation is defined in terms of recursion, a novelty and something we will describe in depth in this paper. We claim that functional programming is especially well suited to implement MetaABS due to its lack of implicit side-effects, natural parallelism and declarative way of describing WHAT systems are instead of HOW they compute. Obviously this new approach has a problem: an agent would need to have complete information about the whole simulation, an assumption which is completely unrealistic. We introduce the term of 'reduced-recursive' which assumes that the next recursive level is a reduced version of the original, very local to the agent, thus solving the problem of complete information.
Our method may look like another optimization method but we will show that this is not the case. Also our motivation and intention is completely different and our intended area of application are the social sciences, artificial life, philosophy and maybe religious studies. Thus we must be very careful to not confuse it with e.g. game-theory where one makes assumptions about the other players - this may seem to be the same but we are not interested in this approach and we argue it is very different: we simulate instead of rational calculation. We test our method using the famous \textit{Sugarscape} model.
\end{abstract}

TODO:  need a very good reasoning /hypothesis why free will is dangerous outside of the simulation
TODO: free will in the bible: adam and eve

\subsection{Motivation}
We build on the discussion of the \textit{simulation argument} proposed by \cite{bostrom_are_2003} which states that we \textit{may} live in a simulation, created by transhumans (TODO: explain). Although we neither can proof or disproof the \textit{simulation argument}, we find the idea of existence being a simulation highly intriguing as it finally allows us to \textit{investigate how existence can be understood from the perspective of computer science in a scientific way}. This was so far not possible due to the lack of a scientific context, which is now given through the \textit{simulation argument}. The obvious question which raises, is then \textit{why are we simulated?}. Bostrom does not give any reason for it in his paper as this obviously touches on ideological and religious ground. We try to develop hypothesis for why we are being simulated and approach it scientifically by drawing parallels to agent-based social simulation.
\cite{steinhart_theological_2010} discusses theological implications of the simulation argument and looks into reasons why we are simulated. He mentions 'evolution of complexity' and gaining of knowledge for the transhumans as the reasons why theses simulations are enacted. We agree with Steinhart but our central point is that we hypothesize that these reasons are not first-order and that there is a first-order reason of this simulation. Our reasoning is as follows:

\begin{theorem}
Free Will leads to ideologies.
\end{theorem}

\begin{theorem}
Ideologies result in separation, inequality, suffering and ultimately in death.
\end{theorem}

\begin{theorem}
To prevent destruction of all life Free Will needs to be withdrawn from life-forms.
\end{theorem}

\begin{theorem}
Learning can only happen through Free Will.
\end{theorem}

\begin{theorem}
To enable life-forms to learn, they are granted Free Will but in a protected environment.
\end{theorem}

\begin{corollary}
This existence is a simulation which allows conciousness to express Free Will. The emergent property of this simulation are ideologies.
\end{corollary}

\cite{irtem_simulation_1978} defines artificial free will of a machine to be: TODO

TODO: 'the quantum system develops according to a wave-function with superposed states, and something, that isn't micro-physics, causes the wave to collapse into one of the superposed states. This something might very well be mental states'. Thus we draw the parallels to our ABS: the description, of the simulation is its unrealized state of superposition. when we run the simulation we realize superposed states. 

NOTE:  i dont believe we are in a simulation but that we are separated from something which is beyond our existence. the simulation argument only serves as a metaphor to be able to address metaphysical issues with scientific tools and language from computer science, physics and ultimately mathematics

\subsection{Research Questions}

\subsubsection{How can Free Will be defined in ABS?}
\paragraph{Hypothesis} It is the ability of an agent to simulate on a meta-level and adjust itself based on these insights.

\subsubsection{How can we model Free Will in ABS?}
\paragraph{Hypothesis} we need a meta-model: a recursive, declarative description of the simulation.

\subsubsection{How can we implement Free Will in ABS?}
\paragraph{Hypothesis} pure functional programming is suitable due to its inherent recursive, declarative nature, which should allow a direct mapping of features of this paradigm to the specification of the meta-model.

\subsubsection{What happens if this kind of Free Will is applied to existing models in social simulation?}
\paragraph{Hypothesis} Chaos, separation, destruction

\cite{epstein_growing_1996} introduced the famous \textit{Sugarscape} model which is widely researched in the social sciences, can it be applied?

\begin{itemize}
	\item does free will lead to ideologies (what are ideologies?)? if yes, how many agents are sufficient?
	\item are there differences between having free will and using it AND of having free and not using it?
	\item agents try to find out if they are in a simulation or not
	\item agents are ordered on a scale in their beliefs about argument 1-3 of the simulation hypothesis 
	\item agents exhibit a range of primal fear between 0 and 1 where 0 means they are enlightened and 1 means they are completely stuck in the simulation
	\item some agents have free will (can run a meta-simulation) and some don't but the free-will agents don't know which the others are. it is the goal of the free-will agents to find out the non-free will ones. BUT HOW? free-will agents can go on a meta-level and compare their result with the actual result, if there is a mismatch then they know the other agent has free will and otherwise not.
\end{itemize}

\bibliographystyle{acm}
\bibliography{../../references/phdReferences.bib}

\end{document}