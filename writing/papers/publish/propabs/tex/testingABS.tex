\section{Testing ABS implementations}
\label{sec:testingABS}

Generally we need to distinguish between two types of testing / verification in ABS.

\begin{enumerate}
	\item Testing / verification of models for which we have real-world data or an analytical solution which can act as a ground-truth - examples for such models are the SIR model, stock-market simulations, social simulations of all kind.
	\item Testing / verification of models which are of exploratory nature, inspired by real-world phenomena but for which no ground-truth per se exists - examples for such models is the Sugarscape \cite{epstein_growing_1996} or Agent\_Zero model \cite{epstein_agent_zero:_2014}.
\end{enumerate}

The baseline is that either one has an analytical model as the foundation of an agent-based model or one does not. In the former case, e.g. the SIR model, one can very easily validate the dynamics generated by the simulation to the one generated by the analytical solution through System Dynamics. In the latter case one has basically no idea or description of the emergent behaviour of the system prior to its execution e.g. SugarScape. In this case it is important to have some hypothesis about the emergent property / dynamics. The question is how verification / validation works in this setting as there is no formal description of the expected behaviour: we don't have a ground-truth against which we can compare our simulation dynamics.

One distinguishes between black-box and white-box verification where in white-box verification one looks directly at code and reasons about it whereas in black-box verification one generally feeds input to the software / functions / methods and compares it to expected output. Black-box verification is our primary concern in this paper as property-based testing is an instance of black-box verification. In the case of ABS we have the following levels of black-box tests:
\begin{enumerate}
	\item Isolated and interacting agent behaviour parts - test the individual parts which make up the agent behaviour under given inputs and test if interaction between agents are correct. For this we can use traditional unit-tests as shown by \cite{collier_test-driven_2013} and also property-based testing as we will show in the use-cases.
	\item Simulation dynamics - compare emergent dynamics of the ABS as a whole under given inputs to an analytical solution or real-world dynamics in case there exists some, using statistical tests. We see this type of tests conceptually as property-tests as well because we are testing properties of the model / simulation as we will see in the use-cases. Technically speaking we can both use traditional unit-tests and also property-based tests to implement them - conceptually they are property-tests.
	\item Hypotheses - test whether hypotheses about the model are valid or invalid. This is very similar to the previous point but without comparing it to analytical solutions or real-world dynamics but only to some hypothetical values.
\end{enumerate}