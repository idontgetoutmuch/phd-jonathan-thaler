\section{Introduction}
Agent-based simulation (ABS) is  a method for simulating the emergent behaviour of a system by modelling and simulating the interactions of its sub-parts, called agents. Examples for an ABS is simulating the spread of an epidemic throughout a population or simulating the dynamics of segregation within a city. Central to ABS is the concept of an agent who needs to be updated in regular intervals during the simulation so it can interact with other agents and its environment. In this paper we are looking at two different kind of simulations to show the differences update-strategies can make in ABS and support our main message that \textit{when developing a model for an ABS it is of most importance to select the right update-strategy which reflects and supports the corresponding semantics of the model}. As we will show due to conflicting ideas about update-strategies this awareness is yet still under-represented in the field of ABS and is lacking a systematic treatment. As a remedy we undertake such a systematic treatment in proposing a new terminology by identifying properties of ABS and deriving all possible update-strategies. The outcome is a terminology to communicate in a unified way about this very important matter, so researchers and implementers can talk about it in a common way, enabling better discussions, same understanding and better reproducibility and continuity in research.
The two simulations we use are discrete and continuous games where in the former one the agents act synchronized at discrete time-steps whereas in the later one they act continuously in continuous time. We show that in the case of simulating the discrete game the update-strategies have a huge impact on the final result whereas our continuous game seems to be stable under different update-strategies. The contribution of this paper is: Identifying general properties of ABS, deriving update-strategies from these properties and establishing a general terminology for talking about these update-strategies.