% FunctionalReactiveAgentBa-JF.tex
\begin{hcarentry}[new]{Functional Reactive Agent-Based Simulation}
\report{Jonathan Thaler}%05/17
\status{Experimental, active development}
\makeheader

Implementations of Agent-Based Simulation (ABS) have so far been reduced to
the context of Object-Orientation (OO). We investigate how ABS can be
implemented in a pure functional language like Haskell. The fundamental
problem is that unlike in OO there are no objects and no implicit aliases
through which to access and change data: method calls are not available in FP.
We solve the problem of how to represent an agent and how agents can interact
with each other. We build on the concept of Functional Reactive Programming
for which we use the library Yampa. This allows us to represent agents as
signal-functions with different types as input and output. In each time-step
an agent gets fed in an input and creates an output which is the input for the
next time-step, creating a feedback. The input- and output-types contain
incoming and outgoing messages and various events like start, termination,
kill. Also we build on the facilities Yampa provides for time-flow of a system
be it continuous or discrete. For interactions between Agents we implemented
messages, which are interactions over simulated time and a novel concept which
we termed 'conversations', which are restricted interactions during which the
simulation-time is halted. It is of most importance to us to keep our code
pure - except from the reactive Yampa-loop all our code is pure and does not
make use of the IO-Monad. Currently we are implementing the full
\textit{SugarScape} model of J.M.Epstein and R.Axtell as described in their
book, which serves as the initial use-case to drive the implementation of our
library. In the next step we also want to implement \textit{Agent\_Zero}, the
most recent Agent-Model developed by J.M.Epstein. Further research will go
into validation \& verification of ABS in our library using QuickCheck and
algebraic reasoning. The code is freely available but not stable as it
currently serves for prototyping for gaining insights into the problems faced
when implementing ABS in Haskell. The plan is to release the final
implementation at the end of the PhD as a stable and full-featured library on
Hackage. If you are interested in the on-going research please contact
Jonathan Thaler (jonathan.thaler@@nottingham.ac.uk).

\FurtherReading
  Repository: \url{https://github.com/thalerjonathan/phd/tree/master/coding/libraries/frABS}
\end{hcarentry}
