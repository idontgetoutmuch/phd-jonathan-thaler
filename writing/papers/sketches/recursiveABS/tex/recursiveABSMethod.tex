\section{Meta ABS}
Informally, Meta-ABS can be understood as giving the agents the ability to project the outcome of their actions into the future. They are able to halt time and 'play through' an arbitrary number of actions, compare their outcome and then to resume time and continue with a specifically chosen action e.g. the best performing or the one in which they haven't died. 
More precisely, what we want is to give an agent the ability to run the simulation recursively a number of times where the this number is not determined initially but can depend on the outcome of the recursive simulation. 

\subsection{Functional description}
In this section we will give a formal description of how Meta-ABS works. Although we look at it from a programming-language agnostic way, we follow a functional description (rather than object-oriented) both in our pseudo-code and description because we think it allows for a much clearer formalization.

We assume an Agent-Behaviour Function with an argument of type AgentIn and a return value of type AgentOut.

TODO: for haskell-code use \url{https://www.andres-loeh.de/lhs2tex/}

\begin{lstlisting}[frame=single]
agentBeh :: AgentIn -> AgentOut
\end{lstlisting}

We will keep the AgentIn and AgentOut types opaque and provide a few functions to work on them. We assume some domain-specific function which can create a new domain-specific solution by taking an AgentIn and returning an AgentOut. This can be used to iteratively update the agent

\begin{lstlisting}[frame=single]
createNewSolution :: AgentIn -> AgentOut

agentBeh :: AgentIn -> AgentOut
agentBeh ain = createNewSolution ain
\end{lstlisting}

If an agent wants to initiate recursive simulations it calls the function recursive 

\begin{lstlisting}[frame=single]
recursive :: AgentOut -> AgentOut

agentBeh :: AgentIn -> AgentOut
agentBeh ain = recursive createNewSolution ain
\end{lstlisting}

This will tell the simulation system to initiate a recursive simulation. The system will then call the agent-behaviour function again but provides information that the agent is running recursively through the AgentIn type. To check this it is possible to query it by using the isRecursive function:

\begin{lstlisting}[frame=single]
isRecursive :: AgentIn -> Bool

agentBeh :: AgentIn -> AgentOut
agentBeh ain 
	| isRecursive ain = handleRecursion ain
	| otherwise = recursive createNewSolution ain
	
handleRecursion :: AgentIn -> AgentOut
\end{lstlisting}

The AgentIn type provides an additional function to get a list of AgentOut of all the past recursions.  invariant: when in recursion and returning an aout and staying in recursion the aout returned will show up in the next recursive calls list at the top

\begin{lstlisting}[frame=single]
recursiveAgentOuts :: AgentIn -> [AgentOut]
\end{lstlisting}

Further we need a function to stop the recursion if we are happy with one of the AgentOut:

\begin{lstlisting}[frame=single]
unrecursive :: AgentOut -> AgentOut
\end{lstlisting}

Putting all together we can continue the calculation of recursions until the agent is satisfied with one AgentOut. 

\begin{lstlisting}[frame=single]
handleRecursion :: AgentIn -> AgentOut
handleRecursion ain 
	| satisfies recursiveAgentOuts ain = unrecursive $ pickBest recursiveAgentOuts ain
	| otherwise = recursive createNewSolution ain

satisfies :: [AgentOut] -> Bool
pickBest :: [AgentOut] -> AgentOut
\end{lstlisting}

We need to establish a very important invariant of time. We can query the simulation-time using the function
\begin{lstlisting}[frame=single]
simTime :: AgentIn -> Real
\end{lstlisting}

The following invariant must hold: when calling simTime during recursion it must return constantly the same time as when starting the recursion. TODO: more formal

Now when following this approach and running more than one agent we will end up in an infinite regress as every agent will run the other agents which will lead them to initiate a recursion on their own. We need a mechanism to prevent agents which are recursively simulated by the initiating agent to run recursive simulations by their own. TODO: we need to define the terms precisely (initiating agent, recursively simulated agent)

TODO:  how do we restrict recursion to only the recurring agent? either all or none or a list of agentids? the problem is that this is model dependent

So far we assumed (TODO: make that clear in the up description) that each recursive simulation is executed for 1 step and thus returning only one AgentOut. What if we want to run a recursive simulation for an arbitrary number of time-steps? We need again the mechanism to prevent 'other' agents from initiating recursive simulations but that is not enough, we need some mechanism of preventing the initiating agent of recursively initiating simulations in every time-step of the recursive simulation - ideally we should leave the choice to the initiating agent but it needs to know that it is in an evolving recursive simulation. within it it can initiate an additional recursive simulation or not - thus an agent can spawn an arbitrary number of recursive simulations within recursive simulations within recursive simulations...

First we need to establish a little bit of terminology to be able to unambiguously discuss the formal approach of Meta-ABS
TODO: recursion-depth:
TODO: recursion-replications
TODO: recursion-length

WARNING WARNING WARNING WARNING WARNING 
TODO: it is yet unclear how the environment is being handled between recursive calls
WARNING WARNING WARNING WARNING WARNING 

\subsection{Deterministic vs. Non-Deterministic future}
The model as described in Background section is completely deterministic once it is running because it makes no use of a random-number generator and there are no other sources of non-determinism - the next move of an agent is always completely predictable. If we introduce randomness through a random-number generator into our model then the future becomes non-deterministic \textit{if the state of random-number generator when running recursive simulations is different from when the simulation is run non-recursively.}

TODO: What if the agents are shuffled every time before being traversed sequentially? The deterministic iteration is of importance here!

\subsection{Computational complexity}
the computation power grows exponentially with the number of recursion: give a formula depending on number of agents, recursion depth, independent moves of an agent and number of time-steps 


\subsection{Philosophical implications}

\subsubsection{Omega Point: the limit case}
tiplers omega point and paper about god and the simulation argument
accelerating turing machine: finishes after 1 time steps
what would be the outcome in a zeno machine/accelerated turing machine if we don't restrict the recursions of 'others' and self?

\url{https://en.m.wikipedia.org/wiki/Omega_Point}
\url{https://en.m.wikipedia.org/wiki/Zeno_machine}
\url{https://en.m.wikipedia.org/wiki/Hypercomputation}

\subsubsection{Multidimensional Computation}
we are spanning up 3 dimensions: recursion-depth, replications, and time-steps

\subsubsection{Emergent Non-Determinism}
the prediction may work for a single agent but what if more and more agents predict their future? within the prediction no recursion is run so no 2nd level anticipation. 
hypothesis: increasing the ratio of predicting agents will decrease the effectiveness of the predictions because the future becomes then in effect non-deterministic => non-determinism as emerging property? is there a limit e.g. up until which ratio does the average utility of the predicting agents increase?

the agent who is initiating the recursion can be seen as 'knowing' that it is running inside a simulation, but the other agents are not able to distinguish between them running on the base level of the simulation or on a recursive level

\subsubsection{Perfect Information}
The main problem of our approach is that, depending on ones view-point, it is violating the principles of locality of information and limit of computing power. To recursively run the simulation the agent which initiates the recursion is feeding in all the states of the other agents and calculates the outcome of potentially multiple of its own steps, each potentially multiple recursion-layers deep and each recursion-layer multiple time-steps long. Both requires that each agent has perfect information about the complete simulation \textit{and} can compute these 3-dimensional recursions, which scale exponentially.
In the social sciences where agents are often designed to have only very local information and perform low-cost computations it is very difficult or impossible to motivate the usage of recursive simulations - it simply does not match the assumptions of the real world, the social sciences want to model.
In general simulations, with no direct link to the real world, where it is much more commonly accepted to assume perfect information and potentially infinite amount of computing power this approach is easily motivated by a constructive argument: it is possible to build, thus we build it.
What we are ultimately interested in is the influence on the dynamics.
Note that we identified the future-optimization technique as being locally. This is still the case despite of using global information for recurring the simulation - the reason for this is that we are talking about two different contexts here.

