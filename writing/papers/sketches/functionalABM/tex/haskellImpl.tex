\section{Implementation in Haskell}
It is important to prevent oneself to implement an object-oriented approach in Haskell because then we would loose the power of pure functional programming. Thus the main design-considerations were:

=> it seems that par-version is slower than sequential ??? maybe because of evaluating all data?

\begin{enumerate}
\item Implement all 4 approaches mentioned in General Implementation for maximum flexibility for users of the library so they can choose which semantics of the simulation they want.
\item Pure simulation to keep up the ability to reason about it. This means to avoid running in IO Monad at all costs (except for the main-loop).
\item Utilizing STM which allows very flexible handling of state and allows running.
\item Run inside Yampa/Dunai to leverage the EDSL and continuous/discrete time-implementation.
\end{enumerate}

take haskell, add yampa and dunai and implement ActorModel on top using STM => have an ABM library in Haskell, put on hackage. NO IO!

\subsection{Utilizing STM}
with stm one can determine when everything is visible: at the end of the step or after each agent update. but this applies only to parallelism! if running agents in parallel they must execute atomically, otherwise the probability for concurrent read/writes goes to 100\% also as the number of agents goes up. if no parallelism then one final atomically at the end of each step enough. but in any case: using STM leads to the effect that agents see later updates. so really 2 cases where STM is of use in haskell: global, sequential and global concurrent

how can we implement true parallelism? can we use STM somehow or do we need local mailboxes?

=> STM solution is much slower than the par-version without yet using parallel-execution!

\subsection{Structuring the Program}
Of course the basic pure functional primitives alone do not make a well structured functional program by themselves as the usage of classes, interfaces, objects and inheritance alone does not make a well structured object-oriented program. What is needed are \textit{patterns} how to use the primitives available in pure functional programs to arrive at well structure programs. In object-orientation much work has been done in the 90s by the highly influential book \cite{gamma_design_1994} whereas in functional programming the major inventions were also done in the 90s by the invention of Monads through \cite{Moggi1989}, \cite{Wadler1990} and \cite{Wadler1995} and beginning of the 2000s by the invention of Arrows through \cite{Hughes2000}.

\subsubsection{Higher Order Functions \& Monads}
map \& fmap, foldl, applicatives
\cite{hutton_programming_2007} gives a great overview and motivation for using fmap, applicatives and Monads. TODO: explain Monads

\subsubsection{Arrows}
\cite{Hughes2004} is a great tutorial about \textit{Arrows} which are very well suited for structuring functional programs with effects.

\begin{quote}
Just like monads, arrow types are useful for the additional operations they support, over and above those that every arrow provides.
\end{quote}

The main difference between Monads and Arrows are that where monadic computations are parameterized only over their output-type, Arrows computations are parameterised both over their input- and output-type thus making Arrows more general.

\begin{quote}
In real applications an arrow often represents some kind of a process, with an input channel of type a, and an output channel of type b.
\end{quote}

In the work \cite{Hughes2004} an example for the usage for Arrows is given in the field of circuit simulation. They use previously introduced streams to advance the simulation in discrete steps to calculate values of circuits thus the implementation is a form of \textit{discrete event simulation} - which is in the direction we are heading already with ABM/S. Also the paper mentions Yampa which is introduced in the section (TODO: reference) on functional reactive programming.

\subsubsection{Functional reactive programming (FRP)}
FRP is a paradigm for programming hybrid systems which combine continuous and discrete components. Time is explicitly modelled: there is a continuous and synchronous time flow.  \\

there have been many attempts to implement FRP in frameworks which each has its own pro and contra. all started with fran, a domain specific language for graphics and animation and at yale FAL, Frob, Fvision and Fruit were developed. The ideas of them all have then culminated in Yampa which is the reason why it was chosen as the FRP framework. Also, compared to other frameworks it does not distinguish between discrete and synchronous time but leaves that to the user of the framework how the time flow should be sampled (e.g. if the sampling is discrete or continuous - of course sampling always happens at discrete times but when we speak about discrete sampling we mean that time advances in natural numbers: 1,2,3,4,... and when speaking of continuous sampling then time advances in fractions of the natural numbers where the difference between each step is a real number in the range of [0..1]) \\

time- and space-leak: when a time-dependent computation falls behind the current time. TODO: give reason why and how this is solved through Yampa. \\
Yampa solves this by not allowing signals as first-class values but only allowing signal functions which are signal transformers which can be viewed as a function that maps signals to signals. A signal function is of type SF which is abstract, thus it is not possible to build arbitrary signal functions. Yampa provides primitive signal functions to define more complex ones and utilizes arrows \cite{Hughes2004} to structure them where Yampa itself is built upon the arrows: SF is an instance of the Arrow class. \\

Fran, Frob and FAL made a significant distinction between continuous values and discrete signals. Yampas distinction between them is not as great. Yampas signal-functions can return an Event which makes them then to a signal-stream - the event is then similar to the Maybe type of Haskell: if the event does not signal then it is NoEvent but if it Signals it is Event with the given data. Thus the signal function always outputs something and thus care must be taken that the frequency of events should not exceed the sampling rate of the system (sampling the continuous time-flow). TODO: why? what happens if events occur more often than the sampling interval? will they disappear or will the show up every time? \\

switches allow to change behaviour of signal functions when an event occurs. there are multiple types of switches: immediate or delayed, once-only and recurring - all of them can be combined thus making 4 types. It is important to note that time starts with 0 and does not continue the global time when a switch occurs. TODO: why was this decided? \\

\cite{Nilsson2002} give a good overview of Yampa and FRP. Quote: "The essential abstraction that our system captures is time flow". Two \textit{semantic} domains for progress of time: continuous and discrete. \\

The first implementations of FRP (Fran) implemented FRP with synchronized stream processors which was also followed by \cite{Wan2000}. Yampa is but using continuations inspired by Fudgets. In the stream processors approach "signals are represented as time-stamped streams, and signal functions are just functions from streams to streams", where "the Stream type can be implemented directly as (lazy) list in Haskell...":
\begin{lstlisting}[frame=single]
type Time = Double
type SP a b = Stream a -> Stream b
newtype SF a b = SF (SP (Time, a) b)
\end{lstlisting}
Continuations on the other hand allow to freeze program-state e.g. through closures and partial applications in functions which can be continued later. This requires an indirection in the Signal-Functions which is introduced in Yampa in the following manner. 
\begin{lstlisting}[frame=single]
type DTime = Double

data SF a b = 
	SF { sfTF :: DTime -> a -> (SF a b, b)
\end{lstlisting}
The implementer of Yampa call a signal function in this implementation a \textit{transition function}. It takes the amount of time which has passed since the previous time step and the durrent input signal (a). It returns a \textit{continuation} of type SF a b determining the behaviour of the signal function on the next step (note that exactly this is the place where how one can introduce stateful functions like integral: one just returns a new function which encloses inputs from the previous time-step) and an \textit{output sample} of the current time-step. \\

When visualizing a simulation one has in fact two flows of time: the one of the user-interface which always follows real-time flow, and the one of the simulation which could be sped up or slowed down. Thus it is important to note that if I/O of the user-interface (rendering, user-input) occurs within the simulations time-frame then the user-interfaces real-time flow becomes the limiting factor. Yampa provides the function embedSync which allows to embed a signal function within another one which is then run at a given ratio of the outer SF. This allows to give the simulation its own time-flow which is independent of the user-interface. \\

One may be initially want to reject Yampa as being suitable for ABM/S because one is tempted to believe that due to its focus on continuous, time-changing signals, Yampa is only suitable for physical simulations modelled explicitly using mathematical formulas (integrals, differential equations,...) but that is not the case. Yampa has been used in multiple agent-based applications: \cite{Hudak2003} uses Yampa for implementing a robot-simulation, \cite{Courtney2003} implement the classical Space Invaders game using Yampa, the thesis of \cite{Meisinger2010} shows how Yampa can be used for implementing a Game-Engine, \cite{Frag2005} implemented a 3D first-person shooter game with the style of Quake 3 in Yampa. Note that although all these applications don't focus explicitly on agents and agent-based modelling / simulation all of them inherently deal with kinds of agents which share properties of classical agents: game-entities, robots,... Other fields in which Yampa was successfully used were programming of synthesizers (TODO: cite), Network Routers, Computer Music Development and various other computer-games. This leads to the conclusion that Yampa is mature, stable and suitable to be used in functional ABM/S. \\
Jason Gregory (Game Engine Architecture) defines Computer-Games as "soft real-time interactive agent-based computer simulations".

To conclude: when programming systems in Haskell and Yampa one describes the system in terms of signal functions in a declarative manner (functional programming) using the EDSL of Yampa. During execution the top level signal functions will then be evaluated and return new signal functions (transition functions) which act as continuations: "every signal function in the dataflow graph returns a new continuation at every time step".

"A major design goal for FRP is to free the programmer from 'presentation' details by providing the ability to think in terms of 'modeling'. It is common that an FRP program is concise enough to also serve as a specification for the problem it solves" \cite{Wan2000}. This quotation describes exactly one of the strengths using FRP in ACE \\

\subsubsection{Dunai}
TODO: \cite{perez_functional_2016}

\subsection{Putting it all together}