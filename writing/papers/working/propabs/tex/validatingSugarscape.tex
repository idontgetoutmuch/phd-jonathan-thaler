\section{Validating Sugarscape in Haskell}
\label{app:validation}
Obviously we wanted our implementation to be correct, which means we validated it against the informal reports in the book. Also we use the work of \cite{weaver_replicating_2009} \footnote{\url{https://www2.le.ac.uk/departments/interdisciplinary-science/research/replicating-sugarscape}} which replicated Sugarscape in NetLogo and reported on it \footnote{Note that lending didn't properly work in their NetLogo code and that they didn't implement Combat}.

In addition to the informal descriptions of the dynamics, we implemented tests which conceptually check the model for emergent properties with hypotheses shown and expressed in the book. Technically speaking we have implemented that with unit-tests where in general we run the whole simulation with a fixed scenario and test the output for statistical properties which, in some cases is straight forward e.g. in case of Trading the authors of the Sugarscape model explicitly state that the standard deviation is below 0.05 after 1000 ticks.
Obviously one needs to run multiple replications of the same simulation, each with a different random-number generator and perform a statistical test depending on what one is checking: in case of an expected mean one utilises a t-test and in case of standard-deviations a chi-squared test. 

\subsection{Terracing}
Our implementation reproduces the terracing phenomenon as described on page TODO in Animation and as can be seen in the NetLogo implementation as well. We implemented a property-test in which we measure the closeness of agents to the ridge: counting the number of same-level sugars cells around them and if there is at least one lower then they are at the edge. If a certain percentage is at the edge then we accept terracing. The question is just how much, which we estimated from tests and resulted in 45\%. Also, in the terracing animation the agents actually never move which is because sugar immediately grows back thus there is no incentive for an agent to actually move after it has moved to the nearest largest cite in can see. Therefore we test that the coordinates of the agents after 50 steps are the same for the remaining steps.

\subsection{Carrying Capacity}
Our simulation reached a steady state (variance < 4 after 100 steps) with a mean around ~182. Epstein reported a carrying capacity of 224 (page 30) and the NetLogo implementations' \cite{weaver_replicating_2009} carrying capacity fluctuates around 205 which both are significantly higher than ours. Something was definitely wrong - the carrying capacity has to be around 200 (we trust in this case the NetLogo implementation and deem 224 an outlier).

After inspection of the NetLogo model we realised that we implicitly assumed that the metabolism range is \textit{continuously} uniformly randomized between 1 and 4 but this seemed not what the original authors intended: in the NetLogo model there were a few agents surviving on sugarlevel 1 which was never the case in ours as the probability of drawing a metabolism of exactly 1 is practically zero when drawing from a continuous range. We thus changed our implementation to draw a discrete value as the metabolism. %Note that this actually makes sense as massive floating-point number calculations were quite expensive in the mid 90s (e.g. computer games ran still on CPU only and exploited various  clever tricks to avoid the need of floating point calculations whenever possible) when SugarScape was implemented which might have been a reason for the authors to assume it implicitly.

This partly solved the problem, the carrying capacity was now around 204 which is much better than 182 but still a far cry from 210 or even 224. After adjusting the order in which agents apply the Sugarscape rules, by looking at the code of the NetLogo implementation, we arrived at a comparable carrying capacity of the NetLogo implementation: agents first make their move and harvest sugar and only after this the agents metabolism is applied (and ageing in subsequent experiments).

For regression-tests we implemented a property-test which tests that the carrying capacity of 100 simulation runs lies within a 95\% confidence interval of a 210 mean. These values are quite reasonable to assume, when looking at the NetLogo implementation - again we deem the reported Carrying Capacity of 224 in the Book to be an outlier / part of other details we don't know.

One lesson learned is that even such seemingly minor things like continuous vs. discrete or order of actions an agent makes, can have substantial impact on the dynamics of a simulation.

\subsection{Wealth Distribution}
By visual comparison we validated that the wealth distribution (page 32-37) becomes strongly skewed with a histogram showing a fat tail, power-law distribution where very few agents are very rich and most of the agents are quite poor. We compute the skewness and kurtosis of the distribution which is around a skewness of 1.5, clearly indicating a right skewed distribution and a kurtosis which is around 2.0 which clearly indicates the 1st histogram of Animation II-3 on page 34. Also we compute the Gini coefficient and it varies between 0.47 and 0.5 - this is accordance with Animation II-4 on page 38 which shows a gini-coefficient which stabilises around 0.5 after. 
We implemented a regression-test testing skewness, kurtosis and gini-coefficients of 100 runs to be within a 95\% confidence interval of a two-sided t-test using an expected skewness of 1.5, kurtosis of 2.0 and gini-coefficient of 0.48.

\subsection{Migration}
With the information provided by \cite{weaver_replicating_2009} we could replicate the waves as visible in the NetLogo implementation as well. Also we propose that a vision of 10 is not enough yet and shall be increased to 15 which makes the waves very prominent and keeps them up for much longer - agent waves are travelling back and forth between both Sugarscape peaks. We haven't implemented a regression-test for this property as we couldn't come up with a reasonable straight forward approach to implement it.

\subsection{Pollution and Diffusion}
With the information provided by \cite{weaver_replicating_2009} we could replicate the pollution behaviour as visible in the NetLogo implementation as well. We haven't implemented a regression-test for this property as we couldn't come up with a reasonable straight forward approach to implement it.

%Note that we spent quite a lot of time of getting this and the terracing properties right because they form the very basics of the other ones which follow so we had to be sure that those were correct otherwise validating would have been much more difficult.

%\subsection{Order of Rules}
%order in which rules are applied is not specified and might have an impact on dynamics e.g. when does the agent mate with others: is it after it has harvested but before metabolism kicks in?

\subsection{Mating}
We could not replicate Figure III-1 (TODO: page) - our dynamics first raised and then plunged to about 100 agents and go then on to recover and fluctuate around 300. This findings are in accordance with \cite{weaver_replicating_2009}, where they report similar findings - also when running their NetLogo code we find the dynamics to be qualitatively the same.

Also at first we weren't able to reproduce the cycles of population sizes. Then we realised that our agent-behaviour was not correct: agents which died from age or metabolism could still engage in mating before actually dying - fixing this to the behaviour, that agents which died from age or metabolism won't engage in mating solved that and produces the same swings as in \cite{weaver_replicating_2009}. Although our bug might be obvious, the lack of specification of the order of the application of the rules is an issue in the SugarScape book.

\subsection{Inheritance}
We couldn't replicate the findings of the Sugarscape book regarding the Gini coefficient with inheritance. The authors report that they reach a gini coefficient of 0.7 and above in Animation III-4. Our Gini coefficient fluctuated around 0.35. Compared to the same configuration but without inheritance (Animation III-1) which reached a Gini coefficient of about 0.21, this is indeed a substantial increase - also with inheritance we reach a larger number of agents of around 1,000 as compared to around 300 without inheritance.
The Sugarscape book compares this to chapter II, Animation II-4 for which they report a Gini coefficient of around 0.5 which we could reproduce as well. The question remains, why it is lower (lower inequality) with inheritance?

The baseline is that this shows that inheritance indeed has an influence on the inequality in a population. Thus we deemed that our results are qualitatively the same as the make the same point. Still there must be some mechanisms going on behind the scenes which are unspecified in the original Sugarscape.

\subsection{Cultural Dynamics}
We could replicate the cultural dynamics of AnimationIII-6 / Figure III-8: after 2700 steps either one culture (red / blue) dominates both hills or each hill is dominated by a different ulture. We wrote a test for it in which we run the simulation for 2.700 steps and then check if either culture dominates with a ratio of 95\% or if they are equal dominant with 45\%. Because always a few agents stay stationary on sugarlevel 1 (they have a metabolism of 1 and cant see far enough to move towards the hills, thus stay always on same spot because no improvement and grow back to 1 after 1 step), there are a few agents which never participate in the cultural process and thus no complete convergence can happen. This is accordance with \cite{weaver_replicating_2009}.

\subsection{Combat}
Unfortunately \cite{weaver_replicating_2009} didn't implement combat, so we couldn't compare it to their dynamics. Also, we weren't able to replicate the dynamics found in the Sugarscape book: the two tribes always formed a clear battlefront where some agents engage in combat e.g. when one single agent strays too far from its tribe and comes into vision of the other tribe it will be killed almost always immediately. This is because crossing the sugar valley is costly: this agent wont harvest as much as the agents staying on their hill thus will be less wealthy and thus easier killed off. Also retaliation is not possible without any of its own tribe anywhere near.

We didn't see a single run where an agent of an opposite tribe "invaded" the other tribes hill and ran havoc killing off the entire tribe. We don't see how this can happen: the two tribes start in opposite corners and quickly occupy the respective sugar hills. So both tribes are acting on average the same and also because of the number of agents no single agent can gather extreme amounts of wealth - the wealth should rise in both tribes equally on average. Thus it is very unlikely that a super-wealthy agent emerges, which makes the transition to the other side and starts killing off agents at large. First: a super-wealthy agent is unlikely to emerge, second making the transition to the other side is costly and also low probability, third the other tribe is quite wealthy as well having harvested for the same time the sugar hill, thus it might be that the agent might kill a few but the closer it gets to the center of the tribe the less like is a kill due to retaliation avoidance - the agent will simply get killed by others.

Also it is unclear in case of AnimationIII-11 if the R rule also applies to agents which get killed in combat. Nothing in the book makes this clear and we left it untouched so that agents who only die from age (original R rule) are replaced. This will lead to a near-extinction of the whole population quite quickly as agents kill each other off until 1 single agent is left which will never get killed in combat because there are no other agents who could kill it - instead it will enter an infinite die and  reborn cycle thanks to the R rule.

\subsection{\, Spice}
The book specifies for AnimationIV-1 a vision between 1-10 and a metabolism between 1-5. The last one seems to be quite strange because the maximum sugar / spice an agent can find is 4 which means that agents with metabolism of either 5 will die no matter what they do because the can never harvest enough to satisfy their metabolism. When running our implementation with this configuration the number of agents quickly drops from 400 to 105 and continues to slowly degrade below 90 after around 1000 steps.
The implementation of \cite{weaver_replicating_2009} used a slightly different configuration for AnimationIV-1, where they set vision to 1-6 and metabolism to 1-4. Their dynamics stabilise to 97 agents after around 500+ steps. When we use the same configuration as theirs, we produce the same dynamics.
Also it is worth nothing that our visual output is strikingly similar to both the book AnimationIV-1 and \cite{weaver_replicating_2009}.

\subsection{\, Trading}
For trading we had a look at the NetLogo implementation of \cite{weaver_replicating_2009}: there an agent engages in trading with its neighbours \textit{over multiple rounds} until either MRSs cross over or no trade has happened anymore. Because \cite{weaver_replicating_2009} were able to exactly replicate the dynamics of the trading time-series we assume that their implementation is correct. We think that the fact that an agent interact with its neighbours over multiple rounds is made not very clear in the book. The only hint is found on page 102: \textit{"This process is repeated until no further gains from trades are possible."} which is not very clear and does not specify exactly what is going on: does the agent engage with all neighbours again? is the ordering random? Another hint is found on page 105 where trading is to be stopped after MRS cross-over to prevent an infinite loop. Unfortunately this is missing in the Agent trade rule T on page 105. Additional information on this is found in footnote 23 on page 107. Further on page 107: \textit{"If exchange of the commodities will not cause the agents' MRSs to cross over then the transaction occurs, the agents recompute their MRSs, and bargaining begins anew."}. This is probably the clearest hint that trading could occur over multiple rounds.

We still managed to exactly replicate the trading-dynamics as shown in the book in Figure IV-3, Figure IV-4 and Figure IV-5. The book is also pretty specific on the dynamics of the trading-prices standard-deviation: on page 109 the authors specify that at t=1000 the standard deviation will have always fallen below 0.05 (Figure IV-5), thus we implemented a property-test which tests for exactly that property. Unfortunately we didn't reach the same magnitude of the trading volume where ours is much lower around 50 but it is equally erratic, so we attribute these differences to other missing specifications or different measurements because the price-dynamics match that well already so we can safely assume that our trading implementation is correct.

According to the book, Carrying Capacity (Animation II-2) is increased by Trade (page 111/112). To check this it is important to compare it not against AnimationII-2 but a variation of the configuration for it where spice is enabled, otherwise the results are not comparable because carrying capacity changes substantially when spice is on the environment and trade turned off. We could replicate the findings of the book: the carrying capacity increases slightly when trading is turned on. Also does the average vision decrease and the average metabolism increase. This makes perfect sense: trading allows genetically weaker agents to survive which results in a slightly higher carrying capacity but shows a weaker genetic performance of the population.

According to the book, increasing the agent vision leads to a faster convergence towards the (near) equilibrium price (page 117/118/119, Figure IV-8 and Figure IV-9). We could replicate this behaviour as well.

According to the book, when enabling R rule and giving agents a finite life span between 60 and 100 this will lead to price dispersion: the trading prices won't converge around the equilibrium and the standard deviation will fluctuate wildly (page 120, Figure IV-10 and Figure IV-11). We could replicate this behaviour as well.

The Gini coefficient should be higher when trading is enabled (page 122, Figure IV-13) - We could replicate this behaviour.

Finite lives with sexual reproduction lead to prices which don't converge (page 123, Figure IV-14). We could reproduce this as well but it was important to re-set the parameters to reasonable values: increasing number of agents from 200 to 400, metabolism to 1-4 and vision to 1-6, most important the initial endowments back to 5-25 (both sugar and spice) otherwise hardly any mating would happen because the agents need too much wealth to engage (only fertile when have gathered more than initial endowment). What was kind of interesting is that in this scenario the trading volume of sugar is substantially higher than the spice volume - about 3 times as high. 

From this part, we didn't implement: Effect of Culturally Varying Preferences, page 124 - 126, Externalities and Price Disequilibrium: The effect of Pollution, page 126 - 118, On The Evolution of Foresight page 129 / 130. 

%\subsection{Lending (Credit)}
%Not really much information to validate was available and the \cite{weaver_replicating_2009} implementation ran into an exception so there was not much to validate against. What was unexpected was that this was the most complex behaviour to implement, with lots of subtle details to take care of (spice on/off, inheritance,...).
%Note that we implemented lending of sugar and spice, although it looks from the book (Animation IV-5) that they only implemented it for sugar.

\subsection{\, Diseases}
We were able to exactly replicate the behaviour of Animation V-1 and Animation V-2: in the first case the population rids itself of all diseases (maximum 10) which happens pretty quickly, in less than 100 ticks. In the second case the population fails to do so because of the much larger number of diseases (25) in circulation. We used the same parameters as in the book. 
The authors of \cite{weaver_replicating_2009} could only replicate the first animation exactly and the second was only deemed "good". Their implementation differs slightly from ours: In their case a disease can be passed to an agent who is immune to it - this is not possible in ours. In their case if an agent has already the disease, the transmitting agent selects a new disease, the other agent has not yet - this is not the case in our implementation and we think this is unreasonable to follow: it would require too much information and is also unrealistic.
We wrote regression tests which check for animation V-1 that after 100 ticks there are no more infected agents and for animation V-2 that after 1000 ticks there are still infected agents left and they dominate: there are more infected than recovered agents.