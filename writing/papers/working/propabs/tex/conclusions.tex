\section{Conclusions}
\label{sec:conclusions}

We found property-based testing particularly well suited for ABS firstly due to ABS stochastic nature and second because we can formulate specifications, meaning we describe \textit{what} to test instead of \textit{how} to test. Also the deductive nature of falsification in property-based testing suits very well the constructive and often exploratory nature of ABS. 

Although it is now available in a wide range of programming languages and paradigms, propert-based testing has its origins in Haskell \cite{claessen_quickcheck_2000,claessen_testing_2002} and we argue that for that reason it really shines in pure functional programming. Property-based testing libraries exist now in other languages as well e.g. Java, Pyhton, C++ and we hope that our research sparked an interest in applying property-based testing to the established object-oriented languages in ABS as well.

%
%\subsection{Further Research}
%\label{sec:further}
%We didn't look into testing full agent and interacting agent behaviour using property-tests.