\section{Verifying ABS implementations}
\label{sec:verifyingABS}

Generally we need to distinguish between two types of testing/verification: 1. testing/verification of models for which we have real-world data or an analytical solution which can act as a ground-truth - examples for such models are the SIR model, stock-market simulations, social simulations of all kind and 2. testing/verification of models which are just exploratory and which are only be inspired by real-world phenomena - examples for such models are Epsteins Sugarscape and Agent\_Zero.

\subsection{Black-Box Verification}
In black-box Verification one generally feeds input and compares it to expected output. In the case of ABS we have the following examples of black-box test:
\begin{enumerate}
	\item Isolated Agent Behaviour - test isolated agent behaviour under given inputs using and property-based testing.
	\item Interacting Agent Behaviour - test if interaction between agents are correct .
	\item Simulation Dynamics - compare emergent dynamics of the ABS as a whole under given inputs to an analytical solution or real-world dynamics in case there exists some using statistical tests.
	\item Hypotheses- test whether hypotheses are valid or invalid using and property-based testing. % TODO: how can we formulate hypotheses in unit- and/or property-based tests?
\end{enumerate}

%- testing of the final dynamics: how close do they match the analytical solution
%- can we express model properties in tests e.g. quickcheck?
%- property-testing shines here
%- isolated tests: how easy can we test parts of an agent / simulation?

Using black-box verification and property-based testing we can apply for the following use cases for testing ABS in FRP:

\subsection{White-Box Verification}
White-Box verification is necessary when we need to reason about properties like \textit{forever}, \textit{never}, which cannot be guaranteed from black-box tests. Additional help can be coverage tests with which we can show that all code paths have been covered in our tests.

TODO: List of Common Bugs and Programming Practices to avoid them \cite{vipindeep_list_2005}

We have discussed in this section \textit{how} to approach an ABS implementation from a pure functional perspective using Haskell where we have also briefly touched on \textit{why} one should do so and what the benefits and drawbacks are. In the next two sections we will expand on the \textit{why} by presenting two case-studies which show the benefits of using Haskell in regards of testing and increasing the confidence in the correctness of the implementation.