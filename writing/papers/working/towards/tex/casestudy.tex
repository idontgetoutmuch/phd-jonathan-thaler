\section{Case-Study: Pure Functional SugarScape}
\label{sec:case_study}

To explore how to approach ABS based on pure FP concepts as introduced before, we did a full implementation \footnote{The code is freely accessible from \url{https://github.com/thalerjonathan/phd/tree/master/public/towards/SugarScape}} of the seminal Sugarscape model \cite{epstein_growing_1996}. It was one of the first models in ABS, with the aim to \textit{grow} an artificial society by simulation and connect observations in their simulation to phenomenon observed in real-world societies. In this model a population of agents move around in a discrete 2D environment, where sugar grows, and interact with each other and the environment in many different ways. The main features of this model are (amongst others): searching, harvesting and consuming of resources, wealth and age distributions, population dynamics under sexual reproduction, cultural processes and transmission, combat and assimilation, bilateral decentralized trading (bartering) between agents with endogenous demand and supply, disease processes transmission and immunology.

We chose this model because it is quite well known in the ABS community, it was highly influential in sparking the interest in ABS, it is quite complex with non-trivial agent-interactions and it used object-oriented techniques and explicitly advocates them as a good fit to ABS, which begged the question whether and how well we could do an FP implementation of it.

Our goal was first to develop techniques and concepts to show \textit{how} to engineer a clean, maintainable and robust ABS in Haskell. The second step was then to discuss \textit{why} and why not one would follow such an approach by identifying benefits and drawbacks. Absolutely paramount in our research was being \textit{pure}, avoiding the \textit{IO} effect type under all circumstances. More specifically, in this case-study:

\begin{itemize}
	\item We developed techniques for engineering a clean, maintainable, general and reusable \textit{sequential} implementation in Haskell. Those techniques are directly applicable to other ABS implementations as well and are discussed in this paper.
	
	\item We explored techniques for a \textit{concurrent} implementation using Software Transactional Memory in Haskell \cite{harris_composable_2005} and data-parallelism to speed up the execution in our sequential implementation. Due to lack of space, we refer to (TODO: cite our own TOMACS paper) for an in-depth discussion of this work.

	\item We explored ways of fully validating our implementation using code-testing. We showed how to apply property-based testing from Haskell \cite{claessen_quickcheck_2000} to ABS for a specification based checking of the correctness of the simulation. Due to lack of space, we refer to (TODO: cite our other submission to this conference) for a more detailed discussion of this work.
\end{itemize}

\subsection{A Functional View}
Due to the fundamentally different approaches of FP, an ABS needs to be implemented fundamentally differently, compared to established OOP approaches. We face the following challenges:

\begin{enumerate}
	\item How can we represent an Agent, its local state and its interface?
	\item How can we implement direct agent-to-agent interactions?
	\item How can we implement an environment and agent-to-environment interactions? 
\end{enumerate}

\subsubsection{Agent representation}
The fundamental building blocks to solve these problems are \textit{recursion} and \textit{continuations}. In recursion a function is defined in terms of itself: in the process of computing the output it \textit{might} call itself with changed input data. Continuations are functions which allow to encapsulate the execution state of a program by capturing local variables (known as closure) and pick up computation from that point later on by returning a new function. As an illustratory example, we implement a continuation in Haskell which sums up integers and stores the sum locally as well as returning it as return value for the current step:

\begin{HaskellCode}
-- define the type of the continuation: it takes an arbitrary type a 
-- and returns a type a with a new continuation
newtype Cont a = Cont (a -> (a, Cont a))

-- an instance of a continuation with type a fixed to Int
-- takes an initial value x and sums up the values passed to it
-- note that it returns adder with the new sum recursively as 
-- the new continuation
adder :: Int -> Cont Int
adder x = Cont (\x' -> (x + x', adder (x + x')))

-- this function runs the given continuation for a given number of steps
-- and always passes 1 as input and prints the continuations output
runCont :: Int -> Cont Int -> IO ()
runCont 0 _ = return () -- finished
runCont n (Cont cont) = do -- pattern match to extract the function
  -- run the continuation with 1 as input, cont' is the new continuation
  let (x, cont') = cont 1
  print x
  -- recursive call, run next step
  runCont (n-1) cont'

-- main entry point of a Haskell program
-- run the continuation adder with initial value of 0 for 100 steps 
main :: IO ()
main = runCont 100 (adder 0)
\end{HaskellCode}

We implement an agent as a continuation: this lets us encapsulate arbitrary complex agent-state which is only visible and accessible from within the continuation - the agent has exclusive access to it. Further, with a continuation it becomes possible to switch behaviour dynamically e.g. switching from one mode of behaviour to another like in a state-machine, simply by returning new functions which encapsulate the new behaviour. If no change in behaviour should occur, the continuation simply recursively returns itself with the new state captured as seen in the example above.

The fact that we design an agent as a function, raises the question of the interface of it: what are the inputs and the output? Note that the type of the function has to stay the same (type \textit{a} in the example above) although we might switch into different continuations - our interface needs to capture all possible cases of behaviour. The way we define the interface is strongly determined by the direct agent-agent interaction. In case of Sugarscape, agents need to be able to conduct two types of direct agent-agent interaction: 1. one-directional, where agent A sends a message to agent B without requiring agent B to synchronously reply to that message e.g. repaying a loan or inheriting money to children; 2. bi-directional, where two agents negotiate over multiple steps e.g. accepting a trade, mating or lending. Thus it seems reasonable to define as input type an enumeration (algebraic data-type in Haskell, see example below) which defines all possible incoming messages the agent can handle. The agents continuation is then called every time the agent receives a message and can process it, update its local state and might change its behaviour.

As output we define a data-structure which allows the agent to communicate to the simulation kernel 1. whether it wants to be removed from the system, 2. a list of new agents it wants to spawn, 3. a list of messages the agent wants to send to other agents. Further because the agents data is completely local, it also returns a data-structure which holds all \textit{observable} information the agent wants to share with the outside world. Together with the continuation this guarantees that the agent is in full control over its local state, no one can mutate or access from outside. This also implies that information can only get out of the agent by actually running its continuation. It also means that the output type of the function has to cover all possible input cases - it cannot change or depend on the input. 

\begin{HaskellCode}
type AgentId    = Int
data Message    = Tick Int | MatingRequest AgentGender ... 
data AgentState = AgentState { agentAge :: Int, ... }             
data Observable = Observable { agentAgeObs :: Int, ... } 
data AgentOut   = AgentOut
  { kill       :: Bool
  , observable :: Observable
  , messages   :: [(AgentId, Message)] -- list of messages with receiver
  }
-- agent continuation has different types for input and output
newtype AgentCont inp out = AgentCont (in -> (out, AgentCont inp out))
-- taking the initial AgentState as input and returns the continuation
sugarscapeAgent :: AgentState -> AgentCont (AgentId, Message) AgentOut
sugarscapeAgent asInit = AgentCont (\ (sender, msg) -> 
  case msg of
    agentCont (sender, Tick t) = ... handle tick
    agentCont (sender, MatingRequest otherGender) = ... handle mating request)
\end{HaskellCode}

\subsubsection{Stepping the simulation}
The simulation kernel keeps track of the existing agents and the message-queue and processes the queue one element at a time. The new messages of an agent are inserted \textit{at the front} of the queue, ensuring that synchronous bi-directional messages are possible without violating resources constraints. The Sugarscape model specifies that in each tick all agents run in random order, thus to start the agent-behaviour in a new time-step, the core inserts a \textit{Tick} message to each agent in random order which then results in them being executed and emitting new messages. The current time-step has finished when all messages in the queue have been processed. See algorithm \ref{alg:stepping_simulation} for the pseudo-code for the simulation stepping.

\begin{algorithm}
\SetKwInOut{Input}{input}\SetKwInOut{Output}{output}
\Input{All agents \textit{as}}
\Input{List of agent observables}
shuffle all agents as\;
messageQueue = schedule Tick to all agents\;
agentObservables = empty List\;
\While{messageQueue not empty} {
  msg = pop message from messageQueue\;
  a = lookup receiving agent in as\;
  (out, a') = runAgent a msg\;
  update agent with continuation a' in as\;
  add agent observable from out to agentObservables\;
  add messages of agent at front of messageQueue\;
}
return agentObservables\;
\caption{Stepping the simulation.}
\end{algorithm}
\label{alg:stepping_simulation}

\subsubsection{Environment and agent-environment interaction}
The agents in the Sugarscape are located in a discrete 2d environment where they move around and harvest resources, which means the need to read and write data of environment. This is conveniently implemented by adding a State side-effect type to the agent continuation function. Further we also add a Random effect type because dynamics in most ABS in general and Sugarscapes in particular are driven by random number streams, so our agent needs to have access to one as well. All of this low level continuation plumbing exists already as a high quality library called Dunai, based on research on Functional Reactive Programming  \cite{hudak_arrows_2003} and Monadic Stream Functions \cite{perez_functional_2016,perez_extensible_2017}.