\section{Case-Study: Pure Functional SugarScape}
TODO

why sugarscape
- original sugarscape sparked ABS and use of OOP, therefore 
- quite complex model, will challenge implementation techniques

\footnote{The code is freely accessible from \url{https://github.com/thalerjonathan/phd/tree/master/public/towards/code}}

\cite{weaver_replicating_nodate}

page 28, footnote 16: we can guarantee that in haskell at compile time

\subsection{Terracing}
Our implementation reproduce the terracing phenomenon as described on page TODO in Animation and as can be seen in the NetLogo implementation as well. We implemented a property-test in which we measure the closeness of agents to the ridge: counting the number of same-level sugarscells around them and if there is at least one lower then they are at the edge. If a certain percentage is at the edge then we accept terracing. The question is just how much, this we estimated from tests and resulted in 45\%. Also, in the terracing animation the agents actually never move which is because sugar immediately grows back thus there is no need for an agent to actually move after it has moved to the nearest largest cite in can see. Therefore we test that the coordinates of the agents after 50 steps are the same for the remaining steps.

\subsection{Carrying Capacity}
Our simulation reached a steady state (variance < 4 after 100 steps) with a mean around ~182. Epstein reported a carrying capacity of 224 (page 30) and the NetLogo implementations carrying capacity fluctuates around 210 which both were thus significantly higher than ours. Something was definitely wrong - the carrying capacity has to be around 200 (we trust in this case the NetLogo implementation and deem 224 an outlier).

After inspection of the netlogo model we realised that we implicitly assumed that the metabolism range is \textit{continuously} uniformly randomized between 1 and 4 but this seemed not what the original authors intended: in the netlogo model there were a few agents surviving on sugarlevel 1 which was never the case in ours as the probability of drawing a metabolism of exactly 1 is 0 when drawing from a continuous range. We thus changed our implementation to draw discrete. Note that this actually makes sense as massive floating-point number calculations were quite expensive in the mid 90s (e.g. computer games ran still on CPU only and exploited various  clever tricks to avoid the need of floating point calculations whenever possible) when SugarScape was implemented which might have been a reason for the authors to assume it implicitly.

This partly solved the problem, the carrying capacity was now around 204 which is much better than 182 but still a far cry from 210 or even 224. After adjusting the order in which agents apply the sugarscape rules, (by looking at the code of the netlogo implementation), we arrived at a comparable carrying capacity of the netlogo implementation: agents first make their move and harvest sugar and only after this the agents metabolism is applied (and ageing in subsequent experiments).

For regression-tests we implemented a property-test which tests that the carrying capacity of 100 simulation runs lies within a 95\% confidence interval of a 210 mean. TODO: variance test. These values are quite reasonable to assume, when looking at NetLogo - again we deem the reported Carrying Capacity of 224 in the Book to be an outlier / part of other details we don't know.

TODO: do a replication experiment with NetLogo (is it possible?)

\subsection{Wealth Distribution}
By visual comparison we validated that the wealth distribution (page 32-37) becomes strongly skewed with a Histogram showing a fat tail, power-law distribution where very few agents are very rich and most of the agents are quite poor. We compute the skewness and kurtosis of the distribution which is around a skewness of 1.5, clearly indicating a right skewed distribution and a kurtosis which is around 2.0 which clearly indicates the lst histogram of Animation II-3 on page 34. Also we compute the gini-coefficient and it varies between 0.47 and 0.5 - this is accordance with Animation II-4 on page 38 which shows a gini-coefficient which stabilises around 0.5 after. 
We implemented a regression-test testing skewness, kurtosis and gini-coefficients of 100 runs to be within a 95\% confidence interval of a two-sided t-test using an expected skewness of 1.5, kurtosis of 2.0 and gini-coefficient of 0.48.

\subsection{Migration}
With the information provided by \cite{weaver_replicating_nodate} we could replicate the waves as visible in the NetLogo implementation as well. Also we propose that a vision of 10 is not enough yet and shall be increased to 15 which makes the waves very prominent and keeps them up for much longer - agent waves are travelling back and forth between both sugarscape peaks. We haven't implemented a regression-test for this property as we couldn't come up with a reasonable straight forward approach to implement it.

Note that we spent quite a lot of time of getting this and the terracing properties right because they form the very basics of the other ones which follow so we had to be sure that those were correct otherwise validating would have been much more difficult.
