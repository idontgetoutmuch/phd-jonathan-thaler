\section{Case-Study: Pure Functional SugarScape}
TODO

why sugarscape
- original sugarscape sparked ABS and use of OOP, therefore 
- quite complex model, will challenge implementation techniques

\footnote{The code is freely accessible from \url{https://github.com/thalerjonathan/phd/tree/master/public/towards/SugarScape}}

\cite{weaver_replicating_nodate}

page 28, footnote 16: we can guarantee that in haskell at compile time

TODO: investigate where data-parallelisation is possible. concurrency has been dealt with in the STM paper already.

\section{Chapter II}
each agent is a Signal Function with no input and outputs an AgentOut which contains a list of agents it wants to spawn, a flag if the agent is to be removed from the simulation (e.g. starved to death) and observable properties the agent exhibits to the outside world. All the agents properties are encapsulated in the SF continuation and there is no way to access and manipulate the data from outside without running the SF itself which will produce an AgentOut.

An agent has access to the shared environment state, a random-number generator and a shared ABS-system state which contains the next agent-id when birthing a new agent. All this is implemented by sharing the data-structures amongst all agents which can read/write it - this is possible in functional programming using Monadic Programming which can simulate a global state, accessible from within a function which can then read/write this state. The fundamental difference to imperative oop-programming is that all reads / writes are explicit using functions (no assignment).

Updating the agents is straight-forward because in this chapter, the agents interact with each other indirectly through the environment. In each step the agents are shuffled and updated one after another, where agents can see actions of agents updated before. 

Our approach of sharing the environment globally and the agent-state locally works but immediately creates potential problems like: ordering of updates matter - in the end we are implementing a kind of an imperative approach but embedded in a functional language. The benefits are that we have much stronger type-safety and that the access and modification of the states is much more explicit than in imperative approaches - also we dont have mutable references.

We implemented a different approach to iterating: instead of running the agents one after another and interacting through a globally shared environment all agents are now run \textit{conceptually} at the same time and receive the current environment as additional input and have to provide it in the output. This has the following implications: we end up with n copies of the environment where n is the number of agents, agents are not able to see the actions of others until the next step, there can be conflicts where multiple agents end up on the same position. 
Obviously, positional conflicts need to be solved as the sugarscape specification clearly states that only one agent stays on a site at a time. Functional programming makes solving such conflicts easy: we pick a winning agent and rollback the other agents by re-running them with their SF at the beginning of the step - this will undo all changes within the encapsulation. Obviously it would be possible to have conflicts again thus one needs to recursively run the conflict-resolving process until no more conflicts are present.
Although this solution is much slower and more complex to implement and thus not feasible to use in practice but we wanted to explore it for the following reasons:
- it is "closer" to functional programming in spirit because programming with globally mutable state (even if its restricted, explicit and only simulated) should be avoided as far as possible.
- we can exploit data-parallelism (but in this case its not possible anyway because of monadic computations: need mapM which can by definition not be parallel because ordering matters)
- it serves more as a study to what different approaches are possible and how difficult / easy it is to implement them in FP, in this case, "rolling back" the actions of an agent is trivial in FP as long as the underlying monadic context is immune to rollbacks, in our case we argue that it is: incrementing agentids in ABSState does not matter, as it doesnt matter that we have a changed random-number stream. It would be a different matter if there is a global shared state which was modified by the agent.
- in the extreme case this degenerates to a (much more expensive) sequential update 

\section{Chapter III}
This chapter reveals the fundamental difference and difficulty in pure functional programming over established OOP approaches in the field: direct agent-interaction e.g. in mating where 2 agents interact synchronously with each other and might updated their internal state. These interactions \textit{must} happen synchronously because there are resource constraints in place which could be violated if an agent interacts with multiple agents virtually at the same time.

In established OOP approaches this is nearly trivial and straight forward: the agent which initiates the direct interaction holds or looks up (e.g. through a central simulation management object) a reference to the other agent (e.g. by neighbourhood) and then makes direct method calls to the other agent where internal agent-states of both agents may be mutated.
This approach is not possible in pure functional programming because: 1. there are no objects which encapsulate state and behaviour and 2. there are not side-effects possible which would allow such a mutation of local state \footnote{Relaxing our constraint by also allowing \textit{impure} functional features so we can workaround the limitation of not being able to locally mutate state but this is not what we are interested in in this paper because we lose all relevant guarantees which make FP relevant and of benefit.}. 

This makes implementation of direct agent-interactions utterly difficult.
If we build on the approach we used for Chapter II (and which worked very well there!) we quickly run into painful problems:
\begin{itemize}
	\item To mutate local agent state or to generate an output / seeing local properties requires to run the SF. 
	\item Running the SF is intrinsically linked in stepping the simulation forward from t to t+1. Currently the agent has no means to distinguish between different reasons why the SF is being run.
	\item The agents are run after another (after being shuffled) and cannot make invokations of other agents SF during being executed due to pure functional programming.
\end{itemize}

A solution is to change to an event-driven approach: SF now have an input, which indicates an EventType and Agents need some way of initiating a multi-step interaction where a reply can lead to a new event and so on. In case of a simple time-advancement the SF is run with a "TimeStep" event, if an agent requests mating, then it sends "MatingRequest" to the other SF. This requires a completely different approach to iterating the agents.

Stateful programming (or programming that \textit{feels} stateful) comes inherently with difficulties where one can forget to update a state or mutate state where not appropriate. A pure functional approach to that is no exception and shows the same problems. In our case we ran into a bug where the trading agent saw an outdated MRS value of the trading-partner resulting into two different trading-prices which obviously must be prevented under all circumstances because it would destroy / create wealth. The origin of the bug was that MRS depends on the wealth (sugar and spice) of the agent and we simply forgot to update the MRS in the environment from which the offering agent can read it when the trading agents wealth changed (e.g. through harvesting, inheritance,...).

\subsection{Performance}
Lazyness led to space-leaks which led to awful performance. It took us quite a while to figure out what was the problem