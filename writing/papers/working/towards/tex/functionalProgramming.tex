\section{Functional Programming}
The reason why functional programming is called \textit{functional} is because because it makes functions the main concept of programming, promoting them to first-class citizens as we will describe later on. Its roots lie in the Lambda Calculus which was first described by Alonzo Church \citep{church_unsolvable_1936}. This is a fundamentally different approach to computation than imperative and object-oriented programming which roots lie in the Turing Machine \citep{turing_computable_1937}. Rather than describing \textit{how} something is computed as in the more operational approach of the Turing Machine, due to the more declarative nature of the Lambda Calculus, code in functional programming describes \textit{what} is computed.

In our research we are using the functional programming language Haskell. The paper of \citep{hudak_history_2007} gives a comprehensive overview over the history of the language, how it developed and its features and is very interesting to read and get accustomed to the background of the language. The main points why we decided to go for Haskell are:

\begin{itemize}
	\item Rich Feature-Set - it has all fundamental concepts of the pure functional programming paradigm of which we explain the most important below.
	\item Real-World applications - the strength of Haskell has been proven through a vast amount of highly diverse real-world applications \cite{hudak_history_2007}, is applicable to a number of real-world problems \cite{osullivan_real_2008} and has a large number of libraries available \footnote{\url{https://wiki.haskell.org/Applications_and_libraries}}.
	\item Modern - Haskell is constantly evolving through its community and adapting to keep up with the fast changing field of computer science. Further, the community is the main source of high-quality libraries.
\end{itemize}

As a short example we give an implementation of the factorial function in Haskell:
\begin{HaskellCode}
factorial :: Integer -> Integer
factorial 0 = 1
factorial n = n * factorial (n-1)
\end{HaskellCode}

When looking at this function we can already see the central concepts of functional programming: 
\begin{enumerate}
	\item Declarative - we describe \textit{what} the factorial function is rather than how to compute it. This is supported by \textit{pattern matching} which allows to give multiple equations for the same function, matching on its input. 
	\item Immutable data - in functional programming we don't have mutable variables - after a variable is assigned, it cannot change its contents. This also means that there is no destructive assignment operator which can re-assign values to a variable. To change values, we employ recursion.
	\item Recursion - the function calls itself with a smaller argument and will eventually reach the case of 0. Recursion is the very meat of functional programming because they are the only way to implement loops in this paradigm due to immutable data.
	\item Static Types - the first line indicates the name and the types of the function. In this case the function takes one Integer as input and returns an Integer as output. Types are static in Haskell which means that there can be no type-errors at run-time e.g. when one tries to cast one type into another because this is not supported by this kind of type-system.
	\item Explicit input and output - all data which are required and produced by the function have to be explicitly passed in and out of it. There exists no global mutable data whatsoever and data-flow is always explicit.
	\item Referential transparency - calling this function with the same argument will \textit{always} lead to the same result, meaning one can replace this function by its value. This means that when implementing this function one can not read from a file or open a connection to a server. This is also known as \textit{purity} and is indicated in Haskell in the types which means that it is also guaranteed by the compiler.
\end{enumerate}

It may seem that one runs into efficiency-problems in Haskell when using algorithms which are implemented in imperative languages through mutable data which allows in-place update of memory. The seminal work of \cite{okasaki_purely_1999} showed that when approaching this problem from a functional mind-set this does not necessarily be the case. The author presents functional data structures which are asymptotically as efficient as the best imperative implementations and discusses the estimation of the complexity of lazy programs.

For an excellent and widely used introduction to programming in Haskell we refer to \cite{hutton_programming_2016}. Other, more exhaustive books on learning Haskell are \cite{lipovaca_learn_2011, allen_haskell_2016}. For an introduction to programming with the Lambda-Calculus we refer to \cite{michaelson_introduction_2011}. For more general discussion of functional programming we refer to \cite{hughes_why_1989, maclennan_functional_1990, hudak_history_2007}.

\subsection{Lazy evaluation, Higher Order Functions and Currying}
TODO: 

\subsection{Side-Effects}
One of the fundamental strengths of functional programming and Haskell is their way of dealing with side-effects in functions. A function with side-effects has observable interactions with some state outside of its explicit scope. This means that the behaviour it depends on history and that it loses its referential transparency character, which makes understanding and debugging much harder. Examples for side-effects are (amongst others): modifying a global variable, await an input from the keyboard, read or write to a file, open a connection to a server, drawing random-numbers,...

Obviously, to write real-world programs which interact with the outside-world we need side-effects. Haskell allows to indicate in the \textit{type} of a function that it does or does \textit{not} have side-effects. Further there are a broad range of different effect types available, to restrict the possible effects a function can have to only the required type. This is then ensured by the compiler which means that a program in which one tries to e.g. read a file in a function which only allows drawing random-numbers will fail to compile. Haskell also provides mechanisms to combine multiple effects e.g. one can define a function which can draw random-numbers and modify some global data. The most common side-effect types are:
\begin{itemize}
	\item IO - Allows all kind of I/O related side-effects: reading/writing a file, creating threads, write to the standard output, read from the keyboard, opening network-connections, mutable references,... 
	\item Rand - Allows to draw random-numbers.
	\item Reader - Allows to read from an environment.
	\item Writer - Allows to write to an environment.
	\item State - Allows to read and write an environment.
\end{itemize}

A function with side-effects has to indicate this in their type e.g. if we want to give our factorial function for debugging purposes the ability to write to the standard output, we add IO to its type: factorial :: Integer -> IO Integer. A function without any side-effect type is called \textit{pure}. A function with a given effect-type needs to be executed with a given effect-runner which takes all necessary parameters depending on the effect and runs a given effectful function returning its return value and depending on the effect also an effect-related result. For example when running a function with a State-effect one needs to specify the initial environment which can be read and written. After running such a function with a State-effect the effect-runner returns the changed environment in addition with the return value of the function itself. Note that we cannot call functions of different effect-types from a function with another effect-type, which would violate the guarantees. Calling a \textit{pure} function though is always allowed because it has by definition no side-effects. An effect-runner itself is a \textit{pure} function. The exception to this is the IO effect type which does not have a runner but originates from the \textit{main} function which is always of type IO.

Although it might seem very restrictive at first, we get a number of benefits from making the type of effects we can use explicit. First we can restrict the side-effects a function can have to a very specific type which is guaranteed at compile time. This means we can have much stronger guarantees about our program and the absence of potential errors already at compile-time which implies that we don't need test them with e.g. unit-tests. Second, because effect-runners are themselves \textit{pure}, we can execute effectful functions in a very controlled way by making the effect-context explicit in the parameters to the effect-runner. This allows a much easier approach to isolated testing because the history of the system is made explicit.

For a technical, in-depth discussion of the concept of side-effects and how they are implemented in Haskell using Monads, we refer to the following papers: \cite{moggi_computational_1989, wadler_essence_1992, wadler_monads_1995, wadler_how_1997, jones_tackling_2002}.

\subsection{Parallelism and Concurrency}
TODO: write this section

Also Haskell makes a very clear distinction between parallelism and concurrency. Parallelism is always deterministic and thus pure without side-effects because although parallel code runs concurrently, it does by definition not interact with data of other threads. This can be indicated through types: we can run pure functions in parallel because for them it doesn't matter in which order they are executed, the result will always be the same due to the concept of referential transparency. Concurrency is potentially non-deterministic because of non-deterministic interactions of concurrently running threads through shared data. Although data in functional programming is immutable, Haskell provides primitives which allow to share immutable data between threads. Accessing these primitives is but only possible from within an IO or STM context which means that when we are using concurrency in our program, the types of our functions change from pure to either IO or STM effect context.

Spawning thousands of threads in Haskell is no problem and has very low memory footprint because they are lightweight user-space threads, managed by the Haskell Runtime System which maps them to physical operating-system threads. 

TODO: in haskell we can distinguish between parallelism and concurrency in the types: parallelism is pure, concurrency is impure
TODO: parallelism for free because all isolated e.g. running multiple replications or parameter-variations

TODO: For a technical, in-depth discussion on parallelism and concurrency in Haskell we refer to the excellent book \cite{marlow_parallel_2013}.

TODO: explain STM, Problem: live locks, For a technical, in-depth discussion on Software Transactional Memory in Haskell we refer to the following papers: \citep{harris_composable_2005, discolo_lock_2006, osullivan_real_2008, perfumo_limits_2008}. TODO: need much more papers on STM, parallelism and concurrency

\subsection{Functional Reactive Programming}
\label{sec:frp}
Functional Reactive Programming (FRP) is a way to implement systems with continuous and discrete time-semantics in functional programming. The central concept in FRP is the Signal Function which can be understood as a process over time which maps an input- to an output-signal. A signal in turn, can be understood as a value
which varies over time. Thus, signal functions have an awareness of the passing of time by having access to a $\Delta t$ which are positive time-steps with which the system is sampled. In general, signal functions can be understood to be computations that represent processes, which have an input of a specific type, process it and output a new type. Note that this is an important building block to represent agents in functional programming: by implementing agents as signal functions allows us to implement them as processes which act continuously over time, which implies a time-driven approach to ABS. We have also applied the concept of FRP to event-driven ABS \citep{meyer_event-driven_2014}.

FRP provides a number of functions for expressing time-semantics, generating events and making state-changes of the system. They allow to change system behaviour in case of events, run signal functions, generate deterministic (after fixed time) and stochastic (exponential arrival rate) events and provide random-number streams. 

TODO: libraries Yampa and Dunai

For a technical, in-depth discussion on FRP in Haskell we refer to the following papers: \citep{wan_functional_2000, hughes_generalising_2000, hughes_programming_2005, nilsson_functional_2002, hudak_arrows_2003, courtney_yampa_2003, perez_functional_2016, perez_extensible_2017}

\subsection{Property-Based Testing}
TODO: write this section

Although property-based testing has been brought to non-functional languages like Java and Python as well, it has its origins in Haskell and it is here where it truly shines.

We found property-based testing particularly well suited for ABS. Although it is now available in a wide range of programming languages and paradigms, propert-based testing has its origins in Haskell \citep{claessen_quickcheck:_2000, claessen_testing_2002} and we argue that for that reason it really shines in pure functional programming. Property-based testing allows to formulate \textit{functional specifications} in code which then the property-testing library (e.g. QuickCheck \citep{claessen_quickcheck:_2000}) tries to falsify by automatically generating random test-data covering as much cases as possible. When an input is found for which the property fails, the library then reduces it to the most simple one. It is clear to see that this kind of testing is especially suited to ABS, because we can formulate specifications, meaning we describe \textit{what} to test instead of \textit{how} to test (again the declarative nature of functional programming shines through). Also the deductive nature of falsification in property-based testing suits very well the constructive nature of ABS.

TODO: use the reverse or the reverse is the list again

For a technical, in-depth discussion on property-based testing in Haskell we refer to the following papers: \citep{claessen_quickcheck:_2000, claessen_testing_2002}.

%\subsection{Software Transactional Memory}
%Although there exist STM implementations in non-functional languages like Java and Python, due to the nature of Haskells type-system, the use of STM has unique benefits in this setting.
%
%Concurrent programming is notoriously difficult to get right because reasoning about the interactions of multiple concurrently running threads and low level operational details of synchronisation primitives and locks is \textit{very hard}. The main problems are:
%
%\begin{itemize}
%	\item Race conditions due to forgotten locks.
%	\item Deadlocks resulting from inconsistent lock ordering.
%	\item Corruption caused by uncaught exceptions.
%	\item Lost wakeups induced by omitted notifications.
%\end{itemize}
%
%Worse, concurrency does not compose. It is utterly difficult to write two functions (or methods in an object) acting on concurrent data which can be composed into a larger concurrent behaviour. The reason for it is that one has to know about internal details of locking, which breaks encapsulation and makes composition depend on knowledge about their implementation. Also it is impossible to compose two functions e.g. where one withdraws some amount of money from an account and the other deposits this amount of money into a different account: one ends up with a temporary state where the money is in none of either accounts, creating an inconsistency - a potential source for errors because threads can be rescheduled at any time.
%
%STM promises to solve all these problems for a very low cost. In STM one executes actions atomically where modifications made in such an action are invisible to other threads until the action is performed. Also the thread in which this action is run, doesn't see changes made by other threads - thus execution of STM actions are isolated. When a transaction exists one of the following things will occur:
%
%\begin{enumerate}
%	\item If no other thread concurrently modified the same data as us, all of our modifications will simultaneously become visible to other threads.
%	\item Otherwise, our modifications are discarded without being performed, and our block of actions is automatically restarted.
%\end{enumerate}
%
%Note that the ability to \textit{restart} a block of actions without any visible effects is only possible due to the nature of Haskells type-system which allows being explicit about side-effects: by restricting the effects to STM only ensures that no uncontrolled effects, which cannot be rolled-back, occur.
%
%STM is implemented using optimistic synchronisation. This means that instead of locking access to shared data, each thread keeps a transaction log for each read and write to shared data it makes. When the transaction exists, this log is checked whether other threads have written to memory it has read - it checks whether it has a consistent view to the shared data or not. This might look like a serious overhead but the implementations are very mature by now, being very performant and the benefits outweigh its costs by far.
%
%Applying this to our agents is very simple: because we already use Dunai / BearRiver as our FRP library, we can run in arbitrary Monadic contexts. This allows us to simply run agents within an STM Monad and execute each agent in their own thread. This allows then the agents to communicate concurrently with each other using the STM primitives without problems of explicit concurrency, making the concurrent nature of an implementation very transparent. Further through optimistic synchronisation we should arrive at a much better performance than with low level locking.
%
%Problem: live locks
%
%For a technical, in-depth discussion on Software Transactional Memory in Haskell we refer to the following papers: \citep{harris_composable_2005, osullivan_real_2008}.