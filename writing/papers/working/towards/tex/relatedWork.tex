\section{Related Work}
\label{sec:related_work}

The amount of research on using FP with Haskell in the field of ABS has been moderate so far. Most of the papers are related to the field of Multi Agent Systems and look into how agents can be specified using the belief-desire-intention paradigm \cite{de_jong_suitability_2014,sulzmann_specifying_2007,jankovic_functional_2007}.

A multi-method simulation library in Haskell called \textit{Aivika 3} is described in the technical report \cite{sorokin_aivika_2015}. It supports implementing Discrete Event Simulations (DES), System Dynamics and comes with basic features for event-driven ABS which is realised using DES under the hood. Further it provides functionality for adding GPSS to models and supports parallel and distributed simulations. It runs within the IO effect type for realising parallel and distributed simulation but also discusses generalising their approach to avoid running in IO.

In his master thesis \cite{bezirgiannis_improving_2013} the author investigated Haskells parallel and concurrency features to implement (amongst others) \textit{HLogo}, a Haskell clone of the NetLogo simulation package, focusing on using Software Transactional Memory for a limited form of agent-interactions. \textit{HLogo} is basically a re-implementation of NetLogos API in Haskell where agents run within the IO effect type.

% SHORTENING
%There exists some research \cite{di_stefano_using_2005,varela_modelling_2004,sher_agent-based_2013} of using the functional programming language Erlang \cite{armstrong_erlang_2010} to implement ABS. The language is inspired by the actor model \cite{agha_actors:_1986} and was created in the 1986 by Joe Armstrong for Eriksson for developing distributed high reliability software in telecommunications. The actor model can be seen as quite influential to the development of the concept of agents in ABS which borrowed it from Multi Agent Systems \cite{wooldridge_introduction_2009}. It emphasises message-passing concurrency with share-nothing semantics (no shared state between agents) which maps nicely to functional programming concepts. The mentioned papers investigate how the actor model can be used to close the conceptual gap between agent-specifications which focus on message-passing and their implementation. Further they also showed that using this kind of concurrency allows to overcome some problems of low level concurrent programming as well.

Using functional programming for DES was discussed in \cite{jankovic_functional_2007} where the authors explicitly mention the paradigm of Functional Reactive Programming (FRP) to be very suitable to DES.

A domain-specific language for developing functional reactive agent-based simulations was presented in \cite{schneider_towards_2012,vendrov_frabjous_2014}. This language called FRABJOUS is human readable and easily understandable by domain-experts. It is not directly implemented in FRP/Haskell but is compiled to Haskell code which they claim is also readable. This supports that FRP is a suitable approach to implement ABS in Haskell. Unfortunately, the authors do not discuss their mapping of ABS to FRP on a technical level, which would be of most interest to functional programmers.

The authors of \cite{botta_time_2010} discuss the problem of advancing time in message-driven agent-based socio-economic models. They formulate purely functional definitions for agents and their interactions through messages. Our architecture for synchronous agent-interaction as discussed in the case-study was not directly inspired by their work but has some similarities: the use of messages and the problem of when to advance time in models with arbitrary number synchronised agent-interactions.