\section{Conclusions}
\label{sec:conclusions}

Our results support the claim that pure functional programming has indeed its place in ABS but only in cases of high-impact and large-scale simulations which results might have far-reaching consequences e.g. influence policy decisions. The reason is that engineering a proper implementation of a non-trivial ABS model takes substantial effort in pure functional programming due to different approaches. This is almost never necessary for quick prototyping and small case-studies of ABS models for which NetLogo and the established object-oriented approaches using Python, Java, C++ are perfectly suitable.

\subsection{Further Research}
- distil the developed techniques into a general purpose ABS library so implementing models becomes much easier and quicker and makes using Haskell attractive for prototyping models as well.

- distributed
- DES and PDES
- gintis-case study
- recursive simulation