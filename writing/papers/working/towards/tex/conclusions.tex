\section{Conclusions}
\label{sec:conclusions}
Our results support the claim that pure functional programming has indeed its place in ABS but only in cases of high-impact and large-scale simulations which results might have far-reaching consequences e.g. influence policy decisions. The reason is that engineering a proper implementation of a non-trivial ABS model takes substantial effort in pure functional programming due to different approaches. This is almost never necessary for quick prototyping and small case-studies of ABS models for which NetLogo and the established object-oriented approaches using Python, Java, C++ are perfectly suitable.

This approach can support TOMACS Reproducibility Board

\subsection{Further Research}
We plan on distilling the developed techniques of the case-study into a general purpose ABS library. This should make implementing models much easier and quicker and using Haskell attractive for prototyping models as well. 

Often, ABS models build on an underlying DES core, especially when they follow an event-driven approach \cite{meyer_event-driven_2014} where they need to schedule actions into the future. Due to the time-driven approach of the Sugarscape model, we didn't implement this in our case-study but our pure functional simulation core is conceptually already very close to a DES approach. We plan on extending it to allow true event-based ABS, which we want to feed into our general purpose ABS library as well.

- recursive simulation