\section{Conclusions}
\label{sec:conclusions}
Our results support the claim that pure functional programming has indeed its place in ABS but only in cases of high-impact and large-scale simulations which results might have far-reaching consequences e.g. influence policy decisions. The reason is that engineering a proper implementation of a non-trivial ABS model takes substantial effort in pure functional programming due to different approaches. This is almost never necessary for quick prototyping and small case-studies of ABS models for which NetLogo and the established object-oriented approaches using Python, Java, C++ are perfectly suitable.

TODO: a bit more conclusion
This approach can support TOMACS Reproducibility Board

\subsection{Further Research}
We plan on distilling the developed techniques of the case-study into a general purpose ABS library. This should make implementing models much easier and quicker and using Haskell attractive for prototyping models as well. 

Often, ABS models build on an underlying DES core, especially when they follow an event-driven approach \cite{meyer_event-driven_2014} where they need to schedule actions into the future. Due to the time-driven approach of the Sugarscape model, we didn't implement this in our case-study but our pure functional simulation core is conceptually already very close to a DES approach. We plan on extending it to allow true event-based ABS, which we want to feed into our general purpose ABS library as well.

Due to the recursive nature of functional programming we believe that it is also a natural fit to implement recursive simulations as the one discussed in \cite{gilmer_recursive_2000}. In recursive ABS agents are able to halt time and 'play through' an arbitrary number of actions, compare their outcome and then to resume time and continue with a specifically chosen action e.g. the best performing or the one in which they haven't died. More precisely, an agent has the ability to run the simulation recursively a number of times where the this number is not determined initially but can depend on the outcome of the recursive simulation. So recursive ABS gives each Agent the ability to run the simulation locally from its point of view to anticipate its actions in the future and change them in the present. Due to the lack of implicit side-effects and referential transparency, combined with the recursive nature of pure functional programming, we think that implementing a recursive simulation in such a setting should be straight-forward.

This research is only a first step towards an even more extensive use of type systems in ABS. The next step is to investigate the use of dependent types in ABS, using the pure functional language Idris. Dependent types allow to express even stronger guarantees at compile time, going as far as programs being valid proofs for the correctness of the implemented feature.