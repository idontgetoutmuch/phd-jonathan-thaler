\section{Background}
In this section we give a short overview of the concepts used throughout the paper.

%\subsection{System Dynamics}
%In System Dynamics (SD) one models a system through differential equations, allowing to conveniently express continuous systems which change over time \cite{porter_industrial_1962}.  TODO: go more into length

\subsection{Haskell}
We are using the functional programming language Haskell. The paper \citep{hudak_history_2007} gives a comprehensive overview over the history of the language, how it developed and its features and is very interesting to read and get accustomed to the background of the language. The main points why we decided to go for Haskell are:

\begin{itemize}
	\item Rich Feature-Set - it has all fundamental concepts of the pure functional programming paradigm of which we explain the most important below.
	\item Real-World applications - the strength of Haskell has been proven through a vast amount of highly diverse real-world applications \cite{hudak_history_2007}, is applicable to a number of real-world problems \cite{osullivan_real_2008} and has a large number of libraries available \footnote{\url{https://wiki.haskell.org/Applications_and_libraries}}.
	\item Modern - Haskell is constantly evolving through its community and adapting to keep up with the fast changing field of computer science. Further, the community is the main source of high-quality libraries.
\end{itemize}

\subsection{Functional Reactive Programming}
Functional Reactive Programming is a way to implement systems with continuous and discrete time-semantics in pure functional languages. There are many different approaches and implementations but in our approach we use \textit{Arrowized} FRP \cite{hughes_generalising_2000, hughes_programming_2005} as implemented in the library Yampa \cite{hudak_arrows_2003, courtney_yampa_2003, nilsson_functional_2002}.

The central concept in Arrowized FRP is the Signal Function (SF) which can be understood as a \textit{process over time} which maps an input- to an output-signal. A signal can be understood as a value which varies over time. Thus, signal functions have an awareness of the passing of time by having access to $\Delta t$ which are positive time-steps with which the system is sampled. 

\begin{flalign*}
Signal \, \alpha \approx Time \rightarrow \alpha \\
SF \, \alpha \, \beta \approx Signal \, \alpha \rightarrow Signal \, \beta 
\end{flalign*}

Yampa provides a number of combinators for expressing time-semantics, events and state-changes of the system. They allow to change system behaviour in case of events, run signal functions and generate stochastic events and random-number streams. We shortly discuss the relevant combinators and concepts we use throughout the paper. For a more in-depth discussion we refer to \cite{hudak_arrows_2003, courtney_yampa_2003, nilsson_functional_2002}.

\subsection{Arrowized programming}
TODO: this is probably by far too much for simulation people

Yampa's signal functions are arrows, requiring us to program with arrows. Arrows are a generalisation of monads which, in addition to the already familiar parameterisation over the output type, allow parameterisation over their input type as well \cite{hughes_generalising_2000, hughes_programming_2005}.

In general, arrows can be understood to be computations that represent processes, which have an input of a specific type, process it and output a new type. This is the reason why Yampa is using arrows to represent their signal functions: the concept of processes, which signal functions are, maps naturally to arrows.

There exists a number of arrow combinators which allow arrowized programing in a point-free style but due to lack of space we will not discuss them here. Instead we make use of Paterson's do-notation for arrows \cite{paterson_new_2001} which makes code more readable as it allows us to program with points.

To show how arrowized programming works, we implement a simple signal function, which calculates the acceleration of a falling mass on its vertical axis as an example \cite{perez_testing_2017}.

\begin{HaskellCode}
fallingMass :: Double -> Double -> SF () Double
fallingMass p0 v0 = proc _ -> do
  v <- arr (+v0) <<< integral -< (-9.8)
  p <- arr (+p0) <<< integral -< v
  returnA -< p
\end{HaskellCode}

To create an arrow, the \textit{proc} keyword is used, which binds a variable after which the \textit{do} of Patersons do-notation \cite{paterson_new_2001} follows. Using the signal function \textit{integral :: SF v v} of Yampa which integrates the input value over time using the rectangle rule, we calculate the current velocity and the position based on the initial position \textit{p0} and velocity \textit{v0}. The $<<<$ is one of the arrow combinators which composes two arrow computations and \textit{arr} simply lifts a pure function into an arrow. To pass an input to an arrow, \textit{-<} is used and \textit{<-} to bind the result of an arrow computation to a variable. Finally to return a value from an arrow, \textit{returnA} is used.

To hide the $\Delta t$ from the type-signature of a signal function Yampa makes use of Haskells strong type system. All signal functions SF are defined for t = 0 and then change their signature into SF' which actually gets passed $\Delta t$ as its first parameter. This happens transparently to the user of the library and is supported by not publicly exporting SF'.