\section{Encoding the agent specification}
\label{sec:method}
We start by encoding the invariants of the susceptible agent directly into Haskell, implementing a function which takes all necessary parameters and returns a \texttt{Bool} indicating whether the invariants hold or not. The encoding is straightforward when using pattern matching and it nearly reads like a formal specification due to the declarative nature of functional programming.

%\begin{HaskellCode}
\begin{footnotesize}
\begin{verbatim}
susceptibleProps :: SIREvent              -- random event sent to agent
                 -> SIRState              -- output state of the agent
                 -> [QueueItem SIREvent]  -- list of events the agent scheduled
                 -> AgentId               -- agent id of the agent
                 -> Bool
-- received Recover => stay Susceptible, no event scheduled
susceptibleProps Recover Susceptible es _ = null es
-- received Contact _ Recovered => stay Susceptible, no event scheduled
susceptibleProps (Contact _ Recovered) Susceptible es _ = null es
-- received Contact _ Susceptible => stay Susceptible, no event scheduled
susceptibleProps (Contact _ Susceptible) Susceptible es _  = null es
-- received Contact _ Infected, didn't get Infected, no event scheduled
susceptibleProps (Contact _ Infected) Susceptible es _ = null es
-- received Contact _ Infected AND got infected, check events
susceptibleProps (Contact _ Infected) Infected es ai
  = checkInfectedInvariants ai es
-- received MakeContact => stay Susceptible, check events
susceptibleProps MakeContact Susceptible es ai
  = checkMakeContactInvariants ai es cor
-- all other cases are invalid and result in a failed test case
susceptibleProps _ _ _ _ = False
\end{verbatim}
\end{footnotesize}
%\end{HaskellCode}

Next, we give the implementation for the \texttt{checkInfectedInvariants} function. We omit a detailed implementation of \texttt{checkMakeContactInvariants} as it works in a similar way and its details do not add anything conceptually new. The function \texttt{checkInfectedInvariants} encodes the invariants which have to hold when the susceptible agent receives a \texttt{(Contact \_ Infected)} event from an infected agent and becomes infected:

%\begin{HaskellCode}
\begin{footnotesize}
\begin{verbatim}
checkInfectedInvariants :: AgentId               -- agent id of the agent 
                        -> [QueueItem SIREvent]  -- list of scheduled events
                        -> Bool
checkInfectedInvariants sender 
  -- expect exactly one Recovery event
  [QueueItem receiver (Event Recover) t'] 
  -- receiver is sender (self) and scheduled into the future
  = sender == receiver && t' >= t 
-- all other cases are invalid
checkInfectedInvariants _ _ = False
\end{verbatim}
\end{footnotesize}
%\end{HaskellCode}

%The \texttt{checkMakeContactInvariants} is a bit more complex:
%
%%%\begin{HaskellCode}
%checkMakeContactInvariants :: AgentId              -- ^ Agent id of the agent 
%                           -> [QueueItem SIREvent] -- ^ Events the agent scheduled
%                           -> Int                  -- ^ Contact Rate
%                           -> Bool
%checkMakeContactInvariants sender es contactRate
%    -- make sure there has to be exactly one MakeContact event and
%    -- exactly contactRate Contact events
%    = invOK && hasMakeCont && numCont == contactRate
%  where
%    (invOK, hasMakeCont, numCont) 
%      = foldr checkMakeContactInvariantsAux (True, False, 0) es
%
%    checkMakeContactInvariantsAux :: QueueItem SIREvent 
%                                  -> (Bool, Bool, Int)
%                                  -> (Bool, Bool, Int)
%    checkMakeContactInvariantsAux 
%        (QueueItem (Contact sender' Susceptible) receiver t') (b, mkb, n)
%      = (b && sender == sender'   -- sender in Contact must be self
%           && receiver `elem` ais -- receiver of Contact must be in agent ids
%           && t == t', mkb, n+1)  -- Contact event is scheduled immediately
%    checkMakeContactInvariantsAux 
%        (QueueItem MakeContact receiver t') (b, mkb, n) 
%      = (b && receiver == sender  -- receiver of MakeContact is agent itself
%           && t' == t + 1         -- MakeContact scheduled 1 timeunit into future
%           &&  not mkb, True, n)  -- there can only be one MakeContact event
%    checkMakeContactInvariantsAux _ (_, _, _) 
%      = (False, False, 0)         -- other patterns are invalid
%%\end{HaskellCode}

\subsection{Writing a property test}
After having encoded the invariants into a function, we need to write a QuickCheck property test which calls this function with random test data. Although QuickCheck comes with a lot of data generators for existing types like Strings, Integers, Double, List, it obviously does not have generators for custom types, like the \texttt{SIRState} and \texttt{SIREvent}. Thus, the first step is to write custom data generators, which are suitable for our problem at hand.

There are two ways to do this by either fixing them at compile time writing an \texttt{Arbitrary} instance or writing a run-time generator running in the \texttt{Gen} context. The advantage of having an \texttt{Arbitrary} instance is that the custom data type can then be used as an argument to a function, the advantage of writing a run-time generator is that it can depend on values computed during run time.

First, we write an \texttt{Arbitrary} instance for the \texttt{SIRState}. When writing an \texttt{Arbitrary} instance one needs to provide an implementation for the \texttt{arbitrary} method, which returns a value of the given type for which the instance was implemented for - in our case it is \texttt{SIRState}. This implementation makes use of the \texttt{elements} function, which picks a random element from a non-empty list with uniform probability.

%\begin{HaskellCode}
\begin{footnotesize}
\begin{verbatim}
instance Arbitrary SIRState where
  arbitrary = elements [Susceptible, Infected, Recovered]
\end{verbatim}
\end{footnotesize}
%\end{HaskellCode}

Next we write a custom generator to capture the concept of probabilities. This is necessary because if using a \texttt{Double} as input to a function it would cover the whole range of the type but we want to restrict its random range to $(0,1)$ reflecting a probability. We achieve this by introducing a new type and writing an \texttt{Arbitrary} instance for it. This implementation makes use of the \texttt{choose} function, which returns a random value within the given bounds.

%\begin{HaskellCode}
\begin{footnotesize}
\begin{verbatim}
-- define a new type, representing a probability
newtype Probability = P Double
instance Arbitrary Probability where
  arbitrary = P <$> choose (0, 1)
\end{verbatim}
\end{footnotesize}
%\end{HaskellCode}
%$

What is left is to write a custom generator for generating random events. In this case we want to be able to control the distribution of events generated. Instead of using \texttt{elements}, which picks uniformly, if a skewed distribution is needed, one can use the \texttt{frequency} function, where a frequency can be specified for each element. Due to the fact, that event generation is parametrised by frequencies and requires the population ids to pick from in the \texttt{Contact} event, it needs to be a run-time generator, thus running in \texttt{Gen} context. The function takes the frequencies for all three types of events, the frequencies of the \texttt{SIRState} in the \texttt{Contact} event and the agent population to pick the sender id from in the \texttt{Contact} event.

%\begin{HaskellCode}
\begin{footnotesize}
\begin{verbatim}
genEventFreq :: Int             -- MakeContact frequency
             -> Int             -- Contact frequency
             -> Int             -- Recover frequency
             -> (Int, Int, Int) -- Susceptible, Infected, Recovered frequency
             -> [AgentId]       -- population agent ids
             -> Gen SIREvent
genEventFreq mf cf rf (s,i,r) ais
  -- generate one of the three events with given frequency
  = frequency [ (mf, return MakeContact)
              , (cf, do
                  -- pick SIRState for Contact event
                  ss <- frequency [ (s, return Susceptible)
                                  , (i, return Infected)
                                  , (r, return Recovered)]
                  -- pick sender id for Contact event
                  ai <- elements ais
                  return (Contact ai ss))
              , (rf, return Recover)]
         
-- helper function with uniform frequencies for all events and SIRState
genEvent :: [AgentId] -> Gen SIREvent
genEvent = genEventFreq 1 1 1 (1,1,1) 
\end{verbatim}
\end{footnotesize}
%\end{HaskellCode}

We are now equipped with all functionality to implement the property test. All parameters to the property test are generated randomly, which expresses that the properties encoded in the previous section have to hold invariant of the model parameters. We make use of additional data generator modifiers: \texttt{Positive} ensures that a value generated is positive; \texttt{NonEmptyList} ensures that a randomly generated list is not empty. Further, we use the function \texttt{label}, which can be used to generate labels for a test case, counting occurrences of same labels, generating a histogram. This is useful to get an insight into the distribution of the property test data and we use it to get an understanding for the distribution of the transitions. The case where the agents output state is \texttt{Recovered} is marked as "INVALID" as it must never occur, otherwise the test will fail, due to the invariants encoded in the previous section.

%\begin{HaskellCode}
\begin{footnotesize}
\begin{verbatim}
prop_susceptible :: Positive Int         -- beta (contact rate)
                 -> Probability          -- gamma (infectivity)
                 -> Positive Double      -- delta (illness duration)
                 -> Positive Double      -- current simulation time
                 -> NonEmptyList AgentId -- population agent ids
                 -> Gen Bool
prop_susceptible 
  (Positive beta) (P gamma) (Positive delta) (Positive t) (NonEmpty ais) = do
  -- generate random event, requires the population agent ids
  evt <- genEvent ais
  -- run susceptible random agent with given parameters (implementation omitted)
  (ai, ao, es) <- genRunSusceptibleAgent beta gamma delta t ais evt
  -- check properties
  return (label (labelTestCase ao) (susceptibleProps evt ao es ai))
  where
    labelTestCase :: SIRState -> String
    labelTestCase Infected    = "Susceptible -> Infected"
    labelTestCase Susceptible = "Susceptible"
    labelTestCase Recovered   = "INVALID"
\end{verbatim}
\end{footnotesize}
%\end{HaskellCode}

We have omitted the implementation of \texttt{genRunSusceptibleAgent} as it would require the discussion of implementation details of the agent. Conceptually speaking, it executes the agent with the respective arguments with a fresh random-number generator and returns the agent id, its state and scheduled events it has output.

Finally we can run the test using QuickCheck. Due to the large random sampling space with 5 parameters, we increase the number of test cases to generate to 100,000.

\begin{footnotesize}
\begin{verbatim}
> quickCheckWith (stdArgs {maxSuccess=100000}) prop_susceptible
+++ OK, passed 100000 tests (6.77s):
94.522% Susceptible
 5.478% Susceptible -> Infected
\end{verbatim}
\end{footnotesize}

All 100,000 test cases pass, taking 6.7 seconds to run on our machine. The distribution of the transitions shows that we indeed cover both cases a susceptible agent can exhibit within one event. It either stays susceptible or makes the transition to infection. The fact that there is no transition to recovered shows that the implementation is correct - for a transition to recovered we would need to send an additional, second event to the agent.

Encoding of the invariants and writing property tests for the infected agents follows the same idea and is not repeated here. Next, we show how to test transition probabilities using the powerful statistical hypothesis testing feature of QuickCheck.

\subsection{Encoding transition probabilities}
In the specifications from the previous section there are probabilistic state transitions, for example the susceptible agent \textit{might} become infected, depending on the events it receives and the infectivity ($\gamma$) parameter. We look now into how we can encode these probabilistic properties using the powerful \texttt{cover} feature of QuickCheck.

The function \texttt{cover :: Double $\rightarrow$ Bool $\rightarrow$ String $\rightarrow$ prop $\rightarrow$ Property} allows to explicitly specify that a given percentage of successful test cases belong to a given class. The first argument is the expected percentage; the second argument indicates whether the current test case belongs to the class or not; the third argument is a label for the coverage; the fourth argument is the property which needs to hold for the test case to succeed.

An example would be to use it to check whether at least 15\% of all test cases of the \texttt{prop\_reverse\_reverse} property of section \ref{sec:proptesting} has a length of at least 50. This would be encoded in the following way:

% stack ghci --package QuickCheck
%\begin{HaskellCode}
\begin{footnotesize}
\begin{verbatim}
prop_reverse_reverse_cover :: [Int] -> Property
prop_reverse_reverse_cover xs  =  
  cover 15 (length xs >= 50) "length at least 50" (reverse (reverse xs) == xs)
\end{verbatim}
\end{footnotesize}
%\end{HaskellCode}

%When running it, the output indicates that although all tests passes, as the property is satisfied in all 100 test cases but the expected coverage differs from the actual one.
%
%\begin{footnotesize}
%\begin{verbatim}
%> quickCheck prop_reverse_reverse_cover
%+++ OK, passed 100 tests (0.01s)
%Only 9% length of list at least 50, but expected 15%
%\end{verbatim}
%\end{footnotesize}

For our case we follow a slightly different approach: we include all test cases into the expected coverage, setting the second parameter always to \texttt{True} and we also set the last argument to \texttt{True} as we are only interested in testing the coverage, which is in fact the property we want to test. Implementing this property test is then simply a matter of computing the probabilities and of case analysis over the random input event and the agents output. It is important to note that in this property test we cannot randomise the model parameters $\beta$, $\gamma$ and $\delta$ because this would lead to random coverage. This might seem like a disadvantage but we do not really have a choice here, still the fixed model parameters can be adjusted arbitrarily and the property must still hold. We could have combined this test into the previous one but then we couldn't have used randomised model parameters. For this reason, and to keep the concerns separated we opted for two different tests, which makes them also much more readable. 

%\begin{HaskellCode}
\begin{footnotesize}
\begin{verbatim}
prop_susceptible_prob :: Positive Double       -- current simulation time
                      -> NonEmptyList AgentId  -- population agent ids 
                      -> Property
prop_susceptible_prob (Positive t) (NonEmpty ais) = do
  -- fixed model parameters, otherwise random coverage
  let cor = 5     -- contact rate (beta)
      inf = 0.05  -- infectivity (gamma)
      ild = 15.0  -- illness duration (delta)
  -- compute distributions for all cases depending on event and SIRState
  -- frequencies; technical detail, omitted for clarity reasons
  let recoverPerc       = ...
      makeContPerc      = ...
      contactRecPerc    = ...
      contactSusPerc    = ...
      contactInfSusPerc = ...
      contactInfInfPerc = ...
  -- generate a random event
  evt <- genEvent ais
  -- run susceptible random agent with given parameters, only
  -- interested in its output SIRState, ignore id and events
  (_, ao, _) <- genRunSusceptibleAgent cor inf ild t ais evt
  -- encode expected distributions
  return $ property $ case evt of 
    Recover -> 
      cover recoverPerc True "Susceptible recv Recover" True
    MakeContact -> 
      cover makeContPerc True "Susceptible recv MakeContact" True
    (Contact _ Recovered) -> 
      cover contactRecPerc True "Susceptible recv Contact * Recovered" True
    (Contact _ Susceptible) -> 
      cover contactSusPerc True "Susceptible recv Contact * Susceptible" True
    (Contact _ Infected) -> 
      case ao of
        Susceptible ->
          cover contactInfSusPerc True 
            "Susceptible recv Contact * Infected, stays Susceptible" True
        Infected ->
          cover contactInfInfPerc True 
            "Susceptible recv Contact * Infected, becomes Infected" True
        _ ->
          cover 0 True "INVALID" True
\end{verbatim}
\end{footnotesize}
%\end{HaskellCode}

We have omitted the details of computing the respective distributions of the cases, which depend on the frequencies of the events and the occurrences of \texttt{SIRState} within the \texttt{Contact} event. By varying different distributions in the \texttt{genEvent} function, we can change the distribution of the test cases, leading to a more general test than just using uniform distributed events. When running the property test we get the following output:

\begin{footnotesize}
\begin{verbatim}
+++ OK, passed 100 tests (0.01s):
40% Susceptible recv MakeContact
25% Susceptible recv Recover
14% Susceptible recv Contact * Infected, stays Susceptible
12% Susceptible recv Contact * Susceptible
 9% Susceptible recv Contact * Recovered
    
Only 9% Susceptible recv Contact * Recovered, but expected 11%
Only 25% Susceptible recv Recover, but expected 33%
\end{verbatim}
\end{footnotesize}

QuickCheck runs 100 test cases, prints the distribution of our labels and issues warnings in the last two lines that generated and expected coverages differ in these cases. Further, not all cases are covered, for example the contact with an infected and becoming infected. The reason for these issues is insufficient testing coverage as 100 test cases are simply not enough for a statistically robust result. We could increase the number of test cases to run to 100,000 which will cover then all cases but still QuickCheck is not satisfied as the expected and generated coverage differs in some fractions. % The usage pattern of \texttt{cover}, where we unconditionally include the test case into the coverage class so all test cases pass, emulates failure. 

Fortunately, as a remedy QuickCheck provides the function \texttt{checkCoverage :: prop $\rightarrow$ Property} which solves this problem. When \texttt{checkCoverage} is added to the top of the function before the \texttt{do}, QuickCheck will run as many test cases necessary until it is clear whether the percentage in \texttt{cover} was reached or cannot be reached at all. The way QuickCheck does it, is by using sequential statistical hypothesis testing \cite{wald_sequential_1992}. Thus, if QuickCheck comes to the conclusion that the given percentage can or cannot be reached, it is based on a robust statistical test giving strong confidence in the result. With the usage of \texttt{checkCoverage} we get the following output:

\begin{footnotesize}
\begin{verbatim}
+++ OK, passed 819200 tests (7.32s):
33.3292% Susceptible recv Recover
33.2697% Susceptible recv MakeContact
11.1921% Susceptible recv Contact * Susceptible
11.1213% Susceptible recv Contact * Recovered
10.5356% Susceptible recv Contact * Infected, stays Susceptible
 0.5520% Susceptible recv Contact * Infected, becomes Infected
\end{verbatim}
\end{footnotesize}

After 819,200 (!) test cases, run in 7.32 seconds on our machine, QuickCheck comes to the statistically robust conclusion that the distributions generated by the test cases reflect the expected distributions and passes the property test.