% interactapasample.tex
% v1.05 - August 2017

\documentclass[]{interact}

\usepackage{epstopdf}% To incorporate .eps illustrations using PDFLaTeX, etc.
\usepackage[caption=false]{subfig}% Support for small, `sub' figures and tables
%\usepackage[nolists,tablesfirst]{endfloat}% To `separate' figures and tables from text if required
%\usepackage[doublespacing]{setspace}% To produce a `double spaced' document if required
%\setlength\parindent{24pt}% To increase paragraph indentation when line spacing is doubled

\usepackage[longnamesfirst,sort]{natbib}% Citation support using natbib.sty
\bibpunct[, ]{(}{)}{;}{a}{,}{,}% Citation support using natbib.sty
\renewcommand\bibfont{\fontsize{10}{12}\selectfont}% To set the list of references in 10 point font using natbib.sty

%\usepackage[natbibapa,nodoi]{apacite}% Citation support using apacite.sty. Commands using natbib.sty MUST be deactivated first!
%\setlength\bibhang{12pt}% To set the indentation in the list of references using apacite.sty. Commands using natbib.sty MUST be deactivated first!
%\renewcommand\bibliographytypesize{\fontsize{10}{12}\selectfont}% To set the list of references in 10 point font using apacite.sty. Commands using natbib.sty MUST be deactivated first!

\theoremstyle{plain}% Theorem-like structures provided by amsthm.sty
\newtheorem{theorem}{Theorem}[section]
\newtheorem{lemma}[theorem]{Lemma}
\newtheorem{corollary}[theorem]{Corollary}
\newtheorem{proposition}[theorem]{Proposition}

\theoremstyle{definition}
\newtheorem{definition}[theorem]{Definition}
\newtheorem{example}[theorem]{Example}

\theoremstyle{remark}
\newtheorem{remark}{Remark}
\newtheorem{notation}{Notation}

\begin{document}

\articletype{ARTICLE TEMPLATE}% Specify the article type or omit as appropriate

% 1. Author details. All authors of a manuscript should include their full name and affiliation on the cover page of the manuscript. Where available, please also include ORCiDs and social media handles (Facebook, Twitter or LinkedIn). One author will need to be identified as the corresponding author, with their email address normally displayed in the article PDF (depending on the journal) and the online article. Authors’ affiliations are the affiliations where the research was conducted. If any of the named co-authors moves affiliation during the peer-review process, the new affiliation can be given as a footnote. Please note that no changes to affiliation can be made after your paper is accepted. Read more on authorship.
\title{Specification Testing of Agent-Based Simulation using Property-Based Testing.}

\author{
\name{Jonathan Thaler \textsuperscript{a}\thanks{CONTACT Jonathan Thaler. Email: jonathan.thaler@nottingham.ac.uk} and Peer-Olaf Siebers\textsuperscript{a}}
\affil{\textsuperscript{a}School Of Computer Science, University of Nottingham, 7301 Wollaton Rd, Nottingham, UK;}
}



\maketitle

% 2. Should contain an unstructured abstract of 200 words. 
% 3. You can opt to include a video abstract with your article. Find out how these can help your work reach a wider audience, and what to think about when filming.
\begin{abstract}
This paper shows how to apply random property-based testing on a technical level to encode and test specifications of agent-based simulations. It is shown that as opposed to unit testing, random property-based testing is a natural match to test stochastic agent-based simulation and is able to encode full agent and model specifications including probabilities. The outcome are specifications expressed directly in code, which relate whole classes of random input to expected classes of output. During test execution, random test data is generated automatically, potentially covering the equivalent of thousands of unit tests. The expressiveness and power of property-based testing is not limited to be part of a test driven development process where they act as specifications, verification and regression tests but can be integrated as a fundamental part of the model development process, supporting the hypothesis and discovery process, where the process of specifying the model is actually already the implementation process. This should result in high confidence that the model at hand is very likely to be correct, something of fundamental importance in ABS.
\end{abstract}

% 4. Between 3 and 6 keywords. Read making your article more discoverable, including information on choosing a title and search engine optimization.
\begin{keywords}
Testing; Test Driven Development;
\end{keywords}

% 5 Funding details - not required for this paper

% 6. Disclosure statement. This is to acknowledge any financial interest or benefit that has arisen from the direct applications of your research. - not required in this paper

% 7. Supplemental online material. Supplemental material can be a video, dataset, fileset, sound file or anything which supports (and is pertinent to) your paper. We publish supplemental material online via Figshare. 
% TODO

% 8. Figures. Figures should be high quality (1200 dpi for line art, 600 dpi for grayscale and 300 dpi for colour, at the correct size). Figures should be supplied in one of our preferred file formats: EPS, PS, JPEG, GIF, or Microsoft Word (DOC or DOCX). For information relating to other file types, please consult our Submission of electronic artwork document.

% 9. Tables. Tables should present new information rather than duplicating what is in the text. Readers should be able to interpret the table without reference to the text. Please supply editable files.

% 10. Equations. If you are submitting your manuscript as a Word document, please ensure that equations are editable. 

% 11. Units. Please use SI units (non-italicized).

% WORD LIMITS
% Please include a word count for your paper. A typical article for this journal should be no more than 6000 words; this limit does not include the abstract, endnotes, tables, figures, figure captions and legends, or references. 

\section{Introduction}
 aim of the paper is to build on the rather conceptual paper by me at summersim19 and show property-based testing in ABS on a much more technical level by the case studies of:
	- encoding agent-specifications of time- and event-driven ABS SIR into property-based tests
	- random event sampling
	- compare two different implementations
 	- and encode model invariants 
 	
hypothesise that a strong reason for why testing in ABS is not very widely used and adopted is that unit testing is not able to deal very well with the stochastic nature of ABS in general. random property-based testing is a remedy to that problem as it allows to relate whole classes of inputs to specific classes of output for which then randomised test cases are automatically generated, covering potentially thousands of unit tests.

TODO: it would be great if i can show how property-based testing found a bug in an implementation

\end{document}
