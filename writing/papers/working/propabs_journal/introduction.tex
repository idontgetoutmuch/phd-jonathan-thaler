\section{Introduction}
\label{sec:introduction}
TODO: don't use citations as part of sentences. This is bad style. Use the name of the author?

% MAKE MORE CLEAR WHAT THE CONTRIBUTION OF THIS PAPER IS WRT TO MY SummerSim Paper:
%— After reading the introduction, I was left wondering about two things:
%        — Is this incremental work? How much more is done wrt. to the previous paper?
%        — What are the main contributions?

Since its inception in the early 1990s \cite{epstein_growing_1996,siebers_introduction_2008,wooldridge_introduction_2009}, Agent-Based Simulation (ABS) as a third way of doing science \cite{axelrod_advancing_1997,axelrod_guide_2006} has matured substantially and has found its way into the mainstream of science \cite{macal_everything_2016}. Further, a number of ABS frameworks and tools like RePast, AnyLogic and NetLogo as well as open databases of ABS models %(TODO: https://www.comses.net/codebases/)
have been developed, allowing for quick and robust prototyping and development of models.  

However, despite the broad acceptance and adoption of ABS as methodology and \textit{generative} way of doing science, there have been struggles as described in \cite{axelrod_chapter_2006}, where the author reports the vulnerability of ABS to misunderstanding. Due to informal specifications of models and change requests amongst members of a research team, bugs are very likely to be introduced. He also reported how difficult it was to reproduce the work of \cite{axelrod_convergence_1995}, which took the team four months, due to inconsistencies between the original code and the published paper. The consequence is that counter-intuitive simulation results can lead to weeks of checking whether the code matches the model and is bug-free as reported in \cite{axelrod_advancing_1997}.

The same problem was reported in \cite{ionescu_dependently-typed_2012}, which tried to reproduce the work of Gintis \cite{gintis_emergence_2006}. In his work, Gintis claimed to have found a mechanism in bilateral decentralized exchange, which resulted in Walrasian General Equilibrium without the neo-classical approach of a tatonement process through a central auctioneer \cite{colell_microeconomic_1995}. This was a major breakthrough for economics as the theory of Walrasian General Equilibrium is non-constructive. It only postulates the properties and existence of the equilibrium but does not explain the process and dynamics through which this equilibrium can be reached or constructed. Gintis seemed to have found this very process.

The authors \cite{ionescu_dependently-typed_2012} failed to reproduce the results and were only able to solve the problem by directly contacting Gintis, which provided the code, the definitive formal reference %\footnote{It seems that at this point Gintis has made his code written in Object Pascal publicly available through his \href{https://people.umass.edu/gintis/}{website}~\cite{gintis_herbert_website}.}.
It was found that there was a bug in the code leading to unexpected results, which were seriously damaged through this error. They also reported ambiguity between the informal model description in Gintis' paper and the actual implementation.
This discovery lead to research in a functional framework for agent-based models of exchange as described in \cite{botta_functional_2011}, which tried to give a very formal functional specification of the model, coming very close to an implementation in Haskell. The failure of Gintis was investigated in more depth in the thesis by \cite{evensen_extensible_2010} who got access to Gintis' code of \cite{gintis_emergence_2006}. They found that the code did not follow good object-oriented design principles (all of it was public, code duplication) and - in accordance with \cite{ionescu_dependently-typed_2012} - discovered a number of bugs serious enough to invalidate the results.

However, due to the fact that ABS is primarily used for scientific research, producing often break-through scientific results, besides on converging both on standards for testing the robustness of implementations and on its tools, ABS more importantly needs to be \textit{free of bugs}, \textit{verified against their specification}, \textit{validated against hypotheses} and ultimately be \textit{reproducible} \cite{axelrod_chapter_2006}. Further, a special issue with ABS is that the emergent behaviour of the system is generally not known in advance and researchers look for some \textit{unique} emergent pattern in the dynamics. Whether the emergent pattern is then truly due to the system working correctly, or a bug in disguise is often not obvious and becomes increasingly difficult to assess with increasing system complexity. 

This is supported by the paper \cite{hammer_tongs_north_2018} which summarises various ABS development methods where it puts fundamental emphasis on the Verification and Validation (V\&V) processes for ABS. Although there exist methods and research of V\&V in ABS, unfortunately, as section \ref{sec:related} shows, there does not exist much research on the issue of testing and verifying the actual code of an ABS implementation. In Software Engineering, this task has been traditionally achieved by unit testing, as introduced by K. Beck in the seminal work on Test Driven Development (TDD) \cite{beck_test_2002}. Unit tests are themselves code pieces which test a given unit of functionality of some given feature. Generally, this results in hundreds or sometimes thousands of unit tests as all paths through the software should be covered.

We hypothesise that the reason why unit testing is strongly neglected in the field of ABS V\&V research is a conceptual mismatch between unit testings' deterministic and ABS rather stochastic nature. The fact that a unit test needs to be written for each edge case does not scale up to the stochastic nature of ABS, where the agent and model behaviour in general is often characterised by probabilistic distributions instead of deterministic rules.

Our work \cite{thaler_show_2019} was the first to propose \textit{property-based testing} as an alternative to unit testing for code testing ABS implementations. The main idea of property-based testing is to express model specifications and invariants directly in code and test them through \textit{automated} and \textit{randomised} test data generation.

This paper picks up this conceptual work \cite{thaler_show_2019}, puts it into a much more technical perspective and demonstrates additional techniques of property-based testing in the context of ABS, which was not covered in the conceptual paper. More specifically, this paper shows how to encode agent specifications into property tests, using an agent-based SIR model inspired by the work \cite{macal_agent-based_2010} as use case. Following an event-driven approach it is shown how to express an agent specification in code by relating random input events to specific output events. Further, showing the use of specific property-based testing features which allow expressing expected coverage of data distributions, it is shown how transition probabilities can be tested. By doing this, this paper demonstrates how property-based testing works on a technical level, how specifications can be put into code and how probabilities can be expressed and tested using statistically robust verification. This underlines the conclusion of \cite{thaler_show_2019}, that property-based testing maps naturally to ABS. Further, this work shows that in the context of ABS, property-based testing is strictly more powerful than unit testing as it allows to run thousands of test cases automatically instead of constructing each manually and because it is able to encode probabilities, something unit testing is not capable of in general.

The paper is structured as follows: section \ref{sec:related} presents related work. In section \ref{sec:proptesting} property-based testing is introduced on a technical level. In section \ref{sec:sirmodel} the agent-based SIR model is introduced, together with its informal event-driven specification. Section \ref{sec:method} contains the main contribution of the paper, where it is shown how to encode agent specifications and transition probabilities with property-based testing. Section \ref{sec:enc_model_inv} shows how to encode model invariants with property-based testing. In section \ref{sec:discussion} the approach is discussed and related to the work of \cite{thaler_show_2019} and other use cases. Finally, section \ref{sec:conclusions} concludes and points out further research.