\section{Conclusions}
hypothesise that a strong reason for why testing in ABS is not very widely used and adopted is that unit testing is not able to deal very well with the stochastic nature of ABS in general. random property-based testing is a remedy to that problem as it allows to relate whole classes of inputs to specific classes of output for which then randomised test cases are automatically generated, covering potentially thousands of unit tests.

benefits:
- express specifications rather than individual test cases which makes it much more general than unit testing
- expressing probabilities of various types (hypotheses, transitions, outputs) and perform statistical robust testing by sequential hypothesis testing
- relates whole classes of inputs to whole classes of outputs, automatically generating thousands of tests if necessary, 

drawbacks:
- coverage but smallcheck could be a remedy
As a remedy for the potential coverage problems of QuickCheck, there exists also a deterministic property-testing library called SmallCheck \cite{runciman_smallcheck_2008} which instead of randomly sampling the test space, enumerates test cases exhaustively up to some depth