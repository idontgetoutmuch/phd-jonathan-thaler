\section{Introduction}
When implementing an ABS it is of fundamental importance that the implementation is correct up to some specification and that this specification matches the real world in some way. This process is called verification and validation (V\&V), where \textit{validation} is the process of ensuring that a model or specification is sufficiently accurate for the purpose at hand whereas \textit{verification} is the process of ensuring that the model design has been transformed into a computer model with sufficient accuracy \cite{robinson_simulation:_2014}. In other words, validation determines if we are we building the \textit{right model}, and verification if we are building the \textit{model right} up to some specification \cite{balci_verification_1998}.

The work \cite{collier_test-driven_2013} was the first to discuss how to do verification of an ABS implementation, using unit testing with the RePast Framework \cite{north_complex_2013}, to verify the correctness of the implementation up to a certain level. Unit testing is a technique where additional tests are written in code by constructing individual test cases, to test a specific unit of the implementation. A different approach to testing an implementation of an ABS was investigated by the rather conceptual paper of \cite{thaler_show_2019}. In this work the authors introduced \textit{property-based testing} to ABS and showed that it allows to do both verification and validation of an implementation, on the code level. The main idea of property-based testing is to express model specifications and laws directly in code and test them through \textit{automated} and \textit{randomised} test data generation. The authors showed that due to ABS' \textit{stochastic}, \textit{exploratory}, \textit{generative} and \textit{constructive} nature, property-based testing is a much more natural fit for testing both explanatory and exploratory ABS than unit testing. Property-based testing has its origins in Haskell \cite{claessen_quickcheck_2000,claessen_testing_2002}, where it was first conceived and implemented and has been successfully used for testing Haskell code in the industry for years \cite{hughes_quickcheck_2007}.

This paper picks up the conceptual work of \cite{thaler_show_2019}, puts it into a much more technical perspective and demonstrates additional techniques of property-based testing in the context of ABS which was not covered in the conceptual paper.

- full agent specification of explanatory SIR model inspired by macal
- showing how to encode transitions and probabilities and test them using statistical robust verification, random event sampling