\section{Discussion}
\label{sec:discussion}
We have shown how to encode the \textit{informal} specification of the susceptible agent behaviour into a \textit{formal} one through the use of functions and property tests. We have omitted tests for the infected agents as they follow conceptually the same patterns. The testing of transitions of the infected agents work slightly different though as they follow an exponential distribution but are encoded in an equal fashion as demonstrated with the susceptible agent. The case for the recovered agent is a bit more subtle, due to its behaviour: it simply stays \texttt{Recovered} \textit{forever}. This is a property which cannot be tested in a reasonable way with neither property-based nor unit testing as it is simply not computable. Only a white box test in the case of a code review would reveal whether the implementation is actually correct or not.
 
Statistical sequential hypothesis testing can also be applied to the example of hypothesis testing in exploratory models in the paper of \cite{thaler_show_2019}. In their case the expected coverage is encoded as in our case but instead of passing all tests unconditionally, a property is checked and the outcome is passed as the last argument to \texttt{cover}.

Another useful application for \texttt{cover} and \texttt{checkCoverage} are checking whether two different implementations of the same model result in the same distributions \textit{under random model parameters}. Another use case is encoding model invariants of the distribution generated, for example in the case of the SIR model, the number of agents stays always constant, the number of susceptible is monotonic decreasing and the number of recovered monotonic decreasing \textit{under random model parameters}.