\section{Property-based testing}
\label{sec:proptesting}
Property-based testing allows to formulate \textit{functional specifications} in code which then a property-based testing library tries to falsify by \textit{automatically} generating test data, covering as much cases as possible. When a case is found for which the property fails, the library then reduces the test data to its simplest form for which the test still fails, for example shrinking a list to a smaller size. It is clear to see that this kind of testing is especially suited to ABS, because we can formulate specifications, meaning we describe \textit{what} to test instead of \textit{how} to test. Also the deductive nature of falsification in property-based testing suits very well the constructive and exploratory nature of ABS. Further, the automatic test generation can make testing of large scenarios in ABS feasible because it does not require the programmer to specify all test cases by hand, as is required in traditional unit tests.

Property-based testing was introduced in \cite{claessen_quickcheck_2000,claessen_testing_2002} where the authors present the QuickCheck library in Haskell, which tries to falsify the specifications by \textit{randomly} sampling the test space. According to the authors of QuickCheck \textit{"The major limitation is that there is no measurement of test coverage."} \cite{claessen_quickcheck_2000}. Although QuickCheck provides help to report the distribution of test cases it is not able to measure the coverage of tests in general. This could lead to the case that test cases which would fail are never tested because of the stochastic nature of QuickCheck. Fortunately, the library provides mechanisms for the developer to measure coverage in specific test cases where the data and its expected distribution is known to the developer. This is a powerful tool for testing randomness in ABS as will be shown in the next chapters.

\medskip

To give a good understanding of how property-based testing works with \\ QuickCheck, we give a few examples of property tests on lists, which are directly expressed as functions in Haskell. Such a function has to return a \texttt{Bool} which indicates \texttt{True} in case the test succeeds or \texttt{False} if not and can take input arguments which data is automatically generated by QuickCheck.

\begin{HaskellCode}
-- append operator (++) is associative
append_associative :: [Int] -> [Int] -> [Int] -> Bool
append_associative xs ys zs = (xs ++ ys) ++ zs == xs ++ (ys ++ zs)

-- The reverse of a reversed list is the original list
reverse_reverse :: [Int] -> Bool
reverse_reverse xs = reverse (reverse xs) == xs

-- reverse is distributive over append (++)
-- This test fails for explanatory reasons, for a correct 
-- property xs and ys need to be swapped on the right-hand side!
reverse_distributive :: [Int] -> [Int] -> Bool
reverse_distributive xs ys = reverse (xs ++ ys) == reverse xs ++ reverse ys

-- running the tests
main :: IO ()
main = do
  quickCheck append_associative
  quickCheck reverse_reverse
  quickCheck reverse_distributive
\end{HaskellCode}

When we run the tests using \textit{main}, we get the following output:

\begin{verbatim}
+++ OK, passed 100 tests.
+++ OK, passed 100 tests.
*** Failed! Falsifiable (after 5 tests and 6 shrinks):    
[0]
[1]
\end{verbatim}

We see that QuickCheck generates 100 test cases for each property test and it does this by generating random data for the input arguments. We have not specified any data for our input arguments because QuickCheck is able to provide a suitable data generator through type inference. For lists and all the existing Haskell types there exist custom data generators already. We have to use a monomorphic list, in our case \texttt{Int}, and cannot use polymorphic lists because QuickCheck would not know how to generate data for a polymorphic type. Still, by appealing to genericity and polymorphism, we get the guarantee that the test case is the same for all types of a lists.

QuickCheck generates 100 test cases by default and requires all of them to pass. If there is a test case which fails, the overall property test fails and QuickCheck shrinks the input to a minimal size, which still fails and reports it as a counter example. This is the case in the last property test \texttt{reverse\_distributive} which is wrong as \textit{xs} and \textit{ys} need to be swapped on the right-hand side. In this run, QuickCheck found a counter example to the property after 5 tests and applied 6 shrinks to find the minimal failing example of \texttt{xs = [0]} and \texttt{ys = [1]}. If we swap \texttt{xs} and \texttt{ys}, the property test passes 100 test cases just like the other two did. It is possible to configure QuickCheck to generate more or less random test cases, which can be used to increase the coverage if the sampling space is quite large - this will become useful later.