\section{Related Work}
\label{sec:related}
Research on code testing of ABS is quite new with few publications so far. Our own work \cite{thaler_show_2019} is the first paper to introduce property-based testing to ABS. In it we show on a conceptual level that property-based testing allows to do both verification and validation of an implementation. However, we do not go into technical details of actual implementations nor how to use property-based testing on a technical level.

The use of unit testing in the context of ABS was first discussed by Collier et al. \cite{collier_test-driven_2013}. The authors introduce Test-Driven Development to ABS and use RePast to show how to verify the correctness of an implementation with unit tests. A similar approach has been discussed for Discrete Event Simulation in the AnyLogic software toolkit \cite{asta_investigation_2014}. 

Unit tests to verify an ABS implementation of maritime search operations was mentioned in \cite{onggo_test-driven_2016}. The authors validate their model against an analytical solution from theory by running the simulation with unit tests and then performing a statistical comparison against the formal specification.

Property-based testing has also connections to data-generators \cite{gurcan_generic_2013} and load generators and random testing \cite{burnstein_practical_2010} with the important benefit that property-based testing allows to express them directly in code.

The authors of \cite{gurcan_generic_2013} provide a case study of an agent-based simulation of synaptic connectivity, for demonstrating their generic testing framework in RePast and MASON, which rely on JUnit to run automated tests.