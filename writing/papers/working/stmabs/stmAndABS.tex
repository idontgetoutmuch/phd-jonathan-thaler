\section{STM in ABS}
\label{sec:stm_abs}
In this section we give a short overview of how to apply STM in ABS. We fundamentally follow a time-driven approach in both case studies where the simulation is advanced by some given $\Delta t$ and in each step all agents are executed. To employ parallelism, each agent runs within its own thread and agents are executed in lock-step, synchronising between each $\Delta t$ which is controlled by the main thread. This way of stepping the simulation is introduced in \cite{thaler_art_2017} on a conceptual level, where the authors name it \textit{concurrent update-strategy}. See Figure \ref{fig:stm_abs_structure} for a visualisation of our concurrent, time-driven lock-step approach. 

An agent thread will block until the main thread sends the next $\Delta t$ and runs the \texttt{STM} action atomically with the given $\Delta t$. When the \texttt{STM} action has been committed, the thread will send the output of the agent action to the main thread to signal it has finished. The main thread awaits the results of all agents to collect them for output of the current step e.g. visualisation or writing to a file.

As will be described in subsequent sections, central to both case studies is an environment which is shared between the agents using a \texttt{TVar} or \texttt{TArray} primitive through which the agents communicate concurrently with each other. To get the environment in each step for visualisation purposes, the main thread can access the \texttt{TVar} and \texttt{TArray} as well. 

\subsection{Adding STM to agents}
A detailed discussion of how to add STM to agents on a technical level is beyond the focus of this paper as it would require to give an in-depth technical explanation of how our agents are actually implemented \cite{thaler_pure_2018}.

However, the concepts are similar to the example in Section \ref{sub:stm_example}. The agent behaviour is an \texttt{STM} action and has access to the environment either through a \texttt{TVar} or \texttt{TArray} and performs read and write operations directly on it. Each agent itself is run within its own thread, and synchronises with the main thread. Thus, it takes Haskells \texttt{MVar} synchronisation primitives to synchronise with the main thread and simply runs the \texttt{STM} agent behaviour each time it receives the next tick \texttt{DTime}:

%\begin{HaskellCode}
\begin{footnotesize}
\begin{verbatim}
agentThread :: RandomGen g 
            => Int             -- Number of steps to compute
            -> SIRAgent g      -- Agent behaviour 
            -> g               -- Random-number generator of the agent
            -> MVar SIRState   -- Synchronisation back to main thread
            -> MVar DTime      -- Receiving DTime for next tick
            -> IO ()
agentThread 0 _ _ _ _ = return () -- all steps computed, terminate thread
agentThread n agent rng retVar dtVar = do
  -- wait for dt to compute current step
  dt <- takeMVar dtVar
  -- compute output of current step
  let agentSTMAction = runAgent agent
  -- run the agents STM action atomically within IO
  ((ret, agent'), rng') <- atomically agentSTMAction 
  -- post result to main thread
  putMVar retVar ret
  -- tail recursion to next step 
  agentThread (n - 1) agent' rng retVar dtVar
\end{verbatim}
\end{footnotesize}
%\end{HaskellCode}

Computing a simulation step is quite trivial within the main thread. All agent threads \texttt{MVars} are signalled to unblock followed by an immediate block on the \texttt{MVars} into which the agent threads post back their result. The state of the current step is then extracted from the environment, which is stored within the \texttt{TVar} which the agent threads have updated:

%\begin{HaskellCode}
\begin{footnotesize}
\begin{verbatim}
simulationStep :: TVar SIREnv     -- environment 
               -> [MVar DTime]    -- sync dt to threads
               -> [MVar SIRState] -- sync output from threads
               -> DTime           -- time delta
               -> IO SIREnv
simulationStep env dtVars retVars dt = do
  -- tell all threads to compute next tick with the corresponding DTime
  mapM_ (`putMVar` dt) dtVars
  -- wait for results but ignore them, SIREnv contains all states
  mapM_ takeMVar retVars
  -- return state of environment when step has finished
  readTVarIO env
\end{verbatim}
\end{footnotesize}
%\end{HaskellCode}