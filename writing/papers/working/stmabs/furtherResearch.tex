\section{Further Research} % (Lived happily ever after...)
\label{sec:further}

So far we only implemented a tiny bit of the Sugarscape model and left out the later chapters which are more involved as they incorporate direct synchronous communication between agents. Such mechanisms are very difficult to approach in GPU based approaches \cite{lysenko_framework_2008} but should be quite straightforward in STM using \textit{TChan} and retries. However, we have yet to prove how to implement reliable synchronous agent interactions without deadlocks in STM. It might be very well the case that a truly concurrent approach is doomed due to the following \cite{marlow_parallel_2013} (Chapter 10. Software Transactional Memory, \textit{What Can We Not Do with STM?}): \textit{"In general, the class of operations that STM cannot express are those that involve multi-way communication between threads. The simplest example is a synchronous channel, in which both the reader and the writer must be present simultaneously for the operation to go ahead. We cannot implement this in STM, at least compositionally [..]: the operations need to block and have a visible effect — advertise that there is a blocked thread — simultaneously."}. 

A drawback of STM is that it is not fair because \textit{all} threads, which block on a transactional primitive, have to be woken up upon a change of the primitive, thus a FIFO guarantee cannot be given. We hypothesise that for most models, where the STM approach is applicable, this has no qualitative influence on the dynamics as agents are assumed to act conceptually at the same time and no fairness is needed. We leave the test of this hypothesis for future research. This is connected to our assumption that concurrent execution has no qualitative influence on the dynamics. Although repeated runs with same initial conditions might lead to different results due to nondeterminism, the dynamics follow still the same distribution as the one from the sequential implementation. To verify this we can make use the techniques of property-based testing as shown in \cite{thaler_show_2019} but we leave it for further research.
