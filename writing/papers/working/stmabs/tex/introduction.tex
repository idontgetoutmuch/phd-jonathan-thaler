\section{Introduction} % (once upon a time...)
The future of scientific computing in general and Agent-Based Simulation (ABS) in particular is parallelism: Moore's law is declining as we are reaching the physical limits of CPU clocks. The only option is going massively parallel due to availability of cheap massive parallel local hardware with many cores, or cloud services like Amazon S2. Unfortunately the established imperative languages in the ABS field, Python, Java, C++, follow mostly a lock-based approach to concurrency which is error prone and does not compose. Further, data-parallelism in an imperative language is susceptible to side-effects because these languages cannot distinguish between data-parallelism and concurrency in the types. 

Functional programming such as Haskell \citep{hudak_history_2007} can help to improve on both problems. Data-parallelism falls into place very naturally in it because of the way side-effects are handled and because data is immutable. Also, the problems of lock-based approaches can be overcome by using Software Transactional Memory (STM). Although STM exists in other languages as well, Haskell was one of the first to natively build it into its core. Further, it has the unique benefit that it can guarantee the lack of persistent side-effects at compile time, allowing unproblematic retries of transactions, something of fundamental importance in STM. This makes the use of STM in Haskell very compelling.

The paper \cite{discolo_lock_2006} gives a good indication how difficult and complex constructing a correct concurrent program is and shows how much easier, concise and less error-prone an STM implementation is over traditional locking with mutexes and semaphores. Further it shows that STM consistently outperforms the lock-based implementation. We follow this work and compare the performance of lock-based and STM implementations and hypothesise that the reduced complexity and increased performance will be directly applicable to ABS as well.

We present two case-studies in which we employ an agent-based spatial SIR \citep{macal_agent-based_2010, thaler_pure_2019} and the well known SugarScape \citep{epstein_growing_1996} model to test our hypothesis. The latter model can be seen as one of the most influential exploratory models in ABS which laid the foundations of object-oriented implementation of agent-based models. The former one is an easy-to-understand explanatory model which has the advantage that it has an analytical theory behind it which can be used for verification and validation. 

The aim of this paper is to experimentally investigate the benefit of using STM over lock-based approaches for concurrent ABS models. Our contribution is that to the best of our knowledge we are the first to \textit{systematically} investigate the use of STM in ABS and compare its performance with sequential, lock-based and imperative implementations both on local and Amazon Cloud Service machinery.

We start with Section \ref{sec:rel_work} where we present related work. Then we discuss the concepts of STM and Haskell in Section \ref{sec:background}. In Section \ref{sec:stm_abs} we show how to apply STM to ABS in general. Section \ref{sec:cs_sir} contains the first case-study using a spatial SIR mode whereas Section \ref{sec:cs_sugarscape} presents the second case-study using the SugarScape model. We conclude in Section \ref{sec:conclusion} and give further research directions in Section \ref{sec:further}.