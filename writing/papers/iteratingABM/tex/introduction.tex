\section{Introduction}
In the paper of \cite{huberman_evolutionary_1993} the authors showed that the results of the simulation of the classic prisoners-dilemma on a 2D-grid reported in \cite{nowak_evolutionary_1992} depends on a a very specific strategy of iterating this simulation and show that the beautiful patterns as reported by \cite{nowak_evolutionary_1992} will not form when selecting a different iteration-strategy. Although the authors differentiated between two strategies, their description still lacks precision and generality which we will try to repair in this paper. Although they too discussed philosophical aspects of choosing one strategy over the other, they lacked to generalize their observation. We will do so in the central message of our paper by stressing that when doing Agent-Based Simulation \& Modelling (ABM/S) \textit{it is of most importance to select the right iteration-strategy which reflects and supports the corresponding semantics of the model}. We find that this awareness is yet still under-represented in the literature of ABM/S and most important of all is lacking a systematic treatment. Thus our contribution in this paper is to provide such a systematic treatment by
\begin{itemize}
	\item Presenting all the general iteration-strategies which are possible in an ABM/S.
	\item Developing a systematic terminology of talking about them.
	\item Giving the philosophical interpretation and meaning of each of them.
	\item Comparing the 3 programming languages Java, Haskell and Scala in regard of their suitability to implement each of these strategies.
\end{itemize}

Besides the systematic treatment of all the general iteration-strategies the paper presents another novelty which is its inclusion of the pure functional declarative language Haskell in the comparison. This language has so far been neglected by the ABM/S community which is dominated by object-oriented (OO) programming languages like Java thus the usage of Haskell presents a real, original novelty in this paper.