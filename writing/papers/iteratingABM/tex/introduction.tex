\section{Introduction}
In this paper we are looking at two different kind of games to show the differences update-strategies can make in ABS and supporting our main message that \textit{when developing a model for an ABS it is of most importance to select the right update-strategy which reflects and supports the corresponding semantics of the model}. As we will show due to conflicting ideas about update-strategies this awareness is yet still under-represented in the field of ABS and is lacking a systematic treatment. As a remedy we undertake such a systematic treatment in proposing a new terminology by identifying properties of ABS and deriving all possible update-strategies. The by-product is a framework to talk in a unified form about this very important matter, so to enable researchers and implementers to talk about it in a common way, enabling better discussions, same understanding and better reproduceability and continuity in research.
The two games we use are discrete and continuous games where in the former one the agents act synchronized at discrete time-steps whereas in the later one they act continuously in continuous time. We show that in the case of simulating the discrete game the update-strategies have a huge impact on the final result whereas our continuous game seems to be stable under different update-strategies.
Because the selection of an update-strategy has profound implications for the implementation of an ABS we investigate the three different programming-paradigms of \textit{object-orientation} (OO), \textit{pure functional} and \textit{multi-paradigm} in the form of the programming languages Java, Haskell and Scala in their suitability of implementing each update-strategy. As it turns out the paradigms can't capture all the update-strategies equally well thus one should be careful when selecting the implementation language for the ABS, reflecting its suitability for implementing the selected updates-strategy. The contribution of this paper is:
\begin{itemize}
	\item Present general properties of ABS.
	\item Derive update-strategies from these properties.
	\item Establish a general terminology of talking about these update-strategies.
	\item Compare the three programming languages Java, Haskell and Scala in regard of their suitability to implement each of these strategies.
\end{itemize}