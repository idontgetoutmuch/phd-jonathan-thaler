\section{Related Research}
\begin{itemize}
	\item \cite{huberman_evolutionary_1993}
	\item \cite{a_framework_2008}
\end{itemize} 

\cite{botta_time_2010} sketch a minimal agent-framework in Haskell which is very similar in the basic structure of ours, also utilizing an agent-transforming function which consumes incoming messages and produces outgoing ones. This proofs that this approach, very well developed in ABM/S, seems to be a very natural one also to apply to Haskell. Their focus is more on economic simulations and instead of iterating a simulation with a global time, their focus is on how to synchronize agents which have internal, local transition times. They introduce a time-keeper agent which synchronizes the actions of all of the agents thus we argue that our framework is able to capture it faithfully using the \textit{Actor-Strategy} utilizing either a timer-keeper as they do or through the access of the global shared-environment. Although their work uses Haskell as well, this does not diminish the novelty of our approach using Haskell because our focus is very different from them.