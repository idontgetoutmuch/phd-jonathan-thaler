\section{Further Research}

\subsection{Theory of Models}
Theory of model-properties connecting them to the different update-strategies: why do some models require a given update-strategy and some not?

\subsection{Functional Reactive Programming}
\cite{jankovic_functional_2007} discuss using functional programming for discrete event simulation (DES) and mention the paradigm of Functional Reactive Programming (FRP) to be very suitable to DES. We were aware of the existence of this paradigm and have experimented with it using the library Yampa, but decided to leave that topic to a side and really keep our implementation clear and very basic. The next step would be to fusion ABS, which can be understood as a variant of DES, with Yampa thus leveraging both approaches from which we hope to gain the ability to develop much more complex models with heterogeneous agents.

\subsection{Functional Model-Specification Language}
After showing that Haskell is a very attractive alternative to existing OO approaches in implementing ABS we are interested whether the declarative power of the pure functional language can be utilized to write specifications for simple ABS models. Such a language would be equals or very close to the program-code thus eliminate the gap between specification and programming.

\subsection{Actor Model in ABS}
Although we showed that the Act-Strategy implemented in Scala with Actors can implement very different kind of Models we barely scratched the surface. There already exists research using the Actor Model for ABS but in the context of Erlang \cite{varela_modelling_2004}, \cite{di_stefano_using_2005}, \cite{di_stefano_exat:_2007}, \cite{sher_agent-based_2013} but we find that the Actor Model should get more attention in ABS. We feel that this research-field is nowhere near exhaustion and we hope that more research is going into this topic as we feel that the Actor-Model has a bright future ahead due to the ever increasing availability of massively parallel computing machinery.