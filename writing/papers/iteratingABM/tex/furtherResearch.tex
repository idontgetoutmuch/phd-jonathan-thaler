\section{Further Research}

\subsection{Functional Reactive Programming}
The implemented framework in Haskell is lacking features like TODO and is basically an attempt of reinventing Functional Reactive Programming (FRP). We were aware of the existence of this paradigm, especially the library Yampa, but decided to leave that one to a side and really keep our implementation clear and very basic. The next step would be to fusion our implementations with Yampa thus leveraging both approaches from which we hypothesize to gain the ability to develop much more complex models with heterogeneous agents.

\subsection{Develop a small modelling-language which is close to the Haskell-Version of modelling agents therefore specification and implementation match}
TODO: see paper-abstract

\subsection{Reasoning in Haskell about the Model \& Simulation}
TODO: sketch ideas

\subsection{Immediate Message-Handling in Haskell}
This is the single main drawback of the Haskell implementation and although it only shows up in the Seq Strategy it would be of interest if there is an elegant, pure functional software-architecture which allows messages sent from Agent A to Agent B be immediately - as in: the same execution thread - handled by Agent B. 

\subsection{The Actor Model in ABM/S}
We find that the Actor Model should get more attention in ABM/S and although we showed that the Act-Strategy implemented in Scala with Actors can implement very different kind of Models we barely scratched the surface. We hope that more research is going into this topic as we feel that the Actor-Model has a bright future ahead due to the ever increasing availability of massively parallel computing machinery.