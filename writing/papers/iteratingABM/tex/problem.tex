\section{Background}
ABM/S is a method of modelling and simulating a system where the global behaviour may be unknown but the behaviour and interactions of the parts making up the system is of knowledge \cite{wooldridge_introduction_2009}. Those parts, called Agents, are modelled and simulated out of which then the aggregate global behaviour of the whole system emerges. Thus the central aspect of ABM/S is the concept of an Agent which can be understood as a metaphor for a unique pro-active unit, able to spawn new Agents, interacting with other Agents in a network of neighbours by exchange of messages which are situated in a generic environment. Thus we informally assume the following about our Agents:

\begin{itemize}
	\item They have a unique identifier and some internal state
	\item They can initiate actions on their own e.g. change their internal state, send messages, create new agents, kill themselves,...
	\item They can react to messages they receive with actions (see above)
	\item They can interact with a generic environment they are situated in
\end{itemize} 

An implementation of an ABS must solve thus solve three fundamental problems:
\begin{enumerate}
	\item Source of pro-activity \\ How can an Agent initiate actions without the external stimuli of messages?
	\item Message-Visibility \\ When is a message \textit{m}, visible to Agent \textit{B}, processed by it?
	\item Semantics of Message-Delivery \\ When is a message \textit{m}, sent by Agent \textit{A} to Agent \textit{B}, visible to \textit{B}?
\end{enumerate}

TODO: add "own thread of execution" or "shared thread of execution". still this does not allow us to generalize the notion of global time because in PAR and CON we have own threads of execution but global time. this brings us to:
TODO: synchronized vs asynchronous iteration. 

In computer systems, pro-activity, the ability to initiate actions on its own without external stimuli, is only possible when there is some internal stimuli, most naturally represented by some generic notion of monotonic increasing time-flow. This can either be some physical real-time system-clock which counts the milliseconds since 1970 (thus binding the time-flow of the system to the one of the 'real-world') or a virtual simulation-clock which is just a monotonic increasing natural number. As we are in a discrete computer-system, this time-flow must be discretized as well in discrete steps and each step must be made available to the Agent, acting as one internal stimuli. This allows the Agent then to perceive time and become pro-active depending on time (NOTE: we could argue that this is not really pro-activity because it depends always on time, but there is really now other way of doing this in our current implementation of computer-systems.). Independent of the representation of the time-flow we have the two fundamental choices whether the time-flow is local to the Agent or whether it is a system-global time-flow.

\medskip 

The semantics of message-delivery define when sent messages are visible to the receivers so they can process them and react to them. The only two ways of implementing them are that messages are visible either \textit{immediately} or after a synchronization point between the sender and receiver. Such a synchronization point can be a local one, just between the two or a global one between all Agents in the system.

\medskip 

Basically we can say that we want to process a message as soon as it is visible to us but this is not how real computer-systems can work. In a real system each Agent would have a message-box into which the messages are posted so the Agent can then check its mail-box for new messages. The question is then when the Agent is going to poll for new messages? Clearly what we need is a recurring, regular trigger which allows the Agent to poll the mail-box and process all queued messages. We argue that the most natural approach is to bind this trigger to the time-flow step which provides pro-activity. Note that all explicit communication between Agents (there could be implicit communication through the environment) is message-based where messages are only processed at specified times in execution which is the time-step of an Agent. Thus a message sent to another Agent is not executed by the Agent at the sending time but queued and executed at the next internal stimulus: the next time-step. This prevents recursive calls due to an agent could never know who initiated the message-chain, something an agent does not want to care for.
TODO: maybe this is the difference between sync and async meant by some: sync is if messages are executed immediately, async is if message-execution is postponed 

\medskip 

To solve these problems an update-strategy is implemented which will iterate through the Agents in \textit{some} way and allow the Agents to perform these steps. It is immediately clear that different choices in the specific problems will lead to different system behaviour. To discuss this we will first present all possible update-strategies and their details in the next section and then outline how they influence the system behaviour.

