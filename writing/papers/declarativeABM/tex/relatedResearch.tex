\section{Related Research}

\cite{sulzmann_specifying_2007} present an EDSL for Haskell allowing to specify Agents using the BDI model. We don't go there, thats not our intention. 

\cite{schneider_towards_2012} and \cite{vendrov_frabjous:_2014} present a domain-specific language for developing functional reactive agent-based simulations. This language called FRABJOUS is very human readable and easily understandable by domain-experts. It is not directly implemented in FRP/Haskell/Yampa but is compiled to Haskell/Yampa code which they claim is also readable. This is the direction we want to head but we don't want this intermediate step but look for how a most simple domain-specific language embedded in Haskell would look like. We also don't touch upon FRP and Yampa yet but leave this to further research for another paper of ours.

TODO: cite julie greensmith paper on haskell

We don't focus on BDI or similar but want to rely much more on low-level basic messaging. We can also draw strong relations to Hoare's Communicating Sequential Processes (CSP), Milner's Calculus of Communicating Systems (CCS) and Pi-Calculus. By mapping the EDSL to CSP/CCS/Pi-Calculus we achieve to be able to algebraic reasoning in our EDSL. TODO: hasn't Agha done something similar in connecting Actors to the Pi-Calculus?