\section{Introduction}
 the specification language should not be too technical, its focus should be on non-technical expressiveness. The question is: can we abstract away the technicalities and still translate it directly to haskell (more or less)? If not, can be adjust our Haskell implementation to come closer to our specification language? Thus it is a two-fold approach: both languages need to come closer to each other if we want to close the gap
 
 note that the difference between SEQ and PAR in Haskell is in the end a 'fold' over the agents in the case of SEQ and a 'map' in the case of PAR
 the conc-version and the act-version of the agent-implementations look EXACTLY the same	 BUT we lost the ability to step the simulation!!!

\cite{sulzmann_specifying_2007} present an EDSL for Haskell allowing to specify Agents using the BDI model.

TODO: \cite{schneider_towards_2012}

TODO: \cite{vendrov_frabjous:_2014}

We don't focus on BDI or similar but want to rely much more on low-level basic messaging. We can also draw strong relations to Hoare's Communicating Sequential Processes (CSP), Milner's Calculus of Communicating Systems (CCS) and Pi-Calculus. By mapping the EDSL to CSP/CCS/Pi-Calculus we achieve to be able to algebraic reasoning in our EDSL. TODO: hasn't Agha done something similar in connecting Actors to the Pi-Calculus?