\section{Introduction}
 the specification language should not be too technical, its focus should be on non-technical expressiveness. The question is: can we abstract away the technicalities and still translate it directly to haskell (more or less)? If not, can be adjust our Haskell implementation to come closer to our specification language? Thus it is a two-fold approach: both languages need to come closer to each other if we want to close the gap
 
TODO: \cite{nowak_evolutionary_1992}, \cite{huberman_evolutionary_1993}
it is still not exactly clear => we present a formal specification using ABS 

it is not trivial to reproduce the results as there is only very informal descriptions in \cite{nowak_evolutionary_1992}. \cite{huberman_evolutionary_1993} give a few more details but also stay quite informal. Thus here we represent a pure functional formulation of the original program which makes it formally extactly clear how the simulation should work. we then look how it can be translated to an ABM specification