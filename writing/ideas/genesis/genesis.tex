%%%%%%%%%%%%%%%%%%%%%%%%%%%%%%%%%%%%%%%%%
% University/School Laboratory Report
% LaTeX Template
% Version 3.1 (25/3/14)
%
% This template has been downloaded from:
% http://www.LaTeXTemplates.com
%
% Original author:
% Linux and Unix Users Group at Virginia Tech Wiki 
% (https://vtluug.org/wiki/Example_LaTeX_chem_lab_report)
%
% License:
% CC BY-NC-SA 3.0 (http://creativecommons.org/licenses/by-nc-sa/3.0/)
%
%%%%%%%%%%%%%%%%%%%%%%%%%%%%%%%%%%%%%%%%%

%----------------------------------------------------------------------------------------
%	PACKAGES AND DOCUMENT CONFIGURATIONS
%----------------------------------------------------------------------------------------

\documentclass{article}

\usepackage[utf8]{inputenc}
\usepackage{graphicx} % Required for the inclusion of images
\usepackage{natbib} % Required to change bibliography style to APA
\usepackage{amsmath} % Required for some math elements 

\setlength\parindent{0pt} % Removes all indentation from paragraphs

%\usepackage{times} % Uncomment to use the Times New Roman font

%----------------------------------------------------------------------------------------
%	DOCUMENT INFORMATION
%----------------------------------------------------------------------------------------

\title{The genesis according to computer science:\\Reality as an agent-based simulation of free will} % Title

\author{Jonathan \textsc{Thaler}} % Author name

\date{\today} % Date for the report

\begin{document}

\maketitle % Insert the title, author and date

% If you wish to include an abstract, uncomment the lines below
\begin{abstract}
TODO: maybe remove "agent-based" and talk more generally about a simulation of free will

All sciences have their genesis-model which are basically explanations of how the world did come into existence, what the reason for existence is and who or what God is. Unfortunately computer science has none so far, so the aim of this paper is to set out to develop such genesis-model from a computer-science perspective. The model is motivated from the perspective that the world and humankind is a simulation to see free will in action in a sand-boxed environment as opposed to the afterlife / afterworld or just the Beyond, which is an outer level of simulation and itself again a simulation as will be shown in subsequent sections. \\
This paper addresses important fundamental questions of belief and religion and tries to explain them using this model - which it does surprisingly well. Thus this paper has a keen aim: it wants to amalgamate religious concepts with concepts of theoretical computer science. It is an attempt to think out-of-the-box, having fun looking at dogmatic things from a total different perspective, breaking down conventional view on old religious things and does not take itself too serious - after all the least thing we need is a new dogma or idealism, the world is full of it.

NOTE: not falsifyable, thus no scientific theory but its a framework within theories/statements which are falsifiable can be formulated

\bigskip

"In the beginning there was nothing, which exploded" - Terry Pratchett
\end{abstract}

\section{Introduction}

The following questions will be addressed and explained in this new context

\begin{itemize}
\item Who or what is God?
\item Who or what is Christ, Buddah, Vishnu, Mohammed,...?
\item What is free will? 
\item What is meditation?
\item What is conciousness?
\item What is life after death?
\item Where do we come from?
\item What is the omega point?
\item What is spiritual enlightenment?
\end{itemize}

Approaching the subject from a more technical view-point:

\begin{itemize}
\item Who or what implemented the simulation?
\item What is outside this simulation?
\item What is free will in this context? Can it be defined formally?
\item On which hardware does this simulation run? Where does the energy come from?
\item What is the computational complexity of this simulation?
\item What are the memory-requirements of this simulation?
\end{itemize}

karma: cause and effect. is a controlling factor in the simulation to prevebt the dynamics to get out of hand

The nothing was something: it was indeed nothingness, a

free will on a machine is a contradiction. the machine works according to very strict rules. free will can be completely unpredictable. or is free will just an imagination? if one confronts a decision maker within short time with too much information then the outcome it is unpredictable 

parallels to matrix

TODO: lucifer, satan, out-of-the-creation, left-handed vs. right-handed path

pro-activity possible through conciousness: the brain produces thoughts and the conciousness can observe these and decide to follow them or not. This is observable on oneself during meditation!

Free will: deliberately ignore thoughts

\section{Self-Conciousness \& Free-Will}
self-conciousness: the ability to observe ones thoughts on a metalevel: more or less pronounced. this meta-observation allows to intervene. also origin and unfolding is then possible. thus one can observe oneself from an outer perspective \\

freely choose NOT to obey some impulse. requires self-conciousness \\

computers have neither and cant have neither. why? thus computers as we know them cant be source of true intelligence as they are not able to introspection, to self-reflection. \\
they don't have this ability because the have no ability to \textit{imagine or anticipate the outcome of their actions without actually computing them}

\section{Simulation}
We as humans constantly run simulations in our minds when thinking and perceiving the reality: we anticipate our actions, envision what we want to do,... all by \textit{simulating} them in our mind. This is probably the most powerful tool of our intelligence which separates us (probably) from the animal kingdom. This ability to simulate potential / future realities but also changes us, there is a feedback.  So in a case there are 2 levels: reality and the simulations of reality in our minds.
I claim that these simulations may be as real as the reality we are living in where "only" the mind in which the simulation runs differs: in the case of our humans it is ourself, in case of the reality we are situated in it is an entity we would like to call God.
Both spawn a reality which are bewohnt by entities. But were God allows the entities free will, we haven't managed to do that yet. I postulate some stage in human development where we are able to create simulations which are able to simulate all free will outcomes. Tthe entities in the simulation need free will, just as we do. For this to happen they need the ability to simulate their reality as well - this creates a cascade. But the whole point is that the free will and conciousness \textit{has always been there}, passed down from the initial \textit{first} simulation initiator - which we refer to God but which may be just a level on a range of infinite many levels. 

\section{Cascading Simulations}
At some point in the existence of a free-will intelligence, it starts to asking for the future. First using religion, then mathematics, then finally computer simulation. But the problem is that such a simulation is too weak to forecast the future because out of simple computation no free will is born. thus the solution of the free-will intelligence is to put itself in a simulation-environment as a seed of free will. this simulation will then play through every decision branches an thus be able to predict possible futures. because within such a simulation the same thing can AND WILL happen at some point, we arrive at cascading simulations within simulations. thus we are at one level of this cascade where our direct outer level is god.

\section{On parallel universes, existence as simulation, free will}
We cannot predict the future due to complex interaction of free will of Humankind. To predict it we would have to spawn a new universe running in parallel if a free-will choice occurs. Then again, maybe this is already the case and the whole existence is an extremely huge tree of parallel universes being created from each other and collapsing back into others or being completely determined. \\
The question is then: Where in this tree am I? And maybe time does only advance in discrete steps after a spawn/collapse? \\
When one looks at the existence as a simulation then one can say that it has become unstable because too many actors with free will and too many variables producing unforeseeable consequences. But then, can we make predictions about a simulation from within? Can we talk about the meaning and meta-workings of a system from within it? \\
We always try to treat reality as smooth and predictable without outliers but ignoring catastrophic events - this is what the book "Black Swan" says. My own point of view is that the problem is the way we do science: "we divide and put reality into small boxes of labels/categories and then pile them up, adding piles of theories describing it creating a mountain of unbearable complexity - just to be caught by surprise by the next catastrophic event no one could predict despite the overwhelming amount of complex theories. \\
What's the problem? Theories describe the past. Science needs to move on to the now letting go of the myriads of categories and look at it all as a single complex system/simulation - the world as a simulation, simulating the interaction of free will, allowing it to unfold and see the effects in all facets. \\
The question is whether "Black Swans" are an emergent system property coming from within the simulation or whether they are created from steering forces e.g. God.

\section{A magical approach as remedy of the dilemma}
Just as we try to manifest our thoughts and desires using magic we need devices which can do so with our thoughts in a structured way. Computers can be seen as a kind of attempt to achieve these devices but are not able to manifest real creational and metaphysical thoughts but only allow to execute formal models which can be mapped to a specific kind of symbol-manipulation. We need something more powerful: a magic computer. We need to learn how to think in its language but it will allow us to manifest thoughts in a virtual reality. \\
Thus we can say: Programming = Magic. It is a systematic altering reality and manifesting thoughts by encoding them in a systematic way in a system of symbols and rules how to change/apply them (=language).
[ ] we imagine something and then create it
[ ] its purely virtual
[ ] we are naming things
[ ] results can be unpredictable

\section{How can humankind survive?}
remove all ideologies
is it possible to live without an ideology?
love is the answer: it is more radical and allows for more change than anything else
free will without love ultimately leads to destruction. this would be the hypothesis of the simulation. 
but then again: what is love? it accepts all live as equal and same value with no right of one to judge and rule over another. even more: it also attributes this to live which kills the loving one

\section{The problem of Ideologies}
what happens if one has unlimited power at hand and wants to totally wipe out evil and unjustice?
write episodes which narrate without judgement important developments of the protagonist on the way to achieving the goal

\section{An agent-based model \& simulation of this meta-world view}
Simulate the flow of karama (Cause-And-Effect), the moving up through the spheres (12), the dynamics of free will, the dynamics of carma, the staying in the "hell"-sphere and return to earth.

%----------------------------------------------------------------------------------------
%	BIBLIOGRAPHY
%----------------------------------------------------------------------------------------

\bibliographystyle{apalike}
\bibliography{../../references/phdReferences.bib}

\end{document}