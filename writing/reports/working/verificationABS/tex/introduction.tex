\section{Introduction}
Previous research has shown that the pure functional programming paradigm as in Haskell is very suitable to implement agent-based simulations. Building on FRP and MSFs the work developed an elegant implementation of an agent-based SIR model which was pure. By statically removing all external influences of randomness already at compile time through types, this guarantees that repeated simulation runs with the same starting conditions will always result in the same dynamics - guaranteed at compile time. This previous research focused only on establishing the basic concepts of ABS in functional programming but it did not explore the inherent strength of functional programming for verification and correctness any further than guaranteeing the reproducibility of the simulation at compile time.

This paper picks up where the previous research has left and wants to investigate the usefulness of pure and dependently typed functional programming for verification and correctness of agent-based simulation. We are especially interested if requirements of an ABS can be guaranteed on a stronger level by those paradigms, if a larger class of bugs can be excluded already at compile time and whether we can express model properties and invariants already at compile time on a type level. Further we are interested in how far we can reason about an agent-based model in a dependently typed implementation. Also we investigate the use of QuickCheck for code- and model-testing.


the key points to investigate in this report are:
- testing of abs: unit- \& property testing
- using types for invariants / bug free
- reasoning about correctness in code
- reasoning about dynamics in code


Independent of the programming paradigm, there exist fundamentally two approaches implementing agent-based simulation: time- and event-driven. In the time-driven approach, the simulation is stepped in fixed $\Delta t$ and all agents are executed at each time-step - they act virtually in lock-step at the same time. The approach is inspired by the theory of continuous system dynamics (TODO: cite).
In the event-driven approach, the system is advanced through events, generated by the agents, and the global system state changes by jumping from event to event, where the state is held constant in between. The approach is inspired by discrete event simulation (DES) (TODO: citation) which is formalized in the DEVS formalism \cite{zeigler_theory_2000}.

In a preceding paper we investigated how to derive a time-driven pure functional ABS approach in Haskell (TODO: cite my paper). We came to quite satisfactory results and implemented also a number of agent-based models of various complexity (TODO: cite schelling, sugarscape, agent zero). Still we identified weaknesses due to the underlying functional reactive programming (FRP) approach. It is possible to define partial implementations which diverge during runtime, which may be difficult to determine for complex models for a programmer at compile time. Also sampling the system with fixed $\Delta t$ can lead to severe performance problems when small $\Delta t$ are required, as was shown in our paper. The later problem is well known in the simulation community and thus as a remedy an event-driven approach was suggested \cite{meyer_event-driven_2014}.
In this paper for the first time, we derive a pure functional event-driven agent-based simulation. Instead of using Haskell, which provides already libraries for DES \cite{sorokin_aivika_2015}, we focus on the dependently typed pure functional programming language Idris. In our previous paper we hypothesised that dependent types may offer interesting new insights and approaches to ABS but it was unclear how exactly we can make use of them, which was left for further research. In this paper we hypothesise that, as opposed to a time-driven approach, the even-driven approach is especially suited to make proper use of dependent types due to its different nature. Note that both a pure functional event-driven approach to ABS \textit{and} the use of dependent types in ABS has so far never been investigated, which is the unique contribution of this paper.
If we can construct a dependently typed program of the SIR ABM which is total, then we have a proof-by-construction that the SIR model reaches a steady-state after finite time

Dependent Types are the holy grail in functional programming as they allow to express even stronger guarantees about the correctness of programs and go as far where programs and types become constructive proofs \cite{wadler_propositions_2015} which must be total by definition \cite{thompson_type_1991}, \cite{altenkirch_why_2005}, \cite{altenkirch_pi_sigma:_2010}, \cite{program_homotopy_2013}. Thus the next obvious step is to apply them to our pure functional approach of agent-based simulation. So far no research in applying dependent types to agent-based simulation exists at all and it is not clear whether dependent types do make sense in this setting. We explore this for the first time and ask more specifically how we can add dependent types to our pure functional approach, which conceptual implications this has for ABS and what we gain from doing so. Note that we can only scratch the surface and lay down basic ideas and leave a proper in-depth treatment of this topic for further research. We use Idris \cite{brady_idris_2013}, \cite{brady_type-driven_2017} as language of choice as it is very close to Haskell with focus on real-world application and running programs as opposed to other languages with dependent types e.g. Agda and Coq which serve primarily as proof assistants.

Dependent Types promise the following:

\begin{enumerate}
	\item Types as proofs - In dependently types languages, types can depend on any values and are first-class objects themselves. TODO: make more clear

	\item Totality and termination - Constructive proofs must terminate, this means a well-typed program (which is itself a proof) is always terminating which in turn means that it must consist out of total functions. A total function is defined by \cite{brady_type-driven_2017} as: it terminates with a well-typed result or produces a non-empty finite prefix of a well-typed infinite result in finite time. Idris is turing complete but is able to check the totality of a function under some circumstances but not in general as it would imply that it can solve the halting problem. Other dependently typed languages like Agda or Coq restrict recursion to ensure totality of all their functions - this makes them non turing complete.
\end{enumerate}

%An agent can be seen as a potentially infinite stream of continuations which at some point could return information to stop evaluating the next item of the stream which allows an agent to terminate.
%correspondence between temporal logics and FRP due to jeffery: is abs just another temporal logic?

%Ionesus talk on dependently typed programming in scientific computing
%https://www.pik-potsdam.de/members/ionescu/cezar-ifl2012-slides.pdf
%Ionescus talk on Increasingly Correct Scientific Computing
%https://www.cicm-conference.org/2012/slides/CezarIonescu.pdf
%Ionescus talk on Economic Equilibria in Type Theory
%https://www.pik-potsdam.de/members/ionescu/cezar-types11-slides.pdf
%Ionescus talk on Dependently-Typed Programming in Economic Modelling
%https://www.pik-potsdam.de/members/ionescu/ee-tt.pdf

Validation \& Verification in ABS
\url{http://www2.econ.iastate.edu/tesfatsi/VVAccreditationSimModels.OBalci1998.pdf} : verification = are we building the model right? validation = are we building the right model?

good paper \url{http://www2.econ.iastate.edu/tesfatsi/VVAccreditationSimModels.OBalci1998.pdf} : very nice 15 guidelines and life cycles, VERY valuable for background and introduction

\url{http://www2.econ.iastate.edu/tesfatsi/VVSimulationModels.JKleijnen1995.pdf} : suggests good programming practice which is extremely important for high quality code and reduces bugs but real world practice and experience shows that this alone is not enough, even the best programmers make mistakes which often can be prevented through a strong static or a dependent type system already at compile time. What we can guarantee already at compile time, doesn't need to be checked at run-time which saves substantial amount of time as at run-time there may be a huge number of execution paths through the simulation which is almost always simply not feasible to check (note that we also need to check all combinations). This paper also cites modularity as very important for verification: divide and conquer and test all modules separately. this is especially easy in functional programming as composability is much better than with traditional oop due to the lack of interdependence between data and code as in objects and the lack of global mutable state (e.g. class variables or global variables) - this makes code extremely convenient to test. The paper also discusses statistical tests (the t test) to check if the outcome of a simulation is sufficiently close to real-world dynamics. Also the paper suggests using animations to visualise the processes within the simulation for verification purposes (of course they note that animation may be misleading when one focuses on too short simulation runs).

good paper:\url{ https://link.springer.com/chapter/10.1007/978-3-642-01109-2_10}
	-> verification. "This is essentially the question: does the model do what we think it is supposed to do? Whenever a model has an analytical solution, a condition which embraces almost all conventional economic theory, verification is a matter of checking the mathematics."
	-> validation: "In an important sense, the current process of building ABMs is a discovery process, of discovering the types of behavioural rules for agents which appear to be consistent with phenomena we observe."
		=> can we encode phenomena we observe in the types? can we use types for the discovery process as well? can dependent types guide our exploratory approach to ABS?
	-> "Because such models are based on simulation, the lack of an analytical solution (in
general) means that verification is harder, since there is no single result the model
must match. Moreover, testing the range of model outcomes provides a test only in
respect to a prior judgment on the plausibility of the potential range of outcomes.
In this sense, verification blends into validation."

either one has an analytical model as the basis of an agent-based model (ABM) or one does not.
In the former case, e.g. the SIR model, one can very easily validate the dynamcis generated by the ABM to the one generated by the analytical solution (e.g. through System Dynamics). Of course the dynamics wont be exactly the same as ABS discretisizes the approach and introduces stochastics which means, one must validate averaged dynamics.
In the latter case one has basically no idea or description of the emergent behaviour of the system prior to its execution. It is important to have some hypothesis about the emergent property / dynamics. The question is how verification / validation works in this setting as there is no formal description of the expected behaviour: we don't have a ground-truth against which we can compare our simulation dynamics. (eventuell hilft hier hans vollbrecht weiter: Simulation hat hier den Sinn, die Controller anhand der Roboteraufgabe zu validieren, Bei solchen Simulationen ist man interessiert an allen möglichen Sequenzen, und da das meist zu viele sind, an einer möglichst gut verteilten Stichprobenmenge. Hier geht es weniger um richtige Zeitmodellierung, sondern um den Test aller möglichen Ereignissequenzen.)

look into DEVS

TODO: the implementation phase is just one stage in a longer process \url{http://jasss.soc.surrey.ac.uk/12/1/1.html}

WE FOCUS ON VERIFICATION
important: we are not concerned here with validating a model with the real world system it simulates. this is an entirely different problem and focuses on the questions if we have built the right model.
we are interested here in extremely strong verification: have we built the model right? we are especially interested in to which extend purely and dependently-typed functional programming can support us in this task.

\url{http://jasss.soc.surrey.ac.uk/8/1/5.html}: "For some time now, Agent Based Modelling has been used to simulate and explore complex systems, which have proved intractable to other modelling approaches such as mathematical modelling. More generally, computer modelling offers a greater flexibility and scope to represent phenomena that do not naturally translate into an analytical framework. Agent Based Models however, by their very nature, require more rigorous programming standards than other computer simulations. This is because researchers are cued to expect the unexpected in the output of their simulations: they are looking for the 'surprise' that shows an interesting emergent effect in the complex system. It is important, then, to be absolutely clear that the model running in the computer is behaving exactly as specified in the design. It is very easy, in the several thousand lines of code that are involved in programming an Agent Based Model, for bugs to creep in. Unlike mathematical models, where the derivations are open to scrutiny in the publication of the work, the code used for an Agent Based Model is not checked as part of the peer-review process, and there may even be Intellectual Property Rights issues with providing the source code in an accompanying web page."

\url{http://jasss.soc.surrey.ac.uk/12/1/1.html}: "a prerequisite to understanding a simulation is to make sure that there is no significant disparity between what we think the computer code is doing and what is actually doing. One could be tempted to think that, given that the code has been programmed by someone, surely there is always at least one person - the programmer - who knows precisely what the code does. Unfortunately, the truth tends to be quite different, as the leading figures in the field report, including the following: You should assume that, no matter how carefully you have designed and built your simulation, it will contain bugs (code that does something different to what you wanted and expected), "Achieving internal validity is harder than it might seem. The problem is knowing whether an unexpected result is a reflection of a mistake in the programming, or a surprising consequence of the model itself. […] As is often the case, confirming that the model was correctly programmed was substantially more work than programming the model in the first place. This problem is particularly acute in the case of agent-based simulation. The complex and exploratory nature of most agent-based models implies that, before running a model, there is some uncertainty about what the model will produce. Not knowing a priori what to expect makes it difficult to discern whether an unexpected outcome has been generated as a legitimate result of the assumptions embedded in the model or, on the contrary, it is due to an error or an artefact created in the model design, its implementation, or its execution."


general requirements to ABS
- modelling progress of time (steward robinson simulation book)
- modelling variability (steward robinson simulation book)
- fixing random number stream (steward robinson simulation book)
- only rely on past
	-> solved with Arrowized FRP
- bugs due to implicitly mutable state
	-> can be ensured by pure functional programming
- ruling out external sources of non-determinism / randomness
	-> can be ensured by pure functional programming
- correct interaction protocols
	-> can be ensured by dependent state machines
- deterministic time-delta
	-> TODO: can we ensure it through dependent-types at type-level?
- repeated runs lead to same dynamics
	-> can be ensured by pure functional programming

dependent-types:
-> encode model-invariants on a meta-level
-> encode dynamics (what? feedbacks? positive/negative) on a meta-level
-> totality equals steady-state of a simulation, can enforce totality if required through type-level programming