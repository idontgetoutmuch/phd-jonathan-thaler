\section{Testing / Verification}
TODO: explore ABS testing in pure functional Haskell
- we need to distinguish between two types of testing/verification
	-> 1. testing/verification of models for which we have real-world data or an analytical solution which can act as a ground-truth. examples for such models are the SIR model, stock-market simulations, social simulations of all kind
	-> 2. testing/verification of models which are just exploratory and which are only be inspired by real-world phenomena. examples for such models are Epsteins Sugarscape and Agent\_Zero

\subsection{Black Box Verification}
Defined as treating the functionality to test as a black box with inputs and outputs and comparing controlled inputs to expected outputs.

- testing of distributions e.g. all agents recover on average after illness duration
- testing of the final dynamics: how close do they match the analytical solution
- can we express model properties in tests e.g. quickcheck?
- isolated tests: how easy can we test parts of an agent / simulation?

\subsubsection{Comparison against SD}
- population-size matters: different population-size results in slightly different dynamics in SD => need same population size in ABS (probably...?)
- utilise a statistical test with H0 "ABS and comparison is not the same" and H1 "ABS and comparison is the same"
- how many replications and how do we average?
- which statistical test do we implement? (steward robinson simulation book, chapter 12.4.4)
	-> Normalizsed Mean Squared Error (NMSE)
	-> TODO: implement confidence interval 
	-> TODO: what about chi-squared?
	-> TODO: what about paired-t confidence interval
	
\subsection{White Box Verification}
Defined as directly looking at the code and reasoning that the code does really what it has to implement.

- coverage testing