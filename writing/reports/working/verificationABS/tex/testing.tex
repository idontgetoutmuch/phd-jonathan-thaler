\section{Testing / Verification}
TODO: explore ABS testing in pure functional Haskell
- we need to distinguish between two types of testing/verification
	-> 1. testing/verification of models for which we have real-world data or an analytical solution which can act as a ground-truth. examples for such models are the SIR model, stock-market simulations, social simulations of all kind
	-> 2. testing/verification of models which are just exploratory and which are only be inspired by real-world phenomena. examples for such models are Epsteins Sugarscape and Agent\_Zero
	
\subsection{Black Box Verification}
Defined as treating the functionality to test as a black box with inputs and outputs and comparing controlled inputs to expected outputs.

In Black Box Verification one generally feeds input and compares it to expected output. In the case of ABS we have two things to black-box test:
\begin{enumerate}
	\item Isolated Agent Behaviour - test isolated agent behaviour under given inputs using unit- and property-based testing
	\item Interacting Agent Behaviour - test if interaction between agents are correct 
	\item Simulation Dynamics - compare emergent dynamics of the ABS as a whole under given inputs to an analytical solution / real-world dynamics in case there exists some using statistical tests
	\item Hypotheses- test whether hypotheses are valid / invalid using unit- and property-based testing. TODO: how can we formulate hypotheses in unit- and/or property-based tests?
\end{enumerate}

- testing of the final dynamics: how close do they match the analytical solution
- can we express model properties in tests e.g. quickcheck?
- property-testing shines here
- isolated tests: how easy can we test parts of an agent / simulation?

\subsubsection{Finding optimal $\Delta t$}
The selection of the right $\Delta t$ can be quite difficult in FRP because we have to make assumptions about the system a priori. One could just play it safe with a very conservatively selected small $\Delta t <= 0.1$ but the smaller $\Delta t$, the lower the performance as it quickly multiplies the number of steps to calculate. Obviously one wants to select the \textit{optimal} $\Delta t$, which in the case of ABS is the largest possible $\Delta t$ for which we still get the correct simulation dynamics.
To find out the \textit{optimal} $\Delta t$ one can make direct use of the Black Box tests: start with a large $\Delta t = 1.0$ and reduce it by half every time the tests fail until no more tests fail - if for $\Delta t = 1.0$ tests already pass, increasing it may be an option. It is important to note that although isolated agent behaviour tests might result in larger $\Delta t$, in the end when they are run in the aggregate system, one needs to sample the whole system with the smallest $\Delta t$ found amongst all tests. Another option would be to apply super-sampling to just the parts which need a very small $\Delta t$ but this is out of scope of this paper.

\subsubsection{Agents as signals}
Agents \textit{might} behave as signals in FRP which means that their behaviour is completely determined by the passing of time: they only change when time changes thus if they are a signal they should stay constant if time stays constant. This means that they should not change in case one is sampling the system with $\Delta t = 0$. Of course to prove whether this will \textit{always} be the case is strictly speaking impossible with a Black Box verification but we can gain a good level of confidence with them also because we are staying pure. It is only through white box verification that we can really guarantee and prove this property.

\subsubsection{Comparison of dynamics against existing data}
- utilise a statistical test with H0 "ABS and comparison is not the same" and H1 "ABS and comparison is the same"
- how many replications and how do we average?
- which statistical test do we implement? (steward robinson simulation book, chapter 12.4.4)
	-> Normalizsed Mean Squared Error (NMSE)
	-> TODO: implement confidence interval 
	-> TODO: what about chi-squared?
	-> TODO: what about paired-t confidence interval

IMPORTANT: this is not what we are after here in this paper, statistical tests are a science on their own and there actually exists quite a large amount of literature for conducting statistical tests on ABS dynamics: Robinson Book (TODO: find additional literature)	

\subsection{White Box Verification}
White-Box verification is necessary when we need to reason about properties like \textit{forever}, \textit{never}, which cannot be guaranteed from black-box tests. Additional help can be coverage tests with which we can show that all code paths have been covered in our tests.