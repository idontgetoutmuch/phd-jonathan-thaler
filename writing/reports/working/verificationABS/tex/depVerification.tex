\section{Towards Dependently Typed Verification}
Independent of the programming paradigm, there exist fundamentally two approaches implementing agent-based simulation: time- and event-driven. In the time-driven approach, the simulation is stepped in fixed $\Delta t$ and all agents are executed at each time-step - they act virtually in lock-step at the same time. The approach is inspired by the theory of continuous system dynamics (TODO: cite).
In the event-driven approach, the system is advanced through events, generated by the agents, and the global system state changes by jumping from event to event, where the state is held constant in between. The approach is inspired by discrete event simulation (DES) (TODO: citation) which is formalized in the DEVS formalism \cite{zeigler_theory_2000}.

In a preceding paper we investigated how to derive a time-driven pure functional ABS approach in Haskell (TODO: cite my paper). We came to quite satisfactory results and implemented also a number of agent-based models of various complexity (TODO: cite schelling, sugarscape, agent zero). Still we identified weaknesses due to the underlying functional reactive programming (FRP) approach. It is possible to define partial implementations which diverge during runtime, which may be difficult to determine for complex models for a programmer at compile time. Also sampling the system with fixed $\Delta t$ can lead to severe performance problems when small $\Delta t$ are required, as was shown in our paper. The later problem is well known in the simulation community and thus as a remedy an event-driven approach was suggested \cite{meyer_event-driven_2014}.
In this paper for the first time, we derive a pure functional event-driven agent-based simulation. Instead of using Haskell, which provides already libraries for DES \cite{sorokin_aivika_2015}, we focus on the dependently typed pure functional programming language Idris. In our previous paper we hypothesised that dependent types may offer interesting new insights and approaches to ABS but it was unclear how exactly we can make use of them, which was left for further research. In this paper we hypothesise that, as opposed to a time-driven approach, the even-driven approach is especially suited to make proper use of dependent types due to its different nature. Note that both a pure functional event-driven approach to ABS \textit{and} the use of dependent types in ABS has so far never been investigated, which is the unique contribution of this paper.
If we can construct a dependently typed program of the SIR ABM which is total, then we have a proof-by-construction that the SIR model reaches a steady-state after finite time

Dependent Types are the holy grail in functional programming as they allow to express even stronger guarantees about the correctness of programs and go as far where programs and types become constructive proofs \cite{wadler_propositions_2015} which must be total by definition \cite{thompson_type_1991}, \cite{altenkirch_why_2005}, \cite{altenkirch_pi_sigma:_2010}, \cite{program_homotopy_2013}. Thus the next obvious step is to apply them to our pure functional approach of agent-based simulation. So far no research in applying dependent types to agent-based simulation exists at all and it is not clear whether dependent types do make sense in this setting. We explore this for the first time and ask more specifically how we can add dependent types to our pure functional approach, which conceptual implications this has for ABS and what we gain from doing so. Note that we can only scratch the surface and lay down basic ideas and leave a proper in-depth treatment of this topic for further research. We use Idris \cite{brady_idris_2013}, \cite{brady_type-driven_2017} as language of choice as it is very close to Haskell with focus on real-world application and running programs as opposed to other languages with dependent types e.g. Agda and Coq which serve primarily as proof assistants.

Dependent Types promise the following:

\begin{enumerate}
	\item Types as proofs - In dependently types languages, types can depend on any values and are first-class objects themselves. TODO: make more clear

	\item Totality and termination - Constructive proofs must terminate, this means a well-typed program (which is itself a proof) is always terminating which in turn means that it must consist out of total functions. A total function is defined by \cite{brady_type-driven_2017} as: it terminates with a well-typed result or produces a non-empty finite prefix of a well-typed infinite result in finite time. Idris is turing complete but is able to check the totality of a function under some circumstances but not in general as it would imply that it can solve the halting problem. Other dependently typed languages like Agda or Coq restrict recursion to ensure totality of all their functions - this makes them non turing complete.
\end{enumerate}

%An agent can be seen as a potentially infinite stream of continuations which at some point could return information to stop evaluating the next item of the stream which allows an agent to terminate.
%correspondence between temporal logics and FRP due to jeffery: is abs just another temporal logic?

%Ionesus talk on dependently typed programming in scientific computing
%https://www.pik-potsdam.de/members/ionescu/cezar-ifl2012-slides.pdf
%Ionescus talk on Increasingly Correct Scientific Computing
%https://www.cicm-conference.org/2012/slides/CezarIonescu.pdf
%Ionescus talk on Economic Equilibria in Type Theory
%https://www.pik-potsdam.de/members/ionescu/cezar-types11-slides.pdf
%Ionescus talk on Dependently-Typed Programming in Economic Modelling
%https://www.pik-potsdam.de/members/ionescu/ee-tt.pdf

dependent-types:
-> encode model-invariants on a meta-level
-> encode dynamics (what? feedbacks? positive/negative) on a meta-level
-> totality equals steady-state of a simulation, can enforce totality if required through type-level programming
-> probabilistic types can encode probability distributions in types already about which we can then reason
-> can we encode objectives in types?

\subsection{Dependently Typed SIR}
Intuitively, based upon our model and the equations we can argue that the SIR model enters a steady state as soon as there are no more infected agents. Thus we can informally argue that a SIR model must always terminate as:
\begin{enumerate}
	\item Only infected agents can infect susceptible agents.
	\item Eventually after a finite time every infected agent will recover.
	\item There is no way to move from the consuming \textit{recovered} state back into the \textit{infected} state \footnote{There exists an extended SIR model, called SIRS which adds a cycle to the state-machine by introducing a transition from recovered to susceptible but we don't consider that here.}.
\end{enumerate}

Thus a SIR model must enter a steady state after finite steps / in finite time. 

This result gives us the confidence, that the agent-based approach will terminate, given it is really a correct implementation of the SD model. Still this does not proof that the agent-based approach itself will terminate and so far no proof of the totality of it was given. Dependent Types and Idris ability for totality and termination checking should theoretically allow us to proof that an agent-based SIR implementation terminates after finite time: if an implementation of the agent-based SIR model in Idris is total it is a proof by construction. Note that such an implementation should not run for a limited virtual time but run unrestricted of the time and the simulation should terminate as soon as there are no more infected agents. We hypothesize that it should be possible due to the nature of the state transitions where there are no cycles and that all infected agents will eventually reach the recovered state. 
Abandoning the FRP approach and starting fresh, the question is how we implement a \textit{total} agent-based SIR model in Idris. Note that in the SIR model an agent is in the end just a state-machine thus the model consists of communicating / interacting state-machines. In the book \cite{brady_type-driven_2017} the author discusses using dependent types for implementing type-safe state-machines, so we investigate if and how we can apply this to our model. We face the following questions: how can we be total? can we even be total when drawing random-numbers? Also a fundamental question we need to solve then is how we represent time: can we get both the time-semantics of the FRP approach of Haskell AND the type-dependent expressivity or will there be a trade-off between the two?

-- TODO: express in the types
-- SUSCEPTIBLE: MAY become infected when making contact with another agent
-- INFECTED:    WILL recover after a finite number of time-steps
-- RECOVERED:   STAYS recovered all the time

-- SIMULATION:  advanced in steps, time represented as Nat, as real numbers are not constructive and we want to be total
--              terminates when there are no more INFECTED agents



%A susceptible agent can only become infected when it comes into contact with an infected agent. The probability of a susceptible agent making contact with an infected one is naturally (number of infected agents) / (total number of agents). For the infection to occur we multiply the contact with the infectivity parameter \Gamma. A susceptible agent makes on average \Beta contacts per time-unit. This results in the following formula:
%
%\begin{align}
%prob &= \frac{I \beta \gamma}{N} \\
%\end{align}
%
%This is for a single agent, which we then need to multiply by the number of susceptible agents because all of them make contact.
%
%TODO: implement sir with state-machine approach from Idris. an idea would be to let infected agents generate infection- actions: the more infected agents the more infection-actions => zero infected agents mean zero infection actions. this list can then be reduced?
%
%can we also emulate SD in Idris and formulate positive/negative feedback loops in types?