\chapter{Going Large-Scale with Software Transactional Memory}
\label{chap:stm}

In this chapter we present a technique which allows to scale the approach of functional ABS as presented in the paper in Appendix \ref{app:pfe} to massively large-scale simulations. This is a very recent development which didn't make it into the paper but will make an appearance in the final thesis in the form of this chapter. Due to the increasing need for massively large-scale ABS in recent years (todo: cite), making this possible with our functional approach as well, gives our research a much larger impact.

The whole concept of our approach is built on the usage of Software Transactional Memory (STM), which we will introduce in Section \ref{sect:stm_intro}. In Section \ref{sect:stm_impl} we give a short overview of our implementation, presenting and discussing code. In Section \ref{sect:stm_perf} we present performance comparisons and discuss the implications more in-depth in Section \ref{sect:stm_discussion}. Finally we will give a short outline over further research in \ref{sect:stm_further}.

\section{Introduction}
\label{sect:stm_intro}
TODO: short introduction to the concepts of STM and what it is capable of.

The main benefits are:
\begin{itemize}
	\item \textit{composable} concurrency!
\end{itemize}

\section{Implementation}
\label{sect:stm_impl}
TODO: brief overview of the implementation of the SIR model with code.

\section{Performance Comparison}
\label{sect:stm_perf}
experiment: SIR on a 51x51 grid (single infected at center) with dt=0.1 until t=100, exporting the dynamics to a matlab file in the end which is the exact same, non-parallel implementation for all implementations 

\paragraph{State Monad, single core (as in Paper in Appendix \ref{app:pfe})}
\begin{minted}[fontsize=\footnotesize]{console}
----------------------------------------------------------------------------------
 159,168,992,488 bytes allocated in the heap
  50,355,058,208 bytes copied during GC
      18,184,960 bytes maximum residency (2257 sample(s))
          97,928 bytes maximum slop
              47 MB total memory in use (0 MB lost due to fragmentation)

                                     Tot time (elapsed)  Avg pause  Max pause
  Gen  0    140816 colls,     0 par   48.832s  49.114s     0.0003s    0.0135s
  Gen  1      2257 colls,     0 par    0.761s   0.765s     0.0003s    0.0006s

  INIT    time    0.000s  (  0.000s elapsed)
  MUT     time   51.196s  ( 51.468s elapsed)
  GC      time   49.593s  ( 49.879s elapsed)
  EXIT    time    0.002s  (  0.002s elapsed)
  Total   time  100.790s  (101.348s elapsed)

  %GC     time      49.2%  (49.2% elapsed)

  Alloc rate    3,109,032,896 bytes per MUT second

  Productivity  50.8% of total user, 50.8% of total elapsed
----------------------------------------------------------------------------------
\end{minted}

Averaging the total time elapsed of 4 runs (101.348, 101.581, 100.991, 100.681) results in a mean of 101.15 and a standard deviation of 0.395.

\paragraph{STM, single core}
\begin{minted}[fontsize=\footnotesize]{console}
----------------------------------------------------------------------------------
  98,857,350,264 bytes allocated in the heap
  20,265,438,776 bytes copied during GC
      98,908,552 bytes maximum residency (156 sample(s))
         996,984 bytes maximum slop
             203 MB total memory in use (0 MB lost due to fragmentation)

                                     Tot time (elapsed)  Avg pause  Max pause
  Gen  0     94776 colls,     0 par   20.305s  20.382s     0.0002s    0.0138s
  Gen  1       156 colls,     0 par    0.032s   0.032s     0.0002s    0.0008s

  TASKS: 4 (1 bound, 3 peak workers (3 total), using -N1)

  SPARKS: 0 (0 converted, 0 overflowed, 0 dud, 0 GC'd, 0 fizzled)

  INIT    time    0.000s  (  0.000s elapsed)
  MUT     time   31.757s  ( 31.878s elapsed)
  GC      time   20.337s  ( 20.414s elapsed)
  EXIT    time    0.006s  (  0.008s elapsed)
  Total   time   52.101s  ( 52.300s elapsed)

  Alloc rate    3,112,910,665 bytes per MUT second

  Productivity  61.0% of total user, 61.0% of total elapsed
----------------------------------------------------------------------------------
\end{minted}

Averaging the total time elapsed of 4 runs (52.300, 52.27, 52.720, 53.740) results in a mean of 52.752 and a standard deviation of 0.676.

\paragraph{STM, 4 cores}
\begin{minted}[fontsize=\footnotesize]{console}
----------------------------------------------------------------------------------
  98,956,602,152 bytes allocated in the heap
  20,182,157,176 bytes copied during GC
      99,043,872 bytes maximum residency (155 sample(s))
       1,135,272 bytes maximum slop
             207 MB total memory in use (0 MB lost due to fragmentation)

                                     Tot time (elapsed)  Avg pause  Max pause
  Gen  0     26564 colls, 26564 par   47.795s   9.277s     0.0003s    0.0157s
  Gen  1       155 colls,   154 par    0.253s   0.047s     0.0003s    0.0009s

  Parallel GC work balance: 87.47% (serial 0%, perfect 100%)

  TASKS: 10 (1 bound, 9 peak workers (9 total), using -N4)

  SPARKS: 0 (0 converted, 0 overflowed, 0 dud, 0 GC'd, 0 fizzled)

  INIT    time    0.001s  (  0.001s elapsed)
  MUT     time   25.821s  ( 10.571s elapsed)
  GC      time   48.048s  (  9.324s elapsed)
  EXIT    time    0.006s  (  0.014s elapsed)
  Total   time   73.877s  ( 19.910s elapsed)

  Alloc rate    3,832,340,955 bytes per MUT second

  Productivity  35.0% of total user, 53.2% of total elapsed
----------------------------------------------------------------------------------
\end{minted}
 
Averaging the total time elapsed of 4 runs (19.910, 20.290, 20.420, 21.690) results in a mean of 20.578 with standard deviation of 0.772.

When we scale up the grid-size for the 4-core version we get the following results:

\begin{center}
  \begin{tabular}{ c || c | c | c }
    Grid-Size & Duration \\ \hline \hline 
    51 x 51 & 21 sec \\ \hline
    101 x 101 & 92 sec \\ \hline
    151 x 151 & 188 sec \\ \hline
    201 x 201 & 305 sec \\ \hline
    251 x 251 & 530 sec \\ \hline
  \end{tabular}
\end{center}

\subsection{Performance Interpretation}
TODO: does the duration grow linearly with increasing agents?

TODO: visual outputs and dynamics look qualitatively the same, so it should be ok

Interpretation of the data leads to the following conclusions:
\begin{enumerate}
	\item On a single core, no transaction retries should happen, the results support that assumption.
	\item Sharing state using an STM variable is much more efficient than using the State Monad.
	\item Running STM on multiple cores concurrently leads to a significant performance improvement. TODO: does the duration grow linearly with increasing agents?
\end{enumerate}

\subsection{Comparison to Java RePast single core}
To have an idea where the functional implementation is performance-wise compared to the established object-oriented methods, we conducted a performance comparison with a Java implementation using RePast, running on a single-core. All parameters were the same and the simulation was run until virtual time t=100 was reached, on various grid-sizes. Due to the lack of proper timing facilities in RePast we measured the time by hand using a stopwatch. Although this is not very precise it gives a rough estimate and allows a very basic comparison, being precise enough if the difference is larger than 1 second. We measured the following:

\begin{center}
  \begin{tabular}{ l || c | r }
    Grid-Size & Java Repast & 4-Core Haskell \\ \hline \hline 
    51 x 51 & 10 sec & 20 sec \\ \hline
    101 x 101 & 110 sec & 103 sec \\ \hline
    201 x 201 & 1260 sec & 305 sec \\ \hline
  \end{tabular}
\end{center}

While on a 51x51 grid the single-core Java RePast version outperforms the 4-core Haskell STM version by about 200\%, the figure is inverted on a 201x201 grid where the 4-core Haskell STM version outperforms the single core Java Repast version by 400\%. We can conclude that the single-core Java RePast version clearly outperforms the Haskell STM 4-core version on small grid-sizes but that the Haskell STM version scales up with increasing grid-sizes and clearly outperforms the RePast version with increasing number of agents.

\section{Discussion}
\label{sect:stm_discussion}
In-depth discussion of the STM concept for ABS.


TODO: implement message-boxed STM (TChan) SIR model: based on data-flow implementation
TODO: can we make synchronous Interactions work with STM? e.g. wormholes in STM

TODO: downside we lose reproducibility due to concurrency (non-deterministic elements: race-conditions), but we still can guarantee that the agents don't do random IO stuff as it is restricted to STM operations only

TODO: talk about retries of transactions which could become a bottle neck, central problem is to keep the retries low, which is directly influenced by the read/writes on the concurrent data-structures. By choosing more fine-grained / suitable data-structures e.g. using a TArray instead of an Array within a TVar, one can reduce retries significantly. We tracked the retries in the above example and arrived at a ratio of 0.0\%, note that there were some retries but they were so low that they didn't significant.

TODO: TChan is a perfect match for PERISTENT agent-message queues. we have broadcast chan functionality which allows convenient environment behaviour

\section{Further Research}
\label{sect:stm_further}
TODO: future research can we apply STM to an even-driven approach as well?
TODO: synchronous agent-interactions with STM