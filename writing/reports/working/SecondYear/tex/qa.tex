\chapter{Questions \& Answers}
\label{chap:qa}

TODO: update and adopt to 2nd year

In this chapter I give answers to anticipated questions and objections about my research direction and vision of doing pure functional ABS \footnote{They are not always posed in a dead-serious way but as it is a quite controversial topic - ABS should be done object-orientated after all huh? - I think it is appropriate. Also some objections were raised in exactly this way.}.

\paragraph{So you had this hypothesis, that pure functional programming and dependent types lead to simulation software which is more likely correct and is easier to verify and validate, right from the beginning?}
Not at all. I even had no deep knowledge of functional programming at the start of my PhD, I've just worked through the 1st edition of Grahams book "Programming in Haskell" and that's it. I had no clear understanding of purity, side-effects and Monads and I didn't know a bit about functional reactive programming. I knew that something like Dependent Types exist because Thorsten (2nd Supervisor) has sent me an email before the start of my PhD in which he pointed at Agda, so I started reading a bit about intuitionistic / constructivistic math, tried out a little bit of Agda but quickly gave up because it was way too far away (without really having mastered pure functional programming in Haskell, I believe it is nearly impossible / too difficult / makes no sense going into dependent types).
So in the beginning there was pure \textit{curiosity} about functional programming in combination with ABS because I knew nothing of FP at all and wanted to understand it (after getting bored by OO) and applying FP to ABS seemed so crazy (because everyone claims OO to be 'natural' for it) that it must be an extremely interesting challenge. I guess this is very often the case with research: there is 'just' curiosity in the beginning and then during the research process a hypothesis falls into place.
