\chapter{Generalising Research}
\label{chap:generalising}

We hypothesize that our research can be transferred to other, related fields as well, which puts our contributions into a much broader perspective, giving it much more impact than restricting it just to the very narrow field of Agent-Based Simulation. Although we don't have the time to back up our claims with in-depth research, we argue that our findings might be applicable to the following fields at least on a conceptual level:

\section{Simulations}
We already showed in the paper found in Appendix \ref{app:pfe}, that purity in a simulation leads to repeatability which is of utmost importance in scientific computation. These insights are easily transferable to simulation software in general and might be of huge benefit there. Also my approach to dependent types in ABS might be applicable to simulations in general due to the correspondence between equilibrium \& totality and in use for hypotheses formulation as pointed out in section \ref{sub:dep_abs_generalconcepts}. 

\section{Multi Agent Systems}
The fields of Multi Agent Systems (MAS) and ABS are closely related where ABS has drawn much inspiration from MAS \cite{wooldridge_introduction_2009}, \cite{weiss_multiagent_2013}. It is important to understand that MAS and ABS are two different fields where in MAS the focus is much more on technical details, implementing a system of interacting intelligent agents within a highly complex environment with the focus primarily on solving AI problems.

Because in both fields, the concept of interacting agents is of fundamental importance, we expect our research also to be applicable in parts to the field of MAS. Especially the work on dependent types should be very useful there because MAS is very interested in correctness, verification and formally reasoning about a system / their agents, to show that a system follows a formal specifications.
% also distributed MAS systems can make use of dependent typed message passing (but this is really a different topic)

\section{Distributed search}
TODO: Julie Greensmith mentioned distributed search algorithms as an example where localized entities (speak: agents) are interacting with each other.