\chapter{Aims and Objectives}
\label{chap:aimsObj}

This chapter gives a compact and concise overview of the aims and objectives. Chapter \ref{chap:future} gives a more in-depth plan and details in how we will approach the aim in general and the objectives in particular.

\section{Aim}
The aim of this Ph.D. is to investigate how the pure functional programming paradigm with and without dependent types can be used to increase the robustness and confidence in the correctness of Agent-Based Simulations.

\subsection{Hypothesis}
We hypothesise that using pure functional programming and dependent types in agent-based simulation narrow the gap between model specification and implementation substantially, which by definition, leads to a simulation which is more likely to be correct, has less sources of bugs and is easier to verify. %TODO: again we are having a problematic measure: how do we measure 'more likely', 'less sources' and 'easier'?

\section{Objectives}
\begin{enumerate}
	% NOTE: we have done that in our 2nd paper: Pure Functional Epidemics
	\item Develop a library for general-purpose Agent-Based Simulation in Haskell to have a tool in the pure functional programming paradigm to be used for conducting the research. This will allow us to understand the benefits and drawbacks of pure functional programming in the context of Agent-Based Simulation.

	% NOTE: we have done that in our 2nd paper: Pure Functional Epidemics
	\item Explore the benefits and drawbacks of using \textit{pure} functional programming without dependent types in Agent-Based Simulations.

	\item Implement a dependently typed agent-based simulation of the SIR and SugarScape model. This will allow us to explore the concepts of dependent types in Agent-Based Simulation in general.
	
	% NOTE: we have done that in our 2nd paper: Pure Functional Epidemics
	%\item Explore the benefits and drawbacks of using \textit{pure and dependently typed} functional programming in Agent-Based Simulations.

	% TODO: how can we quantify that?
	%\item Explore how pure functional programming with dependent types can increase the correctness of Agent-Based Simulations.

	\item Investigate how far we can close the gap between model specification and implementation using dependent types in Agent-Based Simulation. %, exploring the correspondence between the constructive nature of ABS and dependent types.
\end{enumerate}