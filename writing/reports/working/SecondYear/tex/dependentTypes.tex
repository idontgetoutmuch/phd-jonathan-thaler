\chapter{Dependent Types}
Independent of the programming paradigm, there exist fundamentally two approaches implementing agent-based simulation: time- and event-driven. In the time-driven approach, the simulation is stepped in fixed $\Delta t$ and all agents are executed at each time-step - they act virtually in lock-step at the same time. The approach is inspired by the theory of continuous system dynamics (TODO: cite).
In the event-driven approach, the system is advanced through events, generated by the agents, and the global system state changes by jumping from event to event, where the state is held constant in between. The approach is inspired by discrete event simulation (DES) (TODO: citation) which is formalized in the DEVS formalism \cite{zeigler_theory_2000}.

In a preceding paper we investigated how to derive a time-driven pure functional ABS approach in Haskell (TODO: cite my paper). We came to quite satisfactory results and implemented also a number of agent-based models of various complexity (TODO: cite schelling, sugarscape, agent zero). Still we identified weaknesses due to the underlying functional reactive programming (FRP) approach. It is possible to define partial implementations which diverge during runtime, which may be difficult to determine for complex models for a programmer at compile time. Also sampling the system with fixed $\Delta t$ can lead to severe performance problems when small $\Delta t$ are required, as was shown in our paper. The later problem is well known in the simulation community and thus as a remedy an event-driven approach was suggested \cite{meyer_event-driven_2014}.
In this paper for the first time, we derive a pure functional event-driven agent-based simulation. Instead of using Haskell, which provides already libraries for DES \cite{sorokin_aivika_2015}, we focus on the dependently typed pure functional programming language Idris. In our previous paper we hypothesised that dependent types may offer interesting new insights and approaches to ABS but it was unclear how exactly we can make use of them, which was left for further research. In this paper we hypothesise that, as opposed to a time-driven approach, the even-driven approach is especially suited to make proper use of dependent types due to its different nature. Note that both a pure functional event-driven approach to ABS \textit{and} the use of dependent types in ABS has so far never been investigated, which is the unique contribution of this paper.
If we can construct a dependently typed program of the SIR ABM which is total, then we have a proof-by-construction that the SIR model reaches a steady-state after finite time

Dependent Types are the holy grail in functional programming as they allow to express even stronger guarantees about the correctness of programs and go as far where programs and types become constructive proofs \cite{wadler_propositions_2015} which must be total by definition \cite{thompson_type_1991}, \cite{altenkirch_why_2005}, \cite{altenkirch_pi_sigma:_2010}, \cite{program_homotopy_2013}. Thus the next obvious step is to apply them to our pure functional approach of agent-based simulation. So far no research in applying dependent types to agent-based simulation exists at all and it is not clear whether dependent types do make sense in this setting. We explore this for the first time and ask more specifically how we can add dependent types to our pure functional approach, which conceptual implications this has for ABS and what we gain from doing so. Note that we can only scratch the surface and lay down basic ideas and leave a proper in-depth treatment of this topic for further research. We use Idris \cite{brady_idris_2013}, \cite{brady_type-driven_2017} as language of choice as it is very close to Haskell with focus on real-world application and running programs as opposed to other languages with dependent types e.g. Agda and Coq which serve primarily as proof assistants.

Dependent Types promise the following:

\begin{enumerate}
	\item Types as proofs - In dependently types languages, types can depend on any values and are first-class objects themselves. TODO: make more clear

	\item Totality and termination - Constructive proofs must terminate, this means a well-typed program (which is itself a proof) is always terminating which in turn means that it must consist out of total functions. A total function is defined by \cite{brady_type-driven_2017} as: it terminates with a well-typed result or produces a non-empty finite prefix of a well-typed infinite result in finite time. Idris is turing complete but is able to check the totality of a function under some circumstances but not in general as it would imply that it can solve the halting problem. Other dependently typed languages like Agda or Coq restrict recursion to ensure totality of all their functions - this makes them non turing complete.
\end{enumerate}

%An agent can be seen as a potentially infinite stream of continuations which at some point could return information to stop evaluating the next item of the stream which allows an agent to terminate.
%correspondence between temporal logics and FRP due to jeffery: is abs just another temporal logic?

%Ionesus talk on dependently typed programming in scientific computing
%https://www.pik-potsdam.de/members/ionescu/cezar-ifl2012-slides.pdf
%Ionescus talk on Increasingly Correct Scientific Computing
%https://www.cicm-conference.org/2012/slides/CezarIonescu.pdf
%Ionescus talk on Economic Equilibria in Type Theory
%https://www.pik-potsdam.de/members/ionescu/cezar-types11-slides.pdf
%Ionescus talk on Dependently-Typed Programming in Economic Modelling
%https://www.pik-potsdam.de/members/ionescu/ee-tt.pdf

dependent-types:
-> encode model-invariants on a meta-level
-> encode dynamics (what? feedbacks? positive/negative) on a meta-level
-> totality equals steady-state of a simulation, can enforce totality if required through type-level programming
-> probabilistic types can encode probability distributions in types already about which we can then reason
-> can we encode objectives in types?
-> agents as dependently typed continuations?: need a dependently typed concept of a process over time

\subsection{Related Work}
The authors of \cite{botta_functional_2011} are using functional programming as a specification for an agent-based model of exchange markets but leave the implementation for further research where they claim that it requires dependent types. This paper is the closest usage of dependent types in agent-based simulation we could find in the existing literature and to our best knowledge there exists no work on general concepts of implementing pure functional agent-based simulations with dependent types. As a remedy to having no related work to build on, we looked into works which apply dependent types to solve real world problems from which we then can draw inspiration from. 

The authors of \cite{brady_correct-by-construction_2010} use depend types to implement correct-by-construction concurrency in the Idris language \cite{brady_idris_2013}. They introduce the concept of a Embedded Domain Specific Language (EDSL) for concurrently locking/unlocking and reading/writing of resources and show that an implementation and formalisation are the same thing when using dependent types. We can draw inspiration from it by taking into consideration that we might develop a EDSL in a similar fashion for specifying general commands which agents can execute. The interpreter of such a EDSL can be pure itself and doesn't have to run in the IO Monad as our previous research (TODO: cite my PFE paper) has shown that ABS can be implemented pure.

In \cite{brady_idris_2011} the authors discuss systems programming with focus on network packet parsing with full dependent types in the Idris language \cite{brady_idris_2013}. Although they use an older version of it where a few features are now deprecated, they follow the same approach as in the previous paper of constructing an EDSL and and writing an interpreter for the EDSL. In a longer introduction of Idris the authors discus its ability for termination checking in case that recursive calls have an argument which is structurally smaller than the input argument in the same position and that these arguments belong to a strictly positive data type. We are particularly interested in whether we can implement an agent-based simulation which termination can be checked at compile time - it is total.

In \cite{brady_programming_2013} the author discusses programming and reasoning with algebraic effects and dependent types in the Idris language \cite{brady_idris_2013}. They claim that monads do not compose very well as monad transformer can quickly become unwieldy when there are lots of effects to manage. As a remedy they propose algebraic effects and implement them in Idris and show how dependent types can be used to reason about states in effectful programs. In our previous research (TODO: cite my PFE paper) we relied heavily on Monads and transformer stacks and we indeed also experienced the difficulty when using them. Algebraic effects might be a promising alternative for handling state as the global environment in which the agents live or threading of random-numbers through the simulation which is of fundamental importance in ABS. Unfortunately algebraic effects cannot express continuations (according to the authors of the paper) which is but of fundamental importance for pure functional ABS as agents are on the lowest level built on continuations - synchronous agent interactions and time-stepping builds directly on continuations. Thus we need to find a different representation of agents - GADTs seem to be a natural choice as all examples build heavily on them and they are very flexible.

In \cite{fowler_dependent_2014} the authors apply dependent types to achieve safe and secure web programming. This paper shows how to implement dependent effects, which we might draw inspiration from of how to implement agent-interactions which, depending on their kind, are effectful e.g. agent-transactions or events.

In \cite{brady_state_2016} the author introduces the ST library in Idris, which allows a new way of implementing dependently typed state machines and compose them vertically (implementing a state machine in terms of others) and horizontally (using multiple state machines within a function). In addition this approach allows to manage stateful resources e.g. create new ones, delete existing ones. We can draw further inspiration from that approach on how to implement dependently typed state machines, especially composing them hierarchically, which is a common use case in agent-based models where agents behaviour is modelled through hierarchical state-machines. As with the Algebraic Effects, this approach doesn't support continuations (TODO: is this so?), so it is not really an option to build our architecture for our agents on it, but it may be used internally to implement agents or other parts of the system. What we definitely can draw inspiration from is the implementation of the indexed Monad \textit{STrans} which is the main building block for the ST library.

The book \cite{brady_type-driven_2017} is a great source to learn pure functional dependently typed programming and in the advanced chapters introduces the fundamental concepts of dependent state machine and dependently typed concurrent programming on a simpler level than the papers above. One chapter discusses on how to implement a messaging protocol for concurrent programming, something we can draw inspiration from for implementing our synchronous agent interaction protocols.

The authors of \cite{sculthorpe_safe_2009} apply dependent types to FRP to avoid some run-time errors and implement a dependently typed version of the Yampa library in Agda. FRP was the underlying concept of implementing agent-based model we took on in our previous approach (TODO: cite). We could have taken the same route and lift FRP into dependent types but we chose explicitly to not go into this direction and look into complementing approaches on how to implement agent-based models.

The fundamental difference to all these real-world examples is that in our approach, the system evolves over time and agents act over time. A fundamental question will be how we encode the monotonous increasing flow of time in types and how we can reflect in the types that agents act over time.

%The authors of \cite{ionescu_dependently-typed_2012} discuss how to use dependent types to specify fundamental theorems of economics, unfortunately they are not computable and thus not constructive, thus leaving it more to a theoretical, specification side.
%Ionesus talk on dependently typed programming in scientific computing
%https://www.pik-potsdam.de/members/ionescu/cezar-ifl2012-slides.pdf
%Ionescus talk on Increasingly Correct Scientific Computing
%%https://www.cicm-conference.org/2012/slides/CezarIonescu.pdf
%Ionescus talk on Economic Equilibria in Type Theory
%https://www.pik-potsdam.de/members/ionescu/cezar-types11-slides.pdf
%Ionescus talk on Dependently-Typed Programming in Economic Modelling
%https://www.pik-potsdam.de/members/ionescu/ee-tt.pdf

\section{Concepts of Dependent Types in Agent-Based Simulation}
Independent of the programming paradigm, there exist fundamentally two approaches implementing agent-based simulation: time- and event-driven. In the time-driven approach, the simulation is stepped in fixed $\Delta t$ and all agents are executed at each time-step - they act virtually in lock-step at the same time. The approach is inspired by the theory of continuous system dynamics (TODO: cite).
In the event-driven approach, the system is advanced through events, generated by the agents, and the global system state changes by jumping from event to event, where the state is held constant in between. The approach is inspired by discrete event simulation (DES) (TODO: citation) which is formalized in the DEVS formalism \cite{zeigler_theory_2000}.

In a preceding paper we investigated how to derive a time-driven pure functional ABS approach in Haskell (TODO: cite my paper). We came to quite satisfactory results and implemented also a number of agent-based models of various complexity (TODO: cite schelling, sugarscape, agent zero). Still we identified weaknesses due to the underlying functional reactive programming (FRP) approach. It is possible to define partial implementations which diverge during runtime, which may be difficult to determine for complex models for a programmer at compile time. Also sampling the system with fixed $\Delta t$ can lead to severe performance problems when small $\Delta t$ are required, as was shown in our paper. The later problem is well known in the simulation community and thus as a remedy an event-driven approach was suggested \cite{meyer_event-driven_2014}.
In this paper for the first time, we derive a pure functional event-driven agent-based simulation. Instead of using Haskell, which provides already libraries for DES \cite{sorokin_aivika_2015}, we focus on the dependently typed pure functional programming language Idris. In our previous paper we hypothesised that dependent types may offer interesting new insights and approaches to ABS but it was unclear how exactly we can make use of them, which was left for further research. In this paper we hypothesise that, as opposed to a time-driven approach, the even-driven approach is especially suited to make proper use of dependent types due to its different nature. Note that both a pure functional event-driven approach to ABS \textit{and} the use of dependent types in ABS has so far never been investigated, which is the unique contribution of this paper.
If we can construct a dependently typed program of the SIR ABM which is total, then we have a proof-by-construction that the SIR model reaches a steady-state after finite time

Dependent Types are the holy grail in functional programming as they allow to express even stronger guarantees about the correctness of programs and go as far where programs and types become constructive proofs \cite{wadler_propositions_2015} which must be total by definition \cite{thompson_type_1991}, \cite{altenkirch_why_2005}, \cite{altenkirch_pi_sigma:_2010}, \cite{program_homotopy_2013}. Thus the next obvious step is to apply them to our pure functional approach of agent-based simulation. So far no research in applying dependent types to agent-based simulation exists at all and it is not clear whether dependent types do make sense in this setting. We explore this for the first time and ask more specifically how we can add dependent types to our pure functional approach, which conceptual implications this has for ABS and what we gain from doing so. Note that we can only scratch the surface and lay down basic ideas and leave a proper in-depth treatment of this topic for further research. We use Idris \cite{brady_idris_2013}, \cite{brady_type-driven_2017} as language of choice as it is very close to Haskell with focus on real-world application and running programs as opposed to other languages with dependent types e.g. Agda and Coq which serve primarily as proof assistants.

Dependent Types promise the following:

\begin{enumerate}
	\item Types as proofs - In dependently types languages, types can depend on any values and are first-class objects themselves. TODO: make more clear

	\item Totality and termination - Constructive proofs must terminate, this means a well-typed program (which is itself a proof) is always terminating which in turn means that it must consist out of total functions. A total function is defined by \cite{brady_type-driven_2017} as: it terminates with a well-typed result or produces a non-empty finite prefix of a well-typed infinite result in finite time. Idris is turing complete but is able to check the totality of a function under some circumstances but not in general as it would imply that it can solve the halting problem. Other dependently typed languages like Agda or Coq restrict recursion to ensure totality of all their functions - this makes them non turing complete.
\end{enumerate}

%An agent can be seen as a potentially infinite stream of continuations which at some point could return information to stop evaluating the next item of the stream which allows an agent to terminate.
%correspondence between temporal logics and FRP due to jeffery: is abs just another temporal logic?

dependent-types:
-> encode model-invariants on a meta-level
-> encode dynamics (what? feedbacks? positive/negative) on a meta-level
-> totality equals steady-state of a simulation, can enforce totality if required through type-level programming
-> probabilistic types can encode probability distributions in types already about which we can then reason
-> can we encode objectives in types?
-> agents as dependently typed continuations?: need a dependently typed concept of a process over time

As shown in our previous research (TODO: cite), the strong static type system of Haskell allows us to guarantee a lot already at compile time:
\begin{itemize}
	\item Purity - no side-effects possible at all 
	\item Monad - controlled, explicit side-effects possible
	\item generic types allow to guarantee for all
\end{itemize}

Dependent Types in Idris bring the strong static type system of Haskell to a new level, which allows us to both guarantee more things at compile time and express things through type-level computations. This means the following at compile time

\begin{itemize}
	\item Ruling out ever larger classes of bugs
	\item Dependent State Machines
	\item Dependent Agent Interactions
	\item Flow Of Time
	\item Totality
	\item Constructive Proofs
\end{itemize}

\subsection{Ruling out Bugs}
\begin{enumerate}
	\item Index out of bounds access of Lists and Vectors can be guaranteed not to happen any more when using proofs of existence of the element in the list or vector.
	\item Size of list or vector stays constant / increases / decreases / sum of length of multiple vectors guaranteed to be of some number
\end{enumerate}

The question is how far we can generalise our approaches because we fear that the downside of using dependently typed abs is that every implementation needs to start from Scratch: we cant write a general library for it like chimera because the more we put into types, the more specific it is => individual implementation which reuses existing 'patterns' like state machines, messages,...

\subsection{General Agent Interface}
using dependent types to specify the general commands available for an agent. here we can follow the approach of an DSEL as described in \cite{brady_correct-by-construction_2010} and write then an interpreter for it. It is of importance that the interpreter shall be pure itself and does not make use of any fancy IO stuff.

\subsection{Dependent State Machines}
dependent state machines in abs for internal state because that is very Common in ABS. Here we can draw inspiration from the paper \cite{brady_state_2016} and book \cite{brady_type-driven_2017}.

\subsection{Environment}
One of the main advantages of Agent-Based Simulation over other simulation methods e.g. System Dynamics is that agents can live within an environment. Many agent-based models place their agents within a 2D discrete NxM environment where agents either stay always on the same cell or can move freely within the environment where a cell has 0, 1 or many occupants. Ultimately this boils down to accessing a NxM matrix represented by arrays or a similar data structure. In imperative languages accessing memory always implies the danger of out-of-bounds exceptions \textit{at run-time}. With dependent types we can represent such a 2d environment using vectors which carry their length in the type (TODO: discuss them in background) thus fixing the dimensions of such a 2D discrete environment in the types. This means that there is no need to drag those bounds around explicitly as data. Also by using dependent types like Fin which depend on the dimensions we can enforce at compile time that we can only access the data structure within bounds. If we want to we can also enforce in the types that the environment will never be an empty one where N, M > 0.

\begin{minted}{haskell}
Disc2dEnv : (w : Nat) -> (h : Nat) -> (e : Type) -> Type
Disc2dEnv w h e = Vect (S w) (Vect (S h) e)

data Disc2dCoords : (w : Nat) -> (h : Nat) -> Type where
  MkDisc2dCoords : Fin (S w) -> Fin (S h) -> Disc2dCoords w h
  
centreCoords : Disc2dEnv w h e -> Disc2dCoords w h
centreCoords {w} {h} _ =
    let x = halfNatToFin w
        y = halfNatToFin h
    in  mkDisc2dCoords x y
  where
    halfNatToFin : (x : Nat) -> Fin (S x)
    halfNatToFin x = 
      let xh   = divNatNZ x 2 SIsNotZ 
          mfin = natToFin xh (S x)
      in  fromMaybe FZ mfin
      
setCell :  Disc2dCoords w h
        -> (elem : e)
        -> Disc2dEnv w h e
        -> Disc2dEnv w h e
setCell (MkDisc2dCoords colIdx rowIdx) elem env 
    = updateAt colIdx (\col => updateAt rowIdx (const elem) col) env
 
getCell :  Disc2dCoords w h
        -> Disc2dEnv w h e
        -> e
getCell (MkDisc2dCoords colIdx rowIdx) env
    = index rowIdx (index colIdx env)
    
neumann : Vect 4 (Integer, Integer)
neumann = [         (0,  1), 
           (-1,  0),         (1,  0),
                    (0, -1)]

moore : Vect 8 (Integer, Integer)
moore = [(-1,  1), (0,  1), (1,  1),
         (-1,  0),          (1,  0),
         (-1, -1), (0, -1), (1, -1)]

-- TODO: can we express that n <= len?
filterNeighbourhood :  Disc2dCoords w h
                    -> Vect len (Integer, Integer)
                    -> Disc2dEnv w h e 
                    -> (n ** Vect n (Disc2dCoords w h, e))
filterNeighbourhood {w} {h} (MkDisc2dCoords x y) ns env =
    let xi = finToInteger x
        yi = finToInteger y
    in  filterNeighbourhood' xi yi ns env
  where
    filterNeighbourhood' :  (xi : Integer)
                         -> (yi : Integer)
                         -> Vect len (Integer, Integer)
                         -> Disc2dEnv w h e 
                         -> (n ** Vect n (Disc2dCoords w h, e))
    filterNeighbourhood' _ _ [] env = (0 ** [])
    filterNeighbourhood' xi yi ((xDelta, yDelta) :: cs) env 
      = let xd = xi - xDelta
            yd = yi - yDelta
            mx = integerToFin xd (S w)
            my = integerToFin yd (S h)
        in case mx of
            Nothing => filterNeighbourhood' xi yi cs env 
            Just x  => (case my of 
                        Nothing => filterNeighbourhood' xi yi cs env 
                        Just y  => let coord      = MkDisc2dCoords x y
                                       c          = getCell coord env
                                       (_ ** ret) = filterNeighbourhood' xi yi cs env
                                   in  (_ ** ((coord, c) :: ret)))
\end{minted}

\subsection{Dependent Agent Interactions}
\subsubsection{Agent Transactions}
dependently typed message protocols in ABS because its very common, and easily done thorugh methods in OOP: sugarscape mating and trading protocol
using a DSEL \cite{brady_correct-by-construction_2010} to restrict the available primitives in the message protocol?

\subsubsection{Data Flow}
TODO: can dependent types be used in the Data Flow Mechanism?
\subsubsection{Event Scheduling}
TODO: can dependent types be used in the event-scheduling mechanism?

\subsection{Flow Of Time}
TODO: can dependent types be used to express the flow of time and its strongly monotonic increasing?

\subsection{Totality}
totality of parts or the whole simulation e.g. in case of the SIR model we can informally reason that the simulation MUST reach an equilibrium (a steady state from which there is no escape: the dynamics wont't change anymore, derivations are 0) after a finite number of steps. if we can construct a total program which expresses this, we have a formal proof of that which is 1) a specification of the model 2) generates the dynamics 3) is a proof that it reaches equilibrium

\subsection{Constructive Proofs}
- An agent-based model and the simulated dynamics of it is itself a constructive proof which explain a real-world phenomenon sufficiently good
- proof of the existence of an agent: holds always only for the current time-step or for all time, depending on the model. e.g. in the SIR model no agents are removed from / added to the system thus a proof holds for all time. In sugarscape agents are removed / added dynamically so a proof might become invalid after a time or one can construct a proof only from a given time on e.g. when one wants to prove that agent X exists but agent X is only created at time t then before time t the prove cannot be constructed and is uninhabited and only inhabited from time t on.

\section{Dependently Typed SIR}
Intuitively, based upon our model and the equations we can argue that the SIR model enters a steady state as soon as there are no more infected agents. Thus we can informally argue that a SIR model must always terminate as:
\begin{enumerate}
	\item Only infected agents can infect susceptible agents.
	\item Eventually after a finite time every infected agent will recover.
	\item There is no way to move from the consuming \textit{recovered} state back into the \textit{infected} or \textit{susceptible} state \footnote{There exists an extended SIR model, called SIRS which adds a cycle to the state-machine by introducing a transition from recovered to susceptible but we don't consider that here.}.
\end{enumerate}

Thus a SIR model must enter a steady state after finite steps / in finite time. 

This result gives us the confidence, that the agent-based approach will terminate, given it is really a correct implementation of the SD model. Still this does not proof that the agent-based approach itself will terminate and so far no proof of the totality of it was given. Dependent Types and Idris ability for totality and termination checking should theoretically allow us to proof that an agent-based SIR implementation terminates after finite time: if an implementation of the agent-based SIR model in Idris is total it is a proof by construction. Note that such an implementation should not run for a limited virtual time but run unrestricted of the time and the simulation should terminate as soon as there are no more infected agents. We hypothesize that it should be possible due to the nature of the state transitions where there are no cycles and that all infected agents will eventually reach the recovered state. 
Abandoning the FRP approach and starting fresh, the question is how we implement a \textit{total} agent-based SIR model in Idris. Note that in the SIR model an agent is in the end just a state-machine thus the model consists of communicating / interacting state-machines. In the book \cite{brady_type-driven_2017} the author discusses using dependent types for implementing type-safe state-machines, so we investigate if and how we can apply this to our model. We face the following questions: how can we be total? can we even be total when drawing random-numbers? Also a fundamental question we need to solve then is how we represent time: can we get both the time-semantics of the FRP approach of Haskell AND the type-dependent expressivity or will there be a trade-off between the two?

-- TODO: express in the types
-- SUSCEPTIBLE: MAY become infected when making contact with another agent
-- INFECTED:    WILL recover after a finite number of time-steps
-- RECOVERED:   STAYS recovered all the time

-- SIMULATION:  advanced in steps, time represented as Nat, as real numbers are not constructive and we want to be total
--              terminates when there are no more INFECTED agents


show formally that abs does resemble the sd approach: need an idea of a proof and then implement it in dependent types: look at 3 agent system: 2 susceptible, 1 infected. or maybe 2 agents only

%A susceptible agent can only become infected when it comes into contact with an infected agent. The probability of a susceptible agent making contact with an infected one is naturally (number of infected agents) / (total number of agents). For the infection to occur we multiply the contact with the infectivity parameter \Gamma. A susceptible agent makes on average \Beta contacts per time-unit. This results in the following formula:
%
%\begin{align}
%prob &= \frac{I \beta \gamma}{N} \\
%\end{align}
%
%This is for a single agent, which we then need to multiply by the number of susceptible agents because all of them make contact.
%
%TODO: implement sir with state-machine approach from Idris. an idea would be to let infected agents generate infection- actions: the more infected agents the more infection-actions => zero infected agents mean zero infection actions. this list can then be reduced?
%
%can we also emulate SD in Idris and formulate positive/negative feedback loops in types?

\subsection{A constructive proof of totality}
The idea is to implement a total agent-based SIR simulation, where the termination does NOT depend on time (is not terminated after a finite number of time-steps, which would be trivial). The dynamics of the system-dynamics SIR model are in equilibrium (won't change anymore) when the infected stock is 0. This can (probably) be shown formally but intuitionistic it is clear because only infected agents can lead to infections of susceptible agents which then make the transition to recovered after having gone through the infection phase. Thus an agent-based implementation of the SIR simulation has to terminate if it is implemented correctly because all infected agents will recover after a finite number of steps after then the dynamics will be in equilibrium.
Thus we need to 'tell' the type-checker the following:
1) no more infected agents is the termination criterion
2) all infected agents will recover after a finite number of time => the simulation will eventually run out of infected agents But when we look at the SIR+S model we have the same termination criterion, but we cannot guarantee that it will run out of infected => we need additional criteria
3) infected agents are 'generated' by susceptible agents
4) susceptible agents are NOT INCREASING (e.g. recovered agents do NOT turn back into susceptibles)
Interesting: can we adopt our solution (if we find it), into a SIRS	implementation? this should then break totality. also how difficult is it?

The HOTT book states that lists, trees,... are inductive types/inductively defined structures where each of them is characterized by a corresponding "induction principle". For a proof of totality of SIR we need to find the "induction principle" of the SIR model and implement it. What is the inductive, defining structure of the SIR model? is it a tree where a path through the tree is one simulation dynamics? or is it something else? it seems that such a tree would grow and then shrink again e.g. infected agents. Can we then apply this further to (agent-based) simulation in general?

TODO: \url{https://stackoverflow.com/questions/19642921/assisting-agdas-termination-checker/39591118}

\section{Constructivism}
TODO: ABS is constructive: "if you can't grow it, you can't explain it" (epstein)
TODO: Dependent Types are constructive
=> there are no excluded middle in both approaches
=> are there deeper, philosophical connections going on? does it have even deeper implications?
TODO: shortly discuss Propositions as types from HOTT 1.11. In the end a dependently typed ABS is then a constructive proof of WHAT? the model? if we have a total SIR implementation its a constructive proof that the agent-based implementation is total / will reach an equilibrium after a finite number of steps. Still it is not entirely clear WHAT WE ARE PROVING when we are constructing dependently typed agent-based simulations. I need to think about this more carefully
TODO: checkout my notes in 1st annual review on constructivism / popper 

Law of excluded middle does not hold anymore because it would require us to be able to effectively compute / decide whether a proposition is true or false - which amounts to solving the halting problem, which is not possible in the general case.