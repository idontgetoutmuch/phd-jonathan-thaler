\chapter{Dependent Types}
\label{chap:dependent_types}
Dependent types are a very powerful addition to functional programming as they allow us to express even stronger guarantees about the correctness of programs \textit{already at compile-time}. They go as far as allowing to formulate programs and types as constructive proofs which must be \textit{total} by definition \cite{thompson_type_1991, mckinna_why_2006, altenkirch_pi_2010}. 

So far no research using dependent types in agent-based simulation exists at all. We have already started to explore this for the first time and ask more specifically how we can add dependent types to our functional approach, which conceptual implications this has for ABS and what we gain from doing so. We are using Idris \cite{brady_idris_2013} as the language of choice as it is very close to Haskell with focus on real-world application and running programs as opposed to other languages with dependent types e.g. Agda and Coq which serve primarily as proof assistants.

We hypothesise, that  dependent types will allow us to push the correctness of agent-based simulations to a new, unprecedented level by narrowing the gap between model specification and implementation. The investigation of dependent types in ABS will be the main unique contribution to knowledge of my Ph.D.

In the following section \ref{sec:dep_background}, we give an introduction of the concepts behind dependent types and what they can do. Further we give a very brief overview of the foundational and philosophical concepts behind dependent types. In Section \ref{sec:dep_absconcepts} we briefly discuss ideas of how the concepts of dependent types could be applied to agent-based simulation and in Section \ref{sec:dep_vav_deptypes} we very shortly discuss the connection between Verification \& Validation and dependent types.

\section{Introduction}
\label{sec:dep_background}

There exist a number of excellent introduction to dependent types which we use as main ressources for this section: \cite{thompson_type_1991, program_homotopy_2013, stump_verified_2016, brady_type-driven_2017, pierce_programming_2018}.

Generally, dependent types add the following concepts to pure functional programming:

\begin{enumerate}
	\item Types are first-class citizen - In dependently types languages, types can depend on any \textit{values}, and can be \textit{computed} at compile-time which makes them first-class citizen.

	\item Totality and termination - A total function is defined in \cite{brady_type-driven_2017} as: it terminates with a well-typed result or produces a non-empty finite prefix of a well-typed infinite result in finite time. Idris is turing-complete but is able to check the totality of a function under some circumstances but not in general as it would imply that it can solve the halting problem. Other dependently typed languages like Agda or Coq restrict recursion to ensure totality of all their functions - this makes them non turing-complete.

	\item Types as \textit{constructive} proofs - Because types can depend on any values and can be computed at compile-time, they can be used as constructive proofs (see \ref{sub:dep_foundations}) which must terminate, this means a well-typed program (which is itself a proof) is always terminating which in turn means that it must consist out of total functions. Note that Idris does not restrict us to total functions but we can enforce it through compiler flags.
\end{enumerate}

\subsection{An example: Vector}
To give a concrete example of dependent types and their concepts, we introduce the canonical example used in all tutorials on dependent types: the Vector.

In all programming languages like Haskell or in Java, there exists a List data-structure which holds a finite number of homogeneous elements, where the type of the elements can be fixed at compile-time. Using dependent types we can implement the same but adding the length of the list to the type - we call this data-structure a vector.

We define the vector as a Generalised Algebraic Data Type (GADT). A vector has a \textit{Nil} element which marks the end of a vector and a \textit{(::)} which is a recursive (inductive) definition of a linked List. We defined some vectors and we see that the length of the vector is directly encoded in its first type-variable of type Nat, natural numbers. Note that the compiler will refuse to accept \textit{testVectFail} because the type specifies that it holds 2 elements but the constructed vector only has 1 element.

\begin{HaskellCode}
data Vect : Nat -> Type -> Type where
     Nil  : Vect Z e
     (::) : (elem : e) -> (xs : Vect n e) -> Vect (S n) e
	
testVect : Vect 3 String
testVect = "Jonathan" :: "Andreas" :: "Thaler" :: Nil

testVectFail : Vect 2 Nat
testVectFail = 42 :: Nil
\end{HaskellCode}

We can now go on and implement a function \textit{append} which simply appends two vectors. Here we directly see \textit{type-level computations} as we compute the length of the resulting vector. Also this function is \textit{total}, as it covers all input cases and recurs on a \textit{structurally smaller argument}:

\begin{HaskellCode}
append : Vect n e -> Vect m e -> Vect (n + m) e
append Nil ys = ys
append (x :: xs) ys = x :: append xs ys

append testVect testVect
["Jonathan", "Andreas", "Thaler", "Jonathan", "Andreas", "Thaler"] : Vect 8 String
\end{HaskellCode}

What if we want to implement a \textit{filter} function, which, depending on a given predicate, returns a new vector which holds only the elements for which the predicates returns true? How can we compute the length of the vector at compile-time? In short: we can't, but we can make us of \textit{dependent pairs} where the \textit{type} of the second element depends on the \textit{value} of the first (dependent pairs are also known as $\Sigma$ types).

The function is total as well and works very similar to \textit{append} but uses dependent types as return, which are indicated by \textit{**}:

\begin{HaskellCode}
filter : Vect n e -> (e -> Bool) -> (k ** Vect k e)
filter [] f = (Z ** Nil)
filter (elem :: xs) f =
  case f elem of
    False => filter xs f
    True  => let (_ ** xs') = filter xs f
             in  (_ ** elem :: xs')
             
filter testVect (=="Jonathan")
(1 ** ["Jonathan"]) : (k : Nat ** Vect k String)
\end{HaskellCode}

It might seem that writing a \textit{reverse} function for a Vector is very easy, and we might give it a go by writing:
\begin{HaskellCode}
reverse : Vect n e -> Vect n e
reverse [] = []
reverse (elem :: xs) = append (reverse xs) [elem]
\end{HaskellCode}

Unfortunately the compiler complains because it cannot unify 'Vect (n + 1) e' and 'Vect (S n) e'. In the end, the compiler tells us that it cannot determine that (n + 1) is the same as (1 + n). The compiler does not know anything about the commutativity of addition which is due to how natural numbers and their addition are defined.

Lets take a detour. The natural numbers can be inductively defined by their initial element zero Z and the successor. The number 3 is then defined as the successor of successor of successor of zero:

\begin{HaskellCode}
data Nat = Z | S Nat

three : Nat 
three = S (S (S Z))
\end{HaskellCode}

Defining addition over the natural numbers is quite easy by pattern-matching over the first argument: 

\begin{HaskellCode}
plus : (n, m : Nat) -> Nat
plus Z right        = right
plus (S left) right = S (plus left right)
\end{HaskellCode}

Now we can see why the compiler cannot infer that (n + 1) is the same as (1 + n). The expression (n + 1) is translated to (plus n 1), where we pattern-match over the first argument, so we cannot reach a case in which (plus n 1) = S n. To do that we would need to define a different plus function which pattern-matches over the second argument - which is clearly the wrong way to go.

To solve this problem we can exploit the fact that dependent types allow us to perform type-level computations. This should allow us to express commutativity of addition over the natural numbers as a type. For that we define a function which takes in two natural numbers and returns a proof that addition commutes. 

\begin{HaskellCode}
plusCommutative : (left : Nat) -> (right : Nat) -> left + right = right + left
\end{HaskellCode}

We now begin to understand what it means when we speak of \textit{types as proofs}: we can actually express e.g. laws of the natural numbers in types and proof them by implementing a program which inhibits the type - we speak then of a constructive proof (see more on that below \ref{sub:dep_foundations}). Note that \textit{plusCommutative} is already implemented in Idris and we omit the actual implementation as it is beyond the scope of this introduction

Having our proof of commutativity of natural numbers, we can now implement a working (speak: correct) version of \textit{reverse}. The function \textit{rewrite} is provided by Idris: if we have a proof for x = y, the 'rewrite expr in' syntax will search for x in the required type of expr and replace it with y:

\begin{HaskellCode}
reverse : Vect n e -> Vect n e
reverse [] = []
reverse (elem :: xs) = reverseProof (append (reverse xs) [elem])
  where
    reverseProof : Vect (k + 1) a -> Vect (S k) a
    reverseProof {k} result = rewrite plusCommutative 1 k in result
\end{HaskellCode}

\subsection{Constructivity: Equality as a type}
On of the most powerful aspects of dependent types is that they allow us to express equality on an unprecedented level. Non-dependently typed languages have only very basic ways of expressing the equality of two elements of same type. Either we use a boolean or another data-structure which can indicate equality or not. Idris supports this type of equality as well through \textit{(==) : Eq ty $\Rightarrow$ ty $\rightarrow$ ty $\rightarrow$ Bool}. The drawback of using a boolean is that, in the end, we don't have a real evidence of equality: it doesn't tell you anything about the relationship between the inputs and the output. Even though the elements might be equal, the compiler has no means of inferring this and we can still make programming mistakes after the equality check because of this lack of compiler support. Even worse, always returning False / True or whether the inputs are \textit{not} equal is a valid implementation of (==), at least as far as the type is concerned.

As an illustrating example we want to write a function which checks if a Vector has a given length. 

\begin{HaskellCode}
exactLength : (len : Nat) -> (input : Vect n k) -> Maybe (Vect len k)
exactLength {n} len input = case n == len of
                                 True  => Just input 
                                 False => Nothing 
\end{HaskellCode}

Unfortunately this doesn't type-check ('type mismatch between n and len') because the compiler has no way of determining that len is equals n at compile-time. Fortunately we can solve this problem using dependent types themselves by defining \textit{decidable} equality as a type (see more on decidable / non-decidable equality below \ref{sub:dep_foundations}).

First we need a decidable property, meaning it either holds given with some \textit{proof} or it does not hold given some proof that it does \textit{not} hold, resulting in a contradiction. Idris defines such a decidable property already as the following:

\begin{HaskellCode}
-- Decidability. A decidable property either holds or is a contradiction.
data Dec : Type -> Type where
  -- The case where the property holds
  -- @ prf the proof
  Yes : (prf : prop) -> Dec prop

  -- The case where the property holding would be a contradiction
  -- @ contra a demonstration that prop would be a contradiction
  No  : (contra : prop -> Void) -> Dec prop
\end{HaskellCode}

With that we can implement a function which constructs a proof that two natural numbers are equal, or not. We do this simply by pattern matching over both numbers with corresponding base cases and inductions. In case they are not equal we need to construct a proof that they are actually not equal which is done by showing that given some property results in a contradiction - indicated by the type \textit{Void}. In case of \textit{zeroNotSuc} the first number is zero (Z) whereas the other one is non-zero (a successor of some k), which can never be equal, thus we return a \textit{No} instance of the decidable property for which we need to provide the contradiction. In case of \textit{sucNotZero} its just the other way around. \textit{noRec} works very similar but here we are in the induction case which says that if k equals j leads to a contradiction, (k + 1) and (j + 1) can't be equal as well (induction hypothesis).

\begin{HaskellCode}
checkEqNat : (num1 : Nat) -> (num2 : Nat) -> Dec (num1 = num2)
checkEqNat Z Z         = Yes Refl
checkEqNat Z (S k)     = No zeroNotSuc
checkEqNat (S k) Z     = No sucNotZero
checkEqNat (S k) (S j) = case checkEqNat k j of
                              Yes prf   => Yes (cong prf)
                              No contra => No (noRec contra)
                              
zeroNotSuc : (0 = S k) -> Void
zeroNotSuc Refl impossible

sucNotZero : (S k = 0) -> Void
sucNotZero Refl impossible

noRec : (contra : (k = j) -> Void) -> (S k = S j) -> Void
noRec contra Refl = contra Refl
\end{HaskellCode}  
                            
%TODO: explain cong and Refl

The important thing to understand here is that our Dec property holds much more information than just a boolean flag which indicates whether Yes/No that two elements of a type are equal: in case of Yes we have a type which says that num1 is equal to num2, which can be directly used by the compiler, both elements are treated as the same. Refl stands for reflexive and is built into Idris syntax, meaning that a value is equal to itself 'Refl : x = x'. %Further, we need to use 'cong' 

Finally we can implement a correct version of our initial \textit{exactLength} function by computing a proof of equality between both lengths at run-time using \textit{checkEqNat}. This proof can then be used by the compiler to infer that the lengths are indeed equal or not.

\begin{HaskellCode}
exactLength : (len : Nat) -> (input : Vect n k) -> Maybe (Vect len k)
exactLength {n} len input = case checkEqNat n len of
                                 -- len vanishes as compiler can unify len to n
                                 Yes Refl  => Just input 
                                 No contra => Nothing
\end{HaskellCode} 

\subsection{Philosophical Foundations: Constructivism}
\label{sub:dep_foundations}

The main theoretical and philosophical underpinnings of dependent types as in Idris are the works of Martin-L\"of intuitionistic type theory. The view of dependently typed programs to be proofs is rooted in a deep philosophical discussion on the foundations of mathematics, which revolve around the existence of mathematical objects, with two conflicting positions known as classic vs. constructive \footnote{We follow the excellent introduction on constructive mathematics \cite{thompson_type_1991}, chapter 3.}. In general, the constructive position has been identified with realism and empirical computational content where the classical one with idealism and pragmatism.

In the classical view, the position is that to prove $\exists x. P(x)$ it is sufficient to prove that $\forall x. \neg P(x)$ leads to a contradiction. The constructive view would claim that only the contradiction is established but that a proof of existence has to supply an evidence of an $x$ and show that $P(x)$ is provable. In the end this boils down whether to use proof by contradiction or not, which is sanctioned by the law of the excluded middle which says that $A \lor \neg A$ must hold. The classic position accepts that it does and such proofs of existential statements as above, which follow directly out of the law of the excluded middle, abound in mathematics \footnote{Polynomial of degree n has n complex roots; continuous functions which change sign over a compact real interval have a zero in that interval,...}. The constructive view rejects the law of the excluded middle and thus the position that every statement is seen as true or false, independently of any evidence either way. \cite{thompson_type_1991} (p. 61): \textit{The constructive view of logic concentrates on what it means to prove or to demonstrate convincingly the validity of a statement, rather than concentrating on the abstract truth conditions which constitute the semantic foundation of classical logic}.

To prove a conjunction $A \land B$ we need prove both $A$ and $B$, to prove $A \lor B$ we need to prove one of $A, B$ and know which we have proved. This shows that the law of the excluded middle can not hold in a constructive approach because we have no means of going from a proof to its negation. Implication $A \Rightarrow B$ in constructive position is a transformation of a proof $A$ into a proof $B$: it is a function which transforms proofs of $A$ into proofs of $B$. The constructive approach also forces us to rethink negation, which is now an implication from some proof to an absurd proposition (bottom): $A \Rightarrow \perp$. Thus a negated formula has no computational context and the classical tautology $\neg \neg A \Rightarrow A$ is then obviously no longer valid.  Constructively solving this would require us to be able to effectively compute / decide whether a proposition is true or false - which amounts to solving the halting problem, which is not possible in the general case.

A very important concept in constructivism is that of finitary representation / description. Objects which are infinite e.g. infinite sets as in classic mathematics, fail to have computational computation, they are not computable. This leads to a fundamental tenet in constructive mathematics: \cite{thompson_type_1991} (p. 62): \textit{Every object in constructive mathematics is either finite [..] or has a finitary description}

%TODO: discuss intensionality vs extensionality with Thorsten
%HOTT book, NOTES on chapter 1: "Extensional theory makes no distinction between judgmental and propositional equality, the intensional theory regards judgmental equality as purely definitional, and admits a much broader proof-relevant interpretation of the identity type that is central to the homotopy interpretation."
%Propositional equality allows to assume that a variable x of type p is equal to y: p : x = y.
%Judgemental equality (or definitional equality) means "equal by definition" e.g. if we have a function $f : N -> N by f(x) = x^2$ then f(3) is equal to $3^2$ by definition. Whether or not two expressions are equal by definition is just a matter of expanding out the definitions, in particular it is algorithmically decidable.

Concluding, we can say that constructive mathematics is based on principles quite different from classical mathematics, with the idealistic aspects of the latter replaced by a finitary system with computational content. Objects like functions are given by rules, and the validity of an assertion is guaranteed by a proof from which we can extract relevant computational information, rather than on idealist semantic principles. 

All this is directly reflected in dependently typed programs as we introduced above: functions need to be total (finitary) and produce proofs like in \textit{checkEqNat} which allows the compiler to extract additional relevant computational information. Also the way we described the (infinite) natural numbers was in an finitary way. In the case of decidable equality, the case where it is not equal, we need to provide an actual proof of contradiction, with the type of Void which is Idris representation of $\perp$. 

%TODO: connect intensionality to above examples, discuss with Thorsten
%Fact: Idris, Agda and Coq are intensional
%extensionality can also be seen as: two functions are equal if they give the same results on every input (extensional equality on the function space)

\section{Dependent Types in Agent-Based Simulation}
\label{sec:dep_absconcepts}

After having established the concepts of dependent types, we want to briefly discuss ideas where and how they could be made of use in ABS. We expect that dependent types will help ruling out even more classes of bugs at compile time and encode even more invariants. Additionally by constructively implementing model specifications on the type level could allow the ABS community to reason about a model directly in code.

By definition, ABS is constructive, as described by Epstein \cite{epstein_chapter_2006}: "If you can't grow it, you can't explain it" - thus an agent-based model and the simulated dynamics of it is itself a constructive proof which explain a real-world phenomenon sufficiently well. Although Epstein certainly wasn't talking about a constructive proof in any mathematical sense in this context, dependent types \textit{might} be a perfect match and correspondence between the constructive nature of ABS and programs as proofs.

When we talk about dependently typed programs to be proofs, then we also must attribute the same to dependently typed agent-based simulations, which are then constructive proofs as well. The question is then: a constructive proof of what? It is not entirely clear \textit{what we are proving} when we are constructing dependently typed agent-based simulations. Probably the answer might be that a dependently typed agent-based simulation is then indeed a constructive proof in a mathematical sense, explaining a real-world phenomenon sufficiently well - we have closed the gap between a rather informal constructivism as mentioned above when citing Epstein who certainly didn't mean it in a constructive mathematical sense, and a formal constructivism, made possible by the use of dependent types.

In the following subsections we will discuss related work in this field \ref{sub:dep_abs_relwork}, discuss general concepts where dependent types might be of benefit in ABS \ref{sub:dep_abs_generalconcepts}, present a dependently typed implementation of a 2d discrete environment \ref{sub:dep_abs_2denv} and finally discuss potential use of dependent types in the SIR model \ref{sub:dep_abs_sir} and SugarScape model \ref{sub:dep_abs_sugarscape}.

\subsection{Related Work}
\label{sub:dep_abs_relwork}
In \cite{botta_functional_2011} the authors are using functional programming as a specification for an agent-based model of exchange markets but leave the implementation for further research where they claim that it requires dependent types. This paper is the closest usage of dependent types in agent-based simulation we could find in the existing literature and to our best knowledge there exists no work on general concepts of implementing pure functional agent-based simulations with dependent types. As a remedy to having no related work to build on, we looked into works which apply dependent types to solve real world problems from which we then can draw inspiration from. 

The paper \cite{brady_correct-by-construction_2010} discusses depend types to implement correct-by-construction concurrency in the Idris language \cite{brady_idris_2013}. The authors introduce the concept of a Embedded Domain Specific Language (EDSL) for concurrently locking/unlocking and reading/writing of resources and show that an implementation and formalisation are the same thing when using dependent types. We can draw inspiration from it by taking into consideration that we might develop a EDSL in a similar fashion for specifying general commands which agents can execute. The interpreter of such a EDSL can be pure itself and doesn't have to run in the IO Monad as our previous research (TODO: cite my PFE paper) has shown that ABS can be implemented pure.

In \cite{brady_idris_2011} the authors discuss systems programming with focus on network packet parsing with full dependent types in the Idris language \cite{brady_idris_2013}. Although they use an older version of it where a few features are now deprecated, they follow the same approach as in the previous paper of constructing an EDSL and and writing an interpreter for the EDSL. In a longer introduction of Idris the authors discus its ability for termination checking in case that recursive calls have an argument which is structurally smaller than the input argument in the same position and that these arguments belong to a strictly positive data type. We are particularly interested in whether we can implement an agent-based simulation which termination can be checked at compile time - it is total.

In \cite{brady_programming_2013} the author discusses programming and reasoning with algebraic effects and dependent types in the Idris language \cite{brady_idris_2013}. They claim that monads do not compose very well as monad transformer can quickly become unwieldy when there are lots of effects to manage. As a remedy they propose algebraic effects and implement them in Idris and show how dependent types can be used to reason about states in effectful programs. In our previous research (TODO: cite my PFE paper) we relied heavily on Monads and transformer stacks and we indeed also experienced the difficulty when using them. Algebraic effects might be a promising alternative for handling state as the global environment in which the agents live or threading of random-numbers through the simulation which is of fundamental importance in ABS. Unfortunately algebraic effects cannot express continuations (according to the authors of the paper) which is but of fundamental importance for pure functional ABS as agents are on the lowest level built on continuations - synchronous agent interactions and time-stepping builds directly on continuations. Thus we need to find a different representation of agents - GADTs seem to be a natural choice as all examples build heavily on them and they are very flexible.

In \cite{fowler_dependent_2014} the authors apply dependent types to achieve safe and secure web programming. This paper shows how to implement dependent effects, which we might draw inspiration from of how to implement agent-interactions which, depending on their kind, are effectful e.g. agent-transactions or events.

In \cite{brady_state_2016} the author introduces the ST library in Idris, which allows a new way of implementing dependently typed state machines and compose them vertically (implementing a state machine in terms of others) and horizontally (using multiple state machines within a function). In addition this approach allows to manage stateful resources e.g. create new ones, delete existing ones. We can draw further inspiration from that approach on how to implement dependently typed state machines, especially composing them hierarchically, which is a common use case in agent-based models where agents behaviour is modelled through hierarchical state-machines. As with the Algebraic Effects, this approach doesn't support continuations (TODO: is this so?), so it is not really an option to build our architecture for our agents on it, but it may be used internally to implement agents or other parts of the system. What we definitely can draw inspiration from is the implementation of the indexed Monad \textit{STrans} which is the main building block for the ST library.

The book \cite{brady_type-driven_2017} is a great source to learn pure functional dependently typed programming and in the advanced chapters introduces the fundamental concepts of dependent state machine and dependently typed concurrent programming on a simpler level than the papers above. One chapter discusses on how to implement a messaging protocol for concurrent programming, something we can draw inspiration from for implementing our synchronous agent interaction protocols.

In \cite{sculthorpe_safe_2009} the authors apply dependent types to FRP to avoid some run-time errors and implement a dependently typed version of the Yampa library in Agda. FRP was the underlying concept of implementing agent-based model we took on in our previous approach (TODO: cite). We could have taken the same route and lift FRP into dependent types but we chose explicitly to not go into this direction and look into complementing approaches on how to implement agent-based models.

The fundamental difference to all these real-world examples is that in our approach, the system evolves over time and agents act over time. A fundamental question will be how we encode the monotonous increasing flow of time in types and how we can reflect in the types that agents act over time.

%An agent can be seen as a potentially infinite stream of continuations which at some point could return information to stop evaluating the next item of the stream which allows an agent to terminate.
%correspondence between temporal logics and FRP due to jeffery: is abs just another temporal logic?

%The authors of \cite{ionescu_dependently-typed_2012} discuss how to use dependent types to specify fundamental theorems of economics, unfortunately they are not computable and thus not constructive, thus leaving it more to a theoretical, specification side.
%Ionesus talk on dependently typed programming in scientific computing
%https://www.pik-potsdam.de/members/ionescu/cezar-ifl2012-slides.pdf
%Ionescus talk on Increasingly Correct Scientific Computing
%%https://www.cicm-conference.org/2012/slides/CezarIonescu.pdf
%Ionescus talk on Economic Equilibria in Type Theory
%https://www.pik-potsdam.de/members/ionescu/cezar-types11-slides.pdf
%Ionescus talk on Dependently-Typed Programming in Economic Modelling
%https://www.pik-potsdam.de/members/ionescu/ee-tt.pdf

\subsection{General Concepts}
\label{sub:dep_abs_generalconcepts}

We came up with the following ideas of how and where to apply dependent types in the context of agent-based simulation:

%Randomness is of central importance in agent-based simulation but nothing enforces from which distribution to draw. With dependent types we might to implement probabilistic types which can encode probability distributions in types already about which we can then reason and guarantee at compile-time that we draw from the correct distribution.

% encode dynamics in the types (what? feedbacks? positive/negative) on a meta-level

\paragraph{Environment Access}
Accessing e.g. discrete 2D environments involves (almost always) indexed array access which is always potentially dangerous as the indices have to be checked at run-time.

Using dependent types it should be possible to encode the environment dimensions into the types. In combination with suitable data types (finite sets) for coordinates one should be able to ensure already at compile time that access happens only within the bounds of the environment. We have implemented this in detail already and describe it in the section \ref{sub:dep_abs_2denv} below.

\paragraph{State-Machines}
Often, Agent-Based Models define their agents in terms of state-machines. It is easy to make wrong state-transitions e.g. in the SIR model when an infected agent should recover, nothing prevents one from making the transition back to susceptible. 

Using dependent types it might be possible to encode invariants and state-machines on the type level which can prevent such invalid transitions already at compile time. This would be a huge benefit for ABS because of the popularity of state-machines in agent-based models. Here we can draw inspiration from the paper \cite{brady_state_2016} and book \cite{brady_type-driven_2017}.

\paragraph{Flow Of Time}
State-Machines often have timed transitions e.g. in the SIR model, an infected agent recovers after a given time. Nothing prevents us from introducing a bug and \textit{never} doing the transition at all.

With dependent types we might be able to encode the passing of time in the types and guarantee on a type level that an infected agent has to recover after a finite number of time steps. Also can dependent types be used to express the flow of time and its strongly monotonic increasing?
	
\paragraph{Existence Of Agents}
In more sophisticated models agents interact in more complex ways with each other e.g. through message exchange using agent IDs to identify target agents. The existence of an agent is not guaranteed and depends on the simulation time because agents can be created or terminated at any point during simulation. 

Dependent types could be used to implement agent IDs as a proof that an agent with the given id exists \textit{at the current time-step}. This also implies that such a proof cannot be used in the future, which is prevented by the type system as it is not safe to assume that the agent will still exist in the next step. So proof of the existence of an agent: holds always only for the current time-step or for all time, depending on the model. e.g. in the SIR model no agents are removed from / added to the system thus a proof holds for all time. 

\paragraph{Agent-Agent Interactions}
dependently typed message protocols in ABS because its very common, and easily done thorugh methods in OOP: sugarscape mating and trading protocol
using a DSEL \cite{brady_correct-by-construction_2010} to restrict the available primitives in the message protocol?
Data Flow: can dependent types be used in the Data Flow Mechanism?
Event Scheduling: ependent types be used in the event-scheduling mechanism?

Using dependent types we might be able to encode a protocol for agent-agent interactions which e.g. ensures on the type-level that an agent has to reply to a request or that a more specific protocol has to be followed e.g. in auction- or trading-simulations.

\paragraph{Equilibrium and Totality}
For some agent-based simulations there exists equilibria, which means that from that point the dynamics won't change any more e.g. when a given type of agents vanishes from the simulation or resources are consumed. This means that at that point the dynamics won't change any more, thus one can safely terminate the simulation. But still such simulations are stepped for a fixed number of time-steps or events or the termination criterion is checked at run-time in the feedback-loop. 
	
Using dependent types it might be possible to encode equilibria properties in the types in a way that the simulation automatically terminates when they are reached. This results then in a \textit{total} simulation, creating a correspondence between the equilibrium of a simulation and the totality of its implementation. Of course this is only possible for models in which we know about their equilibria a priori or in which we can reason somehow that an equilibrium exists.

A central question in tackling this is whether to follow a model- or an agent-centric approach. The former one looks at the model and its specifications as a whole and encodes them e.g. one tries to directly find a total implementation of an agent-based model. The latter one looks only at the agent level and encodes that as dependently typed as possible and hopes that model guarantees emerge on a meta-level - put otherwise: does the totality of an implementation emerge when we follow an agent-centric approach?

\paragraph{Hypotheses}
Models which are exploratory in nature don't have a formal ground truth where one could derive equilibria or dynamics from and validate with. In such models the researchers work with informal hypotheses which they express before running the model and then compare them informally against the resulting dynamics.

It would be of interest if dependent types could be made of use in encoding hypotheses on a more constructive and formal level directly into the implementation code. So far we have no idea how this could be done but it might be a very promising approach as it allows for a more formal and automatic testable approach to hypothesis checking.

\subsection{Dependently Typed Discrete 2D Environment}
\label{sub:dep_abs_2denv}
One of the main advantages of Agent-Based Simulation over other simulation methods e.g. System Dynamics is that agents can live within an environment. Many agent-based models place their agents within a 2D discrete NxM environment where agents either stay always on the same cell or can move freely within the environment where a cell has 0, 1 or many occupants. Ultimately this boils down to accessing a NxM matrix represented by arrays or a similar data structure. In imperative languages accessing memory always implies the danger of out-of-bounds exceptions \textit{at run-time}. With dependent types we can represent such a 2d environment using vectors which carry their length in the type (TODO: discuss them in background) thus fixing the dimensions of such a 2D discrete environment in the types. This means that there is no need to drag those bounds around explicitly as data. Also by using dependent types like Fin which depend on the dimensions we can enforce at compile time that we can only access the data structure within bounds. If we want to we can also enforce in the types that the environment will never be an empty one where $N, M > 0$.

\begin{HaskellCode}
Disc2dEnv : (w : Nat) -> (h : Nat) -> (e : Type) -> Type
Disc2dEnv w h e = Vect (S w) (Vect (S h) e)

data Disc2dCoords : (w : Nat) -> (h : Nat) -> Type where
  MkDisc2dCoords : Fin (S w) -> Fin (S h) -> Disc2dCoords w h
  
centreCoords : Disc2dEnv w h e -> Disc2dCoords w h
centreCoords {w} {h} _ =
    let x = halfNatToFin w
        y = halfNatToFin h
    in  mkDisc2dCoords x y
  where
    halfNatToFin : (x : Nat) -> Fin (S x)
    halfNatToFin x = 
      let xh   = divNatNZ x 2 SIsNotZ 
          mfin = natToFin xh (S x)
      in  fromMaybe FZ mfin
      
setCell :  Disc2dCoords w h
        -> (elem : e)
        -> Disc2dEnv w h e
        -> Disc2dEnv w h e
setCell (MkDisc2dCoords colIdx rowIdx) elem env 
    = updateAt colIdx (\col => updateAt rowIdx (const elem) col) env
 
getCell :  Disc2dCoords w h
        -> Disc2dEnv w h e
        -> e
getCell (MkDisc2dCoords colIdx rowIdx) env
    = index rowIdx (index colIdx env)
    
neumann : Vect 4 (Integer, Integer)
neumann = [         (0,  1), 
           (-1,  0),         (1,  0),
                    (0, -1)]

moore : Vect 8 (Integer, Integer)
moore = [(-1,  1), (0,  1), (1,  1),
         (-1,  0),          (1,  0),
         (-1, -1), (0, -1), (1, -1)]

filterNeighbourhood :  Disc2dCoords w h
                    -> Vect len (Integer, Integer)
                    -> Disc2dEnv w h e 
                    -> (n ** Vect n (Disc2dCoords w h, e))
filterNeighbourhood {w} {h} (MkDisc2dCoords x y) ns env =
    let xi = finToInteger x
        yi = finToInteger y
    in  filterNeighbourhood' xi yi ns env
  where
    filterNeighbourhood' :  (xi : Integer)
                         -> (yi : Integer)
                         -> Vect len (Integer, Integer)
                         -> Disc2dEnv w h e 
                         -> (n ** Vect n (Disc2dCoords w h, e))
    filterNeighbourhood' _ _ [] env = (0 ** [])
    filterNeighbourhood' xi yi ((xDelta, yDelta) :: cs) env 
      = let xd = xi - xDelta
            yd = yi - yDelta
            mx = integerToFin xd (S w)
            my = integerToFin yd (S h)
        in case mx of
            Nothing => filterNeighbourhood' xi yi cs env 
            Just x  => (case my of 
                        Nothing => filterNeighbourhood' xi yi cs env 
                        Just y  => let coord      = MkDisc2dCoords x y
                                       c          = getCell coord env
                                       (_ ** ret) = filterNeighbourhood' xi yi cs env
                                   in  (_ ** ((coord, c) :: ret)))
\end{HaskellCode}

\subsection{Dependently Typed SIR}
\label{sub:dep_abs_sir}
We plan to prototype the concepts of section \ref{sub:dep_abs_generalconcepts} in a dependently typed SIR implementation. One can object that the SIR model \cite{kermack_contribution_1927} is a very simple model but it has a number of advantages. There is a theory behind it with a formal ground-truth for the dynamics which can be generated by differential equations, which allows validation of the simulation. It has already many concepts of ABS in it without making it too complex: agent-behaviour as a state-machine, local agent-state (current SIR state and duration of illness), feedback, very rudimentary interaction with other agents, 2D environment if required and behaviour over time. We will also look into the SugarScape model below \ref{sub:dep_abs_sugarscape}, which is of quite a different type and adds more complexity.
The general approach of using dependent types is to specify the general commands available for an agent, where we can follow the approach of an DSEL as described in \cite{brady_correct-by-construction_2010} and write then an interpreter for it. It is of importance that the interpreter shall be pure itself and does not make use of any IO.

Applying dependent types to the SIR model, we came up with the following use-cases:

\paragraph{Environment access}
We have already introduced an implementation for a dependently typed 2D environment in section \ref{sub:dep_abs_2denv}. This can be directly used to implement a SIR on a 2D environment as we have done in the paper in Appendix \ref{app:pfe}.

\paragraph{State-Machine and Flow Of Time}
The transition through the Susceptible, Infected and Recovered states are a state-machine, thus we want to apply dependent types to restrict the valid transitions and that they are enforced under the given circumstances. The transitions are restricted to Susceptibles can only transition to Infected, Infected only to Recover and Recover stay in that state forever. The transitions obviously depend on the model and should be enforced. A transition from Susceptible to Infected happens with a given probability in case the Susceptible makes contact with an Infected. The transition from Infected to Recover happens after a given number of time-steps.
The tricky thing is that all these transitions ultimately depend on stochastic events: Susceptible pick their contacts at random, uniformly distributed from all agents in the simulation, they get infected with a probability when the contact is Infected and the duration an Infected agent is ill is picked from an exponential distribution.

\paragraph{Equilibrium and totality}
The idea is to implement a total agent-based SIR simulation, where the termination does NOT depend on time (is not terminated after a finite number of time-steps, which would be trivial).  We argue that the underlying SIR model actually has a steady state.

The dynamics of the system-dynamics SIR model are in equilibrium (won't change any more) when the infected stock is 0. This can (probably) be shown formally but intuitively it is clear because only infected agents can lead to infections of susceptible agents which then make the transition to recovered after having gone through the infection phase. 

Thus an agent-based implementation of the SIR simulation has to terminate if it is implemented correctly because all infected agents will recover after a finite number of steps after then the dynamics will be in equilibrium. Thus we have the following conditions for totality:
\begin{enumerate}
	\item The simulation shall terminated when there are no more infected agents.
	\item All infected agents will recover after a finite number of time, which means that the simulation will eventually run out of infected agents. 
	
	Unfortunately this criterion alone does not suffice because when we look at the SIR+S (which adds a cycle from Recovered back to Susceptible) model we have the same termination criterion, but we cannot guarantee that it will run out of infected. We need an additional criteria.
	\item The source of infected agents is the pool of susceptible agents which is monotonous decreasing (not strictly though!) because recovered agents do NOT turn back into susceptibles.
\end{enumerate}

Thus we can conclude that a SIR model must enter a steady state after finite steps / in finite time \footnote{Note that there exists a SIR+S model, which adds a cycle back from Recovered to Susceptible - if we find a total implementation of the SIR model and add this transition then the simulation should become non-total, checked by the compiler.}.

By this reasoning, a non-total, correctly implemented agent-based simulations of the SIR model will eventually terminate (note that this is independent of which environment is used and which parameters are selected). Still this does not proof that the agent-based approach itself will terminate and so far no proof of the totality of it was given.

Dependent Types and Idris ability for totality and termination checking should theoretically allow us to proof that an agent-based SIR implementation terminates after finite time: if an implementation of the agent-based SIR model in Idris is total it is a proof by construction. Note that such an implementation should not run for a limited virtual time but run unrestricted of the time and the simulation should terminate as soon as there are no more infected agents. 

The HOTT book \cite{program_homotopy_2013} states that lists, trees,... are inductive types/inductively defined structures where each of them is characterized by a corresponding "induction principle". For a proof of totality of SIR we need to find the "induction principle" of the SIR model and implement it. What is the inductive, defining structure of the SIR model? is it a tree where a path through the tree is one simulation dynamics? or is it something else? it seems that such a tree would grow and then shrink again e.g. infected agents. Can we then apply this further to (agent-based) simulation in general?

%TODO: \url{https://stackoverflow.com/questions/19642921/assisting-agdas-termination-checker/39591118}

%We hypothesize that it should be possible due to the nature of the state transitions where there are no cycles and that all infected agents will eventually reach the recovered state. 
%
%-- TODO: express in the types
%-- SUSCEPTIBLE: MAY become infected when making contact with another agent
%-- INFECTED:    WILL recover after a finite number of time-steps
%-- RECOVERED:   STAYS recovered all the time
%
%-- SIMULATION:  advanced in steps, time represented as Nat, as real numbers are not constructive and we want to be total
%--              terminates when there are no more INFECTED agents

So far we have no clear idea and understanding how to implement such a total implementation - this will be subject to quite substantial research. One might object to that undertaking and ask what we got from it. We argue that investigating the correspondence between the equilibrium of an agent-based model and the totality of its implementation for the first time is reason enough and we hope to gain a few unique and deep insights from this undertaking.

\subsection{Dependently Typed Sugarscape}
\label{sub:dep_abs_sugarscape}
The other model we will employ as a use-case for the concepts of section \ref{sub:dep_abs_generalconcepts} is the SugarScape model \cite{epstein_growing_1996}. It is an exploratory model by which social scientists tried to explain phenomena found in societies in the real world. The main complexity of this model lies in the much more complex local state of the agents, complex agent-agent interactions e.g. in case of trade and mating and a pro-active environment. Opposed to the SIR model agents behaviour is not modelled as a state-machine and time-semantics is not of that much importance: the simulation is stepped in unit-steps of $\delta t = 1.0$ and in every time-step, all agents act in random order. Although there are equilibria e.g. in case all agents die out or the carrying capacity of an environment, trading prices, this model is too complex for a total implementation in the cases.

\paragraph{Environment}
We have already introduced an implementation for a dependently typed 2D environment in section \ref{sub:dep_abs_2denv}. This can be directly used to implement the pro-active environment of SugarScape.

\paragraph{Existence Of Agents}
TODO: existence of an agent
In SugarScape agents can die and be born thus on a technical level agents are added and removed dynamically during the simulation. This means we can employ proofs of existence of an agent for establishing interactions with another one. Also a proof might become invalid after a time. Also one can construct a proof only from a given time on e.g. when one wants to prove that agent X exists but agent X is only created at time t then before time t the prove cannot be constructed and is uninhabited and only inhabited from time t on.

\paragraph{Agent-Agent interactions}
In SugarScape agents interact with each other on a much more complex way than in SIR due to the complex behaviour. The two main complex use-cases are mating and trading between agents where both require multiple interaction-steps happening instantaneous without delay (within 1.0 time-step). Both use-cases implement a protocol which we might be able to enforce using dependent types.

\paragraph{Hypotheses}
SugarScape is an exploratory model and it has strictly speaking no analytical or theoretical ground truth. Thus there are no means to validate this model and the researcher works by formulating hypotheses about the emergent properties of the model. So the approach the creators of SugarScape took in \cite{epstein_growing_1996} was that they started from real world phenomenon and modelled the agent-interactions for them and hypothesized that out of this the real-world phenomenon will emerge. An example is the carrying capacity of an environment, as described in the first chapter: they hypothesized that the size of the population will reach a state where it will fluctuate around some mean because the environment cannot sustain more than a given number so agents not finding enough resources will simply die. Maybe we can encode such hypotheses using dependent types.

\subsection{Testing, Verification and Dependent Types}
Dependent Types do not mean that tests are obsolete, we still need to test all things at run-time using property-tests which we cannot encode in types.
todo: connection between black-box verification and dependent types
todo: connection between white-box verification and dependent types

\section{Verification, Validation and Dependent Types}
\label{sec:dep_vav_deptypes}
Dependent types allow to encode specifications on an unprecedented level, narrowing the gap between specification and implementation - ideally the code becomes the specification, making it correct-by-construction. The question is ultimately how far we can formulate model specifications in types - how far we can close the gap in the domain of ABS. Unless we cannot close that gap completely, to arrive at a sufficiently confidence in correctness, we still need to test all properties at run-time which we cannot encode at compile-time in types.

Nonetheless, dependent types should allow to substantially reduce the amount of testing which is of immense benefit when testing is costly. Especially in simulations, testing and validating a simulation can often take many hours - thus guaranteeing properties and correctness already at compile time can reduce that bottleneck substantially by reducing the number of test-runs to make.

Ultimately this leads to a very different development process than in the established object-oriented approaches, which follow a test-driven process. There one defines the necessary interface of an object with empty implementations for a given use-case first, then writes tests which cover all possible cases for the given use-case. Obviously all tests should fail because the functionality behind it was not implemented yet. Then one starts to implement the functionality behind it  step-by-step until no test-case fails. This means that one runs all tests repeatedly to both check if the test-case one is working on is not failing anymore and to make sure that old test-cases are not broken by new code. The resulting software is then trusted to be correct because no counter examples through test hypotheses, could be found. The problem is: we could forget / not think of cases, which is the easier the more complex the software becomes (and simulations are quite complex beasts). Thus in the end this is a deductive approach.

With pure functional programming and dependent types the process is now mostly constructive, type-driven (see \cite{brady_type-driven_2017}). In that approach one defines types first and is then guided by these types and the compiler in an interactive fashion towards a correct implementation, ensured at compile-time. As already noted, the ABS methodology is constructive in nature but the established object-oriented test-driven implementation approach not as much, creating an impedance mismatch. We expect that a type-driven approach using dependent types reduces that mismatch by a substantial amount.

Note that \textit{validation} is a different matter here: independent of our implementation approach we still need to validate the simulation against the real-world / ground-truth. This obviously requires to run the full simulation which could take up hours in either programming paradigm, making them absolutely equal in this respect. Also the comparison of the output to the real-world / ground-truth is completely independent to the paradigm. The fundamental difference happens in case of changes made to the code during validation: in case of the established test-driven object-oriented approach for every minor change one (should) re-run all tests, which could take up a substantial amount of additional time. Using a constructive, type-driven approach this is dramatically reduced and can often be completely omitted because the correctness of the change can be either guaranteed in the type or by informally reasoning about the code.

%-------------------------
%TODO: not sure where to put this
%ABS as a constructive / generative science, follows Poperian approach of falsification: we try to construct a model which explains a real-world (empirical) phenomenon - if validation shows that the generated dynamics match the ones of the real-world sufficiently enough, we say that we have found \textit{a} hypothesis (the model) which emergent properties explains the real-world phenomenon sufficiently enough. This is not a proof but only one possible explanation which holds for now and might be falsified in the future.
%
%When we implement our simulation things change a bit as we add another layer: the conceptual model, describing the phenomenon, which is an abstraction of reality. This description can be of many forms but can be regarded on a line between completely formal (economic models) to informal (sociology) but the implementation will follow that description. The fundamental difference here is that in this case we want our implementation to be exactly the same as the conceptual model. Contrary to the real-world, where it is not possible to find a \textit{true} model (as was argued by Popper), on this level we actually can construct an implementation which matches the conceptual model exactly because we have a description of the conceptual model. In the end we transform the conceptual model description in code, which is itself a formal description. In this translation process (speak: implementation / programming), one can make an endless number of mistakes. Generally we can distinguish between two classes of mistakes: 
%1) conceptual mistakes - wrong translation of the model specifications into code due to various reasons e.g. imprecise description, human error. The more precise an unambiguous a model description is, the less probable conceptual mistakes will be.
%2) internal mistakes - normal programming mistakes e.g. access of arrays out of bounds, ... also using correlated Random Number generators.
%
%Level 0: Real-World phenomenon
%Level 1: Conceptual model of the real-world phenomenon
%Level 2: Implementation of the conceptual model
%
%Note that we must speak of falsification and constructiveness on two different levels:
%- validation level: do the results of the conceptual model match the real-world phenomenon? the conceptual model is the hypothesis which says that its mechanics are sufficient to generate / construct the real-world phenomenon. At this level we are not interested in the implementation level anymore - the implemented model \textit{is} (seen as) the conceptual model, and one only compares its output to the real-world. If the dynamics match, then we got a valid hypothesis which works for now. If the dynamics do NOT match, then the hypothesis (the model) is falsified and one needs to adjust / change the hypothesis (model). The validation will happen by tests, there is no other way, we have no formal specification of the real-world, we can only observe empirically the phenomena, so we run tests which try to falsify the outputs of the model: assuming it will generate phenomena of the real-world and test if it does.
%- implementation \& verficiation level: in this step we are matching the code to the conceptual model. Here we are not only restricted to a test-driven approach because we have a more or less formal description of the conceptual model which we directly encode in our programming language. If the language allows to express model specifications already at compile-time then this means that the implementation narrows the gap between model specification and implementation which means it does not need to be tested at run-time because it is guaranteed for all inputs for all time. 
%
%The constructiveness of ABS and impendance mismatch: ABS methodology is constructive but the established implementation approach not too much, creating an impedance mismatch. this is especially visible in the test-driven development dependent types constructive nature could close this mismatch.
%

%todo: connection between black-box verification and dependent types
%todo: connection between white-box verification and dependent types