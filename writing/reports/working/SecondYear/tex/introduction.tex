\chapter{Introduction}
\label{chap:intro}

In the first half of the phd I have investigated \textit{how} to do agent-based simulation in pure functional programming. The remainder of the Phd will be to research \textit{why} this approach is of benefit. In general it is well established, that pure functional programming as in Haskell, allows to express much stronger guarantees about the correctness of a program \textit{already at compile-time}. This is in fundamental contrast to imperative object-oriented languages like Java or Python where only primitive guarantees about types - mostly relationships between type-hierarchies - can be expressed at compile time which directly implies that one needs to perform much more testing (user testing or unit testing) at \textit{run-time} to check whether the model is sufficiently correct. Thus guaranteeing properties already at compile-time frees us from writing unit tests which cover these cases or test them at run time because they are \textit{guaranteed to be correct under all circumstances, for all inputs}. In this regards we see pure functional programming as truly superior to the traditional object oriented approaches: they lead to implementations of models which are more likely correct because we can express more guarantees already at compile time which directly leads to less bugs which directly increases the probability of the software being a correct implementation of the model.
The ultimate level of this is reached when using dependent types in which types are first class citizens and can be computed at compile time. 
Dependent Types allow us to express even more guarantees at compile time making even more run-time tests obsolete (note: this doesn't make testing in its entirety obsolete!). Generally one can say: what can guaranteed at compile-time doesn't need a test and saves one from a potentially infinite amount (TODO: make more clear) of tests as dijkstra points out: "Testing shows the presence, not the absence of bugs".

In general, Types guide us in program construction by restricting the operations we can perform on the data. This means that by choosing types this reveals already a lot of our program and data and prevents us from making mistakes e.g. interpreting some binary data as text instead of a number. In strongly statically typed languages the types can do this already at compile time which allows to rule out certain bugs already at compile time. In general, we can say that for all bugs which can be ruled out at compile time, we don't need to write property- or unit-tests for them, because those bugs cannot - per definition - occur at run-time, so it won't make sense to test their absence at run-time. Also, as Dijkstra famously put it: "Testing shows the presence, not the absence of bugs" - thus by induction we can claim that compile time guarantees save us from a potentially infinite amount of testing.

Dependent types allow to push these guarantees to a new extreme where we can express nearly arbitrary complex guarantees at compile time because we can \textit{compute types at compile time}. The main requirement here is, that the computation is total. So dependent types go as far as seen as equivalent to mathematical proofs, which in the end allows to guarantee so much properties at compile time so as to make it unnecessary to run the program because the compilation already shows its correctness.
\\

Feedback from the recently submitted conference paper will come soon and then we should organise a joint supervision meeting with Thorsten and discuss the comments from the referees.

\subsection{Dependent Types}
We see this paper as an intermediary and necessary step towards dependent types for which we first needed to understand the potential and limitations of a non-dependently typed pure functional approach in Haskell. Dependent types are extremely promising in functional programming as they allow us to express stronger guarantees about the correctness of programs and go as far as allowing to formulate programs and types as constructive proofs which must be total by definition \cite{thompson_type_1991, mckinna_why_2006, altenkirch_pi_2010}.

So far no research using dependent types in agent-based simulation exists at all. In our next paper we want to explore this for the first time and ask more specifically how we can add dependent types to our pure functional approach, which conceptual implications this has for ABS and what we gain from doing so. We plan on using Idris \cite{brady_idris_2013} as the language of choice as it is very close to Haskell with focus on real-world application and running programs as opposed to other languages with dependent types e.g. Agda and Coq which serve primarily as proof assistants.

We hypothesize that dependent types could help ruling out even more classes of bugs at compile time and encode invariants and model specifications on the type level, which implies that we don't need to test them using e.g. property-testing with QuickCheck. This would allow the ABS community to reason about a model directly in code:

\begin{itemize}
	\item Accessing the environment in section \ref{sec:step5_environment} involves indexed array access which is always potentially dangerous as the indices have to be checked at run-time. 
	
	Using dependent types it should be possible to encode the environment dimensions into the types. In combination with suitable data types for coordinates one should be able to ensure already at compile time that access happens only within the bounds of the environment.
	
	\item In the SIR implementation one could make wrong state-transitions e.g. when an infected agent should recover, nothing prevents one from making the transition back to susceptible. 
	
	Using dependent types it might be possible to encode invariants and state-machines on the type level which can prevent such invalid transitions already at compile time. This would be a huge benefit for ABS because many agent-based models define their agents in terms of state-machines.
	
	\item An infected agent recovers after a given time - the transition of infected to recovered is a timed transition. Nothing prevents us from \textit{never} doing the transition at all. 
	
	With dependent types we might be able to encode the passing of time in the types and guarantee on a type level that an infected agent has to recover after a finite number of time steps.
	
	\item In more sophisticated models agents interact in more complex ways with each other e.g. through message exchange using agent IDs to identify target agents. The existence of an agent is not guaranteed and depends on the simulation time because agents can be created or terminated at any point during simulation. 
	
	Dependent types could be used to implement agent IDs as a proof that an agent with the given id exists \textit{at the current time-step}. This also implies that such a proof cannot be used in the future, which is prevented by the type system as it is not safe to assume that the agent will still exist in the next step.
	%\item Using dependent types we should be able to encode a protocol for agent-interactions which e.g. ensures on the type-level that an agent has to reply to a request or that a more specific protocol has to be followed e.g. in auction- or trading-simulations.
	
	\item In our implementation, we terminate the SIR model always after a fixed number of time-steps. We can informally reason that restricting the simulation to a fixed number of time-steps is not necessary because the SIR model \textit{has to} reach a steady state after a finite number of steps. This means that at that point the dynamics won't change any more, thus one can safely terminate the simulation. Informally speaking, the reason for that is that eventually the system will run out of infected agents, which are the drivers of the dynamic. We know that all infected agents will recover after a finite number of time-steps \textit{and} that there is only a finite source for infected agents which is monotonously decreasing. 
	
	Using dependent types it might be possible to encode this in the types, resulting in a total simulation, creating a correspondence between the equilibrium of a simulation and the totality of its implementation. Of course this is only possible for models in which we know about their equilibria a priori or in which we can reason somehow that an equilibrium exists.
\end{itemize}

\section{Motivation}
The observations of the problems presented in the previous section leads us to posing fundamental directions and questions which are the basis of the motivation of our thesis.

\begin{enumerate}
	\item \textbf{Alternative approach to object-oriented ABMS} - Is there an alternative view to the established object-oriented view to ABMS which does treat agents \textit{not} like objects and does not mix the concept of agent and object? Is there an alternative to the established object-oriented implementation approach to ABMS which offers composability, explicit data-flow?
	\item \textbf{Verification \& Validation of ABMS} - Is there a way for formal specification and verification which is still readable and does not fall back to pure mathematics? What exactly is the meaning of validation in ABMS and is there a way to do formal validation in ABMS? 
\end{enumerate}