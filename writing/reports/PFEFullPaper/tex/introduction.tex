\section{Introduction}
The traditional approach to Agent-Based Simulation (ABS) has so far always been object-oriented techniques due to the influence of the seminal work of Epstein et al \cite{epstein_growing_1996} in which the authors claim "[..] object-oriented programming to be a particularly natural development environment for Sugarscape specifically and artificial societies generally [..]" (p. 179). This work established the metaphor in the ABS community, that \textit{agents map naturally to objects} \cite{north_managing_2007} which still holds up today.
In this paper we fundamentally challenge this metaphor and explore ways of approaching ABS in a pure functional way using Haskell. By doing this we expect to leverage the benefits of pure functional programming \cite{hudak_history_2007}: higher expressivity through declarative code, being polymorph and explicit about side-effects through monads, more robust and less susceptible for bugs due to explicit data flow and lack of implicit side-effects.
As use-case we introduce the simple SIR model of epidemiology with which one can simulate epidemics in a realistic way and over the course of five steps we derive all necessary concepts required for a full agent-based implementation. We start from a very simple solution running in the Random Monad which has all general concepts already there and then refine it in various ways, making the transition to Functional Reactive Programming (FRP) \cite{wan_functional_2000} and to Monadic Stream Functions (MSF) \cite{perez_functional_2016}.
The aim of this paper is to show how ABS can be done in \textit{pure} Haskell and what the benefits and drawbacks are. By doing this we give the reader a good understanding of what ABS is, what the challenges are when implementing it and how we solve these in our approach.
The contributions of this paper are:

\begin{itemize}
	\item We are the first to \textit{systematically} introduce the concepts of ABS to the \textit{pure} functional programming paradigm in a step-by-step approach. It is also the first paper to show how to apply arrowized FRP to ABS on a technical level, presenting a new field of application to FRP. Although there exists some work on ABS in Haskell it is not of the conceptual nature this paper exhibits, barely scratches the surface, sacrifices purity and focuses on different problems.
	\item Our approach shows how robustness can be achieved through purity which guarantees reproducibility at compile time, something not possible with traditional object-oriented approaches.
	\item The result of using arrowized FRP is a unique, hybrid approach to ABS because FRP allows expressing continuous time-semantics not possible with the traditional imperative object-oriented approaches.
\end{itemize}

Section \ref{sec:related_work} discusses related work. In section \ref{sec:background} we introduce functional reactive programming, arrowized programming and monadic stream functions because our approach builds heavily on these concepts. Section \ref{sec:defining_abs} defines agent-based simulation and discussese the ideas behind it. In section \ref{sec:sir_model} we introduce the SIR model of epidemiology as an example model to explain the concepts of ABS. The heart of the paper is section \ref{sec:functional_approach} in which we derive the concepts of a pure functional approach to ABS in five steps, using the SIR model. Finally we draw conclusions and discuss issues in section \ref{sec:conclusions} and point to further research \ref{sec:further_research}.

