\chapter{Agent-Agent Interactions}

\section{OO}
In OO we have basically two possibilities to implement Agent-Agent Interactions: either by direct method calls to the other Agent which requires the calling Agent to hold a reference \textit{with the subtype of the callee if using an abstract Agent-Class} OR by adding messages to a message-box (e.g. a HashMap or List) of the receiving Agent. Both have fundamental differences: a direct method call is like transferring the action to the callee: the Agent suddenly becomes active where before the calling Agent was active - as soon as the callee has finished the method, action returns to the callee. Note that the callee can again call other Agents methods which could, if not guarded explicitly against, lead to a cycle if the model-semantics permit it.
When adding a message to a message-box no action is transferred to another Agent. Still a reference to the Agent with type of the abstract Agent-Class must be held if one implements it as a direct access to the mailbox. This has the advantage that it is fast but the disadvantage that we deal again with references and need synchronization in case of parallelism/concurrency. Another option would be to have a outgoing-box into which the Agent adds messages it wants to send and the ABS system handles then the delivery after each step. This has the advantage that no direct references to other Agents are required, only their (numerical) IDs but the disadvantage that this delivery process has O($n^2$) complexity (TODO: prove that or back it up with evidence. could we also reduce it to O (n log n) when using a map? or is it even possible to somehow use O (n)?)

\section{FP}
In FP when staying completely pure the only option we have is an outgoing-box into which messages are queued and the ABS system then distributes them into the ingoing-boxes of the receivers. This is because we have no method calls - of course we could simulate method calls by dragging the complete state around which would allow to execute some message-handler by the receiving Agent within the calling Agent but then the Agents themselves need to be aware of implementation-details and have full access to all other agents - reasoning becomes difficult and robustness will inevitably suffer, so we won't go there.
If we allow explicit side-effects in the form of Monadic programming then we can implement direct access to other Agents mailboxes as in the OO version: either we use references and run in the IO monad or we use STM channels and run in the STM monad. As IO would ruin very much we can reason about the most reasonable approach would be to make use of STM. But as long as there is no real need for concurrency as in the concurrent- and actor-strategies STM is an overkill and we will stick to outgoing boxes. This buys us the reasoning abilities and robustness but hits us with a penalty in performance.