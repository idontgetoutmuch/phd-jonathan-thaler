\chapter{Literature Review}
\label{chap:literature}

In this chapter we present a literature-review which is driven by the motivating questions from the introduction. We split the literature into three sections in which we present relevant sources which pose possible answers and directions of approaches. Based upon this information, in the next chapter we will select our directions and methods for our research, identify the gap we have to bridge with our PhD research and our objectives in achieving to close the gap.

\section{Agent Formalisms}
When looking for ways to verification and validation of ABS, it is of interest to have a more formal representation of agents and ABS. This helps to clarify the key concepts in an unambiguous, formal way which can then be applied to formal methods e.g. algebraic reasoning for proofing properties of an agent system. Wooldridge \cite{wooldridge_introduction_2009}, \cite{wooldridge_intelligent_1995} gives an abstract architecture for intelligent agents, introducing the concept of actions of agents influencing the environment. The book of \cite{weiss_multiagent_2013} offers several chapters on the theoretical foundations and underpinnings of multi-agent systems \footnote{Although all of these references are primarily rooted in the multi-agent system (MAS) field - which is quite distinct from ABS as stressed in the introduction - they can as well be applied to ABS and offer some inspiration from which to draw upon.}
A very useful formalization is given in \cite{klugl_amason:_2013} which applies their formalism as an example to the Sugarscape model \cite{epstein_growing_1996} \footnote{A happy coincidence because we are using Sugarscape as a major use-case for developing our methodology.}.
In \cite{carchiolo_using_2000} the authors use the process calculi CCS and CSP for designing an agent-specification language LOTOS for formal description and verification. This is a hint that process calculi, as presented in the section above, are indeed a feasible tool for formally specifying low-level properties of agents and their interactions.
The authors of \cite{araragi_formal_2000} compare three formal methods for modelling and validating agent systems: Erdös a knowledge-based environment for agent programming, $Nepi^2$ a programming language for agent system based on the $\pi$-calculus implemented in LISP and I/O automata a mathematical framework used for modelling and reasoning about systems with interacting components.
A more software-engineering approach is taken by \cite{dinverno_formal_2000} and provide an agent development line from a formal agent framework, to agent system specification, agent development to agent deployment. 

\subsection{Actor Model}
The Actor-Model, a model of concurrency, was initially conceived by Hewitt in 1973 \cite{hewitt_universal_1973} and refined later on \cite{hewitt_what_2007}, \cite{hewitt_actor_2010}. It was a major influence in designing the concept of agents and although there are important differences between actors and agents there are huge similarities thus the idea to use actors to build agent-based simulations comes quite natural. The theory was put on firm semantic grounds first through Irene Greif by defining its operational semantics \cite{greif_semantics_1975} and then Will Clinger by defining denotational semantics \cite{clinger_foundations_1981}. In the seminal work of Agha \cite{agha_actors:_1986} he developed a semantic mode, he termed \textit{actors} which was then developed further \cite{agha_foundation_1997} into an actor language with operational semantics which made connections to process calculi and functional programming languages (see both below). 

An actor is a uniquely addressable entity which can \textit{in response to a message}
\begin{itemize}
	\item Send an arbitrary number (even infinite) of messages to other actors.
	\item Create an arbitrary number of actors.
	\item Define its own behaviour upon reception of the next message.
\end{itemize}

In the actor model theory there is no restriction on the order of the above actions and so an actor can do all the things above in parallel and concurrently at the same time. This property and that actors are reactive and not pro-active is the fundamental difference between actors and agents, so an agent is \textit{not} an actor. There have been a few attempts on implementing the actor model in real programming languages where the most notable are Erlang and Scala.

\subsubsection{Erlang}
The programming-model of actors \cite{agha_actors:_1986} was the inspiration for the Erlang programming language which was created in the 80s by Eriksson for developing distributed high reliability software in telecommunications. There exists very little research in using Erlang for ABS \cite{varela_modelling_2004}, \cite{di_stefano_using_2005}, \cite{di_stefano_exat:_2007}, \cite{sher_agent-based_2013}, \cite{Bezirgiannis2013}. TODO: short explanation of what they are doing

\subsubsection{Scala}
Scala is a multi-paradigm language which also comes with an implementation of the actor-model as a library which enables to do actor-programming in the way of Erlang. It was developed in 2004 and became popular in recent years due to the increased availability of multi-core CPUs which emphasised the distributed, parallel and concurrent programming for which the actor-model is highly suited.
As for Erlang, there exists even less research in using Scala \& Actors for ABS TODO: cite.  TODO: short explanation of what they are doing

\subsection{Process Calculi}
pi calculus,...
there is a paper on actor model \& pi-calculus

TODO: \cite{padget_pi-calculus_1998}

TODO: check my folders with print-outs and add them as references but dont discuss them

\section{Discrete Event System Specification (DEVS)}
Discrete Event Simulation (DES) is regarded as one of the three disciplines of simulation \footnote{Agent-Based Modelling and Simulation (ABS) and System Dynamics (SD) being the other two.}. In it the system evolves in discrete time-steps cause by events which happen at discrete time. In between these discrete steps the system does not change - the simulation jumps from event to event. The seminal work of Zeigler \cite{zeigler_theory_2000} developed a formalism for specifying and analyzing such discrete event systems which allows a formal modelling and verification of such systems. DEVS has been applied to a number of problems, most notably in chemistry and biology \cite{ewald_discrete_2007}, \cite{uhrmacher_discrete_2005}. DEVS has also been connected to the $\pi$-calculus \cite{wang_pi-calculus_2008}.
DEVS is of importance here because some agent-behaviour and agent-interactions can be modelled through it, giving us a powerful formal tool for model-specification and verification.

\section{Computation}
\section{Pure Functional Programming}
In his 1977 ACM Turing Award Lecture, John Backus \footnote{One of the giants of Computer Science, a main contributor to Fortran - an imperative programming language.} fundamentally critizied imperative programming for its deep flaws and proposed a functional style of programming to overcome the limitations of imperative programming \cite{backus_can_1978}. The main criticism is its use of \textit{state-transition with complex states} and the inherent semantics of state-manipulation. In the end an imperative program consists of a number of assign-statements resulting in side-effects on global mutable state which makes reasoning about programs nearly impossible. Backus proposes the so called \textit{applicative} computing, which he termes \textit{functional programming} which has its foundations in the Lambda Calculus \cite{church_calculi_1941}. The main idea behind it is that programming follows a declarative rather than an imperative style of programming: instead of describing \textit{how} something is computed, one describes \textit{what} is computed. This concept abandons variables, side-effects and (global) mutable state and resorts to the simple core of function application, variable substitution and binding of the Lambda Calculus. Although possible and an important step to understand the very foundations, one does not do functional programming in the Lambda Calculus \cite{michaelson_introduction_2011}, as one does not do imperative programming in a Turing Machine. 
In our thesis we selected Haskell as our functional programming language. \footnote{Although we did a bit of research using Scala (a mixed paradigm functional language) in ABS (see Appendix \ref{app:frABS}), we deliberately ignored other functional languages as it is completely out-of-scope of this thesis to do an in-depth comparison of functional languages for their suitability to implement ABS.}. The paper of \cite{hudak_history_2007} gives a comprehensive overview over the history of the language, how it developed and its features and is very interesting to read and get accustomed to the background of the language. A widely used introduction to programming in Haskell is \cite{hutton_programming_2016}. The main points why we decided to go for Haskell are

\begin{itemize}
	\item Pure, Lazy Evaluation, Higher-Order Functions and Static Typing - these are the most important points for the decision as they form the very foundation for composition, correctness, reasoning and verification. 
	\item Real-World applications - the strength of Haskell has been proven through a vast amount of highly diverse real-world applications \footnote{\url{https://wiki.haskell.org/Applications_and_libraries}} \cite{hudak_history_2007} and is applicable to a number of real-world problems \cite{osullivan_real_2008}.
	\item Modern - Haskell is constantly evolving through its community and adapting to keep up with the fast changing field of computer science e.g. parallelism \& concurrency.
	\item In-house knowledge - the School of Computer Science of the University of Nottingham has a large amount of in-house knowledge in Haskell which can be put to use and leveraged in my thesis.
\end{itemize}

It seems that we are on the right track with pure functional programming in answering the questions in the motivation as it promises to solve all the issues raised in these questions. We will now investigate whether this is really the case by looking into relevant literature. 

The main conclusion of the classical paper \cite{hughes_why_1989} is that \textit{modularity} is the key to successful programming and can be achieved best using higher-order functions and lazy evaluation provided in functional languages like Haskell. The author argues that the ability to divide problems into sub-problems depends on the ability to glue the sub-problems together which depends strongly on the programming-language. He shows that laziness and higher-order functions are in combination a highly powerful glue and identifies this as the reason why functional languages are superior to structure programming. Another property of lazy evaluation is that it allows to describe infinite data-structures, which are computed as currently needed. This makes functions possible which produce an infinite stream which is consumed by another function - the decision of \textit{how many} is decoupled from \textit{how to}.

In the paper \cite{wadler_essence_1992} Wadler describes Monads as the essence of functional programming (in Haskell). Originally inspired by monads from category-theory (see below) through the paper of Moggi \cite{moggi_computational_1989}, Wadler realized that monads can be used to structure functional programs \cite{wadler_comprehending_1990}. A pure functional language like Haskell needs some way to perform impure (side-effects) computations otherwise it has no relevance for solving real-world problems like GUI-programming, graphics, concurrency,... . This is where monads come in, because ultimately they can be seen as a way to make effectful computations explicit \footnote{This is seen as one of the main impacts of Haskell had on the mainstream programming \cite{hudak_history_2007}}. 
In \cite{wadler_essence_1992} Wadler shows how to factor out the error handling in a parser into monads which prevents code to be cluttered by cross-cutting concerns not relevant to the original problem. Other examples Wadler gives are the propagating of mutable state, (debugging) text-output during execution, non-deterministic choice. Further applications of monads are given in \cite{wadler_essence_1992}, \cite{wadler_monads_1995}, \cite{wadler_how_1997} where they are used for array updating, interpreting of a language formed by expressions in algebraic data-types, filters, parsers, exceptions, IO, emulating an imperative-style of programming. This seems to be exactly the way to go, tackling the problems mentioned in the introduction: making data-flow explicit, allowing to factor out cross-cutting concerns and encapsulate side-effects in types thus making them explicit.

The concept of monads was further generalized by Hughes in the concept of arrows \cite{hughes_generalising_2000}. The main difference between Monads and Arrows are that where monadic computations are parameterized only over their output-type, Arrows computations are parametrised both over their input- and output-type thus making Arrows more general. In \cite{hughes_programming_2005} Hughes gives an example for the usage for Arrows in the field of circuit simulation. Streams are used to advance the simulation in discrete steps to calculate values of circuits thus the implementation is a form of \textit{discrete event simulation} - which is in the direction we are heading already with ABS. As will be shown below, the concept of arrows is essential for Functional Reactive Programming a potential way to do ABS in pure functional programming.

One of the most compelling example to utilize pure functional programming is the reporting of \cite{hudak_haskell_1994} where in a prototyping contest of DARPA the Haskell prototype was by far the shortest with 85 lines of code (LoC) as compared to the C++ solution with 1105 LoC. The remarkable thing is that the Jury mistook the Haskell code as specification because its approach was to implement a small embedded domain specific language (EDSL) to solve the problem - this is a perfect proof how close an EDSL can get to a specification. When implementing an EDSL one develops and programs primitives e.g. types and functions in a host language (embed) in a way that they can be combined. The combination of these primitives then looks like a language specific to a given domain. The ease of development of EDSLs in pure functional programming is also a proof of the superior extensibility and composability of pure functional languages over object-orientation and is definitely one of its major strength. The classic paper \cite{henderson_functional_1982} gives a wonderful way of constructing an EDSL to denotationally construct a picture reminiscent of the works of Escher.
A major strength of developing an EDSL is that one can reason about and do formal verification. A nice introduction how to do reasoning in Haskell is given in \cite{hutton_tutorial_1999}. Also the testing-library QuickCheck \cite{claessen_quickcheck:_2000}, \cite{claessen_testing_2002} defines an EDSL which allows to formulate a specification in the QuickCheck- EDSL and domain-EDSL and test the code against this specification - testing code happens by writing formal specifications which is the very heart of verification. 
It seems that in EDSL we have found a way to tackle the problem of verification and close the gap between specification and implementation at least conceptually - whether this is really possible will be subject of the research conducted in the thesis.

\subsection{ABS}
The amount of research on using the pure functional paradigm using Haskell in the field of ABS has been moderate so far. Most of the papers look into how agents can be specified using the belief-desire-intention paradigm \cite{de_jong_suitability_2014}, \cite{sulzmann_specifying_2007}, \cite{jankovic_functional_2007}. A library for Discrete Event Simulation (DES) and System Dynamics (SD) in Haskell called \textit{Aivika 3} is described in \cite{sorokin_aivika_2015}. It comes with very basic features for ABS but only allows to specify simple state-based agents with timed transitions.
\cite{jankovic_functional_2007} which discuss using functional programming for DES mention the paradigm of functional reactive programming (FRP) to be very suitable to DES. \cite{schneider_towards_2012} and \cite{vendrov_frabjous:_2014} present a domain-specific language for developing functional reactive agent-based simulations. This language called FRABJOUS is human readable and easily understandable by domain-experts. It is not directly implemented in FRP/Haskell but is compiled to Yampa code - a FRP library for Haskell - which they claim is also readable. It seems that FRP is a promising approach to ABS in Haskell, an important hint we will follow in the section below.

Tim Sweeney, CTO of Epic Games gave an invited talk in which he talked about programming languages in the development of game-engines and scripting of game-logic \cite{sweeney_next_2006}. Although the fields of games and ABS seem to be very different, in the end they have also very important similarities: both are simulations which perform numerical computations and update objects in a loop either concurrently or sequential \footnote{Gregory \cite{gregory_game_2018} defines computer-games as \textit{soft real-time interactive agent-based computer simulations}}. In games these objects are called \textit{game-objects} and in ABS they are called \textit{agents} but they are conceptually the same thing. The two main points Sweeney made were that dependent types could solve most of the run-time failures and that parallelism is the future for performance improvement in games. He distinguishes between pure functional algorithms which can be parallelized easily in a pure functional language and updating game-objects concurrently using software transactional memory (STM).

The thesis of \cite{bezirgiannis_improving_2013} constructs two frameworks: an agent-modelling framework and a DES framework, both written in Haskell. They put special emphasis on parallel and concurrency in their work. The author develops two programs with strong emphasis on parallelism: HLogo which is a clone of the NetLogo agent-modelling framework and HDES, a framework for discrete event simulation.

Although probably the most important selling point of a pure functional language is its ease of parallelizing code due to lack of side-effects \cite{peyton_jones_concurrent_1996}, \cite{osullivan_real_2008}, \cite{jones_tutorial_2009}, \cite{marlow_parallel_2013} we don't go into this direction in our thesis and consider this just to be a by-product which luckily just falls out of the language itself \footnote{We did some research of implementing concurrent agents with STM as proposed by \cite{sweeney_next_2006} and \cite{bezirgiannis_improving_2013} in our research on programming paradigms as can be seen in Appendix \ref{app:paradigms}. We think that STM is probably the single major feature which is \textit{only} possible in a pure functional language because only in a pure functional language with explicit side-effects it is possible to \textit{compose} concurrency.}.

%TODO: this seems all to be focused on MAS
%\url{http://haskell-distributed.github.io/wiki.html} looks good but too big and not well suited for simulations
%\url{https://code.google.com/archive/p/haskellactor/} makes heavy use of IORef and running in IO-Monad, something we deliberately want to avoid to keep the ability to reason about the program.
%TODO: \url{https://github.com/fizruk/free-agent} look into

\subsection{Functional Reactive Programming}
So far we have considered only quite low-level approaches to structuring and composing functional programming: higher-order functions, laziness, monads and arrows. What we need is a programming paradigm built into pure functional programming which we can leverage to implement ABS. As already mentioned above, functional reactive programming (FRP) seems to be a highly promising approach. It is rather a lucky coincidence that Henrik Nilsson, one of the major contributor to the library Yampa, an implementation of FRP, is situated at the School of Computer Science of the University of Nottingham.

FRP is a paradigm for programming hybrid systems which combine continuous and discrete components. Time is explicitly modelled: there is a continuous and synchronous time flow. There have been many attempts to implement FRP in libraries which each has its benefits and deficits. The very first functional reactive language was Fran, a domain specific language for graphics and animation. At Yale FAL, Frob, Fvision and Fruit were developed. The ideas of them all have then culminated in Yampa, the most recent FRP library \cite{nilsson_functional_2002}. The essence of FRP with Yampa is that one describes the system in terms of signal functions in a declarative manner using the EDSL of Yampa. During execution the top level signal functions will then be evaluated and return new signal functions which act as continuations. A major design goal for FRP is to free the programmer from 'presentation' details by providing the ability to think in terms of 'modeling'. It is common that an FRP program is concise enough to also serve as a specification for the problem it solves \cite{wan_functional_2000}.

Yampa has been used in multiple agent-based applications: \cite{hudak_arrows_2003} uses Yampa for implementing a robot-simulation, \cite{courtney_yampa_2003} implement the classical Space Invaders game using Yampa, \cite{nilsson_declarative_2014} implements a Pong-clone, the thesis of \cite{meisinger_game-engine-architektur_2010} shows how Yampa can be used for implementing a Game-Engine, \cite{mun_hon_functional_2005} implemented a 3D first-person shooter game with the style of Quake 3 in Yampa. Note that although all these applications don't focus explicitly on agents all of them inherently deal with kinds of agents which share properties of classical agents: game-entities, robots,... Other fields in which Yampa was successfully used were programming of synthesizers, network routers, computer music development and has been successfully combined with monads \cite{perez_functional_2016}.

This leads to the conclusion that Yampa is mature, stable and suitable to be used in functional ABS. This and the reason that we have the in-house knowledge lets us focus on Yampa. Also it is out-of-scope to do a in-depth comparison of the many existing FRP libraries.

\subsection{Dependent Types}
As already pointed out by Sweeney in \cite{sweeney_next_2006}, dependent types could remove an important class of run-time errors which in the end means that using them allows to push correctness even further because type-invariants are statically checked at compile time. As correctness and verification is our major concern, dependent types seem to be attractive. The papers of \cite{norell_dependently_2009}, \cite{bove_brief_2009} and \cite{bove_dependent_2009} give a good introduction of what dependent types are and how to program with them in Agda, a dependently typed pure functional programming language, closely related to Haskell. For now this approach seems to be too early to follow as we haven't yet laid the basic groundwork: an non-dependently typed pure functional implementation of ABS in Haskell.
%The delicate point is, that programming in Agda becomes a proof-assistant: a proof is then a constructed programm. The work of \cite{ionescu_dependently-typed_2012} which was mentioned in the introduction uses dependent types for proving fundamental theorems in economics. 

\subsection{Foundations in Category-Theory}
With the advent of monads the interest in category-theory surged and it was discovered that many computational concepts can be expressed through category-theory \cite{pierce_basic_1991}, \cite{spivak_category_2014}. Because many concepts of functional programming were put on firm grounds it would be only consequential to find a category-theoretical approach to functional ABS and put its foundations on firm category-theoretical grounds as well. So far only two papers looked into category-theoretical approaches to agent-based models and simulating \cite{beheshti_analyzing_2013}, \cite{lloyd_category-theoretic_2010} but none of them is really satisfying, making this an ideal research topic for the thesis.

%ADOM: Agent Domain of Monads: https://www.haskell.org/communities/11-2006/html/report.html

\subsection{A final word on LISP}
Being the oldest functional programming language and the 2nd oldest high-level programming language ever created, at one point we considered using LISP in our research due to its immensely powerful feature of homoiconicity. The idea was to investigate if this could be made useful for ABS and bring it to a new level. We abandoned this quickly as it would have led to a total different approach. Besides, it would have definitely not solved the issues the questions raised in the introduction because of its imperative nature. Still there exists a paper \cite{kawabe_nepi2programming_2000} which implements a MAS in LISP.

\section{Verification \& Validation of ABS}
Verification \& Validation are, generally speaking, independent processes to check whether a product meets its requirements and fulfills its intended purpose TODO: cite?. Here we focus explicitly on software verification \& validation where we identify \textit{Verification} to be the process of checking whether an implementation matches a given specification without any bugs or missing parts and \textit{Validation} to be the process of checking if the implementation meets high level requirements. TODO: cite?
To put short the difference between verification \& validation is that verification tries to answer "are we building the product right?" and validation "are we building the right product?" \cite{boehm_software_1989}. For (most of) the software built in the industry in well-defined software-development processes with its own quality control and quality assurance to answer these questions is rather straight-forward. Numerous techniques like checklists, software-tests, integration-tests,... have been developed to deal with Verification \& Validation. TODO: cite?

The question is how the above applies to ABS and from reviewing the literature, it becomes evident that in the context of ABS it is a much more difficult challenge and is still open research \footnote{Leigh Tesfatsion has, as part of her internet-presence on ACE, set up a whole site devoted just to this topic where she cites major references \url{http://www2.econ.iastate.edu/tesfatsi/empvalid.htm}. Most of the literature we have investigated  is drawn from there.}. 

In this thesis we aim for replicating both the Sugarscape \cite{epstein_growing_1996} and Agent\_Zero \cite{epstein_agent_zero:_2014} models. In doing so we also perform verification \& validation on these models, or at least on parts of them. Also we aim to develop new methods of verification and validation of ABS based on theses models as use-cases. 

The issue of validation \& verification is also very closely related to the problem of replication. In the paper \cite{wilensky_making_2007} Wilensky discusses the issues with replicating a model and gives advices to modellers what details (e.g. order of events) to publish to support replication of models. They recommend to make pseudo-code publicly available and to converge to a common standard form for model publication in the long run. This is our approach: an EDSL can act both as a pseudo-code specification which is in fact already runnable code and due to its declarative and concise nature can act as a full description.
 
\cite{galan_errors_2009} makes the critical point that the dynamics of a model in ABS are almost always so complex, that the creator of the is not able to exactly explain what the deeper reasons are for them and which aspects of the model are responsible for their exhibition. If this were so, one would not need to resort to simulation. This implies that one can not know in advance what exactly to expect and which part of the emergent behaviour of the system connects to which local interaction amongst agents. Also it is very hard to check if the emergent behaviour is not due to some bug in the implementation.









TODO: baas: emergence, hierarchies and hyperstructures
TODO: \cite{baas_emergence_1997}
TODO: Burton and Obel 1995 "The validity of computational models in organization science: from model realism to purpose of the model"
TODO: Knepell 1993 "Simulation validation, a confidence assessment methodology."

TODO: \url{http://dspace.stir.ac.uk/handle/1893/3365#.WNjO1DsrKM8}
TODO: \cite{klugl_amason:_2013}

\subsection{Verification}
TODO: \cite{axelrod_advancing_1997}

In the work of \cite{axtell_aligning_1996} the authors tried to see whether the more complex Sugarscape model can be used to reproduce the results of \cite{axelrod_convergence_1995}. In both models agents have a tag for cultural identification which is comprised of a string of symbols. The question was whether Sugarscape, focusing on generating a complete artifical society which incorporates many more mechanisms like trading, war, ressources can reproduce the results of \cite{axelrod_convergence_1995} which only focuses on transmission of these cultural tags. Although interesting the question if two models are qualitatively equivalent is not what we want to pursue in our thesis as it requires a complete different direction of research.
The cooperative work of \cite{axtell_aligning_1996} gives insights into validation of computational models, in a process what they call "alignment". They try to determine if two models deal with the same phenomena. For this they tried to qualtiatively reproduce the same results of \cite{axelrod_convergence_1995} in the Sugarscape model of \cite{epstein_growing_1996}. Both models are of very different nature but try to investigate the qualitatively same phenomoenon: that of cultural processes. TODO: read

TODO: write about ABS as a new tool and  generative as opposed to the classical inductive and deductive sciences. major sources: 
TODO fully read \cite{epstein_chapter_2006}
TODO fully read \cite{epstein_generative_2012}

TODO: \cite{galan_errors_2009}
TODO: \cite{windrum_empirical_2007}

model checking and reasoning in \cite{hutton_tutorial_1999}\\

\subsection{Validation}
The question is what the \textit{meaning} of validation - are we building the right simulation? - in the context of ABS is. Wilensky defines model validation in \cite{wilensky_making_2007} to be the process of determining whether the simulation explains and corresponds to phenomenon of the real world.
\cite{galan_errors_2009} define validation of a model to check if it is consistent with the intended application of the model, how well the model captures its empirical referent.

\section{Category-Theory}
The field of Category-Theory formalizes mathematical structure by abstracting away from internal structure of mathematical ojbects and only looking at relations between them. It developed out of algebraic topology in the 1940s by the work of Samuel Eilenberg and Saunders Mac Lane with the goal of understanding the processes that preserve mathematical structure.
The central concepts are collection of objects and arrows, called morphisms which are structure preserving mapings between these objects. An example for a category is the category of sets with sets being the objects and the arrows are functions from one set to another. It is important to make it clear that in this case an objects is a set which itself consists of elements but in category-theory one does and must not look at these elements - only the extensional properties of the object are relevenat: how does it relate to other objects through morphisms.

The classic introductory text for computer scientists is \cite{pierce_basic_1991}. A more comprehensive text with more emphasis on applications to the (computer) scientist is \cite{spivak_category_2014}.

In the work of  \cite{baez_physics_2009} the author has shown that category-theory can be used to link concepts of computation, physics and logic. As our thesis is multi-disciplinary our aim is also to use category-theory to abstract from these concrete fields and derive common structure.

\paragraph{Computation}
With the advent of monads in functional programming, the interest in category-theory surged and it was discovered that many computational concepts can be expressed through category-theory.

\paragraph{ABS}
So far only two papers looked into category-theoretical approaches to agent-based models and simulating \cite{beheshti_analyzing_2013}, \cite{lloyd_category-theoretic_2010} but none of them is really satisfying as they lack . The lack of a proper treatment of ABS and because many concepts of functional programming were put on firm grounds it would be only consequential to find a category-theoretical approach to functional ABS and put its foundations on firm category-theoretical grounds as well.

\paragraph{Emergence}
We already mentioned the work of Baas \cite{baas_emergence_1994}, \cite{baas_emergence_1997} on formalizing emergence through category-theory in the section on Verification \& Validation.

Concluding, we can say that the essence of category-theory is to derive structural concepts and compare them. This gives us a high-level structural view to compare concepts which may seem be completely different at first and see it all through one lens. Thus our central hypothesis is that if we formulate the real-world phenomenon, the model-specification, ABS and the functional implementation in category-theory we may be able to show that they all represent the same thing. So we arrive at the  very essence of validation by showing that the functional program / simulation is indeed a faithful implementation of the real-world phenomenon. To our knowledge such an approach to validating ABS has not been attempted yet.

\section{Applications}
\section{Agent-Based Social Simulation (ABSS)}
The field of social simulation can be traced back to self-replicating von Neumann machines, cellular automata and Conway's Game of Life. The famous Schelling segregation model \cite{schelling_dynamic_1971} is regarded as a pioneering example. The most prominent topics which are explored in social simulation are social norms, institutions, reputation, elections and economics.

Axelrod \cite{axelrod_advancing_1997}, \cite{axelrod_guide_2006} has called social simulation the third way of doing science, which he termed the \textit{generative} approach which is in opposition to the classical inductive (finding patterns in empirical data) and deductive (proving theorems). Thus the generative approach can be seen as a form of empirical research and is a natural environment for studying social and interdisciplinary phenomena as discussed more in-depth in the work of Epstein \cite{epstein_chapter_2006}, \cite{epstein_generative_2012}. He gives a fundamental introduction to agent-based social social simulation and makes the strong claim that \textit{"If you didn't grow it, you didn't explain its emergence"} \footnote{Emergence is treated more in-depth in the Verification \& Validation section.} \footnote{Note the fundamental constructivist approach to social science, which implies that the emergent properties are actually computable. This applies to ACE as well, which can be seen to be its most fundamental difference to general equilibrium theory of neo-classical economics which is non-constructive. When making connections from the simulation to reality (as in validation, see below), constructible emergence raises the question whether our existence is computable or not. When pushing this further, we can conjecture that the future of simulation will be simulated copies of our own existence which potentially allows to simulate \textit{everything}. An interesting treatment of this can be found in \cite{bostrom_are_2003} and \cite{steinhart_theological_2010}.}. Epstein puts much emphasis on the claim that ABSS is indeed a scientific instrument as hypotheses which are investigated are empirical falsifiable: the simulation exhibits the emergent pattern in which case the model is \textit{one} way of explaining it or it simply does not show the emergent pattern, in which case the hypothesis, that the model (the micro-interactions amongst the agents) generates the emergent pattern is falsified \footnote{This is fundamentally following Poppers theory of science \cite{popper_logic_2002}.} - we haven't found an explanation \textit{yet}. So in summary, growing a phenomena is a necessary, but not sufficient condition for explanation \cite{epstein_chapter_2006}.

% NOTE: incorporate this only when there is enough time (and energy) to go through the 3 references cited here
%This raises a number of philosophical questions \cite{frigg_philosophy_2009}, \cite{grune-yanoff_philosophy_2010}, \cite{borrill_agent-based_2011}. Although we don't want to give an in-depth discussion of the questions raised, we want to have a quick look at them as this is a foundational research-proposal for a Doctor in \textit{Philosophy} (Ph.D.).
%TODO: read above papers and give short outline philosophical questions

The first large scale ABSS model which rose to some prominence was the \textit{Sugarscape} model developed by Epstein and Axtell in 1996 \cite{epstein_growing_1996}. Their aim was to \textit{grow} an artificial society by simulation and connect observations in their simulation to phenomenon of real-world societies. The main features of this model are:

\begin{itemize}
	\item Searching, harvesting and consuming of resources.
	\item Wealth and age distributions.
	\item Seasons in the environment and migration of agents.
	\item Pollution of the environment.
	\item Population dynamics under sexual reproduction.
	\item Cultural processes and transmission.
	\item Combat and assimilation.
	\item Bilateral decentralized trading (bartering) between agents with endogenous demand and supply.
	\item Emergent Credit-Networks.
	\item Disease Processes, Transmission and immunology.
\end{itemize}

Because of its essential importance to this field, its complexity, number of features and allowing us to bridge the gap to ACE, we select it as the first of two central models, which will serve as use-case to develop our methods. The idea is to formally specify and then verify the process of bilateral decentralized trading because it is the most complex of the features and connects directly to ACE.

In 2013 Epstein introduced the \textit{Agent\_Zero} model \cite{epstein_agent_zero:_2014} in which the author approaches the generative social sciences from a neurocognitive perspective \footnote{Epstein termed this work Volume III in the triology on generative social science. Volume I is the Sugarscape book mentioned above \cite{epstein_growing_1996}. Volume II is a collection of papers published in the book \cite{epstein_generative_2012} which applied agent-based modelling to the fields of economics, archeology, conflict, epidemiology, spatial games and the dynamics of norms.}.
\textit{Agent\_Zero} is an agent which is endowed with emotional/affective (emotional/gefühlsbezogen), cognitive/deliberative (wahrnehmung/abwägend) and social modules which are all interconnected and interact with each other. Also Agent\_Zero is always part of a social network through which it is influenced by other Agent\_Zero and can influence them. The core behaviour Epstein wants to "grow" in this model is \textit{"the person who feels no aversion to black people, who has never had any direct evidence or experience of black wrongdoing [...], and who yet initiates the lynching"} \footnote{\cite{epstein_agent_zero:_2014}, page 2}. The central concept of the model is the one of \textit{dispositional contagion} which allows to replicate and simulate the following scenarios (amongst others):

\begin{itemize}
	\item Fight vs. Flight
	\item Replicating the Latané-Darley experiment
	\item Growing the 2011 Arab Spring
	\item Jury processes
	\item Prices and seasonal economic cycles
	\item Mutual escalation spirals
\end{itemize}

We select \textit{Agent\_Zero} as our second central model serving as use-case to develop our methods because of its in ABSS and offers a very interesting use-case to apply various networks as presented in the ACE section \footnote{In a recent work \cite{epstein_advancing_2016} Epstein offers a range of new research directions for Agent\_Zero, most notably new interactions, empirical testing, replication of historical episodes and formal axioms for modular agents. We include it for completeness but it does not offer fundamentally new insights to Agent\_Zero neither does it approach the lack of a deeper treatment of the influence of networks in the model.}. As Epstein only looks at a network of three agents, the idea is to investigate the effect of various types of networks as presented in the literature-review section on ACE with much more than three agents on the model.

\subsection{Agent-Based Computational Economics (ACE)}
The field of economics is an immensely vast and complex one with many facets to it \cite{bowles_understanding_2005}. Equilibrium-theory is the very foundation of (micro) economics \cite{colell_microeconomic_1995} and central to the way economists think and approach the dynamics of economic processes. This model requires a central \textit{walrasian} auctioneer which has perfect information and assumes homogeneous, rationally acting agents. The theory then postulates the existence of an equilibrium under given properties but does not give a process or the dynamics how this equilibrium can be approached. 
This notion of equilibrium has always been criticized for being not realistic, making impossible assumptions e.g. perfect information and not being able to provide a process under which this equilibrium is reached \cite{kirman_complex_2010}. The problem is that as soon as more realistic assumptions are made, the solutions become analytically intractable. This is where ACE comes in, as it allows to experimentally approach equilibrium-theory from a more realistic point-of-view by removing the central auctioneer and introducing agents with bounded rationality, local information and restricted interactions over networks \cite{farmer_economy_2009}. 
%look into computable economics book: \url{http://www.e-elgar.com/shop/computable-economics}

Tesfatsion defines ACE as \textit{[...] computational modelling of economic processes (including whole economies) as open-ended dynamic systems of interacting agents.} \url{http://www2.econ.iastate.edu/tesfatsi/ace.htm}. She gives a broad overview \cite{tesfatsion_agent-based_2006} of ACE, discusses advantages and disadvantages and giving the four primary objectives of it which are:

\begin{enumerate}
	\item empirical understanding: why have particular global regularities evolved and persisted, despite the absence of centralized planning and control?
	\item normative understanding: how can agent-based models be used as laboratories for the discovery of good economic designs?
	\item qualitative insight and theory generation: how can economic systems be more fully understood through a systematic examination of their potential dynamical behaviors under alternatively specified initial conditions?
	\item methodological advancement: how best to provide ACE researchers with the methods and tools they need to undertake the rigorous study of economic systems through controlled computational experiments?
\end{enumerate}

She introduces a model called \textit{ACE Trading World} in which she shows how an artificial economy can be implemented without the \textit{Walrasian Auctioneer} but just by agents and their interactions. She gives a detailed mathematical specification in the appendix of the paper which should allow others to implement the simulation. Other works which investigate ACE as a discipline and discuss its methodology are \cite{tesfatsion_agent-based_2002}, \cite{richiardi_agent-based_2007}, \cite{ballot_agent-based_2015}, \cite{blume_introduction_2015}

During the reading we became particularly interested in the dynamics of bilateral decentralized bartering \footnote{The Sugarscape book \cite{epstein_growing_1996}, in footnote 14 on page 104, cites a few introductory works on bilateral bartering with incomplete information which we don't want to repeat here} and emerging networks. The reason for this is that the Sugarscape model \cite{epstein_growing_1996}, mentioned in the previous section on Social Simulation, implements this bilateral decentralized bartering and emerging of credit-networks and looks at the out-of-equilibrium dynamics. This allows us to bridge the gap from ACE to Social Simulation because considerable research in ACE goes into out-of-equilibrium models, which try to find processes which lead to the general equilibrium \cite{gintis_emergence_2006}, \cite{gintis_dynamics_2007}, \cite{arthur_out--equilibrium_2006} and \cite{botta_functional_2011}.

Although not directly subject of this research, to better understand trading and bartering, it is quite useful to have a basic understanding of \textit{Market Microstructure} which deals how real markets work \footnote{A topic we deliberately ignore is the one of \textit{Market Design} which deals with problems real markets face. It is a very hot topic at the moment in economics, having received a number of Nobel-prices e.g. Alvin Roth who wrote an introduction to this topic for the non-expert \cite{roth_who_2015}. In the preparation phase we did quite some reading in this field inspired by \cite{sornette_crashes_2011} and \cite{budish_editors_2015} on the problems caused by High-Frequency Trading (HFT). We were able to find quite a few papers which used ABS to research the benefits and downside of HFT: \cite{wah_latency_2013}, \cite{leal_rock_2016}, \cite{yim_effect_2015}. A broad overview of Market Design using Agent-Based Models is given in \cite{marks_chapter_2006}}. Introductory texts to market microstructure are \cite{harris_trading_2003}, \cite{baker_market_2013} and \cite{lehalle_market_2013}. A highly interesting research using ABS for simulating the NASDAQ market was done in \cite{darley_nasdaq_2007}. The authors where approached by NSDAQ to predict the switching to the decimal system which was enforced by the SEC in April 9th 2001. They implemented an ABS in Java to build relevant models and predicted most of the changes correctly. Other works on using ABS in finance and stock markets are \cite{lebaron_agent-based_2000}, \cite{lebaron_building_2002}, \cite{streltchenko_multi-agent_2005} and \cite{panayi_agent-based_2012}. Another sub-field is autonomous and automated trading agents \cite{mackie-mason_chapter_2006}, \cite{toulis_mertacor:_2006}.

Another topic I am particularly interested in is the one of networks as it was of central focus in my master-thesis TODO:cite on continuous double-auctions in networks. Although this topic is not unique to economics, it has received considerable attention in the last years due to the sub-prime mortgage crisis where contagion through networks was one of the primary reasons for its cause.

TODO: add literature on networks from my masterthesis.
TODO cite Jackson Social and Economic networks, cite easley networks, crowds and markets


\begin{enumerate}
	\item Bilateral decentralized bartering \&  trading
	\item Out-Of-Equilibrium dynamics
	\item Networks
\end{enumerate}

 \footnote{It is of very importance to note that this thesis does not attempt to develop or proof some economic theory. Rather the intention is to use ACE as a use-case to develop the tools and apply them directly to ACE to demonstrate the usefulness and benefit of the new tool.}.


