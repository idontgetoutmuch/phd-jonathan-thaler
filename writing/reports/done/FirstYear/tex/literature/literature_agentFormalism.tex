\section{Agent Formalisms}
When looking for ways to verification and validation of ABS, it is of interest to have a more formal representation of agents and ABS. This helps to clarify the key concepts in an unambiguous, formal way which can then be applied to formal methods e.g. algebraic reasoning for proofing properties of an agent system. Wooldridge \cite{wooldridge_introduction_2009}, \cite{wooldridge_intelligent_1995} gives an abstract architecture for intelligent agents, introducing the concept of actions of agents influencing the environment. The book of \cite{weiss_multiagent_2013} offers several chapters on the theoretical foundations and underpinnings of multi-agent systems \footnote{Although all of these references are primarily rooted in the multi-agent system (MAS) field - which is quite distinct from ABS as stressed in the introduction - they can as well be applied to ABS and offer some inspiration from which to draw upon.}
A very useful formalization is given in \cite{klugl_amason:_2013} which applies their formalism as an example to the Sugarscape model \cite{epstein_growing_1996} \footnote{A happy coincidence because we are using Sugarscape as a major use-case for developing our methodology.}.
In \cite{carchiolo_using_2000} the authors use the process calculi CCS and CSP for designing an agent-specification language LOTOS for formal description and verification. This is a hint that process calculi, as presented in the section above, are indeed a feasible tool for formally specifying low-level properties of agents and their interactions.
The authors of \cite{araragi_formal_2000} compare three formal methods for modelling and validating agent systems: Erdös a knowledge-based environment for agent programming, $Nepi^2$ a programming language for agent system based on the $\pi$-calculus implemented in LISP and I/O automata a mathematical framework used for modelling and reasoning about systems with interacting components.
A more software-engineering approach is taken by \cite{dinverno_formal_2000} and provide an agent development line from a formal agent framework, to agent system specification, agent development to agent deployment. 