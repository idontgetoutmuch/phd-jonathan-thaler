\subsection{Agent-Based Computational Economics (ACE)}
The field of economics is an immensely vast and complex one with many facets to it, ranging from firms, to financial markets to whole economies of a country \cite{bowles_understanding_2005}. Today its very foundations rest on rational expectations, optimization and the efficient market hypothesis. The idea is that the macroeconomics are explained by the micro foundations \cite{colell_microeconomic_1995} defined through behaviour of individual agents. These agents are characterized by rational expectations, optimizing behaviour, having perfect information, equilibrium \cite{focardi_is_2015}.
This approach to economics has come under heavy critizism in the last years for being not realistic, making impossible assumptions like perfect information, not being able to provide a process under which equilibrium is reached \cite{kirman_complex_2010} and failing to predict crashes like the sub-prime mortgage crisis despite all the promises - the science of economics is perceived to be detached from reality \cite{focardi_is_2015}. 
ACE is a promise to repair the empirical deficit which (neo-classic) economics seem to exhibit by allowing to make more realistic, empirical assumptions about the agents which form the micro foundations. The ACE agents are characterized by bounded rationality, local information, restricted interactions over networks and out-of-equilibrium behaviour \cite{farmer_economy_2009}. 
%look into computable economics book: \url{http://www.e-elgar.com/shop/computable-economics}
Tesfatsion \cite{tesfatsion_agent-based_2017} defines ACE as \textit{[...] computational modelling of economic processes (including whole economies) as open-ended dynamic systems of interacting agents.}. She gives a broad overview \cite{tesfatsion_agent-based_2006} of ACE, discusses advantages and disadvantages and giving the four primary objectives of it which are:

\begin{enumerate}
	\item Empirical understanding: why have particular global regularities evolved and persisted, despite the absence of centralized planning and control?
	\item Normative understanding: how can agent-based models be used as laboratories for the discovery of good economic designs?
	\item Qualitative insight and theory generation: how can economic systems be more fully understood through a systematic examination of their potential dynamical behaviours under alternatively specified initial conditions?
	\item Methodological advancement: how best to provide ACE researchers with the methods and tools they need to undertake the rigorous study of economic systems through controlled computational experiments?
\end{enumerate}

Other works which investigate ACE as a discipline and discuss its methodology are \cite{tesfatsion_agent-based_2002}, \cite{richiardi_agent-based_2007}, \cite{ballot_agent-based_2015}, \cite{blume_introduction_2015}.

During the reading-process we became particularly interested in the dynamics of bilateral decentralized bartering \footnote{The Sugarscape book \cite{epstein_growing_1996}, in footnote 14 on page 104, cites a few introductory works on bilateral bartering with incomplete information which we don't want to repeat here}. The reason for this is that the Sugarscape model \cite{epstein_growing_1996}, mentioned in the previous section on Social Simulation, implements this bilateral decentralized bartering and looks at the out-of-equilibrium dynamics. This allows us to bridge the gap from ACE to Social Simulation because considerable research in ACE goes into out-of-equilibrium models, which try to find processes which lead to the general equilibrium \cite{gintis_emergence_2006}, \cite{gintis_dynamics_2007}, \cite{arthur_out--equilibrium_2006} and \cite{botta_functional_2011}. %TODO: write about Debt 5000 years book when having read on Amrum \cite{graeber_debt:_2011}

Although not directly subject of this research, to better understand trading and bartering, it is quite useful to have a basic understanding of \textit{Market Microstructure} which deals how real markets work \footnote{A topic we deliberately ignore is the one of \textit{Market Design} which deals with problems real markets face. It is a very hot topic at the moment in economics, having received a number of Nobel-prices e.g. Alvin Roth who wrote an introduction to this topic for the non-expert \cite{roth_who_2015}. In the preparation phase we did quite some reading in this field inspired by \cite{sornette_crashes_2011} and \cite{budish_editors_2015} on the problems caused by High-Frequency Trading (HFT). We were able to find quite a few papers which used ABS to research the benefits and downside of HFT: \cite{wah_latency_2013}, \cite{leal_rock_2016}, \cite{yim_effect_2015}. A broad overview of Market Design using Agent-Based Models is given in \cite{marks_chapter_2006}.}. Introductory texts to market microstructure are \cite{harris_trading_2003}, \cite{baker_market_2013} and \cite{lehalle_market_2013}. A highly interesting research using ABS for simulating the NASDAQ market was done in \cite{darley_nasdaq_2007}. The authors where approached by NASDAQ to predict the switching to the decimal system which was enforced by the SEC in April 9th 2001. They implemented an ABS in Java to build relevant models and predicted most of the changes correctly. Other works on using ABS in finance and stock markets are \cite{lebaron_agent-based_2000}, \cite{lebaron_building_2002}, \cite{streltchenko_multi-agent_2005} and \cite{panayi_agent-based_2012}. Another sub-field is autonomous and automated trading agents \cite{mackie-mason_chapter_2006}, \cite{toulis_mertacor:_2006}.

Another topic highly important to ACE is the one of networks \cite{wilhite_economic_2006}. Although it is not unique to ACE or economics, it has received considerable attention in the last years due to the sub-prime mortgage crisis where contagion through networks was one of the primary reasons for its cause \cite{glasserman_contagion_2015}. Networks also play a very important role in Social Simulation and both Sugarscape \cite{epstein_growing_1996} and Agent\_Zero \cite{epstein_agent_zero:_2014} incorporate them in their model, so this is another bridge from ACE to Social Simulation. In Sugarscape they occur as emerging neighbour-, genealogical-, cultural-, credit- and disease transmission-networks and in Agent\_Zero networks define the influence of agents amongst each other. A short-coming of Agent\_Zero is that only a network of three agents is considered - it would be interesting how dynamics unfold in other types of networks. The books of \cite{jackson_social_2008} and \cite{easley_networks_2010} give a very broad and in-depth overview over social networks with focus on economics. The paper of \cite{newman_structure_2003} is a thorough review of the field of networks focusing on small-world effects, degree distribution, clustering, random graph models, preferential attachment and dynamic processes on networks. A category-theoretical approach to networks is given by Spivak in \cite{spivak_higher-dimensional_2009}, which may bridge the gap to functional programming and its category-theoretical approach.

To conclude, it is of very importance to note that this thesis does not attempt to develop or proof some economic theory. Rather the intention is to use ACE  - together with Social Simulation - as a use-case to develop the tools and apply them directly to ACE to demonstrate the usefulness and benefit of the new tool.