\chapter{Aims and Objectives}
In this chapter we derive the hypotheses to the motivating questions from the introduction based upon our literature-review and present our aims and objectives for our PhD. We present research-questions and hypotheses.

\section{Deriving the research-direction}
TODO: select the direction: Category-Theory and not Actor Model, Haskell with FRP and not Scala\&Actors/Erlang. need a thorough explanation reasoning

emphasize my own research on scala with actors, experiments with erlang and prototyping with haskell.

As becomes evident from the literature-review we advocate pure functional programming in Haskell and its category-theoretic foundations as a solution to the questions posed. The usage of pure functional programming in ABS is also a strong motivation by itself as it hasn't been researched yet and deserves a thorough treatment on its own. Surprisingly there exist hardly any attempts on implementing ABS in pure functional programming as will become clear in the literature-review. Maybe this can also be seen as a hint that ABS lacks a level of deductive formalism which we hope to repair with our thesis. 

So put short the motivation is a twofold direction, referring to each other in a circular way. First, pure functional programming has not been researched for implementing and specifying ABS so far. Second,  the current state-of-the-art seems to be susceptible to flaws and bugs due to the lack of powerful verification. Combining both issues forms the very basic motivation of our thesis: use pure functional programming and its underlying theoretical framework to develop new methods for specifying, implementing, verifying and validating ABS to create simulations which are more reliable, reproducible and communicatable.

\section{Identifying the Gap}
- Functional programming in this area exists but only scratches the surface and focus only on implementing agent-behaviour frameworks like BDI. An in-depth treatise of Agent-Based Modelling and implementing an Agent-Based Simulation in a pure functional language has so far never been attempted.

- There basically exists no approach to Agent-Based Modelling \& Simulation in terms of Category-Theory and Type-Theory

- Verification is an issue in ABS as they are very often described in natural language and supplemented with a few formulas. This leads to implementation-errors, e.g. Gintis Bartering-Paper, and results become hard to reproduce. Such errors become a threatening problem when simulation-results are used in decision making e.g. economics, policy-making, ...

- Validation is basically an untouched topic in ABS: models are formulated, a few hypotheses are formulated, the model is implemented and run, then the results are checked against the hypotheses. What the field of ABS needs is an in-depth discussion on how to rigorously validate a model. Validation is of course only as strong as the verification part: if the implementation is wrong anyway then we can not rely on anything (from false comes nothing)


- developing a category- \& type-theoretical view on Agent-Based Modelling \& Simulation which will 
	-> 1. give a deeper insight into the structure of agents, agent-models and agent-based simulation
	-> 2. serves as the basis for the pure functional implementation
	-> 3. serves as a high-level specification tool for agent-models

- implementing a library called FrABS based upon the FRP paradigm which allows to specify Agent-Based Models in an EDSL and run them

- Verification: closing the gap between specification and implementation through the category- \& type-theoretical view and the EDSL

- Validation: formalizing hypotheses and reasoning about dynamics and expected outcomes of the simulation

Define 5 general research questions for each Research-Context
	\begin{itemize}
    \item 2 related to FP
    \item 1 related to integration of FP to ABM/S
    \item 2 related to ABM/S
    \end{itemize}


\section{Aims}
The pure functional programming language Haskell offers all the required features. Together with the concept of functional reactive programming (FRP) we will show that ABS becomes very well possible to be implemented in a functional language. The declarative nature of Haskell allows us to implement an embedded domain-specific language (EDSL) for ABS based on the FRP paradigm.


we propose a new tool for implementing, testing, verifying and validating ABS: the one of pure functional programming and its underlying theory which we together will term \textit{pure functional methods}. We claim that these methods are the answer to the questions posed above and will reduce or even elimiate the danger of failure considerably as outlined in the hypotheses. 
Of course our motivation is not only to improve verification, validation and reproduceability in ABS but also the one of discovery. So far pure functional programming was not really investigated in the context of ABS as will be shown in the literature-review so this thesis is also a proof-of-concept, an investigation how ABS can be done in a pure functional language.

The central aspect of this thesis is centred around the main question of \textit{How Agent-Based Simulation can be done using pure functional programming and what the benefits and disadvantages are.}. So far functional programming has not got much attention in the field of ABS and implementations always focus on the object-oriented approach. We claim, based upon the research of the first year that functional programming is very well suited for ABS and that it offers methods which are not directly possible and only very difficult to achieve with object-oriented programming.

We claim that to build large and complex agent-based simulations in functional programming is possible using the functional reactive programming (FRP) paradigm. We applied FRP to implementing ABS and developed a library in Haskell called FrABS. We implemented the quite complex model SugarScape from social simulation using FrABS and proofed by that, that applying FRP to ABS enables ABS to happen in pure functional programming.

After having shown how agent-based simulation can be done in functional programming we claim that the major benefit of using it enabled a new way of \textit{verification \& validation} in agent-based simulation. 

Due to the declarative nature of pure functional programming it is an established method of implementing an EDSL to solve a given problem in a specific domain. We followed this approach in FrABS and developed an EDSL for ABS in pure functional programming. Our intention was to develop an EDSL which can be used both as specification- and implementation-language. We show this by specifying all the rules of SugarScape in our EDSL.

Due to the lack of implicit side-effects and the recursive nature of pure functional programming we claim that it is natural to apply it to a novel method we came up with: MetaABS, which allows recursive simulation.

Finally having such an EDSL at hand this will allow us to reason about the programs. This will be applied to specify and reason about the dynamics and emergent properties of decentralized bilateral trading and bartering in agent-based computational economics (ACE) and social simulations like SugarScape.

Disadvantages
- although the lack of side-effects is also a benefit, it is also a weakness as all data needs to be passed in and out explicitly 
- indirection due to the lack of objects \& method calls.
- When not to use it: 
	- if you are not familiar with functional programming
	- when you can solve your problem without programming in a Tool like NetLogo, AnyLogic,...
	- when you don't need to reason about your program
	
Functional approach to Agent-Based Modelling \& Simulation
Because we left the path of OO and want to develop a completely different method we have fundamentally two problems to solve in our functional method:
1. Specifying the Agent-Based Model (ABM): Category-Theory, Type-Theory, EDSL: all this clearly overlaps with the  implementation-aspect because the theory behind pure functional programming in Haskell is exactly this. This is a very strong indication that functional programming may be able to really close the gap between specification and implementation in ABS.
2. Implementing the ABM into an Agent-Based Simulation (ABS): building on FRP paradigm

We show that the implicit assumption that an Agent is \textit{about equal to} an object is not correct and leads to many implicit assumptions in OO implementations of an ABS. When implementing ABS in Haskell these implicit assumptions become explicit and challenge the fundamental assumptions about ABS and Agents. We present these implicit assumption in an explicit way by approaching it through programming, type-theory and category-theory to further deepen the concepts and methods in the field of Agent-Based Modelling \& Simulation.

TODO: is it really valid to bind the sending of messages to the advancing of time?
Downside: we cannot have method-calls as in OO, which allows agents to communicate with each other without time advancing. 

Expected benefits
1. By mapping the concepts of ABS to Category-Theory and Type-Theory we gain a deeper understanding of the deeper structure of Agents, Agent-Models and Agent-Simulations.
2. The declarative nature of pure functional programming will allow to close the gap between specification and code by designing an EDSL for ABS in Haskell building on the previously derived abstractions in Category-Theory and Type-Theory. The abstractions and the EDSL implementation will then serve as a specification tool and at the same time code.
3. The pure functional nature together with the EDSL and abstractions in Category- \& Type-Theory allow for a new level of formal verification \& validation using a combination of mathematical proofs in Category- \& Type-Theory, algebraic reasoning in the EDSL and model-checking using Unit-Tests and QuickCheck. The expectation is that this allows us to formally specify hypotheses about expected outcomes about the dynamics (or emergent patterns) of our simulations which then can be verified.
	
	
\section{Objectives}
\subsection{Functional reactive ABS in Haskell}
implement full-fledged pure functional ABS library on top of Haskell and FRP (Yampa): FrABS
	-> includes a very basic view on ABS from the perspective of category-theory
	
\subsection{Category Theory view on ABS}

\subsection{Case Study in Verification}
Sugarscapes decentralized  bilateral trading 

\subsection{Case Study in Validation}
Agent\_Zero network extension

Semantics for FrABS and our EDSL to reason about the results: are they reasonable? do they match the theory? if yes why? if not why not?
	- Can we define semantics for the EDSL to do reasoning about ABS in general?
	- How can we reason about ABS in general in pure functional programming?
		- dynamics
		- emergent properties
		- deadlocks
		- silence (no messages/agent-agent communication and interaction)
		- define semantics of FrABS based on semantics of FRP and Actors
		- what is emergence in ABS and how can we reason about it? 
			- identify emergent properties: equilibrium, behaviour on macroscale not defined on micro, chaos,...
			- can we anticipate emergent properties / dynamics just by looking at the code and reason about it?
			- can emergence in ABS be formalized?
				- hypothesis: it may be possible through functional programming because of its dual nature of declarative EDSL which awakens to a process during computation
					- what is the relation between emergence and computation? we need change over time (=computation) for emergence
					
		- Can we reason about the dynamics and equilibria of agent-based models of decentralized bilateral trading \& bartering?