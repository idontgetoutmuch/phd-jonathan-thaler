\chapter{Aims and Objectives}
\label{chap:aimsObj}

This chapter gives a very brief and concise overview of the aims and objectives by just stating them without giving further substantiation. Chapter \ref{chap:future} will give then more details in how we plan approaching the aim in general and the objectives in particular.

\section{Aims}
The aim of this Ph.D. is to explore the benefits and drawbacks using the functional programming paradigm for implementing Agent-Based Simulations. 

\section{Hypotheses}
\begin{enumerate}
	\item Yampa is a valid / good approach of implementing ABS in Haskell.
	
	\item Haskell benefits (which are not possible in java) are: Code == Spec, can statically rule out a major and very relevant class of bugs, can perform reasoning and proofing of properties of the program. The main advantages are: time-semantics, recursive simulation, correctness, verification, composability, testability?

	\item Haskell drawbacks over java are: slower, potential for difficult to find space-leaks, much more difficult to reason about performance in general, steeper learning curve, think ABS different 
	
	\item haskell allows to test ABS in a new way, creating more robust and correcter simulations using smallcheck and quickcheck (can you write unit tests in anylogic, repast, netlogo? could you even think of writing specification tests? i very doubt so)
	
	\item Reproducibility of a system is much stronger in Haskell: unpredictable side-effects can be guaranteed to not show up because it is statically enforced through the type-system already at compile-time.
\end{enumerate}

\section{Objectives}
\begin{enumerate}
	\item Implement a library for general-purpose Agent-Based Simulation in Haskell which will serve as the tool for conducting the research 
	
	\item Compare the potential benefits and drawbacks of Haskell and Java in general to implementing ABS.
	
	\item Compare the benefits and drawbacks of the Haskell library and Repast to implement ABS on the examples of Sugarscape and the SIR model.
	
	\item Develop reasoning-techniques using Functional Programming in ABS by comparing the implementations of the SD- and ABS-model of the SIR compartment model in epidemiology.
	
	\item Explore the further potential of formal reasoning and verification in functional ABS on the example of Sugarscape.
\end{enumerate}

\section{Research Questions}
\begin{enumerate}
	\item What are the benefits/drawbacks of Haskell and what are the benefits/drawbacks of Java in implementing ABS? Are they orthogonal to each other e.g. are the weaknesses of one language the other languages strength?

	\item Which is the best / a valid / good working approach of implementing ABS in Haskell?
	
	\item What are the benefits and drawbacks of using Haskell in ABS?

	\item Are there things which are unique when doing Haskell in ABS and cannot be done in a Java approach and vice versa?
	
	\item Both the System-Dynamics and Agent-Based implementation of the SIR compartment model in epidemiology lead to the same dynamics or put different: the Agent-Based implementation shows the same dynamics of the SD implementation when using replications. This is shown by plotting the dynamics as graphs. Can we show that they are equivalent through reasoning about the code?
	
	\item In the Sugarscape model where Agents engage in bilateral decentralized bartering Equilibrium is only reached when neo-classical agents are used which don't die of natural age. The equilibrium is not reached when more realistic assumptions are made. This is shown by plotting the prices over time. Can we show that the equilibrium is reached / not reached when using neo-classical agents / realistic agents through reasoning in the code?
	
	\item What and to which extent can we reason about an Agent-Based Simulation using my implementation in Haskell?
	
	\item Do the findings on "functional ABS" apply only to the field of Agent-Based Social Simulation and Agent-Based Computational Economics or to ABS in general?
\end{enumerate}