\chapter{Functional Reactive ABS (FrABS)}
\label{app:frABS}

TODO: functionsl approach to ABS: parallel application to previous states where only one agent is acting and the others are fixed. per step we have n results. for a full iteration we need $(n-1)^2$ applicatioms


TODO: is it really valid to bind the sending of messages to the advancing of time?
Downside: we cannot have method-calls as in OO, which allows agents to communicate with each other without time advancing. 


\section{General principles}
idea: can we implement a message between two agents through events? thus two states: waiting for messages, processing messages. BUT: then sending a message \textit{will take some time}

NOTE: it is important to make a difference about whether the simulation will dynamically \textit{add} or \textit{remove} agents during execution. If this is not the case, a simple par-switch is possible to run ALL agent SF in parallel. If dynamically changes to the agent-population should be part of the simulation, then the dpSwitch or dpSwitchB should be used. Also it should be possible to start/stop agents: if they are inactive then they should have no running SF because would use up resources. Inactive means: doing nothing, also not awaiting something/"doing nothing in the sense that DOING something which is nothing - the best criteria to decide if an agent can be set inactive is when the event which decides if the agents SF should be started comes from outside e.g. if the agent is just statically "living" but not changing and then another agent will "ignite" the "living" agent then this is a clear criterion for being static without a running SF.

NOTE: the route-function will be used to distribute "messages" to the agents when they are communicating with each other 

TODO: need a mechanism to address agents: if agent A wants to send a message to agent B and agent B wants to react by answering with a message to agent A then they must have a mechanism to address each other 

TODO: design general input/output data-structures 

TODO: design general agent SF 

Wormholes in FRP? 

\section{Yampa}
TODO: why Yampa? There are lots of other FRP-libraries for Haskell. Reason: in-house knowledge (Nilsson, Perez), start with \textit{some} FRP-library to get familiar with the concept and see if FRP is applicable to ABS. TODO: short overview over other FRP-libraries but leave a in-depth evaluation for further-research out of the scope of the PhD as Yampa seems to be suitable. One exception: the extension of Yampa to Dunai to be able to do FRP in Monads, something which will be definitely useful for a better and clearer structuring of the implementation.
TODO: Push vs. Pull

TODO: describe FRP

TODO: 1st year report Ivan: "FPR tries to shift the direction of data-flow, from message passing onto data dependency. This helps reason about what things are over time, as opposed to how changes propagate". QUESTION: Message-passing is an essential concept in ABS, thus is then FRP still the right way to do ABS or DO WE HAVE TO LOOK AT MESSAGE PASSING IN A DIFFERENT WAY IN FRP, TO VIEW AND MODEL IT AS DATA-DEPENDENCY? HOW CAN THIS BE DONE?
BUT: agent-relations in interactions are NEVER FIXED and always completely dynamic, forming a network. The question is: is there a mechanism in which we have explicit data-dependency but which is dynamic like message-passing but does not try to fake method-calls? maybe the conversations come very close

time- and space-leak: when a time-dependent computation falls behind the current time. TODO: give reason why and how this is solved through Yampa.
Yampa solves this by not allowing signals as first-class values but only allowing signal functions which are signal transformers which can be viewed as a function that maps signals to signals. A signal function is of type SF which is abstract, thus it is not possible to build arbitrary signal functions. Yampa provides primitive signal functions to define more complex ones and utilizes arrows \cite{hughes_programming_2005} to structure them where Yampa itself is built upon the arrows: SF is an instance of the Arrow class. 

Fran, Frob and FAL made a significant distinction between continuous values and discrete signals. Yampas distinction between them is not as great. Yampas signal-functions can return an Event which makes them then to a signal-stream - the event is then similar to the Maybe type of Haskell: if the event does not signal then it is NoEvent but if it Signals it is Event with the given data. Thus the signal function always outputs something and thus care must be taken that the frequency of events should not exceed the sampling rate of the system (sampling the continuous time-flow). TODO: why? what happens if events occur more often than the sampling interval? will they disappear or will the show up every time?

switches allow to change behaviour of signal functions when an event occurs. there are multiple types of switches: immediate or delayed, once-only and recurring - all of them can be combined thus making 4 types. It is important to note that time starts with 0 and does not continue the global time when a switch occurs. TODO: why was this decided?

The first implementations of FRP (Fran) implemented FRP with synchronized stream processors which was also followed by \cite{wan_functional_2000}. Yampa is but using continuations inspired by Fudgets. In the stream processors approach "signals are represented as time-stamped streams, and signal functions are just functions from streams to streams", where "the Stream type can be implemented directly as (lazy) list in Haskell...":
\begin{lstlisting}[frame=single]
type Time = Double
type SP a b = Stream a -> Stream b
newtype SF a b = SF (SP (Time, a) b)
\end{lstlisting}
Continuations on the other hand allow to freeze program-state e.g. through closures and partial applications in functions which can be continued later. This requires an indirection in the Signal-Functions which is introduced in Yampa in the following manner. 
\begin{lstlisting}[frame=single]
type DTime = Double

data SF a b = 
	SF { sfTF :: DTime -> a -> (SF a b, b)
\end{lstlisting}
The implementer of Yampa call a signal function in this implementation a \textit{transition function}. It takes the amount of time which has passed since the previous time step and the durrent input signal (a). It returns a \textit{continuation} of type SF a b determining the behaviour of the signal function on the next step (note that exactly this is the place where how one can introduce stateful functions like integral: one just returns a new function which encloses inputs from the previous time-step) and an \textit{output sample} of the current time-step. 

When visualizing a simulation one has in fact two flows of time: the one of the user-interface which always follows real-time flow, and the one of the simulation which could be sped up or slowed down. Thus it is important to note that if I/O of the user-interface (rendering, user-input) occurs within the simulations time-frame then the user-interfaces real-time flow becomes the limiting factor. Yampa provides the function embedSync which allows to embed a signal function within another one which is then run at a given ratio of the outer SF. This allows to give the simulation its own time-flow which is independent of the user-interface. 