\chapter{Literature Review}
Literature Review
- trichter: mit den 3 themen beginnen und dann runterbrechen und ins detail gehen, bis der gap gefunden wurde

% MAJOR TOPIC
\section{Agent-Based Modelling \& Simulation}
Agent Models are NOT specific to any programming language implementation but should in theory be implementable in all languages which support the required primitives of the model or which allows the primitives of the model to be mapped to primitives of the language. Of course this says nothing about how well a language is suited to implement a given agent model and how readable and natural the mapping and implementation is. \\

Start from wooldridge 2.6 (look at the original papers which inspired the 2.6 chapter) and weiss book and the original papers those chapter is based upon. Look into denotational semantics of actor model. Look also in functional models of czesar ionescu. why: this is the major contribution of my thesis and is new knowledge. Must find intuitive, original and creative approach. \\
From functional agent models (e.g. Wooldridge) to implementation of ACE in Haskell (this should go into research-proposal introduction)

TODO: \cite{Genesereth1987}

TODO: \cite{Muller1997}\\

 wooldridge,  will clinger, hewitt

TODO: baas: emergence, hierarchies and hyperstructures
TODO: \cite{baas_emergence_1997}

This method was selected because Scala is an object-oriented functional programming language and has a powerful library included which implements the actor-model. Because actors and agents are closely related this is an obvious method to follow.


The Actor-Model, a model of concurrency, has been around since the paper \cite{Hewitt_1973} in 1973. It was a major influence in designing the concept of Agents and although there are important differences between Actors and Agents there are huge similarities thus the idea to use actors to build agent-based simulations comes quite natural. Although there are papers around using the actor model as basis for their ABMS unfortunately no proper theoretical treatment of using the actor-model in implementing agent-based simulations has been done so far. This paper looks into how the more theoretical foundations of the suitability of actor-model to ABMS and what the upsides and downsides of using it are.

\url{http://www.grids.ac.uk/Complex/ABMS/}

\cite{Bezirgiannis2013} describes in chapter 3.3 a naive clone of NetLogo in the Erlang programming language where each agent was represented as an Erlang process. The author claims the 1:1 mapping between agent and process to "be inherently wrong" because when recursively sending messages (e.g. A to B to A) it will deadlock as A is already awaiting Bs answer. Of course this is one of the problems when adopting Erlang/Scala with Akka/the Actor Model for implementing agents \textit{but it is inherently short-sighted to discharge the actor-model approach just because recursive messaging leads to a deadlock}. It is not a problem of the actor-model but merely a very problem with the communication protocol which needs to be more sophisticated than \cite{Bezirgiannis2013} described. The hypothesis is that the communication protocol will be in fact \textit{very highly application-specific} thus leading to non-reusable agents (across domains, they should but be re-usable within domains e.g. market-simulations) as they only understand the domain-specific protocol. This is definitely NOT a drawback but can't be solved otherwise as in the end (the content of the) communication can be understand to be the very domain of the simulation and is thus not generalizable. Of course specific patterns will show up like "multi-step handshakes" but they are again then specifically applied to the concrete domain.

\cite{jankovic_functional_2007} discuss using functional programming for discrete event simulation (DES) and mention the paradigm of Functional Reactive Programming (FRP) to be very suitable to DES. We were aware of the existence of this paradigm and have experimented with it using the library Yampa, but decided to leave that topic to a side and really keep our implementation clear and very basic.

The amount of research on using the pure functional paradigm using Haskell in the field of ABS has been moderate so far. Though there exist a few papers which look into Haskell and ABS \cite{de_jong_suitability_2014}, \cite{sulzmann_specifying_2007}, \cite{jankovic_functional_2007} they focus primarily on how to specify agents. A library for Discrete Event Simulation (DES) and System Dynamics (SD) in Haskell called \textit{Aivika 3} is described in \cite{sorokin_aivika_2015}. It also comes with very basic features for ABS but only allows to specify simple state-based agents with timed transitions. This papers is investigating Haskell in a different way by looking into its suitability in implementing update-strategies in ABS, something not looked at in the ABS community so far, presenting an original novelty.

There already exists research using the Actor Model \cite{agha_actors:_1986} for ABS in the context of Erlang \cite{varela_modelling_2004}, \cite{di_stefano_using_2005}, \cite{di_stefano_exat:_2007}, \cite{sher_agent-based_2013} but we feel that they barely scratched the surface. We want to renew the interest in this direction of research by incorporating Scala with using the Actor-library in our research because we will show that one update-strategy maps directly to the Actor Model.

% MAJOR TOPIC
\section{Fields of Application}
% SUB TOPIC
\subsection{Social Simulation}
The most influential source on the generative social sciences can be regarded the work of Epstein and Axtell in \cite{epstein_growing_1996} in which they indeed create an artificial society and connect observations in their simulation to phenomenon of real-world societies. They later added more research to this in \cite{epstein_generative_2012} and most recently \cite{epstein_agent_zero:_2014} in which Epstein tries to approach the generative social sciences from a neurocognitive approach.


look closer at the contents of the 3 major books: SugarScape, Generative, Agent\_Zero

Agent\_Zero: \url{http://sites.nationalacademies.org/cs/groups/dbassesite/documents/webpage/dbasse_175078.pdf}


The SugarScape model \cite{epstein_growing_1996} is one of the most influential models of agent-based simulation in the social sciences. The book heavily promotes object-oriented programming (note that in 1996 oop was still in its infancy and not yet very well understood by the mainstream software-engineering industry). We ask how it can be done using pure functional programming paradigm and what the benefits and limits are. We hypothesize that our solution will be shorter (original reported 20.000 LOC), can make use of EDSL thus making it much more expressive, can utilize QuickCheck for a completely new dimension of model-checking and debugging and allows a very natural implementation of MetaABS (see Part III) due to its recursive and declarative nature.

TODO \cite{huberman_evolutionary_1993} 
TODO \cite{nowak_evolutionary_1992}
% SUB TOPIC
\subsection{Computational Economics}
TODO: decentralized bilateral bartering is of primary interest to us in the end because of 1.) interesting problem 2.) master-thesis 3.) it is so fundamental to a society 4.) is a non neo-classical approach to economics 5.) Sugarscape approaches it => can connect Social Simulation and ACE in one model. 
Another field we are particularly interested in, the simulation of decentralized bilateral bartering, which belongs to the field of Agent-Based Computational Economics (ACE) \cite{tesfatsion_agent-based_2006} is covered already in the artifical society of \cite{epstein_growing_1996}. This lucky coincidence let us approach both these very special fields together and apply them as additional use-cases for developing our new methods. 


\cite{tesfatsion_agent-based_2006} gives a broad overview of agent-based computational economics (ACE), gives the four primary objectives of it and discusses advantages and disadvantages. She introduces a model called \textit{ACE Trading World} in which she shows how an artificial economy can be implemented without the \textit{Walrasian Auctioneer} but just by agents and their interactions. She gives a detailed mathematical specification in the appendix of the paper which should allow others to implement the simulation.

- Artificial agent-based economies: \cite{tesfatsion_agent-based_2006}, \cite{gintis_emergence_2006}, \cite{gintis_dynamics_2007}, \cite{gaffeo_adaptive_2008}, \cite{botta_functional_2011}
- Artificial agent-based markets: \cite{mackie-mason_chapter_2006}, \cite{darley_nasdaq_2007}
- Agent-Based Market Design: \cite{marks_chapter_2006}, \cite{budish_editors_2015}

market-microstructure: \cite{LehalleLaruelle2013}, \cite{baker_market_2013}

Basics of Economics \cite{bowles_understanding_2005}, \cite{kirman_complex_2010}

look into computable economics book: \url{http://www.e-elgar.com/shop/computable-economics}

TODO: the reading should pull out the essence of what types of ACE there are and what features each type has (continuous/discrete time, complex agent communication, equilibriua, networks amongst agents,...)

NOTE: I REALLY need to work out what is special in ACE? what is the unique property of ACE AS compared to other ABM/S? Conjecture: equilibrium of dynamics is the central aspect.
\url{http://www2.econ.iastate.edu/tesfatsi/ace.htm}

\cite{mandel_2015} Agent-based modeling and economic theory: where do we stand? - Ballot, Mandel, Vignes \\
\cite{richiardi_2007} Agent-based Computational Economics. A Short Introduction - Richiardi \\
\cite{tesfatsion_2006} Agent-based computational economics: a constructive approach to economic theory - tesfatsion \\
\cite{kleinberg_easley_2015} Introduction to computer science and economic theory - blume, easley, kleinberg \\
\cite{tesfatsion_2002} agent-based computational economics - tesfatsion 



The book \cite{KirmanComplex2010} is a critique of classic economics with the triple of rational agents, "average" inidividuum, equilibrium theory. Although it does not mention ACE it  can be seen as an important introduction to the approach of ACE as it introduces many important concepts and views dominant in ACE. Also ACE can be seen as an approach of tackling the problems introduced in this book: \\

page 6: "the view of economy is much closer to that of social insects than to the traditional view of how economies function." \\
page 7: "... main argument that it is the interaction between individuals that is at the heart of the explanation of many macroeconomic phenomena..." \\
page 15: "problem of equilibrium is information" \\
page 21: "the theme of this book will be that the very fact individuals interact with each other causes aggregate behaviour to be different from that of individuals" \\

TODO: http://www2.econ.iastate.edu/tesfatsi/ace.htm

TODO: \cite{Kirman2001}
TODO: \cite{Kaminski2013}
TODO: how economists can get a life  tesfatsion

"[...] computational modelling of economic processes (including whole economies) as open-ended dynamic systems of interacting agents." Leigh Tesfatsion

TODO: look into the models of agents dominant in ACE. They seem to be more of reactive, continuous nature

depending on the model ACE \\

properties
\begin{itemize}
\item Discrete entities with own goals and behaviour
\item Not necessarily own thread of control
\item Capable to adapt 
\item Capable to modify their behaviour
\item Proactive behaviour: actions depending on motivations generated from their internal state
\end{itemize}

same as in classic ABMs: decentralised: there is no place where global system behaviour (system dynamics) is defined. Instead individual agents interact with each other and their environment to produce complex collective behaviour patterns.

also central to ACE: emergent properties. They show up in the form of equilibria

spatial / geo-spatial aspects not as dominant as in other fields of ABMs. TODO: is this really true?
more important: networks between agents

what are goals in ACE?
what are behaviour in ACE?

behaviour and intelligence is not the main focus

they can be seen as a continuous transformation process

% MAJOR TOPIC
\section{Alternative to object-orientation}
TODO: this is already too specific, we need first to derive why haskell: 1st because its not been done so far and because it solves a lot of problems

The pure functional programming language Haskell offers all the required features. Together with the concept of functional reactive programming (FRP) we will show that ABS becomes very well possible to be implemented in a functional language. The declarative nature of Haskell allows us to implement an embedded domain-specific language (EDSL) for ABS based on the FRP paradigm.

\subsection{The Actor Model}
The Actor-Model, a model of concurrency, has been around since the paper \cite{hewitt_universal_1973} in 1973. It was a major influence in designing the concept of Agents and although there are important differences between Actors and Agents there are huge similarities thus the idea to use actors to build agent-based simulations comes quite natural. Although there are papers around using the actor model as basis for their ABMS unfortunately no proper theoretical treatment of using the actor-model in implementing agent-based simulations has been done so far. This paper looks into how the more theoretical foundations of the suitability of actor-model to ABMS and what the upsides and downsides of using it are.

\url{http://www.grids.ac.uk/Complex/ABMS/}

\begin{enumerate}
\item \cite{Hewitt_1973}
\item \cite{Greif_1975}
\item \cite{Clinger_1981}
\item \cite{Agha_1986}
\item \cite{Agha_1997}
\item \cite{Hewitt_2007}
\item \cite{Hewitt_2010}
\item \cite{Agha_2004}
\end{enumerate}

upside: extreme huge number of agnts possible due to distributed and parallel technology 
downside: depends on system \& hardware: scheduler, system time, systime resolution (not very nice for scientific computation), much more complicated, debugging difficult due to concurrency, no global notion of time appart from systemtime, thus always runs in real-time, but there is no global notion of time in the actor model anyway, no EDSL full of technical details, no determinism, no reasoning

Agents more a high-level concept, Actors low level, technical concurrency primitives\\

This makes simulations very difficult and also due to concurrency implementing a sync conversation among agents is very cumbersome. I have already experience with the Actor Model when implementing a small version of my Master-Thesis Simulation in Erlang which uses the Actor Model as well. For a continuous simulation it was actually not that bad but the problem there was that between a round-trip between 2 agents other messages could have already interfered - this was a problem when agents trade with each other, so one has to implement synchronized trading where only messages from the current agent one trades with are allowed otherwise budget constraints could be violated. Thus I think Erlang/Akka/Actor Model is better suited for distributed high-tolerance concurrent/parallel systems instead for simulations. Note: this is definitely a major point I have to argue in my thesis: why I am rejecting the actor model.

AKKA: thus my prediction is: akka/actor model is very well suited to simulations which 1. dont rely on global time 2. dont have multi-step conversations: interactions among agents which are only question-answer. TODO: find some classical simulation model which satisfies these criterias.

how can we simulate global time? how can we implement multistep conversations (by futures)?

The real problem seems to be concurrency but i feel we can simulate concurrency by synchronizing to continuous time. computations are carried out after another but because time is explicitly modelled they happen logically at the same time. these rules hold: an agent cannot be in two conversations at the same time, the agent can be in only one or none conversation at a given time t.

What if time is of no importance and only the continuous dynamics are of interest?

To put it another way: real concurrency (with threads) makes time implicit which is what one does NOT want in simulation. Maybe FRP is the way to go because it allows to explicitly model continuous and discrete time, but I have to get into FRP first to make a proper judgement about its suitability.
\\

% MAJOR TOPIC
\subsection{Functional Programming}
all FRP, quickcheck, arrows, monads, wadler, hughes

 the specification language should not be too technical, its focus should be on non-technical expressiveness. The question is: can we abstract away the technicalities and still translate it directly to haskell (more or less)? If not, can be adjust our Haskell implementation to come closer to our specification language? Thus it is a two-fold approach: both languages need to come closer to each other if we want to close the gap
 
TODO: \cite{nowak_evolutionary_1992}, \cite{huberman_evolutionary_1993}
it is still not exactly clear => we present a formal specification using ABS 

it is not trivial to reproduce the results as there is only very informal descriptions in \cite{nowak_evolutionary_1992}. \cite{huberman_evolutionary_1993} give a few more details but also stay quite informal. Thus here we represent a pure functional formulation of the original program which makes it formally exactly clear how the simulation should work. we then look how it can be translated to an ABM specification

Object-oriented (OO) programming is the current state-of-the-art method used in implementing ABM/S due to the natural way of mapping concepts and models of ABM/S to an OO-language. Although this dominance in the field we claim that OO has also its serious drawbacks:

\begin{itemize}
\item Mutable State is distributed over multiple objects which is often very difficult to understand, track and control.
\item Inheritance is a dangerous thing if not used properly and with care because it introduces very strong dependencies which cannot be changed during runtime any-more.
\item Objects don't compose very well due to their internal (mutable) state (note that we are aware that there is the concept of immutable objects which are becoming more and more popular but that does not solve the fundamental problem.
\item It is (nearly) impossible to reason about programs.
\end{itemize}

We claim that these drawbacks are non-existent in pure functional programming like Haskell due to the nature of the functional approach. To give an introduction into functional programming is out of scope of this paper but we refer to the classical paper of \cite{hughes_why_1989} which is a great paper explaining to non-functional programmers what the significance of functional programming is and helping functional programmers putting functional languages to maximum use by showing the real power and advantages of functional languages. The main conclusion of this classical paper is that \textit{modularity}, which is the key to successful programming, can be achieved best using higher-order functions and lazy evaluation provided in functional languages like Haskell. \cite{hughes_why_1989} argues that the ability to divide problems into sub-problems depends on the ability to glue the sub-problems together which depends strongly on the programming-language and \cite{hughes_why_1989} argues that in this ability functional languages are superior to structured programming.

\cite{sulzmann_specifying_2007} present an EDSL for Haskell allowing to specify Agents using the BDI model. We don't go there, thats not our intention. 

\cite{schneider_towards_2012} and \cite{vendrov_frabjous:_2014} present a domain-specific language for developing functional reactive agent-based simulations. This language called FRABJOUS is very human readable and easily understandable by domain-experts. It is not directly implemented in FRP/Haskell/Yampa but is compiled to Haskell/Yampa code which they claim is also readable. This is the direction we want to head but we don't want this intermediate step but look for how a most simple domain-specific language embedded in Haskell would look like. We also don't touch upon FRP and Yampa yet but leave this to further research for another paper of ours.

\cite{klugl_amason:_2013} TODO

TODO: cite julie greensmith paper on haskell

The amount of research on using the pure functional paradigm using Haskell in the field of ABS has been moderate so far.
Most of the papers look into how agents can be specified using the belief-desire-intention paradigm \cite{de_jong_suitability_2014}, \cite{sulzmann_specifying_2007}, \cite{jankovic_functional_2007}.
A library for Discrete Event Simulation (DES) and System Dynamics (SD) in Haskell called \textit{Aivika 3} is described in \cite{sorokin_aivika_2015}. It comes with very basic features for ABS but only allows to specify simple state-based agents with timed transitions.
\cite{jankovic_functional_2007} discuss using functional programming for discrete event simulation (DES) and mention the paradigm of FRP to be very suitable to DES.

\cite{sulzmann_specifying_2007} present an EDSL for Haskell allowing to specify Agents using the BDI model. TODO: We don't go there, thats not our intention. 

\cite{schneider_towards_2012} and \cite{vendrov_frabjous:_2014} present a domain-specific language for developing functional reactive agent-based simulations. This language called FRABJOUS is very human readable and easily understandable by domain-experts. It is not directly implemented in FRP/Haskell/Yampa but is compiled to Haskell/Yampa code which they claim is also readable. This is the direction we want to head but we don't want this intermediate step but look for how a most simple domain-specific language embedded in Haskell would look like. We also don't touch upon FRP and Yampa yet but leave this to further research for another paper of ours.

\cite{klugl_amason:_2013} TODO

TODO: cite julie greensmith paper on haskell

We don't focus on BDI or similar but want to rely much more on low-level basic messaging. We can also draw strong relations to Hoare's Communicating Sequential Processes (CSP), Milner's Calculus of Communicating Systems (CCS) and Pi-Calculus. By mapping the EDSL to CSP/CCS/Pi-Calculus we achieve to be able to algebraic reasoning in our EDSL. TODO: hasn't Agha done something similar in connecting Actors to the Pi-Calculus?

\cite{bezirgiannis_improving_2013} constructs two frameworks: an agent-modelling framework and a DES framework, both written in Haskell. They put special emphasis on parallel and concurrency in their work. The author develops two programs: HLogo which is a clone of the NetLogo agent-modelling framework and HDES, a framework for discrete event simulation - where in both implementations is the very strong emphasis on parallelism.  Here only the HLogo implementation is of interest as it is directly related to agent-based simulation. In this implementation the author claims to have implemented an EDSL which tries to be close to the language used for modelling in NetLogo (Logo) "which lifts certain restrictions of the original NetLogo implementation". Also the aim was to be "faster in most circumstances than NetLogo" and "utilizes many processor cores to speedup the execution of Agent Based Models". The author implements a primitive model of concurrent agents which implements a non-blocking concurrent execution of agents which report their results back to the calling agent in a non-blocking manner. The author mentions that a big issue of the implementation is that repeated runs with same inputs could lead to different results due to random event-orderings happening because of synchronization. The problem is that the author does not give a remedy for that and just accepts it as a fact. Of course it would be very difficult, if not impossible, to introduce determinism in an inherently concurrent execution model of agents which may be the reason the author does not even try. Unfortunately the example implementation the author uses for benchmarking is a very simplistic model: the basic pattern is that agent A sends to agent B and thats it - no complex interactions. Of course this lends itself very good to parallel/concurrent execution and does not need a sophisticated communication protocol. The work lacks a proper treatment of the agent-model presented with its advantages and disadvantages and is too sketchy although the author admits that is is just a proof of concept. \\

Tim Sweeney, CTO of Epic Games gave an invited talk about how "future programming languages could help us write better code" by "supplying stronger typing, reduce run-time failures;  and the need for pervasive concurrency support, both implicit and explicit, to effectively exploit the several forms of parallelism present in games and graphics." \cite{sweeney_next_2006}. Although the fields of games and agent-based simulations seem to be very different in the end, they have also very important similarities: both are simulations which perform numerical computations and update objects - in games they are called "game-objects" and in abm they are called agents but they are in fact the same thing - in a loop either concurrently or sequential. His key-points were:

\begin{itemize}
\item Dependent types as the remedy of most of the run-time failures.
\item Parallelism for numerical computation: these are pure functional algorithms, operate locally on mutable state. Haskell ST, STRef solution enables encapsulating local heaps and mutability within referentially transparent code.
\item Updating game-objects (agents) concurrently using STM: update all objects concurrently in arbitrary order, with each update wrapped in atomic block - depends on collisions if performance goes up.
\end{itemize}

TODO: discuss \cite{schneider_towards_2012}
TODO: discuss \cite{vendrov_frabjous:_2014}
TODO: discuss \cite{sulzmann_specifying_2007}
TODO: discuss \cite{jankovic_functional_2007}
TODO: discuss \cite{de_jong_suitability_2014}
TODO: discuss \cite{sorokin_aivika_2015}

TODO: check out the internet for Actors/Agents i Haskell, but havn't found anything promising

\url{http://haskell-distributed.github.io/wiki.html} looks good but too big and not well suited for simulations
\url{https://code.google.com/archive/p/haskellactor/} makes heavy use of IORef and running in IO-Monad, something we deliberately want to avoid to keep the ability to reason about the program.
TODO: \url{https://github.com/fizruk/free-agent} look into

\subsection{Haskell} 
NOTE: this chapter should be the very first implementation chapter as the approach of Haskell lays the very foundations for the functional approach by introducing the basic functional concepts

So far the literature on agent-based modelling \& simulation (ABM/S) hasn't focused much on models for functional agents and is lacking a proper treatment of implementing agents in pure-functional languages like Haskell. This paper looks into how agents can be specified functionally and then be implemented properly in the pure functional language Haskell. The functional agent-model is inspired by wooldridge 2.6. The programming paradigm used to implement the agents in Haskell is functional reactive programming (FRP) where the Yampa framework will be used. The paper will show that specifying and implementing agents in a pure functional language like Haskell has many advantages over classical object-oriented, concurrent ones but needs also more careful considerations to work properly.

TODO: read \cite{backus_can_1978}

The state-of-the-art approach to implementing Agents are object-oriented methods and programming as the metaphor of an Agent as presented above lends itself very naturally to object-orientation (OO). The author of this thesis claims that OO in the hands of inexperienced or ignorant programmers is dangerous, leading to bugs and hardly maintainable and extensible code. The reason for this is that OO provides very powerful techniques of organising and structuring programs through Classes, Type Hierarchies and Objects, which, when misused, lead to the above mentioned problems. Also major problems, which experts face as well as beginners are 1. state is highly scattered across the program which disguises the flow of data in complex simulations and 2. objects don’t compose as well as functions. The reason for this is that objects always carry around some internal state which makes it obviously much more complicated as complex dependencies can be introduced according to the internal state.
All this is tackled by (pure) functional programming which abandons the concept of global state, Objects and Classes and makes data-flow explicit. This then allows to reason about correctness, termination and other properties of the program e.g. if a given function exhibits side-effects or not. Other benefits are fewer lines of code, easier maintainability and ultimately fewer bugs thus making functional programming the ideal choice for scientific computing and simulation and thus also for ACE. A very powerful feature of functional programming is Lazy evaluation. It allows to describe infinite data-structures and functions producing an infinite stream of output but which are only computed as currently needed. Thus the decision of how many is decoupled from how to (Hughes, J. (1989). Why functional programming matters. Comput. J., 32(2):98–107.).
The most powerful aspect using pure functional programming however is that it allows the design of embedded domain specific languages (EDSL). In this case one develops and programs primitives e.g. types and functions in a host language (embed) in a way that they can be combined. The combination of these primitives then looks like a language specific to a given domain, in the case of this thesis ACE. The ease of development of EDSLs in pure functional programming is also a proof of the superior extensibility and composability of pure functional languages over OO (Henderson P. (1982). Functional Geometry. Proceedings of the 1982 ACM Symposium on LISP and Functional Programming.).
One of the most compelling example to utilize pure functional programming is the reporting of Hudak (Hudak P., Jones M. (1994). Haskell vs. Ada vs. C++ vs. Awk vs. ... An Experiment in Software Prototyping Productivity. Department of Computer Science, Yale University.)  where in a prototyping contest of DARPA the Haskell prototype was by far the shortest with 85 lines of code. Also the Jury mistook the code as specification because the prototype did actually implement a small EDSL which is a perfect proof how close EDSL can get to and look like a specification.

Functional languages can best be characterized by their way computation works: instead of \textit{how} something is computed, \textit{what} is computed is described. Thus functional programming follows a declarative instead of an imperative style of programming. The key points are:
\begin{itemize}
\item No assignment statements - variables values can never change once given a value.
\item Function calls have no side-effect and will only compute the results - this makes order of execution irrelevant, as due to the lack of side-effects the logical point in \textit{time} when the function is calculated within the program-execution does not matter.
\item higher-order functions
\item lazy evaluation
\item Looping is achieved using recursion, mostly through the use of the general fold or the more specific map.
\item Pattern-matching
\end{itemize}

% SUB TOPIC
\subsubsection{General principles}
idea: can we implement a message between two agents through events? thus two states: waiting for messages, processing messages. BUT: then sending a message \textit{will take some time}

NOTE: it is important to make a difference about whether the simulation will dynamically \textit{add} or \textit{remove} agents during execution. If this is not the case, a simple par-switch is possible to run ALL agent SF in parallel. If dynamically changes to the agent-population should be part of the simulation, then the dpSwitch or dpSwitchB should be used. Also it should be possible to start/stop agents: if they are inactive then they should have no running SF because would use up resources. Inactive means: doing nothing, also not awaiting something/"doing nothing in the sense that DOING something which is nothing - the best criteria to decide if an agent can be set inactive is when the event which decides if the agents SF should be started comes from outside e.g. if the agent is just statically "living" but not changing and then another agent will "ignite" the "living" agent then this is a clear criterion for being static without a running SF. \\

NOTE: the route-function will be used to distribute "messages" to the agents when they are communicating with each other \\

NOTE: \cite{Meisinger2010} argues that in Game-Engines (it is paraphrased in english, as the thesis was written in german): "communication among Game-Objects is always computer-game specific and must be implemented always new but the functionality of Game-objects can be built by combining independent functions and signal-functions which are fully reuse-able". Game-Objects can be understood as agents thus maybe this also holds true for agent-based simulation. \cite{Meisinger2010} thus distinguishes between normal functions e.g. mathematical functions, signal functions which depend on output since its creation in localtime and game-object functions which output depends on inputs AND time (which is but another input). \\

TODO: need a mechanism to address agents: if agent A wants to send a message to agent B and agent B wants to react by answering with a message to agent A then they must have a mechanism to address each other \\

TODO: design general input/output data-structures \\

TODO: design general agent SF \\

TODO: don't loose STM out of sight!

Wormholes in FRP? \\
\\

% SUB TOPIC
\subsection{Structuring}
Of course the basic pure functional primitives alone do not make a well structured functional program by themselves as the usage of classes, interfaces, objects and inheritance alone does not make a well structured object-oriented program. What is needed are \textit{patterns} how to use the primitives available in pure functional programs to arrive at well structure programs. In object-orientation much work has been done in the 90s by the highly influential book \cite{gamma_design_1994} whereas in functional programming the major inventions were also done in the 90s by the invention of Monads through \cite{Moggi1989}, \cite{Wadler1990} and \cite{Wadler1995} and beginning of the 2000s by the invention of Arrows through \cite{Hughes2000}.


map \& fmap, foldl, applicatives
\cite{hutton_programming_2007} gives a great overview and motivation for using fmap, applicatives and Monads. TODO: explain Monads

\cite{Hughes2004} is a great tutorial about \textit{Arrows} which are very well suited for structuring functional programs with effects.

\begin{quote}
Just like monads, arrow types are useful for the additional operations they support, over and above those that every arrow provides.
\end{quote}

The main difference between Monads and Arrows are that where monadic computations are parameterized only over their output-type, Arrows computations are parameterised both over their input- and output-type thus making Arrows more general.

\begin{quote}
In real applications an arrow often represents some kind of a process, with an input channel of type a, and an output channel of type b.
\end{quote}

In the work \cite{Hughes2004} an example for the usage for Arrows is given in the field of circuit simulation. They use previously introduced streams to advance the simulation in discrete steps to calculate values of circuits thus the implementation is a form of \textit{discrete event simulation} - which is in the direction we are heading already with ABM/S. Also the paper mentions Yampa which is introduced in the section (TODO: reference) on functional reactive programming.


Monads
Arrows
Continuations
% SUB TOPIC
\subsection{Paradigm: FRP}
TODO: why Yampa? There are lots of other FRP-libraries for Haskell. Reason: in-house knowledge (Nilsson, Perez), start with \textit{some} FRP-library to get familiar with the concept and see if FRP is applicable to ABS. TODO: short overview over other FRP-libraries but leave a in-depth evaluation for further-research out of the scope of the PhD as Yampa seems to be suitable. One exception: the extension of Yampa to Dunai to be able to do FRP in Monads, something which will be definitely useful for a better and clearer structuring of the implementation.
TODO: Push vs. Pull

TODO: describe FRP

TODO: 1st year report Ivan: "FPR tries to shift the direction of data-flow, from message passing onto data dependency. This helps reason about what things are over time, as opposed to how changes propagate". QUESTION: Message-passing is an essential concept in ABS, thus is then FRP still the right way to do ABS or DO WE HAVE TO LOOK AT MESSAGE PASSING IN A DIFFERENT WAY IN FRP, TO VIEW AND MODEL IT AS DATA-DEPENDENCY? HOW CAN THIS BE DONE?
BUT: agent-relations in interactions are NEVER FIXED and always completely dynamic, forming a network. The question is: is there a mechanism in which we have explicit data-dependency but which is dynamic like message-passing but does not try to fake method-calls? maybe the conversations come very close


FRP is a paradigm for programming hybrid systems which combine continuous and discrete components. Time is explicitly modelled: there is a continuous and synchronous time flow.  \\

there have been many attempts to implement FRP in frameworks which each has its own pro and contra. all started with fran, a domain specific language for graphics and animation and at yale FAL, Frob, Fvision and Fruit were developed. The ideas of them all have then culminated in Yampa which is the reason why it was chosen as the FRP framework. Also, compared to other frameworks it does not distinguish between discrete and synchronous time but leaves that to the user of the framework how the time flow should be sampled (e.g. if the sampling is discrete or continuous - of course sampling always happens at discrete times but when we speak about discrete sampling we mean that time advances in natural numbers: 1,2,3,4,... and when speaking of continuous sampling then time advances in fractions of the natural numbers where the difference between each step is a real number in the range of [0..1]) \\

time- and space-leak: when a time-dependent computation falls behind the current time. TODO: give reason why and how this is solved through Yampa. \\
Yampa solves this by not allowing signals as first-class values but only allowing signal functions which are signal transformers which can be viewed as a function that maps signals to signals. A signal function is of type SF which is abstract, thus it is not possible to build arbitrary signal functions. Yampa provides primitive signal functions to define more complex ones and utilizes arrows \cite{Hughes2004} to structure them where Yampa itself is built upon the arrows: SF is an instance of the Arrow class. \\

Fran, Frob and FAL made a significant distinction between continuous values and discrete signals. Yampas distinction between them is not as great. Yampas signal-functions can return an Event which makes them then to a signal-stream - the event is then similar to the Maybe type of Haskell: if the event does not signal then it is NoEvent but if it Signals it is Event with the given data. Thus the signal function always outputs something and thus care must be taken that the frequency of events should not exceed the sampling rate of the system (sampling the continuous time-flow). TODO: why? what happens if events occur more often than the sampling interval? will they disappear or will the show up every time? \\

switches allow to change behaviour of signal functions when an event occurs. there are multiple types of switches: immediate or delayed, once-only and recurring - all of them can be combined thus making 4 types. It is important to note that time starts with 0 and does not continue the global time when a switch occurs. TODO: why was this decided? \\

\cite{Nilsson2002} give a good overview of Yampa and FRP. Quote: "The essential abstraction that our system captures is time flow". Two \textit{semantic} domains for progress of time: continuous and discrete. \\

The first implementations of FRP (Fran) implemented FRP with synchronized stream processors which was also followed by \cite{Wan2000}. Yampa is but using continuations inspired by Fudgets. In the stream processors approach "signals are represented as time-stamped streams, and signal functions are just functions from streams to streams", where "the Stream type can be implemented directly as (lazy) list in Haskell...":
\begin{lstlisting}[frame=single]
type Time = Double
type SP a b = Stream a -> Stream b
newtype SF a b = SF (SP (Time, a) b)
\end{lstlisting}
Continuations on the other hand allow to freeze program-state e.g. through closures and partial applications in functions which can be continued later. This requires an indirection in the Signal-Functions which is introduced in Yampa in the following manner. 
\begin{lstlisting}[frame=single]
type DTime = Double

data SF a b = 
	SF { sfTF :: DTime -> a -> (SF a b, b)
\end{lstlisting}
The implementer of Yampa call a signal function in this implementation a \textit{transition function}. It takes the amount of time which has passed since the previous time step and the durrent input signal (a). It returns a \textit{continuation} of type SF a b determining the behaviour of the signal function on the next step (note that exactly this is the place where how one can introduce stateful functions like integral: one just returns a new function which encloses inputs from the previous time-step) and an \textit{output sample} of the current time-step. \\

When visualizing a simulation one has in fact two flows of time: the one of the user-interface which always follows real-time flow, and the one of the simulation which could be sped up or slowed down. Thus it is important to note that if I/O of the user-interface (rendering, user-input) occurs within the simulations time-frame then the user-interfaces real-time flow becomes the limiting factor. Yampa provides the function embedSync which allows to embed a signal function within another one which is then run at a given ratio of the outer SF. This allows to give the simulation its own time-flow which is independent of the user-interface. \\

One may be initially want to reject Yampa as being suitable for ABM/S because one is tempted to believe that due to its focus on continuous, time-changing signals, Yampa is only suitable for physical simulations modelled explicitly using mathematical formulas (integrals, differential equations,...) but that is not the case. Yampa has been used in multiple agent-based applications: \cite{Hudak2003} uses Yampa for implementing a robot-simulation, \cite{Courtney2003} implement the classical Space Invaders game using Yampa, the thesis of \cite{Meisinger2010} shows how Yampa can be used for implementing a Game-Engine, \cite{Frag2005} implemented a 3D first-person shooter game with the style of Quake 3 in Yampa. Note that although all these applications don't focus explicitly on agents and agent-based modelling / simulation all of them inherently deal with kinds of agents which share properties of classical agents: game-entities, robots,... Other fields in which Yampa was successfully used were programming of synthesizers (TODO: cite), Network Routers, Computer Music Development and various other computer-games. This leads to the conclusion that Yampa is mature, stable and suitable to be used in functional ABM/S. \\
Jason Gregory (Game Engine Architecture) defines Computer-Games as "soft real-time interactive agent-based computer simulations".

To conclude: when programming systems in Haskell and Yampa one describes the system in terms of signal functions in a declarative manner (functional programming) using the EDSL of Yampa. During execution the top level signal functions will then be evaluated and return new signal functions (transition functions) which act as continuations: "every signal function in the dataflow graph returns a new continuation at every time step".

"A major design goal for FRP is to free the programmer from 'presentation' details by providing the ability to think in terms of 'modeling'. It is common that an FRP program is concise enough to also serve as a specification for the problem it solves" \cite{Wan2000}. This quotation describes exactly one of the strengths using FRP in ACE \\

TODO: \cite{perez_functional_2016}

\cite{Wan2000} discuss the semantic framework of FRP. Very difficult to understand and full of corollaries and theorems and proofs, have to study in depth at another time.


\subsection{EDSL}
EDSL steht für Embedded Domain Specific Language d.h. man implementiert in Haskell eine Art von 'Spezifikations-Sprache' für eine spezielle Domain (z.b. ABM/S), die - dank der rein funktionalen, deklarativen Natur von Haskell - auch gleichzeitig Haskell Code ist - der Unterschied zwischen Spezifikation und Implementierung verschwindet dann (idealserweise). In diese Richtung arbeite ich erst seit kurzem, durch die Umsetzung von ABS/M mit Yampa. Yampa ist ebenfalls eine EDSL um funktional-reaktive Systeme zu beschreiben/implementieren, ich werde auf dieser EDSL aufsetzen und sie um ABS/M erweitern - so zumindest der Plan. Dann habe ich die theoretische Grundlage von FRP, auf die ich dann auch theorie von ABS/M (z.b. Actor Semantics) setzen kann und somit zum nächsten Punkt komme:

\cite{Henderson1982} gives a wonderful way of constructing an EDSL do denotationally construct an Escher-Picture.

\subsection{Category- \& Type-Theory}
Category-Theory \cite{Pierce1991} \cite{spivak_category_2014}

include paper on arrows my hughes
apply category theory to agent-based simulation: how can a ABS system itself be represented in category theory and can we represent models in this category theory as well?
ADOM: Agent Domain of Monads: https://www.haskell.org/communities/11-2006/html/report.html
develop category theory behind FrABS: look into monads, arrows

% MAJOR TOPIC
\section{Verification \& Validation of ABS}
In the work of \cite{axtell_aligning_1996} the authors tried to see whether the more complex Sugarscape model can be used to reproduce the results of \cite{axelrod_convergence_1995}. In both models agents have a tag for cultural identification which is comprised of a string of symbols. The question was whether Sugarscape, focusing on generating a complete artifical society which incorporates many more mechanisms like trading, war, ressources can reproduce the results of \cite{axelrod_convergence_1995} which only focuses on transmission of these cultural tags. Although interesting the question if two models are qualitatively equivalent is not what we want to pursue in our thesis as it requires a complete different direction of research.

TODO: write about ABS as a new tool and  generative as opposed to the classical inductive and deductive sciences. major sources: 
TODO fully read \cite{epstein_chapter_2006}
TODO fully read \cite{epstein_generative_2012}


To do verification we need a form of formal specification which can be translated easily to the code. Being inspired by the previously mentioned work on a functional framework for agent-based models of exchange in \cite{botta_functional_2011} we opt for a similar direction. Having Haskell as the implementation language instead of an object-oriented one like Java allows us to build on the above proposed EDSL for ABS. Because of the declarative nature of the hypothesized EDSL it can act both as specification- and implementation-language which closes the gap between specification and implementation. This would give us a way of formally specifying the model but still in a more readable way than pure mathematics. This form of formal specification can act easily as a medium for communication between team-memberes and to the scientific audience in papers. Most important the explicit step of verification becomes obsolete as there exists no more difference between specification and implementation. The last point seems to be quite ambitious but this is a hypothesis and we will see in the course of the thesis how far we can close the gap in the end with this approach.

TODO: need to do lots of literature research still
Having the EDSL from the firt two hypotheses at hand it may be possible to extend it through additional primitves which then allow to formulate hypotheses in a formal way which then can be checked automatically.

TODO: Burton and Obel 1995 "The validity of computational models in organization science: from model realism to purpose of the model"
TODO: Knepell 1993 "Simulation validation, a confidence assessment methodology."

TODO: \url{http://jasss.soc.surrey.ac.uk/12/1/1.html}
TODO: \url{http://ieeexplore.ieee.org/stamp/stamp.jsp?arnumber=4419595}
TODO: \url{http://dspace.stir.ac.uk/handle/1893/3365#.WNjO1DsrKM8}
TODO: \url{http://www2.econ.iastate.edu/tesfatsi/DockingSimModels.pdf}
TODO: \url{http://www2.econ.iastate.edu/tesfatsi/empvalid.htm}
TODO: \url{http://jasss.soc.surrey.ac.uk/10/2/8.html}
TODO: \url{https://www.openabm.org/faq/how-validate-and-calibrate-agent-based-models}
TODO: \url{https://link.springer.com/chapter/10.1007%2F978-3-642-01109-2_10}
TODO: \url{http://www3.nd.edu/~nom/Papers/ADS019_Xiang.pdf}

TODO: \url{http://www.cse.nd.edu/~nom/Papers/ads05_kennedy.pdf}

The cooperative work of \cite{axtell_aligning_1996} gives insights into validation of computational models, in a process what they call "alignment". They try to determine if two models deal with the same phenomena. For this they tried to qualtiatively reproduce the same results of \cite{axelrod_convergence_1995} in the Sugarscape model of \cite{epstein_growing_1996}. Both models are of very different nature but try to investigate the qualitatively same phenomoenon: that of cultural processes. TODO: read

model checking and reasoning in \cite{hutton_tutorial_1999}

In \cite{claessen_quickcheck:_2000} introduce \textit{QuickCheck}, a testing-framework which allows to specifify properties and invariants of ones functions and then test them using randomly generated test-data. This is an additional tool of model-specification and increases the power and strength of the verification process and more properties of a model can be expressed which are directly formulated in code through the EDSL of QuickCheck AND the EDSL of FrABS. Of course it also serves for testing (e.g. regression) and points out errors in the implementation e.g. wrong assumptions about input-data. The authors claim that the major advante of QuickCheckj is to formulate formal specifications which help in understanding a program.
TODO: the question is whether it can be used for Validation as well.

Verification/Reasoning ist einer der größten Pluspunkte von rein funktionaler Programmierung, da durch den deklarativen Stil und das Fehlen von Sideeffects und Globalen Daten equational/algebraic/inductive Reasoning betrieben werden kann. Hier habe ich noch garnichts dazu gemacht, aber sollte mit den oben genannten Ideen sicherlich interessant werden - ein interessantes Paper von Graham Hutton (für den ich übrigens dieses Semester ein Tutor in seiner Haskell-Laborübung bin) gibt interessante Richtungen für Reasoning vor: http://dl.acm.org/citation.cfm?id=968579

[ ] deadlock: when messages need to be exchanged but mutual waiting
[ ] silence: no more message exchange
[ ] protocoll: ensure happens before / sequences (like necessary for 2D prisoner dilemma)