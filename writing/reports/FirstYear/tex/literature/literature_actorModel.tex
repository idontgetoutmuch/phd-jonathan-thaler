\subsection{Actor Model}
The Actor-Model, a model of concurrency, was initially conceived by Hewitt in 1973 \cite{hewitt_universal_1973} and refined later on \cite{hewitt_what_2007}, \cite{hewitt_actor_2010}. It was a major influence in designing the concept of agents and although there are important differences between actors and agents there are huge similarities thus the idea to use actors to build agent-based simulations comes quite natural. The theory was put on firm semantic grounds first through Irene Greif by defining its operational semantics \cite{greif_semantics_1975} and then Will Clinger by defining denotational semantics \cite{clinger_foundations_1981}. In the seminal work of Agha \cite{agha_actors:_1986} he developed a semantic mode, he termed \textit{actors} which was then developed further \cite{agha_foundation_1997} into an actor language with operational semantics which made connections to process calculi and functional programming languages (see both below). 

An actor is a uniquely addressable entity which can \textit{in response to a message}
\begin{itemize}
	\item Send an arbitrary number (even infinite) of messages to other actors.
	\item Create an arbitrary number of actors.
	\item Define its own behaviour upon reception of the next message.
\end{itemize}

In the actor model theory there is no restriction on the order of the above actions and so an actor can do all the things above in parallel and concurrently at the same time. This property and that actors are reactive and not pro-active is the fundamental difference between actors and agents, so an agent is \textit{not} an actor but conceptually nearly identical and definitely much closer to an agent in comparison to an object. There have been a few attempts on implementing the actor model in real programming languages where the most notable are Erlang and Scala.

\subsubsection{Erlang}
The programming-model of actors \cite{agha_actors:_1986} was the inspiration for the Erlang programming language which was created in the 80s by Eriksson \cite{armstrong_erlang_2010} for developing distributed high reliability software in telecommunications. There exists very little research in using Erlang for ABS where the following papers are all what could be found on this topic.

In \cite{varela_modelling_2004} the authors introduce reflexive action - sending messages to itself - to make the passive actors pro-active, thus rendering them in fact being agents. Although their work is rather focused on the field of MAS, it is also applicable to ABS where the main contribution of the authors is the implementation of a Belief-Desire-Intention (BDI) architecture for agents built on top of Erlang. 
The work of \cite{sher_agent-based_2013} emphasises the conceptual gap between the programming languages which implement ABS and the concept of an agent and propose using Erlang to close this gap. Although the paper is not very in-depth, it gives extensive code-listings for implementing Schellings Tipping Model and an agent-based representation of a neural network.
In \cite{bezirgiannis_improving_2013} a naive clone of NetLogo in the Erlang programming language is briefly discussed where each agent is represented as an Erlang process. Although the author does not go into much detail, the synchronization problems caused by the inherent asynchronous nature of the actor model become apparent when mapping ABS to the actor model.
The authors of \cite{di_stefano_using_2005} and \cite{di_stefano_exat:_2007} focus more on using Erlang to implement MAS applications instead of ABS and propose a new agent platform called eXAT in Erlang. Besides claiming to be a more natural approach to agent-programming they mention better readability as compared to Java implementations.

\subsubsection{Scala}
Scala is a multi-paradigm language which also comes with an implementation of the actor-model as a library which enables to do actor-programming in the way of Erlang. It was developed in 2004 and became popular in recent years due to the increased availability of multi-core CPUs which emphasised the distributed, parallel and concurrent programming for which the actor-model is highly suited.
As for Erlang, there exists even less research in using Scala \& Actors for ABS. 

In \cite{krzywicki_massively_2015} the authors use Scala to build a massively-concurrent multi-agent system for evolutionary optimization in the domain of biology. They compared their implementation to Erlang and showed that Scala performs better due to a more efficient underlying memory-model.
The paper \cite{todd_multi-agent_nodate} focuses explicitly on simulation using a MAS built in Scala. They compare their implementation in Scala to a threaded version in Java and come to the finding that the Scala-version is slower. This seems to be curious as Scala with Actors should, when not slowed down by synchronization, have about the same performance. The reason for this is that their implementation does exactly this kind of synchronization by sending global time-stamps to all agents which allows them to perform the next step in their calculation thus synchronizing them all in lock-step.