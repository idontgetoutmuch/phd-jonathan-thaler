\section{Discrete Event System Specification (DEVS)}
Discrete Event Simulation (DES) is regarded as one of the three disciplines of simulation \footnote{Agent-Based Modelling and Simulation (ABS) and System Dynamics (SD) being the other two.}. In it the system evolves in discrete time-steps cause by events which happen at discrete time. In between these discrete steps the system does not change - the simulation jumps from event to event. The seminal work of Zeigler \cite{zeigler_theory_2000} developed a formalism for specifying and analyzing such discrete event systems which allows a formal modelling and verification of such systems. DEVS has been applied to a number of problems, most notably in chemistry and biology \cite{ewald_discrete_2007}, \cite{uhrmacher_discrete_2005}. DEVS has also been connected to the $\pi$-calculus \cite{wang_pi-calculus_2008}.
DEVS is of importance here because some agent-behaviour and agent-interactions can be modelled through it, giving us a powerful formal tool for model-specification and verification.