\subsection{Agent-Based Computational Economics (ACE)}
The field of economics and as a consequence ACE is an immensely vast one, so we limit our interests to the one of out-of-equilibrium models in bilateral decentralized bartering and trading and the influence of networks in it \footnote{It is of very importance to note that this thesis does not attempt to develop or proof some economic theory. Rather the intention is to use ACE as a use-case to develop the tools and apply them directly to ACE to demonstrate the usefulness and benefit of the new tool.}. Out of this we can derive the following three major topics:

\begin{enumerate}
	\item Out-Of-Equilibrium dynamics
	\item Bilateral decentralized bartering / trading
	\item Networks
\end{enumerate}

All three topics are highly interrelated and particularly well suited for ACE because they are only possible by representing the system as interconnected autonomous heterogeneous traders, with bounded-rationality and local information. The topics are also applicable to the field of Social Simulation as presented in the previous section and indeed the Sugarscape model incorporates all three of them which allows us to bridge the gap from ACE to Social Simulation.

"[...] computational modelling of economic processes (including whole economies) as open-ended dynamic systems of interacting agents." Leigh Tesfatsion

NOTE: I REALLY need to work out what is special in ACE? what is the unique property of ACE AS compared to other ABM/S? Conjecture: equilibrium of dynamics is the central aspect.
\url{http://www2.econ.iastate.edu/tesfatsi/ace.htm}

\cite{tesfatsion_agent-based_2006} gives a broad overview of agent-based computational economics (ACE), gives the four primary objectives of it and discusses advantages and disadvantages. She introduces a model called \textit{ACE Trading World} in which she shows how an artificial economy can be implemented without the \textit{Walrasian Auctioneer} but just by agents and their interactions. She gives a detailed mathematical specification in the appendix of the paper which should allow others to implement the simulation.

- Artificial agent-based economies: \cite{tesfatsion_agent-based_2006}, \cite{gintis_emergence_2006}, \cite{gintis_dynamics_2007}, \cite{gaffeo_adaptive_2008}, \cite{botta_functional_2011}
- Artificial agent-based markets: \cite{mackie-mason_chapter_2006}, \cite{darley_nasdaq_2007}
- Agent-Based Market Design: \cite{marks_chapter_2006}, \cite{budish_editors_2015}

\cite{mandel_2015} Agent-based modeling and economic theory: where do we stand? - Ballot, Mandel, Vignes \\
\cite{richiardi_2007} Agent-based Computational Economics. A Short Introduction - Richiardi \\
\cite{tesfatsion_2006} Agent-based computational economics: a constructive approach to economic theory - tesfatsion \\
\cite{kleinberg_easley_2015} Introduction to computer science and economic theory - blume, easley, kleinberg \\
\cite{tesfatsion_2002} agent-based computational economics - tesfatsion 

The book \cite{KirmanComplex2010} is a critique of classic economics with the triple of rational agents, "average" inidividuum, equilibrium theory. Although it does not mention ACE it  can be seen as an important introduction to the approach of ACE as it introduces many important concepts and views dominant in ACE. Also ACE can be seen as an approach of tackling the problems introduced in this book: 

page 6: "the view of economy is much closer to that of social insects than to the traditional view of how economies function." \\
page 7: "... main argument that it is the interaction between individuals that is at the heart of the explanation of many macroeconomic phenomena..." \\
page 15: "problem of equilibrium is information" \\
page 21: "the theme of this book will be that the very fact individuals interact with each other causes aggregate behaviour to be different from that of individuals" \\

TODO: http://www2.econ.iastate.edu/tesfatsi/ace.htm

TODO: \cite{Kirman2001}
TODO: \cite{Kaminski2013}
TODO: how economists can get a life  tesfatsion

TODO: look into the models of agents dominant in ACE. They seem to be more of reactive, continuous nature

same as in classic ABMs: decentralised: there is no place where global system behaviour (system dynamics) is defined. Instead individual agents interact with each other and their environment to produce complex collective behaviour patterns.

also central to ACE: emergent properties. They show up in the form of equilibria

spatial / geo-spatial aspects not as dominant as in other fields of ABMs. TODO: is this really true?
more important: networks between agents


\subsubsection{Out-Of-Equilibrium}
Basics of Economics \cite{bowles_understanding_2005}, \cite{kirman_complex_2010}
look into computable economics book: \url{http://www.e-elgar.com/shop/computable-economics}

\subsubsection{Trading, Market Design \& Microstructure}
building the nasdaq stock market
market-microstructure: \cite{LehalleLaruelle2013}, \cite{baker_market_2013}

Another field we are particularly interested in, the simulation of decentralized bilateral bartering, which belongs to the field of Agent-Based Computational Economics (ACE) \cite{tesfatsion_agent-based_2006} is covered already in the artifical society of \cite{epstein_growing_1996}. This lucky coincidence let us approach both these very special fields together and apply them as additional use-cases for developing our new methods.

TODO: A topic we deliberately ignore is the one of \textit{market design}, which is a very hot topic at the moment in economics having received a number of nobel prices (TODO: cite). Why we don't include it: does not allow to combine as easily with Social Simulation, too complex, completely different direction, 

\subsubsection{Networks}
TODO cite Jackson Social and Economic networks, cite easley networks, crowds and markets
and the authors master-thesis TODO: cite had networks in double-auctions as the central focus. 

Although the topic of networks is not unique to economics, it has received considerable attention in the last years due to the sub-prime mortgage crisis where contagion through networks was one of the primary reasons for its cause. \\

