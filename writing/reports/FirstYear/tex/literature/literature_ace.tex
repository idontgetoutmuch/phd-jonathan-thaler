\subsection{Agent-Based Computational Economics (ACE)}
TODO: network structure also of primary interest (following my master-thesis) TODO: cite my master-thesis, cite Jackson Social and Economic networks, cite easley networks, crowds and markets
TODO: decentralized bilateral bartering is of primary interest to us in the end because of 1.) interesting problem 2.) master-thesis 3.) it is so fundamental to a society 4.) is a non neo-classical approach to economics 5.) Sugarscape approaches it => can connect Social Simulation and ACE in one model. 
Another field we are particularly interested in, the simulation of decentralized bilateral bartering, which belongs to the field of Agent-Based Computational Economics (ACE) \cite{tesfatsion_agent-based_2006} is covered already in the artifical society of \cite{epstein_growing_1996}. This lucky coincidence let us approach both these very special fields together and apply them as additional use-cases for developing our new methods. 


\cite{tesfatsion_agent-based_2006} gives a broad overview of agent-based computational economics (ACE), gives the four primary objectives of it and discusses advantages and disadvantages. She introduces a model called \textit{ACE Trading World} in which she shows how an artificial economy can be implemented without the \textit{Walrasian Auctioneer} but just by agents and their interactions. She gives a detailed mathematical specification in the appendix of the paper which should allow others to implement the simulation.

- Artificial agent-based economies: \cite{tesfatsion_agent-based_2006}, \cite{gintis_emergence_2006}, \cite{gintis_dynamics_2007}, \cite{gaffeo_adaptive_2008}, \cite{botta_functional_2011}
- Artificial agent-based markets: \cite{mackie-mason_chapter_2006}, \cite{darley_nasdaq_2007}
- Agent-Based Market Design: \cite{marks_chapter_2006}, \cite{budish_editors_2015}

market-microstructure: \cite{LehalleLaruelle2013}, \cite{baker_market_2013}

Basics of Economics \cite{bowles_understanding_2005}, \cite{kirman_complex_2010}

look into computable economics book: \url{http://www.e-elgar.com/shop/computable-economics}

TODO: the reading should pull out the essence of what types of ACE there are and what features each type has (continuous/discrete time, complex agent communication, equilibriua, networks amongst agents,...)

NOTE: I REALLY need to work out what is special in ACE? what is the unique property of ACE AS compared to other ABM/S? Conjecture: equilibrium of dynamics is the central aspect.
\url{http://www2.econ.iastate.edu/tesfatsi/ace.htm}

\cite{mandel_2015} Agent-based modeling and economic theory: where do we stand? - Ballot, Mandel, Vignes \\
\cite{richiardi_2007} Agent-based Computational Economics. A Short Introduction - Richiardi \\
\cite{tesfatsion_2006} Agent-based computational economics: a constructive approach to economic theory - tesfatsion \\
\cite{kleinberg_easley_2015} Introduction to computer science and economic theory - blume, easley, kleinberg \\
\cite{tesfatsion_2002} agent-based computational economics - tesfatsion 



The book \cite{KirmanComplex2010} is a critique of classic economics with the triple of rational agents, "average" inidividuum, equilibrium theory. Although it does not mention ACE it  can be seen as an important introduction to the approach of ACE as it introduces many important concepts and views dominant in ACE. Also ACE can be seen as an approach of tackling the problems introduced in this book: \\

page 6: "the view of economy is much closer to that of social insects than to the traditional view of how economies function." \\
page 7: "... main argument that it is the interaction between individuals that is at the heart of the explanation of many macroeconomic phenomena..." \\
page 15: "problem of equilibrium is information" \\
page 21: "the theme of this book will be that the very fact individuals interact with each other causes aggregate behaviour to be different from that of individuals" \\

TODO: http://www2.econ.iastate.edu/tesfatsi/ace.htm

TODO: \cite{Kirman2001}
TODO: \cite{Kaminski2013}
TODO: how economists can get a life  tesfatsion

"[...] computational modelling of economic processes (including whole economies) as open-ended dynamic systems of interacting agents." Leigh Tesfatsion

TODO: look into the models of agents dominant in ACE. They seem to be more of reactive, continuous nature

depending on the model ACE \\

properties
\begin{itemize}
\item Discrete entities with own goals and behaviour
\item Not necessarily own thread of control
\item Capable to adapt 
\item Capable to modify their behaviour
\item Proactive behaviour: actions depending on motivations generated from their internal state
\end{itemize}

same as in classic ABMs: decentralised: there is no place where global system behaviour (system dynamics) is defined. Instead individual agents interact with each other and their environment to produce complex collective behaviour patterns.

also central to ACE: emergent properties. They show up in the form of equilibria

spatial / geo-spatial aspects not as dominant as in other fields of ABMs. TODO: is this really true?
more important: networks between agents

what are goals in ACE?
what are behaviour in ACE?

behaviour and intelligence is not the main focus

they can be seen as a continuous transformation process
