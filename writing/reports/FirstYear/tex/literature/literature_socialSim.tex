\section{Agent-Based Social Simulation (ABSS)}
The field of social simulation can be traced back to self-replicating von Neumann machines, cellular automata and Conway's Game of Life. The famous Schelling segregation model \cite{schelling_dynamic_1971} is regarded as a pioneering example. The most prominent topics which are explored in social simulation are social norms, institutions, reputation, elections and economics.

Axelrod \cite{axelrod_advancing_1997}, \cite{axelrod_guide_2006} has called social simulation the third way of doing science, which he termed the \textit{generative} approach which is in opposition to the classical inductive (finding patterns in empirical data) and deductive (proving theorems). Thus the generative approach can be seen as a form of empirical research and is a natural environment for studying social and interdisciplinary phenomena as discussed more in-depth in the work of Epstein \cite{epstein_chapter_2006}, \cite{epstein_generative_2012}. He gives a fundamental introduction to agent-based social social simulation and makes the strong claim that \textit{"If you didn't grow it, you didn't explain its emergence"} \footnote{Emergence is treated more in-depth in the Verification \& Validation section.} \footnote{Note the fundamental constructivist approach to social science, which implies that the emergent properties are actually computable. This applies to ACE as well, which can be seen to be its most fundamental difference to general equilibrium theory of neo-classical economics which is non-constructive. When making connections from the simulation to reality (as in validation, see below), constructible emergence raises the question whether our existence is computable or not. When pushing this further, we can conjecture that the future of simulation will be simulated copies of our own existence which potentially allows to simulate \textit{everything}. An interesting treatment of this can be found in \cite{bostrom_are_2003} and \cite{steinhart_theological_2010}.}. Epstein puts much emphasis on the claim that ABSS is indeed a scientific instrument as hypotheses which are investigated are empirical falsifiable: the simulation exhibits the emergent pattern in which case the model is \textit{one} way of explaining it or it simply does not show the emergent pattern, in which case the hypothesis, that the model (the micro-interactions amongst the agents) generates the emergent pattern is falsified \footnote{This is fundamentally following Poppers theory of science \cite{popper_logic_2002}.} - we haven't found an explanation \textit{yet}. So in summary, growing a phenomena is a necessary, but not sufficient condition for explanation \cite{epstein_chapter_2006}.

% NOTE: incorporate this only when there is enough time (and energy) to go through the 3 references cited here
%This raises a number of philosophical questions \cite{frigg_philosophy_2009}, \cite{grune-yanoff_philosophy_2010}, \cite{borrill_agent-based_2011}. Although we don't want to give an in-depth discussion of the questions raised, we want to have a quick look at them as this is a foundational research-proposal for a Doctor in \textit{Philosophy} (Ph.D.).
%TODO: read above papers and give short outline philosophical questions

The first large scale ABSS model which rose to some prominence was the \textit{Sugarscape} model developed by Epstein and Axtell in 1996 \cite{epstein_growing_1996}. Their aim was to \textit{grow} an artificial society by simulation and connect observations in their simulation to phenomenon of real-world societies. The main features of this model are:

\begin{itemize}
	\item Searching, harvesting and consuming of resources.
	\item Wealth and age distributions.
	\item Seasons in the environment and migration of agents.
	\item Pollution of the environment.
	\item Population dynamics under sexual reproduction.
	\item Cultural processes and transmission.
	\item Combat and assimilation.
	\item Bilateral decentralized trading (bartering) between agents with endogenous demand and supply.
	\item Emergent Credit-Networks.
	\item Disease Processes, Transmission and immunology.
\end{itemize}

Because of its essential importance to this field, its complexity, number of features and allowing us to bridge the gap to ACE, we select it as the first of two central models, which will serve as use-case to develop our methods. The idea is to formally specify and then verify the process of bilateral decentralized trading because it is the most complex of the features and connects directly to ACE.

In 2013 Epstein introduced the \textit{Agent\_Zero} model \cite{epstein_agent_zero:_2014} in which the author approaches the generative social sciences from a neurocognitive perspective \footnote{Epstein termed this work Volume III in the triology on generative social science. Volume I is the Sugarscape book mentioned above \cite{epstein_growing_1996}. Volume II is a collection of papers published in the book \cite{epstein_generative_2012} which applied agent-based modelling to the fields of economics, archeology, conflict, epidemiology, spatial games and the dynamics of norms.}.
\textit{Agent\_Zero} is an agent which is endowed with emotional/affective (emotional/gefühlsbezogen), cognitive/deliberative (wahrnehmung/abwägend) and social modules which are all interconnected and interact with each other. Also Agent\_Zero is always part of a social network through which it is influenced by other Agent\_Zero and can influence them. The core behaviour Epstein wants to "grow" in this model is \textit{"the person who feels no aversion to black people, who has never had any direct evidence or experience of black wrongdoing [...], and who yet initiates the lynching"} \footnote{\cite{epstein_agent_zero:_2014}, page 2}. The central concept of the model is the one of \textit{dispositional contagion} which allows to replicate and simulate the following scenarios (amongst others):

\begin{itemize}
	\item Fight vs. Flight
	\item Replicating the Latané-Darley experiment
	\item Growing the 2011 Arab Spring
	\item Jury processes
	\item Prices and seasonal economic cycles
	\item Mutual escalation spirals
\end{itemize}

We select \textit{Agent\_Zero} as our second central model serving as use-case to develop our methods because of its in ABSS and offers a very interesting use-case to apply various networks as presented in the ACE section \footnote{In a recent work \cite{epstein_advancing_2016} Epstein offers a range of new research directions for Agent\_Zero, most notably new interactions, empirical testing, replication of historical episodes and formal axioms for modular agents. We include it for completeness but it does not offer fundamentally new insights to Agent\_Zero neither does it approach the lack of a deeper treatment of the influence of networks in the model.}. As Epstein only looks at a network of three agents, the idea is to investigate the effect of various types of networks as presented in the literature-review section on ACE with much more than three agents on the model.