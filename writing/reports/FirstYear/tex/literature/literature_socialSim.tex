\subsection{Social Simulation}
The most influential source on the generative social sciences can be regarded the work of Epstein and Axtell in \cite{epstein_growing_1996} in which they indeed create an artificial society and connect observations in their simulation to phenomenon of real-world societies. They later added more research to this in \cite{epstein_generative_2012} and most recently \cite{epstein_agent_zero:_2014} in which Epstein tries to approach the generative social sciences from a neurocognitive approach.

look closer at the contents of the 3 major books: SugarScape, Generative, Agent\_Zero

Agent\_Zero: \url{http://sites.nationalacademies.org/cs/groups/dbassesite/documents/webpage/dbasse_175078.pdf}

The SugarScape model \cite{epstein_growing_1996} is one of the most influential models of agent-based simulation in the social sciences. The book heavily promotes object-oriented programming (note that in 1996 oop was still in its infancy and not yet very well understood by the mainstream software-engineering industry). We ask how it can be done using pure functional programming paradigm and what the benefits and limits are. We hypothesize that our solution will be shorter (original reported 20.000 LOC), can make use of EDSL thus making it much more expressive, can utilize QuickCheck for a completely new dimension of model-checking and debugging and allows a very natural implementation of MetaABS (see Part III) due to its recursive and declarative nature.

TODO \cite{huberman_evolutionary_1993} 
TODO \cite{nowak_evolutionary_1992}