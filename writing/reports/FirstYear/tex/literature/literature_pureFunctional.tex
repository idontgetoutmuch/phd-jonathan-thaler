\subsection{(Pure) Functional Programming}
In his 1977 ACM Turing Award Lecture, John Backus, one of the giants of Computer Science and a main contributor to Fortran, an imperative programming language, fundamentally critizied imperative programming for its deep flaws and proposed a functional style of programming to overcome the limitations of imperative programming \cite{backus_can_1978}. The main criticism is its use of \textit{state-transition with complex states} and the inherent semantics of state-manipulation. In the end an imperative program consists of a number of assign-statements resulting in side-effects on global mutable state which makes reasoning about programs nearly impossible. Backus proposes the so called \textit{applicative} computing, which he termes \textit{functional programming} which has its foundations in the Lambda Calculus \cite{church_calculi_1941}. The main idea behind it is that programming follows a declarative rather than an imperative style of programming: instead of describing \textit{how} something is computed, one describes \textit{what} is computed. This concept abandons variables, side-effects and (global) mutable state and resorts to the simple core of function application, variable substitution and binding of the Lambda Calculus. Although possible and an important step to understand the very foundations, one does not do functional programming in the Lambda Calculus \cite{michaelson_introduction_2011}, as one does not do imperative programming in a Turing Machine. 
In our thesis we opt for Haskell as our choice for a functional programming language and deliberately ignored other functional languages \footnote{We did a bit of research using Scala (a mixed paradigm functional language) in ABS (see Appendix B) but it is completely out-of-scope of this thesis to do an in-depth comparison of functional languages for their suitability to implement ABS.}. The paper of \cite{hudak_history_2007} gives a comprehensive overview over the history of the language, how it developed and its features and is very interesting to read and get accustomed to the background of the language. A widely used introduction to programming in Haskell which is regarded as being one is \cite{hutton_programming_2016}. The main points why we decided to go for Haskell are

\begin{itemize}
	\item Pure, Lazy, Higher-Order and Static Typing - these are the most important points for the decision as they form the very foundation for composition, correctness, reasoning and verification. 
	\item Real-World applications - The strength of Haskell has been proven through a vast amount of highly diverse real-world applications \footnote{\url{https://wiki.haskell.org/Applications_and_libraries}}
	\item Modern - Haskell is constantly evolving through its community and adapting to keep up with the fast changing field of computer science e.g. parallelism \& concurrency.
	\item In-house knowledge - The School of Computer Science of the University of Nottingham has a large amount of in-house knowledge in Haskell which can be put to use and leveraged in my thesis.
\end{itemize}

It seems that we are on the right track with pure functional programming in answering the questions in the motivation as it promises to solve all the issues raised in these questions. We will now investigate by looking into relevant literature if this is really the case.

\subsubsection{Strengths}
TODO: read \cite{wadler_essence_1992}

TODO: read \cite{hudak_haskell_1994}: pure functional haskell is extremely productive and code seems to look like specification 
One of the most compelling example to utilize pure functional programming is the reporting of \cite{hudak_haskell_1994} where in a prototyping contest of DARPA the Haskell prototype was by far the shortest with 85 lines of code. Also the Jury mistook the code as specification because the prototype did actually implement a small EDSL which is a perfect proof how close EDSL can get to and look like a specification.

TODO: read \cite{hughes_why_1989}
The main conclusion of this classical paper is that \textit{modularity}, which is the key to successful programming, can be achieved best using higher-order functions and lazy evaluation provided in functional languages like Haskell. \cite{hughes_why_1989} argues that the ability to divide problems into sub-problems depends on the ability to glue the sub-problems together which depends strongly on the programming-language and \cite{hughes_why_1989} argues that in this ability functional languages are superior to structured programming.
A very powerful feature of functional programming is Lazy evaluation. It allows to describe infinite data-structures and functions producing an infinite stream of output but which are only computed as currently needed. Thus the decision of how many is decoupled from how to.

\subsubsection{Weaknesses}
TODO: look for relevant literature, also required for  supervision-meeting

\subsubsection{ABS}
Functional Programming in ABS
STM Monads, Arrows, FRP, STM
\cite{Bezirgiannis2013}

\cite{sulzmann_specifying_2007} present an EDSL for Haskell allowing to specify Agents using the BDI model. We don't focus on BDI or similar but want to rely much more on low-level basic messaging. We can also draw strong relations to Hoare's Communicating Sequential Processes (CSP), Milner's Calculus of Communicating Systems (CCS) and Pi-Calculus. By mapping the EDSL to CSP/CCS/Pi-Calculus we achieve to be able to algebraic reasoning in our EDSL. TODO: hasn't Agha done something similar in connecting Actors to the Pi-Calculus?

The amount of research on using the pure functional paradigm using Haskell in the field of ABS has been moderate so far. Most of the papers look into how agents can be specified using the belief-desire-intention paradigm \cite{de_jong_suitability_2014}, \cite{sulzmann_specifying_2007}, \cite{jankovic_functional_2007}.
A library for Discrete Event Simulation (DES) and System Dynamics (SD) in Haskell called \textit{Aivika 3} is described in \cite{sorokin_aivika_2015}. It comes with very basic features for ABS but only allows to specify simple state-based agents with timed transitions.
\cite{jankovic_functional_2007} discuss using functional programming for discrete event simulation (DES) and mention the paradigm of FRP to be very suitable to DES.
The amount of research on using the pure functional paradigm using Haskell in the field of ABS has been moderate so far. Though there exist a few papers which look into Haskell and ABS \cite{de_jong_suitability_2014}, \cite{sulzmann_specifying_2007}, \cite{jankovic_functional_2007} they focus primarily on how to specify agents. A library for Discrete Event Simulation (DES) and System Dynamics (SD) in Haskell called \textit{Aivika 3} is described in \cite{sorokin_aivika_2015}. It also comes with very basic features for ABS but only allows to specify simple state-based agents with timed transitions. This papers is investigating Haskell in a different way by looking into its suitability in implementing update-strategies in ABS, something not looked at in the ABS community so far, presenting an original novelty.

\cite{schneider_towards_2012} and \cite{vendrov_frabjous:_2014} present a domain-specific language for developing functional reactive agent-based simulations. This language called FRABJOUS is very human readable and easily understandable by domain-experts. It is not directly implemented in FRP/Haskell/Yampa but is compiled to Haskell/Yampa code which they claim is also readable. This is the direction we want to head but we don't want this intermediate step but look for how a most simple domain-specific language embedded in Haskell would look like. We also don't touch upon FRP and Yampa yet but leave this to further research for another paper of ours.

\cite{klugl_amason:_2013} TODO

\cite{bezirgiannis_improving_2013} constructs two frameworks: an agent-modelling framework and a DES framework, both written in Haskell. They put special emphasis on parallel and concurrency in their work. The author develops two programs: HLogo which is a clone of the NetLogo agent-modelling framework and HDES, a framework for discrete event simulation - where in both implementations is the very strong emphasis on parallelism.  Here only the HLogo implementation is of interest as it is directly related to agent-based simulation. In this implementation the author claims to have implemented an EDSL which tries to be close to the language used for modelling in NetLogo (Logo) "which lifts certain restrictions of the original NetLogo implementation". Also the aim was to be "faster in most circumstances than NetLogo" and "utilizes many processor cores to speedup the execution of Agent Based Models". The author implements a primitive model of concurrent agents which implements a non-blocking concurrent execution of agents which report their results back to the calling agent in a non-blocking manner. The author mentions that a big issue of the implementation is that repeated runs with same inputs could lead to different results due to random event-orderings happening because of synchronization. The problem is that the author does not give a remedy for that and just accepts it as a fact. Of course it would be very difficult, if not impossible, to introduce determinism in an inherently concurrent execution model of agents which may be the reason the author does not even try. Unfortunately the example implementation the author uses for benchmarking is a very simplistic model: the basic pattern is that agent A sends to agent B and thats it - no complex interactions. Of course this lends itself very good to parallel/concurrent execution and does not need a sophisticated communication protocol. The work lacks a proper treatment of the agent-model presented with its advantages and disadvantages and is too sketchy although the author admits that is is just a proof of concept. \\

Tim Sweeney, CTO of Epic Games gave an invited talk about how "future programming languages could help us write better code" by "supplying stronger typing, reduce run-time failures;  and the need for pervasive concurrency support, both implicit and explicit, to effectively exploit the several forms of parallelism present in games and graphics." \cite{sweeney_next_2006}. Although the fields of games and agent-based simulations seem to be very different in the end, they have also very important similarities: both are simulations which perform numerical computations and update objects - in games they are called "game-objects" and in abm they are called agents but they are in fact the same thing - in a loop either concurrently or sequential. His key-points were:

\begin{itemize}
\item Dependent types as the remedy of most of the run-time failures.
\item Parallelism for numerical computation: these are pure functional algorithms, operate locally on mutable state. Haskell ST, STRef solution enables encapsulating local heaps and mutability within referentially transparent code.
\item Updating game-objects (agents) concurrently using STM: update all objects concurrently in arbitrary order, with each update wrapped in atomic block - depends on collisions if performance goes up.
\end{itemize}

\cite{jankovic_functional_2007} discuss using functional programming for discrete event simulation (DES) and mention the paradigm of Functional Reactive Programming (FRP) to be very suitable to DES. We were aware of the existence of this paradigm and have experimented with it using the library Yampa, but decided to leave that topic to a side and really keep our implementation clear and very basic.

TODO: this seems all to be focused on MAS
\url{http://haskell-distributed.github.io/wiki.html} looks good but too big and not well suited for simulations
\url{https://code.google.com/archive/p/haskellactor/} makes heavy use of IORef and running in IO-Monad, something we deliberately want to avoid to keep the ability to reason about the program.
TODO: \url{https://github.com/fizruk/free-agent} look into

\subsubsection{Composition}
\paragraph{Monads}
TODO: \cite{moggi_computational_1989}
TODO: \cite{wadler_comprehending_1990}
TODO: \cite{wadler_monads_1995}
TODO: \cite{wadler_how_1997}

\paragraph{Arrows}
The main difference between Monads and Arrows are that where monadic computations are parameterized only over their output-type, Arrows computations are parametrised both over their input- and output-type thus making Arrows more general.

"Just like monads, arrow types are useful for the additional operations they support, over and above those that every arrow provides."
"In real applications an arrow often represents some kind of a process, with an input channel of type a, and an output channel of type b."

TODO: \cite{hughes_generalising_2000}
TODO: \cite{Hughes2004} -- cant find this reference anymore, maybe in print-outs?
TODO: \cite{hughes_programming_2005}

In the work \cite{Hughes2004} an example for the usage for Arrows is given in the field of circuit simulation. They use previously introduced streams to advance the simulation in discrete steps to calculate values of circuits thus the implementation is a form of \textit{discrete event simulation} - which is in the direction we are heading already with ABM/S. Also the paper mentions Yampa which is introduced in the section (TODO: reference) on functional reactive programming.

\paragraph{FRP}
Luxury of having the know-how in-house
TODO \cite{courtney_yampa_2003}
TODO \cite{nilsson_functional_2002}
TODO \cite{hudak_arrows_2003}
TODO \cite{mun_hon_functional_2005}
TODO \cite{meisinger_game-engine-architektur_2010}
TODO \cite{nilsson_declarative_2014}
TODO \cite{perez_functional_2016}

\subsubsection{EDSL}
TODO: cite julie greensmith paper on haskell
EDSL steht für Embedded Domain Specific Language d.h. man implementiert in Haskell eine Art von 'Spezifikations-Sprache' für eine spezielle Domain (z.b. ABM/S), die - dank der rein funktionalen, deklarativen Natur von Haskell - auch gleichzeitig Haskell Code ist - der Unterschied zwischen Spezifikation und Implementierung verschwindet dann (idealserweise). In diese Richtung arbeite ich erst seit kurzem, durch die Umsetzung von ABS/M mit Yampa. Yampa ist ebenfalls eine EDSL um funktional-reaktive Systeme zu beschreiben/implementieren, ich werde auf dieser EDSL aufsetzen und sie um ABS/M erweitern - so zumindest der Plan. Dann habe ich die theoretische Grundlage von FRP, auf die ich dann auch theorie von ABS/M (z.b. Actor Semantics) setzen kann und somit zum nächsten Punkt komme:

The most powerful aspect using pure functional programming however is that it allows the design of embedded domain specific languages (EDSL). In this case one develops and programs primitives e.g. types and functions in a host language (embed) in a way that they can be combined. The combination of these primitives then looks like a language specific to a given domain, in the case of this thesis ACE. The ease of development of EDSLs in pure functional programming is also a proof of the superior extensibility and composability of pure functional languages over OO. \cite{henderson_functional_1982} gives a wonderful way of constructing an EDSL do denotationally construct an Escher-Picture.

\subsubsection{Reasoning}
TODO \cite{hutton_programming_2007}
TODO \cite{claessen_quickcheck:_2000}

\subsubsection{Foundations in Category-Theory}
Category-Theory \cite{Pierce1991} \cite{spivak_category_2014}
TODO: cite the 2 papers which i found on category-theory in ABS
apply category theory to agent-based simulation: how can a ABS system itself be represented in category theory and can we represent models in this category theory as well?
ADOM: Agent Domain of Monads: https://www.haskell.org/communities/11-2006/html/report.html
develop category theory behind FrABS: look into monads, arrows

\subsubsection{A final word on LISP}
Being the oldest functional programming language and the 2nd oldest high-level programming language ever created, at one point we considered using LISP in our research due to its immensely powerful feature of homoiconicity. The idea was to investigate if this could be made useful for ABS and bring it to a new level. We abandoned this quickly as it would have led to a total different approach and would have definitely not solved the issues the questions raised in the introduction. TODO cite Nepi2 paper, it uses LISP