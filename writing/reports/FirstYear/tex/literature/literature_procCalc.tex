\section{Process Calculi}
Process Calculi were initially invented for algebraically specifying concurrent systems and their interaction. By mapping concurrent computation to the field of algebra allows to reason about whether two processes are equivalent or not, whether deadlocks can occur and to formally verify correctness of concurrent systems. The main concept in process calculi is the one of independently computing processes which interact with each other via messages which act as points of synchronization between these processes. The first process calculi which were created at around the same time are Communicating Sequential Processes (CSP) \cite{hoare_communicating_1985} and Calculus of Communicating Processes (CCS) \cite{milner_calculus_1980}. A problem of these early process calculi was that connections - the direction of message-flow - between processes were always fixed in advance. This was addressed by Milner in the $\pi$-calculus \cite{milner_calculus_1992}, \cite{milner_calculus_1992-1}, \cite{milner_elements_1993}, \cite{milner_communicating_1999} in which processes can be mobile: connections between processes can change over time. 
By introducing a changing network-structure of communication the $\pi$-calculus is reminiscent to agents which can also be seen as independently running processes which interact with each other through messages. Indeed, research exists which tries to specify agent-based systems in terms of the $\pi$-calculus \cite{esterline_using_2001}, \cite{kawabe_nepi2programming_2000} and in \cite{padget_pi-calculus_1998} the authors model an agent-based Spanish fish market. Unfortunately no research was found on using a process calculus in the field of agent-based \textit{simulation}.