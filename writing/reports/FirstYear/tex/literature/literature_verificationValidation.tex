\section{Verification \& Validation of ABS}
Verification \& Validation are, generally speaking, independent processes to check whether a product meets its requirements and fulfills its intended purpose TODO: cite?. Here we focus explicitly on software verification \& validation where we identify \textit{Verification} to be the process of checking whether an implementation matches a given specification without any bugs or missing parts and \textit{Validation} to be the process of checking if the implementation meets high level requirements. TODO: cite?
To put short the difference between verification \& validation is that verification tries to answer "are we building the product right?" and validation "are we building the right product?" \cite{boehm_software_1989}. For (most of) the software built in the industry in well-defined software-development processes with its own quality control and quality assurance to answer these questions is rather straight-forward. Numerous techniques like checklists, software-tests, integration-tests,... have been developed to deal with Verification \& Validation. TODO: cite?

The question is now how the above applies to ABS. TODO: \url{http://www2.econ.iastate.edu/tesfatsi/empvalid.htm}

TODO: baas: emergence, hierarchies and hyperstructures
TODO: \cite{baas_emergence_1997}
TODO: Burton and Obel 1995 "The validity of computational models in organization science: from model realism to purpose of the model"
TODO: Knepell 1993 "Simulation validation, a confidence assessment methodology."

TODO: \url{http://dspace.stir.ac.uk/handle/1893/3365#.WNjO1DsrKM8}
TODO: \cite{klugl_amason:_2013}

\subsection{Verification}
TODO: \cite{axelrod_advancing_1997}

In the work of \cite{axtell_aligning_1996} the authors tried to see whether the more complex Sugarscape model can be used to reproduce the results of \cite{axelrod_convergence_1995}. In both models agents have a tag for cultural identification which is comprised of a string of symbols. The question was whether Sugarscape, focusing on generating a complete artifical society which incorporates many more mechanisms like trading, war, ressources can reproduce the results of \cite{axelrod_convergence_1995} which only focuses on transmission of these cultural tags. Although interesting the question if two models are qualitatively equivalent is not what we want to pursue in our thesis as it requires a complete different direction of research.
The cooperative work of \cite{axtell_aligning_1996} gives insights into validation of computational models, in a process what they call "alignment". They try to determine if two models deal with the same phenomena. For this they tried to qualtiatively reproduce the same results of \cite{axelrod_convergence_1995} in the Sugarscape model of \cite{epstein_growing_1996}. Both models are of very different nature but try to investigate the qualitatively same phenomoenon: that of cultural processes. TODO: read

TODO: write about ABS as a new tool and  generative as opposed to the classical inductive and deductive sciences. major sources: 
TODO fully read \cite{epstein_chapter_2006}
TODO fully read \cite{epstein_generative_2012}

TODO: \cite{galan_errors_2009}
TODO: \cite{windrum_empirical_2007}

model checking and reasoning in \cite{hutton_tutorial_1999}\\

\subsection{Validation}
