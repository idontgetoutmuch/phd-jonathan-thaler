\section{Background}

\subsection{Implementation}

\subsubsection{Monads}

\subsubsection{Arrows}

\subsubsection{Continuations}
 
\subsubsection{FRP}
TODO: why Yampa? There are lots of other FRP-libraries for Haskell. Reason: in-house knowledge (Nilsson, Perez), start with \textit{some} FRP-library to get familiar with the concept and see if FRP is applicable to ABS. TODO: short overview over other FRP-libraries but leave a in-depth evaluation for further-research out of the scope of the PhD as Yampa seems to be suitable. One exception: the extension of Yampa to Dunai to be able to do FRP in Monads, something which will be definitely useful for a better and clearer structuring of the implementation.
TODO: Push vs. Pull

TODO: describe FRP

TODO: 1st year report Ivan: "FPR tries to shift the direction of data-flow, from message passing onto data dependency. This helps reason about what things are over time, as opposed to how changes propagate". QUESTION: Message-passing is an essential concept in ABS, thus is then FRP still the right way to do ABS or DO WE HAVE TO LOOK AT MESSAGE PASSING IN A DIFFERENT WAY IN FRP, TO VIEW AND MODEL IT AS DATA-DEPENDENCY? HOW CAN THIS BE DONE?
BUT: agent-relations in interactions are NEVER FIXED and always completely dynamic, forming a network. The question is: is there a mechanism in which we have explicit data-dependency but which is dynamic like message-passing but does not try to fake method-calls? maybe the conversations come very close

\subsection{Specification}
\subsubsection{Category Theory}
include paper on arrows my hughes

apply category theory to agent-based simulation: how can a ABS system itself be represented in category theory and can we represent models in this category theory as well?

ADOM: Agent Domain of Monads: https://www.haskell.org/communities/11-2006/html/report.html

 develop category theory behind FrABS: look into monads, arrows