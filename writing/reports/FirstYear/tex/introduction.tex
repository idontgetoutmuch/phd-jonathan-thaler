\section{Introduction}

The central aspect of my PhD is centred around the main question of \textit{How can Agent-Based Simulation be done using pure functional programming and what are the benefits and disadvantages of it?}. So far functional programming has not got much attention in the field of ABS and implementations always focus on the object-oriented approach. We claim, based upon the research of the first year that functional programming is very well suited for ABS and that it offers methods which are not directly possible and only very difficult to achieve with object-oriented programming. 

We claim that to build large and complex agent-based simulations in functional programming is possible using the functional reactive programming (FRP) paradigm. We applied FRP to implementing ABS and developed a library in Haskell called FrABS. We implemented the quite complex model SugarScape from social simulation using FrABS and proofed by that, that applying FRP to ABS enables ABS to happen in pure functional programming.

After having shown how agent-based simulation can be done in functional programming we claim that the major benefit of using it enabled a new way of \textit{verification \& validation} in agent-based simulation. 

Due to the declarative nature of pure functional programming it is an established method of implementing an EDSL to solve a given problem in a specific domain. We followed this approach in FrABS and developed an EDSL for ABS in pure functional programming. Our intention was to develop an EDSL which can be used both as specification- and implementation-language. We show this by specifying all the rules of SugarScape in our EDSL.

Due to the lack of implicit side-effects and the recursive nature of pure functional programming we claim that it is natural to apply it to a novel method we came up with: MetaABS, which allows recursive simulation.

Finally having such an EDSL at hand this will allow us to reason about the programs. This will be applied to specify and reason about the dynamics and emergent properties of decentralized bilateral trading and bartering in agent-based computational economics (ACE) and social simulations like SugarScape.

Disadvantages
- although the lack of side-effects is also a benefit, it is also a weakness as all data needs to be passed in and out explicitly 
- indirection due to the lack of objects \& method calls.
- When not to use it: 
	- if you are not familiar with functional programming
	- when you can solve your problem without programming in a Tool like NetLogo, AnyLogic,...
	- when you don't need to reason about your program
	
I noticed that it is pretty hard to convince an agent-based economics specialist who is not a computer scientist about a pure functional approach. My conjecture is that the implementation technique and method does not matter much to them because they have very little knowledge about programming and are almost always self-taught - they don't know about software-engineering, nothing about proper software-design and architecture, nothing about software-maintenance, nothing about unit-testing,... In the end they just "hack" the simulation in whatever language they are able to: C++, Visual Basic, Java or toolboxes like Netlogo. For them it is all about to \textit{get things done somehow} and not to get things done the right way or in a beautiful way - the way and the method doesn't matter, its just a necessary evil which needs to be done. Thus if functional programming could make their lives easier, then they will definitely welcome it. But functional programming is, i think, harder to learn and harder to understand - so one needs to provide an abstraction through EDSL. So I REALLY need to come up with convincing arguments why to use pure functional approaches in ACE THEY can understand, otherwise I will be lost and not heard (not published,...). \\

What ACE economists care for:

\begin{itemize}
\item Very: Qualitative modelling with quantitative results
\item Yes: Easy reproducibility
\item Likely: Reasoning about convergence?
\item Likely: EDSL
\end{itemize}

My contributions are: pure functional framework, functional agent-model for market-simulations, EDSL for market-simulations, qualitative / implicit modelling with quanitative results, reasoning in my framework about convergence \\

IDEA: could I develop non-causal modelling (models are expressed in terms of non-directed equations, modelled in signal-relations) to allow for qualitative modelling for the agent-based economists? See hybrid modelling paper of Yampa. \textbf{THIS WOULD BE A HUGE NOVEL CONTRIBUTION TO ACE ESPECIALLY WHEN COMBINED WITH AN EDSL AND PROVIDING FULL REFERENTIAL TRANSPARENCY TO KEEP THE ABILITY TO REASON ABOUT CONVERGENCE}. This should be covered in the "EDSL"-paper.

TODO: maybe i should really focus only on market models? otherwise too much? \\

central novelty of my PhD: model specification = runnable code. possible through EDSL. but only in specific subfield of ACE: market-models. need a functional description of the model, then translate it to model specification in EDSL and then run it to see dynamics. But: model specification moves closer to functional programming languages. \\

another novelty approach: model specification through qualitative instead of quantiative approaches. is this possible? \\

WHY FUNCTIONAL? "because its the ultimate approach to scientific computing": fewer bugs due to mutable state (why? is thos shown obkectively by someone?), shorter (again as above, productivity), more expressive and closer to math, EDSL, EDSL=model=simulation, better parallelising due to referental transparency, reasoning \\

scientific results need to be reproduced, especially when they have high impact. a more formal approach of specifying the model and the simulation (model=simulation) could lead to easier sharing and easier reporduction without ambigouites \\

pure functional agent-model \& theory, EDSL framework in Haskell for ACE

\begin{enumerate}
\item Which kind of problem do we have?
\item What aim is there? Solving the problem? 
\item How the aim is achieved by enumerating VERY CLEAR objectives.
\item What the impact one expects (hypothesis) and what it is (after results).
\end{enumerate}

Note: It is not in the interest of the researcher to develop new economic theories but to research the use of functional methods (programming and specification) in agent-based computational economics (ACE).

NOTE: Get the reader’s attention early in the introduction: motivation, significance, originality and novelty.

Methods need to be selected to implement the simulations. Special emphasis will be put on functional ones which will then be compared to established methods in the field of ABM/S and ACE. \\

Claim: non-programming environments are considered to be not powerful enough to capture the complexity of ACE implementations thus a programming approach to ACE will be always required.


To apply and test functional methods in ACE, four scenarios of ACE are selected and then the methods applied and compared with each other to see how each of them perform in comparison. The 4 selected scenarios represent a selection of the challenges posed in ACE: from very abstract ones to very operational ones.

Each of the selected scenarios is then implemented using the selected methods where each solution is then compared against the following criteria: 

\begin{enumerate}
\item suitability for scientific computation
\item robustness
\item error-sources
\item testability
\item stability
\item extendability
\item size of code
\item maintainability
\item time taken for development
\item verification \& correctness
\item replications \& parallelism
\item EDSL
\end{enumerate}

This will then allow to compare the different methods against each other and to show under which circumstances functional methods shine and when they should not be used.




We understand ABS as a method of modelling and simulating a system where the global behaviour may be unknown but the behaviour and interactions of the parts making up the system is of knowledge. Those parts, called agents, are modelled and simulated out of which then the aggregate global behaviour of the whole system emerges. So the central aspect of ABS is the concept of an agent which can be understood as a metaphor for a pro-active unit, situated in an environment, able to spawn new agents and interacting with other agents in a network of neighbours by exchange of messages \cite{wooldridge_introduction_2009}. It is important to note that we focus our understanding of ABS on a very specific kind of agents where the focus is on communicating entities with individual, localized behaviour from out of which the global behaviour of the system emerges. We informally assume the following about our agents:

\begin{itemize}
	\item They are uniquely addressable entities with some internal state.
	\item They can initiate actions on their own e.g. change their internal state, send messages, create new agents, kill themselves.
	\item They can react to messages they receive with actions as above.
	\item They can interact with an environment they are situated in.
\end{itemize} 

\subsection{Problems of ABS in OO}
- Objects don't compose
- implicit state, change through effectful computations
- blurring of fundamental difference between agent and object: an agent is a metaphor, it is much more than an object. an object is: a uniquely identifieable compound of functions (=methods) and data

\subsection{Problems of ABS in general}
Specification: how is my model specified? 
Verification: does my implementation really match my specification? 
Validation: how to connect the results to the hypothesis? are the emergent properties the ones anticipated? if it is completely different why? note: we always MUST HAVE a hypothesis regarding the outcome of the simulation, otherwise we leave the path of scientific discovery. But we must admit that sometimes it is extremely hard to anticipate \textit{emergent patterns}. But anyway there must be \textit{some} hypothesis regarding the dynamics of the simulation.
The idea is to express hypotheses directly in the program using QuickCheck and then let the simulation verify it

\subsection{Functional approach to Agent-Based Modelling \& Simulation}
Because we left the path of OO and want to develop a completely different method we have fundamentally two problems to solve in our functional method:
1. Specifying the Agent-Based Model (ABM): Category-Theory, Type-Theory, EDSL: all this clearly overlaps with the  implementation-aspect because the theory behind pure functional programming in Haskell is exactly this. This is a very strong indication that functional programming may be able to really close the gap between specification and implementation in ABS.
2. Implementing the ABM into an Agent-Based Simulation (ABS): building on FRP paradigm

\subsubsection{Challenges}
- how is an agent represented?
- how do agents pro-actively act?
- how do agents interact?
- how is the environment represented?
- how can agents act on the environment?
- how to handle structural dynamism (creation and removal of agents)?

\subsubsection{Expected benefits}
1. By mapping the concepts of ABS to Category-Theory and Type-Theory we gain a deeper understanding of the deeper structure of Agents, Agent-Models and Agent-Simulations.
2. The declarative nature of pure functional programming will allow to close the gap between specification and code by designing an EDSL for ABS in Haskell building on the previously derived abstractions in Category-Theory and Type-Theory. The abstractions and the EDSL implementation will then serve as a specification tool and at the same time code.
3. The pure functional nature together with the EDSL and abstractions in Category- \& Type-Theory allow for a new level of formal verification \& validation using a combination of mathematical proofs in Category- \& Type-Theory, algebraic reasoning in the EDSL and model-checking using Unit-Tests and QuickCheck. The expectation is that this allows us to formally specify hypotheses about expected outcomes about the dynamics (or emergent patterns) of our simulations which then can be verified.

\subsection{Field of Application}
Agent-Based Modelling \& Simulation is a method and tool and thus always applied in a very specific domain in which phenomenon are being researched which can be mapped to ABS. For our PhD we picked the field of Agent-Based Computational Social Sciences (ACS) with slight influences from Agent-Based Computational Economics (ACE). The reason for this is that ACS was one of the first fields to adopt ABS in their research on artificial societies and is still a strong driving force behind the application of ABS. ACE is about the same age as ACS but is not yet nearly as established in Economics as ACS is in the Social Sciences (TODO: can i really back up my claims?). ACS draw upon findings of ACE in the SugarScape Model where Agents engage in bilateral decentralized bartering. The author claims that both fields are highly relevant for the future. Economists start realizing that more heterogenous models need to be studied which can only be done using ABS.
Our primary field of application will be ACS but we will sometimes draw upon ACE when appropriate e.g. in SugarScape. By applying our new method to these fields we hope to bring forward a paradigm-shift which allows to better understand the concepts of ABS and to have more powerful tools for verification \& validation at hand.

TODO: what are the aims of ACS and ACE?

1. Initial Use-Case: SugarScape Model as in "Growing Artificial Societies: Social Science From the Bottom Up"
2. Supplementary Material: "Generative Social Science: Studies in Agent-Based Computational Modelling"
3. Verification \& Validation Use-Case: "Agent\_Zero: Toward Neurocognitive Foundations for Generative Social Science"

