\section{Part III: Phd Research} 
Describe the idea, motivation and research questions of the method I want to research 
 
\subsection{Research Questions} 

\subsubsection{Main} How can Agent-Based Simulation be done using pure functional programming and what are the benefits and disadvantages of it?
this is the main thread of my PhD, everything needs to be connected to this

\paragraph{How?}
\paragraph{Benefits?}
\paragraph{Disadvantages?}



\section{Meta Agent-Based Simulation}
- TODO:  i have only the idea but am lacking a theory or hypothesis for its use

- meta need a kind of decision error measure to distinguish between various meta-simulations. also we need a mechanism to sample the decision space => it can be considered to be an optimization technique.

\subsection{Overview}
\subsubsection{Idea} Give each  Agent the ability to run the simulation locally from its point of view do anticipate its actions and change them in the future thus introducing a meta-level in the simulation, from which the method derives its name.
\subsubsection{Problems} 
\begin{itemize}
	\item Definition of a recursive, declarative description of the Model.
	\item Perfect information about other agents is not realistic and runs counter to agent-based simulation (especially in social sciences) thus an Agent needs to be able to have local, noisy representations of the other agents.
	\item Local representation of other agents could be captured by Hidden Markov Models: observe what other agents do but have hidden interpretation of their internal state - these internal state-representations can be different between the local and the global version whereas the agent learns to represent the global version as best as possible locally.
	\item Infinite regress is theoretically possible but not on computers, we need to terminate at some point
\end{itemize}

\subsubsection{Interpretation} It can be regarded as a Model of Free Will in ABS, which allows learning in an ABS environment in a new way - look on the section of interpretation.
\subsubsection{Application} hypothesis: allows to model social and psychological phenomena like free will. Mostly in social sciences, maybe also in economics. Investigate SugarScape, PrisonersDilemma and ACE Trading World

TODO: question: what is the meaning of an entity running simulations? it strongly depends on the context: in ACE it may be search for optimization behaviour, in Social Simulation it may be interpreted as a kind of free will

\subsubsection{Research Questions} 
\begin{enumerate}
	\item How does deep regression influence the dynamics of a system? Hypothesis: TODO
	\item How do the dynamics of a system change when using perfect information or learning local information? Hypothesis: TODO
	\item Is a hidden markov model suitable for the local learning? Hypothesis: TODO
	\item How can MetaABS best be implemented? Hypothesis: implementing a MetaABS EDSL in a pure functional language like Haskell, should be best suited due to its inherent recursive, declarative nature, which should allow a direct mapping of features of this paradigm to the specification of the meta-model
\end{enumerate}

- functional programming perfect. standard toolkits (anylogic, netlogo, repast) are not capable of doing this
- extend my existing EDSL for functional reactive agent-based simulation \& modelling (FrABS/M) with recursive functionality
 
\subsubsection{Related Research}
TODO: \cite{gilmer_recursive_2000} cite paper of recursive simulation: [ ] military simulation, [ ] not explicitly abs, [ ] implemented in c++, [ ] deterministic models seem to benefit significantly from using recursions of the simulation for the decision making process. when using stochastic models this benefit seems to be lost