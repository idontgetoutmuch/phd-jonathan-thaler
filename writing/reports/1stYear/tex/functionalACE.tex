\section{Functional ACE}

\subsection{Motivation}
\cite{tesfatsion_agent-based_2006} gives a broad overview of agent-based computational economics (ACE), gives the four primary objectives of it and discusses advantages and disadvantages. She introduces a model called \textit{ACE Trading World} in which she shows how an artificial economy can be implemented without the \textit{Walrasian Auctioneer} but just by agents and their interactions. She gives a detailed mathematical specification in the appendix of the paper which should allow others to implement the simulation.

\subsection{Research Questions}

MY INTERESTS:
- Artificial economies: \cite{tesfatsion_agent-based_2006}, \cite{gintis_emergence_2006}, \cite{gintis_dynamics_2007}, \cite{gaffeo_adaptive_2008}
- Artificial markets: \cite{mackie-mason_chapter_2006}, \cite{darley_nasdaq_2007}
- Market Design: \cite{marks_chapter_2006}, \cite{budish_editors_2015}

Benefits of pure functional paradigm:
- no side-effects
- static, strong type-system

Implications of benefits:
- EDSL, expressive
- fewer LoC
- fewer bugs
- parallelism
- Debugging: QuickCheck

Possible ideas / directions
- Software-Quality
	- EDSL for modelling, where model specification = haskell code
	- QuickCheck in ABS (first useage in this field?)

- Qualitative Modelling
	- EDSL for qualitative, descriptive modelling instead of quantitative

\subsection{TODO}
\paragraph{Hypothesis} TODO
