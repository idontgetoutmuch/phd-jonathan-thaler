\chapter{Agent Representation}

\section{OO}
In the OO paradigm an Agent will (almost) always be represented as an object which encapsulates the state of the Agent and implements the behaviour of the Agent into private and public methods. Care must be taken to not confuse the concept of an Agent with the one of an object: an Agent is pro-active and always in full control over its state and the messages sent to it. We have discussed pro-activity in the update-strategies paper already: what is needed is a method to update the Agent which transports some time-delta to allow the Agent perceive time, ultimately allowing it to become pro-active. Other options would be to spawn a thread within the object which then makes the object an \textit{active} object but then one needs to deal with synchronization issues in case of Agent-Agent interactions. 
In OO it is tempting to generate getter and setter for all properties of the Agent state but this would make the Agent vulnerable to changes out of its control - state-changes should always come from the Agent within. Of course when generating text- or visual output then getter are required for the properties which need to be observed.
Agent-creation in OO is then in the end an instantiation of an Agent-Class resulting in an Agent-Object. When one strictly avoids setter-methods then the only way of instantiating the Agent into a consistent state is the constructor. This could lead to a very bloated constructor in the case of a complex Agent with many properties. Still we think this is better than having setter-methods as setters are always tempting to be used outside of the construction phase, especially when multiple persons are working on the implementation or when the original implementer is not available any more. If an Agent construction is really complicated with many constructor-parameters one can resort to the Builder-Pattern \cite{bloch_effective_2014}. Another approach to creation is dependency injection \footnote{See \url{https://martinfowler.com/articles/injection.html}} but then the application would need to run in an IoC container e.g. Spring. We haven't tried this approach but we think it would over-complicate things and is an overkill in the domain of ABS. TODO: add some illustrating code

\section{FP}
Although there exist object-oriented approaches to functional programming (e.g. F\#, OCaml) we assume that there are no classes and inheritance in FP. By a class we understand a collection of functions (called methods in OO) and data (members or properties in OO) where the functions can access this data without the need to explicitly pass it in through arguments.
So we need functions which represent the Agent's behaviour and data which represents the Agents state. The functions need to access this state somehow and be able to change the state. This may seem to be an attempt to emulate OO in FP but this is not the case: functions operating on data are not an OO-exclusive concept - it becomes OO when the data is implicitly bound in the function \footnote{We are aware that OO is characterized by many more features e.g. inheritance, but we don't go into those details here as they are not relevant anyway - we simply want to show the subtle differences in Agent-representation of FP and OO where it suffices to emphasise the concept of implicitly / explicitly bound data}.
FP in general has no notion of a compound data-type but tuples can be used to emulate such. Because it is quite cumbersome to work on tuples or to emulate compound data-types using tuples, FP languages (e.g. Haskell) have built-in features for compound data-types. So we assume that without loss of generality (because compound data-types are in the end tuples with different names for projection-functions) Agent state in FP is represented using a compound data-type.
The relevant function in FP for Agent-Behaviour is the update-function. We have two options:
Either the function arguments are the compound agent-state, time-delta and incoming messages and must return the (changed) compound agent-state and outgoing messages.
Or we use continuation-style programming in which the compound agent-state is updated internally and the only input are the time-delta and incoming messages and the output are outgoing messages, the observable agent-state which can be represented by a different compound data-type AND a continuation function.
In the first approach the full agent-state is available outside and could be changed any time - there is no such thing as data-hiding in this case. In the continuation case the state is bound in a closure which is the newly constructed function which will be returned as continuation. This is only possible in a real functional language which allows the construction of functions through lambdas AND return them as a return value of a function. TODO: add some illustrating code