%%%%%%%%%%%%%%%%%%%%%%%%%%%%%%%%%%%%%%%%%
% University/School Laboratory Report
% LaTeX Template
% Version 3.1 (25/3/14)
%
% This template has been downloaded from:
% http://www.LaTeXTemplates.com
%
% Original author:
% Linux and Unix Users Group at Virginia Tech Wiki 
% (https://vtluug.org/wiki/Example_LaTeX_chem_lab_report)
%
% License:
% CC BY-NC-SA 3.0 (http://creativecommons.org/licenses/by-nc-sa/3.0/)
%
%%%%%%%%%%%%%%%%%%%%%%%%%%%%%%%%%%%%%%%%%

%----------------------------------------------------------------------------------------
%	PACKAGES AND DOCUMENT CONFIGURATIONS
%----------------------------------------------------------------------------------------

\documentclass{article}

\usepackage[utf8]{inputenc}
\usepackage{graphicx} % Required for the inclusion of images
\usepackage{natbib} % Required to change bibliography style to APA
\usepackage{amsmath} % Required for some math elements 

\setlength\parindent{0pt} % Removes all indentation from paragraphs

%\usepackage{times} % Uncomment to use the Times New Roman font

%----------------------------------------------------------------------------------------
%	DOCUMENT INFORMATION
%----------------------------------------------------------------------------------------

\title{Metathoughts} % Title

\author{Jonathan \textsc{Thaler}} % Author name

\date{\today} % Date for the report

\begin{document}

\maketitle % Insert the title, author and date

% If you wish to include an abstract, uncomment the lines below
\begin{abstract}
I started this document on 10th of August 2016 because I wanted to write down meta-thoughts which formed during my PhD studies. These thoughts, which I termed metathoughts du to their nature, touch upon the very basic nature of computation, how it connects to reality, whether computation can produce conciousness, whether computation can produce intelligence, about the the whole existence as a computational model (to achieve what?), about the whole existence as a simulation (to simulate what?), about the future of humankind in connection with it's computation tools, the future of science and a new way of science. The intention of this document is mainly to give the thoughts a clearer structure, to refine and clarify them in the process of writing them down and to be able to share those thoughts and ideas.
\bigskip

"Thou shalt not make a machine in the likeness of a man's mind" - Dune (Book 1)
 
\end{abstract}

\section{Philosophical thoughts on computation and existence}
\begin{itemize}
\item What connects our view of computation to reality? Where does computation occur in reality? is our notion of computation even applicable to reality? what is computation? what is computation in nature? does computation even occur in nature? is reality discrete or continuous?
\item I have the feeling that the way we regard computation works well for our very limited computer tools but has absolutely no connection to reality and that we must look at computation from a complete different approach to connect it to reality.
\item The way we look at reality is not suitable to touch the most fundamental questions. We look at it separated from it. We must look inward ourselves, there we will find answers. This is contrary to existing scientific approaches. Thus a new way of scientific working has to be established. The problem of existing science is that it piles up endless heaps of complexity which is tractable only for specialcases and leads to further distance and separation from reality.
\item When we have conceived a new way of science and a new way of thinking of computation which has actually a connection to reality this will bring us closer to the concepts of "reality", "the creation", "self-conciousness", "existence", "truth" and "god".
\item I believe the future of simulation is that at some point we will create our own creation to run simulations thus ascending to gods. Maybe this is what god is: an entity which created this creation (within a much bigger context/universe) with the intention of "simulating" "something". But what is it that is simulated? A few religions and tell about a cicle of creation and destruction, which would amount to running replications of the same simulation... 
\item calculating device (brain) and conciousness (soul) = true intelligence
\end{itemize}


\section{New Kind of Science}
TODO add table 4 in black swan book from position 5578

\section{Cascading Simulations}
At some point in the existence of a free-will intelligence, it starts to asking for the future. First using religion, then mathematics, then finally computer simulation. But the problem is that such a simulation is too weak to forecast the future because out of simple computation no free will is born. thus the solution of the free-will intelligence is to put itself in a simulation-environment as a seed of free will. this simulation will then play through every decision branches an thus be able to predict possible futures. because within such a simulation the same thing can AND WILL happen at some point, we arrive at cascading simulations within simulations. thus we are at one level of this cascade where our direct outer level is god.
\end{document}

\section{On parallel universes, existence as simulation, free will}
We cannot predict the future due to complex interaction of free will of Humankind. To predict it we would have to spawn a new universe running in parallel if a free-will choice occurs. Then again, maybe this is already the case and the whole existence is an extremely huge tree of parallel universes being created from each other and collapsing back into others or being completely determined. \\
The question is then: Where in this tree am I? And maybe time does only advance in discrete steps after a spawn/collapse? \\
When one looks at the existence as a simulation then one can say that it has become unstable because too many actors with free will and too many variables producing unforeseeable consequences. But then, can we make predictions about a simulation from within? Can we talk about the meaning and meta-workings of a system from within it? \\
We always try to treat reality as smooth and predictable without outliers but ignoring catastrophic events - this is what the book "Black Swan" says. My own point of view is that the problem is the way we do science: "we divide and put reality into small boxes of labels/categories and then pile them up, adding piles of theories describing it creating a mountain of unbearable complexity - just to be caught by surprise by the next catastrophic event no one could predict despite the overwhelming amount of complex theories. \\
What's the problem? Theories describe the past. Science needs to move on to the now letting go of the myriads of categories and look at it all as a single complex system/simulation - the world as a simulation, simulating the interaction of free will, allowing it to unfold and see the effects in all facets. \\
The question is whether "Black Swans" are an emergent system property coming from within the simulation or whether they are created from steering forces e.g. God.

\section{My own expectations about my Ph.D.}
I will be clear about the expectations about my Ph.D.: It won't change the world. It will add another minor note to the ever increasing amount of science produced in this field. With a bit of luck I will find (literally find) something of minor importance but I am realistic enough to not expect my Ph.D. to have any serious impact in the classical scientific community as it exists yet (except with a bit of luck with my ideas about equilibrium processes but that is a different topic). That doesn't matter, I don't really want it. What I really want and what I expect about my Ph.D. is simply a fascinating and funny adventure, a short tour into the strange scientific community, which takes itself but so too serious. \\
I expect to advance my skills in english, formal working, precise thinking and so on... But what really drives me is the by-product of the Ph.D.: the meta-thoughts, which come up during reflecting about my research and results. I hope that I can publish them one day and that they maybe change the world. I will pursue interesting and fascinating ideas and theories in my Ph.D. but I cannot predict their impact nor do I plan it.

\section{Idea for a "philosophical computer science" paper}
All sciences have their genesis-model (how did the world come into existence, what is the reason for existence?) but computer science has not, so I want to set out to develop such from a computer science perspective - actually I want to disguise deep spiritual truths in scientific cloak, that would be really fun. \\
Thus the idea is to write a paper with the title "Reality as an agent-based simulation of free will" in which I motivate that the world and humankind is a simulation to see free will in action in a sandboxed environment (as opposed to spirit, which is an outer level of simulation). Thus I need to develop then a model of free will from a computer science perspective and give an answer WHO implemented the simulation.