%%%%%%%%%%%%%%%%%%%%%%%%%%%%%%%%%%%%%%%%%
% University/School Laboratory Report
% LaTeX Template
% Version 3.1 (25/3/14)
%
% This template has been downloaded from:
% http://www.LaTeXTemplates.com
%
% Original author:
% Linux and Unix Users Group at Virginia Tech Wiki 
% (https://vtluug.org/wiki/Example_LaTeX_chem_lab_report)
%
% License:
% CC BY-NC-SA 3.0 (http://creativecommons.org/licenses/by-nc-sa/3.0/)
%
%%%%%%%%%%%%%%%%%%%%%%%%%%%%%%%%%%%%%%%%%

%----------------------------------------------------------------------------------------
%	PACKAGES AND DOCUMENT CONFIGURATIONS
%----------------------------------------------------------------------------------------

\documentclass{article}

\usepackage{graphicx} % Required for the inclusion of images
\usepackage{natbib} % Required to change bibliography style to APA
\usepackage{amsmath} % Required for some math elements 

\setlength\parindent{0pt} % Removes all indentation from paragraphs

\renewcommand{\labelenumi}{\alph{enumi}.} % Make numbering in the enumerate environment by letter rather than number (e.g. section 6)

%\usepackage{times} % Uncomment to use the Times New Roman font

%----------------------------------------------------------------------------------------
%	DOCUMENT INFORMATION
%----------------------------------------------------------------------------------------

\title{Research Proposal \\ Formal methods and pure functional computation in agent-based computational economics.} % Title

\author{Jonathan \textsc{Thaler}} % Author name

\date{\today} % Date for the report

\begin{document}

\maketitle % Insert the title, author and date

\begin{center}
\begin{tabular}{l r}
Institution: University of Nottingham \\
Department: School of Computer Science \\
Project: PhD Programme 2016-2019 \\
Supervisor: Peer-Olaf Siebers \\
Co-Supersvisor: Thorsten Altenkirch 
\end{tabular}
\end{center}

% If you wish to include an abstract, uncomment the lines below
\begin{abstract}
Agent-Based Modeling and Simulation (ABM/S) is still a young discipline and the dominant approach to it is object oriented computation. This thesis goes into the opposite direction and asks how ABM/S can be mapped to and implemented using pure functional computation and what one gains from doing so. To the best knowledge of the author, so far no proper treatment of ABM/S in pure functional computation exists but only a few papers which only scratch the surface. The author argues that approaching ABM/S from a pure functional direction  offers a wealth of new powerful tools and methods. The most obvious one is that when using pure functional computation (equational) resoning about the correctness and about total and partial correctness of the simulation becomes possible. The ultimate benefit is that Agda becomes applicable which is both a pure functional programming language and a proof assistant allowing both to compute the dynamics of the simulation and to look at meta-level properties of the simulation - termination, convergence, equilibria, domain specific properties - by constructing proofs utilizing computer aided verification. \\
To map ABM/S to pure functional computation the idea is to apply both Robin Milner's PI-calculus and category theory. The PI-calculus will be used for a formal modelling of the problem and allows already a basic form of algebraic reasoning. Then the agents and the process of the agent-simulation will be mapped to category theory because pure functional programming approaches complex problems from the direction of category theory in the form of monadic programming. \\
The application will be in the field of agent-based computational economics where the approach will be to take an established model/theory and then apply the above mentioned methods to it and to show that using them will lead to the same results. 
\end{abstract}

%----------------------------------------------------------------------------------------
%	SECTION 1
%----------------------------------------------------------------------------------------

\section{Application}
Question: It is still open how exactly and to what exactly I want to apply my research. Answer: I take an established theory and then apply my method to it and show that it is useful.
Question: Which theory to take? Answer: best would be an experimental one which is agent-based so I won’t be stuck in modeling agents but can directly apply my method to it.

Topics Barter Economics
- The emergence of a price system from decentralized bilateral exchange
- An extensible and scalable agent-based simulation of barter economics
Topic: Market Design / Auction Design
 - ?

Various auction types will be simulated using an agent-based approach where the following auction types are of interest:
- Continuous Double-Auctions (CDA)
- (Continuous) Batch Auctions ((C)BA)
- Sealed-Bid Auction (SBA)
- Simultaneous Ascending Auction (SAA)
In this research the price-formation during such auctions are of no direct interest e.g. this research tries not to research best bidding strategies but is only interested in investigating the dynamics of the simulation-process as the auction-process unfolds. Thus so called Zero-Intelligence (ZI) Agents are used in the simulation.
ZI-Agents are a concept from computational economics (Gode and Sunders) and model traders which bid (buy) and ask (sell) randomly within constraints to increase their utility (the expected earnings from goods hold by the agent). Constraints may vary but always depend on the auction design and always include the avoidance of bankruptcy and negative holdings to conserve the wealth in the system. ZI-Agents do not learn or adapt over time and are completely deterministic in their behaviour as one always know what the agent will do next depending on its current state.
Both ZI-Agents and the mentioned auction-types are very well researched topics in economics and are used as a tool for studying properties and behaviours of market institutions and designs in computational economics. To emphasise this, it is worth mentioning that Experimental Economists believe that the final state of a CDA comes close to what theoretical equilibrium predicts, thus linking the dynamic process of a CDA to the static description of the equilibrium theory. Besides it was shown that ZI-Agents raise allocative efficiency to nearly 100\% (all products match the consumer’s preferences – if allocative efficiency is at 100\% then there is no pair of two traders which can still increase BOTH their utilities when trading with each other).
So far ZI-Agents have been investigated in CDA only and lack the adoption to other auction-types. One of the challenges of this research will be to develop ZI-Agents for the other auction-types – if possible.
To my best knowledge, no research on formally specifying the auction-types and the ZI-Agents exists yet and no formal method has been developed yet which supports this goal. So far the mechanics and specifications are provided in natural language, thus leaving much room for error and missing details which was experienced by me in my master thesis where I implemented a simulation of a CDA with ZI-Agents. Formal methods mentioned in "Motivation" can be used as a remedy for this problem. 
Note: it is not in the interest of the researcher to proof the match of the outcome of the various auction types with established equilibrium-theorems (which conditions must be satisfied that the dynamic process reaches the static equilibrium?) because this would touch too deep on economics and complex equilibrium theory which is not the focus of this PhD. The focus is much more on the research of a specification tool and then on the dynamics of such an auction- process (e.g. market \& agent activities over time) in the context of the field of Agent-Based Modelling/Simulation.

 
%----------------------------------------------------------------------------------------
%	SECTION 2
%----------------------------------------------------------------------------------------

\section{Goals and Outcome}
The goal is to develop a formal language for formally specifying simulation processes of the given auction-types with ZI-Agents and an accompanying simulation framework implemented in a pure functional language. It should be possible to translate a concrete specification directly without (important) loss of expressiveness to the simulation framework and be executed as a simulation-process to reveal its dynamics over time when actual computation is carried out over the formal language.
An important emphasis will be put on the verification of correctness, whether the implementation matches the formal specification or not, of both the simulation framework and translated auction-type programs.
Research will also go into proofs of termination / convergence-possibility: proving whether a specified auction can ever terminate / converge, ignoring the time-space constraint and the quality of the convergence. Another focus of research will focus on formal testing of an auction specification / implementation for deadlock-free as such auction-processes can get stuck when specifications exhibit configurations which can result in deadlock-situations between traders which are still willing to trade but cannot because of circular dependencies.
Of special interest is also the visualization and the support to investigate the dynamics of such simulation-processes. Part of the research will be how to describe and visualize dynamics of such an auction-process in a meaningful and comprehensive way so to detect patterns in the dynamics and to provide a new way at looking at such dynamics. With the use of a pure functional approach it may be possible to develop a new way of looking at such dynamics which can then also be visualized in a new way.

%----------------------------------------------------------------------------------------
%	SECTION 3
%----------------------------------------------------------------------------------------

\section{Impact}
The resulting auction simulation tool (formal language and simulation framework) allows computational economists to specify problems in this field in a precise and concise formal language thus they can refrain from having to share and explain all painstaking details in natural language. The simulation framework allows them to run formal specifications of auction-processes and investigate the dynamics of them. Also they will be able to proof if their specification is able to converge / terminate or not and that it is deadlock-free or not.
Another advantage of a formal specification is that two specifications can be proofed for bisimilarity where one matches the other which is quite difficult in natural language.


%----------------------------------------------------------------------------------------
%	SECTION 4
%----------------------------------------------------------------------------------------

\section{Approach}
Formal specification is done both through pi-calculus and due to the pure functional approach a mapping of ABM/S to category theory is required. Both approaches are quite novel and not much research has been undertaken in both areas. This work tries to approach both to  TODO
[ ] as current pure functional programming is based upon category theory a suitable representation of the agents and the simulation in category theory must be found/developed
[ ] 1. formal ABM/S allgemein untersuchen, 2. dann auf process-calculi focussieren, 3. dort im konkreten fall pi-calculus ansehen, 4. daraus folgt nur bestimmte arten von ABM/S vernünftig mit pi calculus umsetzbar/spezifizierbar: keine komplexen verhaltensregeln sondern mathematische, funktionale, stetige verhalten wie eben zero-intelligence agents. 5. konkreter anwendungsfall dann auf breuer et al CDA und continuous batch auction und weitere market designs

TODO: it is still unclear how i will model the agents and implement them. Either follow agent-model if provided in theory or have to model myself. Implementing: heavily depends on agent-model but question if we can follow actor model? Would actor model make sense? Then formalize it into pi-calculus?


%----------------------------------------------------------------------------------------
%	SECTION 5
%----------------------------------------------------------------------------------------

\section{Research Questions}
TODO

%----------------------------------------------------------------------------------------
%	BIBLIOGRAPHY
%----------------------------------------------------------------------------------------

\bibliographystyle{apalike}

\bibliography{researchproposal}

%----------------------------------------------------------------------------------------

\end{document}