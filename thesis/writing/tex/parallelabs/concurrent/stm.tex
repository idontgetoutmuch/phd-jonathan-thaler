\section{Software Transactional Memory}
Software Transactional Memory (STM) was introduced by \cite{shavit_software_1995} in 1995 as an alternative to lock-based synchronisation in concurrent programming which, in general, is notoriously difficult to get right. This is because reasoning about the interactions of multiple concurrently running threads and low level operational details of synchronisation primitives is \textit{very hard}. The main problems are:

\begin{itemize}
	\item Race conditions due to forgotten locks;
	\item Deadlocks resulting from inconsistent lock ordering;
	\item Corruption caused by uncaught exceptions;
	\item Lost wake-ups induced by omitted notifications.
\end{itemize}

Worse, concurrency does not compose. It is very difficult to write two functions (or methods in an object) acting on concurrent data which can be composed into a larger concurrent behaviour. The reason for it is that one has to know about internal details of locking, which breaks encapsulation and makes composition dependent on knowledge about their implementation. Therefore, it is impossible to compose two  functions e.g. where one withdraws some amount of money from an account and the other deposits this amount of money into a different account: one ends up with a temporary state where the money is in none of either accounts, creating an inconsistency - a potential source for errors because threads can be rescheduled at any time.

STM promises to solve all these problems for a low cost by executing actions \textit{atomically}, where modifications made in such an action are invisible to other threads and changes by other threads are invisible as well until actions are committed - STM actions are atomic and isolated. When an STM action exits, either one of two outcomes happen: if no other thread has modified the same data as the thread running the STM action, then the modifications performed by the action will be committed and become visible to the other threads. If other threads have modified the data then the modifications will be discarded, the action block rolled-back and automatically restarted.

\subsection{STM in Haskell}
The work of \cite{harris_composable_2005, harris_transactional_2006} added STM to Haskell, which was one of the first programming languages to incorporate STM into its main core and added the ability to composable operations. There exist various implementations of STM in other languages as well (Python, Java, C\#, C/C++, etc) but we argue, that it is in Haskell with its type-system and the way how side-effects are treated where it truly shines.

In the Haskell implementation, STM actions run within the \textit{STM} context. This restricts the operations to only STM primitives as shown below, which allows to enforce that STM actions are always repeatable without persistent side-effects because such persistent side-effects (e.g. writing to a file, launching a missile) are not possible in an \textit{STM} context. This is also the fundamental difference to  \textit{IO}, where all bets are off because \textit{everything} is possible as there are basically no restrictions because \textit{IO} can run everything.

Thus the ability to \textit{restart} a block of actions without any visible effects is only possible due to the nature of Haskells type-system: by restricting the effects to STM only, ensures that no uncontrolled effects, which cannot be rolled-back, occur.

STM comes with a number of primitives to share transactional data. Amongst others the most important ones are:

\begin{itemize}
	\item \textit{TVar} - A transactional variable which can be read and written arbitrarily;
	\item \textit{TArray} - A transactional array where each cell is an individual shared data, allowing much finer-grained transactions instead of e.g. having the whole array in a \textit{TVar};
	\item \textit{TChan} - A transactional channel, representing an unbounded FIFO channel;
	\item \textit{TMVar} - A transactional \textit{synchronising} variable which is either empty or full. To read from an empty or write to a full \textit{TMVar} will cause the current thread to block and retry its transaction when the variable has changed.
\end{itemize}

% NOTE: too technical
To run an \textit{STM} action the function \textit{atomically :: STM a $\to$ IO a} is provided, which can be seen as the STM effect-runner as it performs a series of \textit{STM} actions atomically within an \textit{IO} context. It takes the STM action which returns a value of type \textit{a} and returns an \textit{IO} action which returns a value of type \textit{a}. This \textit{IO} action can only be executed within an \textit{IO} context. TODO: this io context is provided through a thread. TODO: yes we are running in IO but not the agent behaviour itself, which is a fundamental difference.

STM in Haskell is implemented using optimistic synchronisation, which means that instead of locking access to shared data, each thread keeps a transaction log for each read and write to shared data it makes. When the transaction exits, the thread checks whether it has a consistent view to the shared data or not: whether other threads have written to memory it has read. % This might look like a serious overhead but the implementations are very mature by now, being very performant and the benefits outweigh its costs by far.

In the paper \cite{heindl_modeling_2009} the authors use a model of STM to simulate optimistic and pessimistic STM behaviour under various scenarios using the AnyLogic simulation package. They conclude that optimistic STM may lead to 25\% less retries of transactions. The authors of \cite{perfumo_limits_2008} analyse several Haskell STM programs with respect to their transactional behaviour. They identified the roll-back rate as one of the key metric which determines the scalability of an application. Although STM might promise better performance, they also warn of the overhead it introduces which could be quite substantial in particular for programs which do not perform much work inside transactions as their commit overhead appears to be high.