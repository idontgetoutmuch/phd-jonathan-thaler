\subsubsection{Sugarscape}
The same case as in the monadic SIR: running the agents with evaluation or data-flow parallelism is not possible in a monadic context  \textit{in general}. We have shortly discussed how it could be achieved in specific circumstances where then agents are running in the Par monad only, but this is highly model specific and for the Sugarscape this approach does not work. 

There is though one tiny thing we could optimise.

use parmap for updating Pollution/regrow resources. still agents can't be run in parallel because of monadic effects, we show in the concurrent section how we can use concurrency to achieve a substantial speed up using STM.

compare Environment parallelism between sequential and concurrent sugarscape: should see alarger speedup in conc bcs the sequential percentage is larger there

unfortunately its a Map datastructure, so we cannot operate in parallel e.g. map. but we can compute pollution because it uses map:
- (as a side note, folds can be parallelized only when the operation being folded is associative, and then the linear fold can be turned into a tree)
- IntMap traverseWithKey :: Applicative t => (Key -> a -> t b) -> IntMap a -> t (IntMap b):
	m <- Map.traverseWithKey (\i jmap -> spawn (return (shortmap i jmap))) g
	