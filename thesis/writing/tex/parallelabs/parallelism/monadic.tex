\subsection{Monadic SIR}
Unfortunately \textit{parMap} is not applicable to the monadic SIR version of Chapter \ref{sec:adding_env} because of the use of mapM, which cannot be replaced due to its inherent sequential nature: mapM runs monadic actions which have side-effects thus ordering matters. Even if the implementation in that chapter behaves as if the agents are run in parallel, technically they are run sequentially because of the need for the Random Monad effect. This leaves us basically without any options of parallelism for the monadic SIR model, we will come back to this use-case in the concurrency section, where we will show that by using concurrency it is possible to achieve a substantial speed up of orders of magnitude.

We can try to do the same as we did in the non-monadic SIR version but we can expect our approach to be doomed already. Although we collect a list of agent-outputs in the end, the way it is constructed is not lazy but sequential due to monadic execution. Thus we expect to see no speedup but a decreasing of performance due to overhead of parallelism. And indeed: the parallel version runs at the same speed as the non-parallel one. The reason is as follows: the list which is consumed by the mechanism for evaluation parallelism is indeed a lazy one but the elements are created sequentially one after another due to the monadic sequencing of mapM. 
Thus even though parallelism is happening it is of no use because the elements are already in whnf

TODO: the marlow book says: don't do repeated calls to runPar, so although the agent can in fact do that it should avoid it and if there is heavy parallel work in each agent then one should consider running the agent in the par monad with a single runPar outside



TODO: try the same thing as in the non-monadic SIR: parMap rpar id evaluating the output. Hypothesis is that it should not show any parallelism because of monadic code. Indeed it does not show any speedup
