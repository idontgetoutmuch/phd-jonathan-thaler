\section{Parallelism in ABS}
The promise of parallelism in Haskell is compelling: speeding up the execution but retaining all static compile-time guarantees about determinism. In other words, using parallelism could give us a substantial performance improvement without sacrificing the static guarantees of reproducible outputs from repeated runs with initial conditions.

Generally, parallelism can be applied whenever the execution of code is order-independent, that is referential transparent, and has no implicit or explicit side-effects. Without going into too much technical detail, in this section we outline the parallelism techniques available in Haskell and briefly discuss how they can be used in ABS in general. We also discuss if and how parallelism can be added to our previously discussed use-cases of Chapters \ref{sec:timedriven_firststep}, \ref{sec:adding_env} and Sugarscape TODO and report the performance improvements where applicable.

\subsection{Parallelism in Haskell}
parallelism in haskell builds on laziness

We follow the book \cite{marlow_parallel_2013}, which can be seen as the main source for parallelism and concurrency in Haskell and refer to it for an in-depth discussions of parallel Haskell.

\paragraph{Evaluation parallelism}
The basis are the following functions: \textit{rpar :: a -> Eval a} and \textit{rseq :: a -> Eval a}, where Eval is a Monad which can be run with \textit{runEval :: Eval a -> a}. rpar runs the computation in parallel and immediately returns without waiting for the evaluation of the thunk - this will happen behind the sences. rseq runs the computation in parallel as well but waits for the evaluation to WFNF. Using this we can start evaluating multiple expressions in parallel with rpar and then wait for their result with rseq. Note that both cases evaluate their argument to weak head normal form (WHNF), thus if the argument is already in WHNF, then the computation does nothing. This becomes important when to understand how far we can to in evaluation of parallelism. TODO: need to say a bit about haskell as a lazy language. TODO: isnt this all the very basics of haskells parallelism together with lazy evaluation?

Put short, evaluation parallelism allows to build functions which run in parallel e.g. a parallel version of \textit{map}, which is called \textit{parMap}. This is achieved using the \textit{Eval Monad} which is run using a parallel evaluation strategy, arriving at a pure value - the evaluation of the \textit{Eval Monad} itself is pure and does not require the IO (this is exactly what we expect from parallelism: to be deterministic). Obviously this gives huge potential for speeding up programs because maps are omnipresent in a lot of functional code. Not only parmap! explain a little bit more in detail without going into too much technical stuff.


Very important: "In the previous two chapters, we looked at the Eval monad and Strategies, which work in conjunction with lazy evaluation to express parallelism. A Strategy consumes a lazy data structure and evaluates parts of it in parallel. This model has some advantages: it allows the decoupling of the algorithm from the parallelism, and it allows parallel evaluation strategies to be built compositionally. But Strategies and Eval are not always the most convenient or effective way to express parallelism. We might not want to build a lazy data structure, for example. Lazy evaluation brings the nice modularity properties that we get with Strategies, but on the flip side, lazy evaluation can make it tricky to understand and diagnose performance."
Its laziness which allows that.


%https://www.oreilly.com/library/view/parallel-and-concurrent/9781449335939/ch02.html
%https://www.oreilly.com/library/view/parallel-and-concurrent/9781449335939/ch03.html

\paragraph{Data-flow parallelism}
Par Monad: how does it work? can express data-flow networks where tasks are forked and then results are synchronised. all this happens deterministically by building on the same mechanics the Eval monad is using thus technically speaking they are equivalent. 

can use both par and eval monad but which is applicable? par seems to require strict data, eval works on lazy data-structure. can we use both inside an msf? what is the Advantage over simple rpar or rseq?\\

%https://www.oreilly.com/library/view/parallel-and-concurrent/9781449335939/ch04.html

\paragraph{Data-structure parallelism}
An environment could be organised and accessed through such a data-structure, which could potentially lead to big speed ups. Agents could locally read the data-structure data-parallel and the simulation kernel could feed the output of the agents data-parallel back into this structure.

%https://learning.oreilly.com/library/view/parallel-and-concurrent/9781449335939/ch05.html

general solution we opt for is  to run agents in parallel in our approaches. in other abs models we could apply data-structure parallelism and/or data-flow parallelism with huge Performance potential but thats always highly model dependent thus we dont go in depth here

\subsection{Use-Cases}

\subsubsection{Non-Monadic SIR}
Although \textit{parMap} can be applied in all cases where a map us used, we are particularly interested in running agents in parallel. With \textit{parMap} this should become possible in our non-monadic SIR implementation built on Yampa from Chapter \ref{sec:timedriven_firststep}. Even thought the Eval Monad is used under the hood and Yampa is non-monadic, it is still applicable because running the monad is pure, resulting in a pure result - \textit{parMap} is a pure function. TODO: how can we apply this?

Inspired by the work of \cite{perez_60_2014}, which shows the potential of speeding up real-world Haskell programs using Yampa We conducted a comparison of an implementation which makes use of evaluation parallelism to run agents in parallel.

OK, rephrase: compare performance of non-parallel implementation WITH threaded an -N option to non-parallel implementation without threaded and / or N1 to make sure that no performance improvement happens automatically by using threaded e.g. GCs or something else...
I observed the behaviour in the following code: https://github.com/thalerjonathan/phd/tree/master/public/purefunctionalepidemics/code/SIR_Yampa

I analysed a bit more using the threadscope tool. I ran the same program twice with different ghc-options:
1. -O2 -Wall -Werror -eventlog 
2. -O2 -Wall -Werror -eventlog -threaded -with-rtsopts=-N

When looking at the event logs with threadscope it becomes appartent, that parallel garbage collection is the cause of the CPU usage above 100%:
-  In the single-threaded case 0 sparks are created and everything runs indeed only on one core. There are two Garbage Collectors (Gen0 and Gen1) but nothing runs in parallel (Par collections are 0 for both).
- In the multi-threaded case also 0 sparks are created but now 8 cores are used: all 'running' activity happens on only 1 core as expected but garbage collection happens on all 8 cores: the diagrams and the number of Par collections clearly indicates that. The time spent on parallel GC work is 10.76% (0 is completely serial and 100% is completely parallel).

Now when we compare the timing between both runs we see the following: 
- single-threaded: 11.68s total, 7.35s mutator, 4.34s GC,
- multi-threaded: 10.70s total, 7.03s mutator, 3.68s GC

This adds up: the ~ 10\% of parallel GC work done in multi-threaded are also the ~ 10\% it is faster over the single-threaded one. Of course I only did a single run in each case but I think the analysis is still valid and the point was made: when running a Haskell program which does not use any parallel features, running it with the -threaded option can lead to an increase in performance due to parallel GC.

% https://www.reddit.com/r/haskell/comments/2jbl78/from_60_frames_per_second_to_500_in_haskell/\\

TODO: should be applicable to the par monad as well? 

\subsubsection{Monadic SIR}
Unfortunately \textit{parMap} is not applicable to the monadic SIR version of Chapter \ref{sec:adding_env} because of the use of mapM, which cannot be replaced due to its inherent sequential nature: mapM runs monadic actions which have side-effects thus ordering matters. Even if the implementation in that chapter behaves as if the agents are run in parallel, technically they are run sequentially because of the need for the Random Monad effect. This leaves us basically without any options of parallelism for the monadic SIR model, we will come back to this use-case in the concurrency section, where we will show that by using concurrency it is possible to achieve a substantial speed up of orders of magnitude.

TODO: can't we run with the Par monad? in the end its just a data-flow graph! see the update-strategies. Should be possible, only problem is how to deal with rand then? we use split. the marlow book says: don't do repeated calls to runPar. we can because we implemented the agent-running loop  by our selves. so we can use Eval and Par

TODO: following idea: instead of updating the environment, we simply build a new one, no need to update because we update all of them anyway, so very expensive. Also simply split RNGs for every agent, then we can run them in parallel 


NOTE: running the agents in parallel with par doesn't work because we use mapM and are thus monadic, which involves sequencing. so this is really out of the window here. Also we cannot put a Par in a transformer stack because the library doesn't support it, what actually makes sense. But we can do the following: we can run an agents MSF only within the Par monad which gives agents the ability to spawn data-flow parallel computations - random-number streams are handled like in the non-monadic version. Note that this is only possible with the MSFs of dunai and not the SF because the latter one adds already the a ReaderT DTime which makes it impossible already. 
What is actually possible would be to write a combined monad for Par and ReaderT because the latter one is a read-only value and could thus potentially run in parallel - we leave this for further research. There exists also a combination of the Par with the Rand monad, so if the time-driven approach is not needed then this could be used to give the agents the ability to both draw random numbers AND do deterministic data-parallel computations. The agents can then be run in parallel through the par monad.


\subsubsection{Sugarscape}
The same case as in the monadic SIR: running the agents with evaluation or data-flow parallelism is not possible in a monadic context  \textit{in general}. We have shortly discussed how it could be achieved in specific circumstances where then agents are running in the Par monad only, but this is highly model specific and for the Sugarscape this approach does not work. 

There is though one tiny thing we could optimise.

use parmap for updating Pollution/regrow resources. still agents can't be run in parallel because of monadic effects, we show in the concurrent section how we can use concurrency to achieve a substantial speed up using STM.

compare Environment parallelism between sequential and concurrent sugarscape: should see alarger speedup in conc bcs the sequential percentage is larger there

unfortunately its a Map datastructure, so we cannot operate in parallel e.g. map. but we can compute pollution because it uses map

\section{Parallel Runs}
Often one needs to perform a large number of runs of the same simulation. The most prominent use-cases for this are:

\begin{itemize}
	\item Parameter Sweeps / Variations - to explore the parameter space and the dynamics under varying parameter configurations, the same simulation is run with varying parameters and the results recorded for statistical analysis.
	
	\item Stochastic replications - due to ABS stochastic nature, running a simulation only once does not allow to generalise or predict overall behaviour - one might have just hit an (un)fortunate special case. To counter this problem, in ABS multiple replications of the  simulation are run with same initial model parameters but with different random-number streams. All the results are collected and analysed stochastically (averaged, median,...) from which then more general properties can be derived.
\end{itemize}

In each case thousands of runs of the same simulation with different model parameters and / or varying random-number streams are needed, requiring a considerable amount of computing power.

Parallelism is a remedy to this problem because in each of these cases individual runs do not interfere with each other and thus can be seen as isolated from each other, like referential transparent, pure computations. Our approaches shown in Part II make this very explicit: the top level functions can always be made pure computations because we are ruling out \textit{IO} and thus even though Monads are employed in many cases, they are still pure. A benefit of our approach is that it is guaranteed at compile time, that individual runs do not interfere with each other and thus there is no danger that parallel runs influence each other. 

All this allows to implement parameter sweeps and stochastic replications both through evaluation and data-flow parallelism making another very compelling use-case - probably the most striking one - for the use of parallelism in ABS. We hypothesize that data-flow parallelism is better suited for this task because it makes parallelism more explicit as it is indeed a data-flow problem: we pass parameters to single replications which are run and return their results. To apply this we simply run the top level replication logic in the \textit{Par} Monad where replications are run in parallel by forking tasks and results are handed back through \textit{IVars}. If we want the convenience of having a monadic random-number generator within the \textit{Par} Monad, one can use the combined \textit{ParRand} Monad which provides both.

\subsection{Reflection}
Despite high hopes, there were very few opportunities to apply parallelism to our pure functional ABS. This has three reasons: it is often highly model specific and our models simply didn't offer a lot of suitable parallelisations, the data-structures have to support parallelism e.g. map doesn't, but we also have to say that the sequential nature of ABS in general seems to be less suited to parallelism. We will see that concurrency offers a remedy against that.

the difficulties / low ausbeute von parallelism just Shows how difficult it is to parallelise abs. also maybe our approach is not very well.suited e.g. not very functional?

In general we aimed at running agents in parallel using the various techniques. Because of the quite sequential nature of the agent behaviours themselves, there is much less potential for parallelism \textit{within} an agent, thus the obvious idea was to run them all in parallel because they are an obvious unit of partitioning, have considerable workload and can indeed be run in parallel under given circumstances.
Unfortunately it is not possible applying parallelism in case the agents run within a monadic context: we have side-effects which imposes ordering e.g. in the case of a

It becomes apparent, that applying parallelism to our approaches doesn't lead to very much performance increase. This is because in the cases were we can actually run the agents with evaluation parallelism, the performance is not bound by them. As soon as we switch to monadic agents, evaluation parallelism is out of the window, as agents can't be run in parallel anymore because side-effects require to impose a sequential ordering. This can be only tackled using non-deterministic concurrency, which we will show in-depth in the next chapter because it is much more promising than prallelism in terms of performance gain. Further it is also more technically involved and the way we chose to approach it using Software Transactional Memory (STM) hasn't been undertaken in this form ever and to our best knowledge we are the first one to do so.

We see a direct consequence of this that types also reflect the semantics of our model: when our agents are pure they can be run indeed in parallel and independent from each other, if they are monadic, then this is not applicable to parallelism. In the next section, we show how to approach this problem and come up with a solution where we can run monadic agents in parallel. This is obviously only possible within a concurrent setting which means we have to sacrifice determinism in our solution. Still we reach considerable speed ups using Software Transactional Memory.

We didn't discuss data parallelism on large array structure or parallelism on GPU as they are used in massively large numerical computation. These techniques achieve tremendous speed ups but are not applicable to ABS in general but only in model specific cases where e.g. each agent needs to crunch through arrays of numbers to perform numerical computations. We refer to \cite{marlow_parallel_2013} for a more in-depth discussion of both in Haskell and leave the application to pure functional ABS for further research.