\section{Parallelism in ABS}
The promise of parallelism in Haskell is compelling: speeding up the execution but retaining all static compile-time guarantees about determinism. In other words, using parallelism could give us a substantial performance improvement without sacrificing the static guarantees of reproducible outputs from repeated runs with initial conditions.

Generally, parallelism can be applied whenever the execution of code is order-independent, that is referential transparent, and has no implicit or explicit side-effects. Without going into too much technical detail, in this section we outline the parallelism techniques available in Haskell and briefly discuss how they can be used in our approaches or to ABS in general. We follow the book \cite{marlow_parallel_2013}, which can be seen as the main source for parallelism and concurrency in Haskell and refer to it for an in-depth discussions of parallel Haskell.

\subsection{Evaluation parallelism}
Put short, evaluation parallelism allows to build functions which run in parallel e.g. a parallel version of \textit{map}, which is called \textit{parMap}. This is achieved using the \textit{Eval Monad} which is run using a parallel evaluation strategy, arriving at a pure value - the evaluation of the \textit{Eval Monad} itself is pure and does not require the IO (this is exactly what we expect from parallelism: to be deterministic). Obviously this gives huge potential for speeding up programs because maps are omnipresent in a lot of functional code.

Although \textit{parMap} can be applied in all cases where a map us used, we are particularly interested in running agents in parallel. With \textit{parMap} this should become possible in our non-monadic SIR implementation built on Yampa from Chapter \ref{sec:timedriven_firststep}. Even thought the Eval Monad is used under the hood and Yampa is non-monadic, it is still applicable because running the monad is pure, resulting in a pure result - \textit{parMap} is a pure function. TODO: how can we apply this?

Inspired by the work of \cite{perez_60_2014}, which shows the potential of speeding up real-world Haskell programs using Yampa We conducted a comparison of an implementation which uses map vs one that uses parMap.


% http://keera.co.uk/blog/2014/10/15/from-60-fps-to-500/
% https://www.reddit.com/r/haskell/comments/2jbl78/from_60_frames_per_second_to_500_in_haskell/\\

Unfortunately \textit{parMap} is not applicable to the monadic SIR version of Chapter \ref{sec:adding_env} because of the use of mapM, which cannot be replaced due to its inherent sequential nature: mapM runs monadic actions which have side-effects thus ordering matters. Even if the implementation in that chapter behaves as if the agents are run in parallel, technically they are run sequentially because of the need for the Random Monad effect.

\subsection{Data-flow parallelism}
Par Monad: what for? not possible in yampa, but in dunai?
	
\subsection{Data-structure parallelism}
An environment could be organised and accessed through such a data-structure, which could potentially lead to big speed ups. Agents could locally read the data-structure data-parallel and the simulation kernel could feed the output of the agents data-parallel back into this structure.

\section{Parallel Runs}
Often one needs to perform a large number of runs of the same simulation. The most prominent use-cases for this are:

\begin{itemize}
	\item Parameter Sweeps / Variations - to explore the parameter space and the dynamics under varying parameter configurations, the same simulation is run with varying parameters and the results recorded for statistical analysis.
	
	\item Stochastic replications - due to ABS stochastic nature, running a simulation only once does not allow to generalise or predict overall behaviour - one might have just hit an (un)fortunate special case. To counter this problem, in ABS multiple replications of the  simulation are run with same initial model parameters but with different random-number streams. All the results are collected and analysed stochastically (averaged, median,...) from which then more general properties can be derived.
\end{itemize}

In each case thousands of runs of the same simulation with different model parameters and / or varying random-number streams are needed, requiring a considerable amount of computing power.

Parallelism is a remedy to this problem because in each of these cases individual runs do not interfere with each other and thus can be seen as isolated from each other, like referential transparent, pure computations. Our approaches shown in Part II make this very explicit: the top level functions can always be made pure computations because we are ruling out \textit{IO} and thus even though Monads are employed in many cases, they are still pure. A benefit of our approach is that it is guaranteed at compile time, that individual runs do not interfere with each other and thus there is no danger that parallel runs influence each other. 

All this allows to implement parameter sweeps and stochastic replications both through evaluation and data-flow parallelism making another very compelling use-case - probably the most striking one - for the use of parallelism in ABS. We hypothesize that data-flow parallelism is better suited for this task because it makes parallelism more explicit as it is indeed a data-flow problem: we pass parameters to single replications which are run and return their results. To apply this we simply run the top level replication logic in the \textit{Par} Monad where replications are run in parallel by forking tasks and results are handed back through \textit{IVars}. If we want the convenience of having a monadic random-number generator within the \textit{Par} Monad, one can use the combined \textit{ParRand} Monad which provides both.

\subsection{Reflection}
Despite high hopes, there were very few opportunities to apply parallelism to our pure functional ABS. This has three reasons: it is often highly model specific and our models simply didn't offer a lot of suitable parallelisations, the data-structures have to support parallelism e.g. map doesn't, but we also have to say that the sequential nature of ABS in general seems to be less suited to parallelism. We will see that concurrency offers a remedy against that.

the difficulties / low ausbeute von parallelism just Shows how difficult it is to parallelise abs. also maybe our approach is not very well.suited e.g. not very functional?

In general we aimed at running agents in parallel using the various techniques. Because of the quite sequential nature of the agent behaviours themselves, there is much less potential for parallelism \textit{within} an agent, thus the obvious idea was to run them all in parallel because they are an obvious unit of partitioning, have considerable workload and can indeed be run in parallel under given circumstances.
Unfortunately it is not possible applying parallelism in case the agents run within a monadic context: we have side-effects which imposes ordering e.g. in the case of a

It becomes apparent, that applying parallelism to our approaches doesn't lead to very much performance increase. This is because in the cases were we can actually run the agents with evaluation parallelism, the performance is not bound by them. As soon as we switch to monadic agents, evaluation parallelism is out of the window, as agents can't be run in parallel anymore because side-effects require to impose a sequential ordering. This can be only tackled using non-deterministic concurrency, which we will show in-depth in the next chapter because it is much more promising than prallelism in terms of performance gain. Further it is also more technically involved and the way we chose to approach it using Software Transactional Memory (STM) hasn't been undertaken in this form ever and to our best knowledge we are the first one to do so.

We see a direct consequence of this that types also reflect the semantics of our model: when our agents are pure they can be run indeed in parallel and independent from each other, if they are monadic, then this is not applicable to parallelism. In the next section, we show how to approach this problem and come up with a solution where we can run monadic agents in parallel. This is obviously only possible within a concurrent setting which means we have to sacrifice determinism in our solution. Still we reach considerable speed ups using Software Transactional Memory.

We didn't discuss data parallelism on large array structure or parallelism on GPU as they are used in massively large numerical computation. These techniques achieve tremendous speed ups but are not applicable to ABS in general but only in model specific cases where e.g. each agent needs to crunch through arrays of numbers to perform numerical computations. We refer to \cite{marlow_parallel_2013} for a more in-depth discussion of both in Haskell and leave the application to pure functional ABS for further research.