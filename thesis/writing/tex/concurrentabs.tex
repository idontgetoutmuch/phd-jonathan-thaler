\chapter{Concurrent ABS}
\label{ch:concurrent_abs}
Functional programming as in Haskell is well known and accepted as a remedy against the problems of imperative programming in implementing parallel software TODO: cite ?. The reason for it is clear: immutable data and explicit control of side-effects removes a large class of bugs due to data-conflicts, data-races, and blablabla TODO: we are claiming things here, which we need to clearly back up, also data-races ARE possible in Haskell! A fundamental benefit and strength of Haskell is, that it clearly distinguishes between parallelism and concurrency \cite{jones_tackling_2002} and it is very important for us to do so as well:

\begin{itemize}
	\item \textbf{Parallelism} - In parallelism, code runs in parallel without interfering with other code through shared data (references, mutexes, semaphores,...). An example is the function \textit{map :: (a $\rightarrow$ b) $\rightarrow$ [a] $\rightarrow$ [b]}, which maps each element of type \textit{a} to \textit{b} using the function \textit{(a $\rightarrow$ b)}. It is a pure function and thus no sharing of data either through some monadic context or through the function \textit{(a $\rightarrow$ b)} is possible. This allows to run it in parallel: each function evaluation \textit{(a $\rightarrow$ b)} could potentially be executed at the same time, if we had enough CPU cures. Whether it runs actually in parallel or not, has no influence on the outcome, it is not subject to any non-deterministic influences. Thus we identify parallelism with pure and deterministic execution of data-transformations (data-parallelism).
	
	\item \textbf{Concurrency} - In concurrency, code runs in parallel but can potentially interfere with other code through shared data (references, mutexes, semaphores, ...). An example are two threads, running in parallel, which share data through \textit{IORefs}. In concurrency there is no option: code has to run in parallel through the use of threads but now the outcome of the program very much depends on the ordering in which the threads are scheduled. This gives rise to very different access patterns to the shared data, with the potential for race conditions, dirty reads and so on... The challenge of implementing concurrent programs, is to write the program in a way that despite of these non-deterministic influences it is still a correctly working program. Thus we identify concurrency with impure and non-deterministic execution of imperative-style monadic command execution.
\end{itemize}

There is obvious potential for adding (data-)parallelism to ABS e.g. using data-parallel data-structures for the environment so cells can be updated in parallel, in time-driven ABS agents can be updated in parallel using parMap because they all act conceptually at the same time as shown already in Yampa \footnote{\url{https://www.reddit.com/r/haskell/comments/2jbl78/from_60_frames_per_second_to_500_in_haskell/}}.

Despite the potential for parallelism, in this chapter we focus on concurrency only and refer to the book \cite{marlow_parallel_2013} for an in-depth discussions of the mechanism for parallelism in Haskell. The reason for focusing on concurrency and leaving parallelism out is simple: the Ph.D. doesn't provide enough time to explore both in equal depth and the application of STM to implement concurrent ABS looks very much more interesting and challenging probably because it is also a complete novelty.

\section{Concurrency}
This section is a quite extensive one, due to the much complexer nature of the topic. In this chapter we focus on the use of Software Transactional Memory (STM) to implement concurrent ABS in Haskell. There exist mechanisms for traditional lock-based concurrency in Haskell as well, but even though due to Haskells type system and pure functional nature, we would still suffer similar problems, imperative solutions suffer. Thus we focus on the unique way Haskell implements and enables STM and show that it is indeed a very powerful abstraction to implement concurrent ABS. Note that STM is by now available in imperative oop languages as well, but we will show that it is only in Haskell with its type system where it really shines and can guarantee its semantics.


\section{Discussion}
Although there are similarities to the work of \cite{botta_time_2010} (the use of messages and the problem of when to advance time in models with arbitrary number synchronised agent-interactions), we approach our agents differently. First in our approach an agent is only a single MSF and thus can not be directly queried for its internal state / its id or outgoing messages, instead of taking a list of messages, our agents take a single event/message and can produce an arbitrary number of outgoing messages together with an observable state - note that this would allow to query the agent for its id and its state as well by simply sending a corresponding message to the agents MSF and requiring the agent to implement message handling for it. Also the state of our agents is \textit{completely} localised and there is no means of accessing the state from outside the agent, they are thus "fully encapsulated agents" \cite{botta_time_2010}. Note that the authors of \cite{botta_time_2010} define their agents with a polymorphic agent-state type \textit{s}, which implies that without knowledge of the specific type of \textit{s} there would be no way of accessing the state, rendering it in fact also fully encapsulated. The problem of advancing time in our approach is solved not exactly the same but conceptually it is the same: after sending a tick message to each agent (in random order), we process all agents until they are idle: there are no more enqueued messages / events in the queue.

our eventdriven approach makes heavy use of 2 state monads, thus one might ask what the benefits are, after all we seem to fall back into stateful, imperative style programming. we agree that our approach is just one way of implementing abs in fp but we think we have come a long way thus making our approach quite valuable even if there might be other approaches like shallow EDSLs. on the other hand even our stateful programming is highly restricted to only those 2 local datatypes which makes it much more manageable than unrestricted data mutation

quote carmack (\url{http://www.gamasutra.com/view/news/169296/Indepth_Functional_programming_in_C.php}): the main difficulty as a developer in software programming is to keep track of the states a program can be in and reason about them and their Validity

TODO: report LoC and compare it with other implementations we found on the internet