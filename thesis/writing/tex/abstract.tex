\section*{Abstract}
%You need to provide a little bit of motivation/background in your abstract (to start with). Why is this a relevant topic; what are current issues/shortcomings?
% setting the scene
Implementations of Agent-Based Simulations (ABS) primarily use object- \\ oriented programming techniques, as in Python, Java and C++, due to the established opinion that \textit{agents map naturally to objects}. These techniques have seen tremendous success throughout the last two decades in general and were able to provide the ABS community with useful simulation tools in particular.

% problem 1: oop has uncontrolled side effects, which comes with a ton of protential problems
However, the verification process of ensuring the correctness of the implementation up to some specification, has never been an easy task with the established object-oriented techniques, due to their inherent use unrestricted side effects. This is amplified by the inherent difficulties of using automated unit tests for testing simulations in general. Unit tests are conceptually operational in nature, focusing on how to test something, whereas simulations should conceptually be tested rather declarative by describing what to expect, for example in terms of individual agent behaviour or emergent dynamics.

% problem 2: parallelisation (and concurrency) is difficult due to unrestricted side effects
Further, with the shift towards multicore CPUs in recent years, it became clear that the unrestricted side effects of the established object-oriented techniques pose serious difficulties in arriving at a correct parallel or concurrent solution, and more robust techniques to overcome the difficulties of parallel and concurrent programming are needed to fully exploit the full potential of this new technology. 

% why is this a problem: because ABS are scientific computations which need to be correct as they support scientific theories and policy decisions influencing peoples lives
As ABS is almost always used in the context of scientific computation, to test hypotheses, explore dynamics, support scientific theories and also to make informed decisions about policies, potentially influencing many peoples lives, these difficulties are a serious issue because results of ABS thus need to be reproducible given same initial starting conditions and comprehensible by other researchers. The special issue with ABS is that the emergent behaviour of the system is generally not known in advance and researchers look for some \textit{unique} emergent pattern in the dynamics. Whether the emergent pattern is then truly due to the systems correct working or a bug in disguise is often not obvious and becomes increasingly difficult to assess with increasing system complexity. 

\medskip

% what we do in this thesis: a potential solution. also define purity
This thesis proposes pure functional programming with the language Haskell as a potential solution to overcome these difficulties. The fundamental focus is on \textit{purity}, which identifies the \textit{lack of implicit side effects} and \textit{referential transparency}: data is immutable, there are no uncontrolled side effects and a computation does not depend on its context within the system but will produce the same result when run repeatedly with similar inputs. Thus, the central theme of this thesis is exploring how purity and the resulting concepts help in overcoming the issues of the established object-oriented techniques in ensuring the confidence in the correctness of an implementation, by answering the following questions:

%From the abstract it seems like the body of your thesis should be subdivided into two rather than three main sections. If you want to keep three sections, consider the following:
%
%II How can we do it? (approach)
%    Implementation techniques
%III Why should we do it? (benefits)
%   Parallel computing
%   Property based testing
%IV How certain can we be? (confidence)
%
%If you want to keep three sections, you should come up with a third question. These could then be the three questions you are aiming to answer with your PhD studies
%You could list them all together initially in your abstract, before referring to them individually one after one. 

\begin{enumerate}
	\item How can ABS be implemented pure functionally and what are the benefits and drawbacks in doing so?
	\item How can pure functional programming be used for robust parallel and concurrent programming? 
	\item How can pure functional programming be used for testing ABS implementations?
\end{enumerate}

% Then you say what you did.
Thematically, the research of the thesis is split into two parts with the first one dealing with the the \textit{approach} to a pure functional ABS, and the second part with exploring additional \textit{benefits} enabled through pure functional programming. Both parts are permeated by the fundamental theme of purity, which inherently relates to the very problem of ensuring the \textit{confidence} in the results of an implementation. 

First, the thesis explores \textit{how} to implement ABS pure functionally, discussing both a time- and event-driven approach. In each case arrowized Functional Reactive Programming is used to derive fundamental abstractions and concepts. As use cases the well known explanatory agent-based SIR and the exploratory Sugarscape model are used. Then the thesis focuses on \textit{why} it is of benefit to implement ABS pure functionally. It explores robust parallel and concurrent programming where the main focus is on how to speedup the simulation but keeping it still pure, whereas in the concurrency part, Software Transactional Memory is used, sacrificing purity but still retaining certain guarantees about reproducibility. Finally, the thesis explores testing of ABS implementations using property-based testing to show how to encoding agent specifications, model invariants and perform verification of the explanatory model and hypothesis testing of the exploratory model.

%Then you clearly state your contribution.

The contribution of this thesis is threefold:
\begin{enumerate}
	\item Researching new pure functional implementation techniques for ABS \\ through the use of arrowized Functional Reactive Programming.
	\item Researching the use of Software Transactional Memory for a new approach to implement concurrent ABS in a robust way.
	\item Researching the use of property-based testing for a new approach to ABS testing.
\end{enumerate}

The derived concepts in this thesis show at length that \textit{pure} functional programming has indeed its place in ABS as it comes with unique benefits over the established object-oriented approach. A \textit{pure} functional approach in general should lead to implementations which are more likely to be correct due to the use of the strong static type system and focus on purity by avoiding computations with unrestricted side effects under all costs. Further, parallel computation makes it possible to gain substantial speedup by retaining static guarantees. The use of Software Transactional Memory speeds up concurrent implementations by a substantial factor, dramatically outperforming lock-based approaches while still retaining certain guarantees about reproducibility. Finally, the use of property-based testing is shown to be highly effective as it maps naturally to the stochastic nature of ABS and is thus possible to be integrated into the development process as additional tool for testing specifications and hypotheses.

%One final paragraph, making a statement of the usefulness of the research conducted.
%"Overall we have fulfilled the aim, and supported both communities by giving them new technologies (ABM) and application areas (FP) ... We believe that this research will lead to ..."

Overall we have fulfilled the aim, and supported both the functional programming and ABS communities by giving them new technologies and application areas. We believe that this research will lead to an increased interest of purity in ABS, the use of Software Transactional Memory to parallelise ABS implementations and property-based testing to test ABS, to ultimately increase the confidence in the results.