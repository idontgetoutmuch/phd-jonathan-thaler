\section*{Abstract}
%You need to provide a little bit of motivation/background in your abstract (to start with). Why is this a relevant topic; what are current issues/shortcomings?
%Then you say what you did.
%Then you clearly state your contribution.

%From the abstract it seems like the body of your thesis should be subdivided into two rather than three main sections. If you want to keep three sections, consider the following:
%
%II How can we do it? (approach)
%    Implementation techniques
%III Why should we do it? (benefits)
%   Parallel computing
%   Property based testing
%IV How certain can we be? (confidence)
%
%If you want to keep three sections, you should come up with a third question. These could then be the three questions you are aiming to answer with your PhD studies
%
%You could list them all together initially in your abstract, before referring to them individually one after one. 

This thesis systematically investigates the use of the \textit{pure} functional programming paradigm for implementing Agent-Based Simulations (ABS) and the benefits and drawbacks when doing so. As language of choice, Haskell is used due to its modern, \textit{pure} nature and increasing use in real-world applications. First, the thesis explores \textit{how} to implement ABS pure functionally, discussing both a time- and event-driven approach. In each case arrowized Functional Reactive Programming plays a fundamental role to derive fundamental abstractions and concepts. As use cases the well known explanatory agent-based SIR and the exploratory Sugarscape model are used. Then the thesis explores \textit{why} it is of benefit to implement ABS pure functionally, where it focuses on parallelism and concurrency and property-based testing. In the parallelism part, the main focus is on how to speedup the simulation but keeping it still pure, whereas in the concurrency part, Software Transactional Memory is used, sacrificing purity. Property-based testing is used to show how to encode full agent specifications, model invariants and do verification of the explanatory model directly in code. Further, hypothesis testing for the exploratory Sugarscape model is shown.

The derived concepts in this thesis show at length that \textit{pure} functional programming has indeed its place in ABS as it comes with unique benefits over the established object-oriented approach. A \textit{pure} functional approach in general should lead to implementations which are more likely to be correct due to the use of the strong static type system and focus on purity by avoiding \textit{IO} computations under all costs. Further, parallel computation makes it possible to gain substantial speedup by retaining static guarantees. The use of Software Transactional Memory speeds up concurrent implementations by a substantial factor, dramatically outperforms lock-based approaches while still retaining certain guarantees about reproducibility. Finally, the use of randomised property-based testing is shown to be highly effective as it maps naturally to the stochastic nature of ABS and is thus possible to be integrated into the development process as additional tool for testing specifications and hypotheses.

%One final paragraph, making a statement of the usefulness of the research conducted.
%"Overall we have fulfilled the aim, and supported both communities by giving them new technologies (ABM) and application areas (FP) ... We believe that this research will lead to ..."