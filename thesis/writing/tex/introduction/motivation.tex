\chapter{Motivation}
\label{ch:motivation}
% MOTIVATION 
The traditional approach to Agent-Based Simulation (ABS) has so far always been object-oriented, due to the influence of the seminal Sugarscape model \cite{epstein_growing_1996}, in which the authors claim \textit{"[..] object-oriented programming to be a particularly natural development environment for Sugarscape specifically and artificial societies generally [..]"} (p. 179). This work established the metaphor in the ABS community, that \textit{agents map naturally to objects} \cite{north_managing_2007}, which still holds up today.

% THE PROBLEM, THE GINTIS CASE
Despite the broad acceptance and adoption of object-oriented techniques however, there seem to be struggles with established ABS approaches, as described in \cite{axelrod_chapter_2006}, where the author reports the vulnerability of ABS to misunderstanding. Due to informal specifications of models and change requests amongst members of a research team, bugs are very likely to be introduced. He also reported how difficult it was to reproduce the work of \cite{axelrod_convergence_1995}, which took the team four months, due to inconsistencies between the original code and the published paper. The consequence is that counter-intuitive simulation results can lead to weeks of checking whether the code matches the model and is bug-free as reported in \cite{axelrod_advancing_1997}.

The same problem was reported in \cite{ionescu_dependently-typed_2012}, which tried to reproduce the work of Gintis \cite{gintis_emergence_2006}. In his work, Gintis claimed to have found a mechanism in bilateral decentralized exchange, which resulted in Walrasian General Equilibrium without the neo-classical approach of a tatonement process through a central auctioneer. This was a major breakthrough for economics as the theory of Walrasian General Equilibrium is non-constructive. It only postulates the properties and existence of the equilibrium \cite{colell_microeconomic_1995} but does not explain the process and dynamics through which this equilibrium can be reached or constructed - Gintis seemed to have found this very process.

The authors \cite{ionescu_dependently-typed_2012} failed to reproduce the results and were only able to solve the problem by directly contacting Gintis, which provided the code, the definitive formal reference \footnote{It seems that at this point Gintis has made his code written in Object Pascal publicly available through his \href{https://people.umass.edu/gintis/}{website}~\cite{gintis_herbert_website}.}. It was found that there was a bug in the code leading to unexpected results, which were seriously damaged through this error. They also reported ambiguity between the informal model description in Gintis' paper and the actual implementation.
This discovery lead to research in a functional framework for agent-based models of exchange as described in \cite{botta_functional_2011}, which tried to give a very formal functional specification of the model, coming very close to an implementation in Haskell. The failure of Gintis was investigated in more depth in the thesis by \cite{evensen_extensible_2010} who got access to Gintis' code of \cite{gintis_emergence_2006}. They found that the code did not follow good object-oriented design principles (all of it was public, code duplication) and - in accordance with \cite{ionescu_dependently-typed_2012} - discovered a number of bugs serious enough to invalidate the results.
%%%%%%%%%%%%%%%%%%%%%%%%%%%%%%%%%%%%%%%%%%%%%%%%%%%%%%%%%%%%%%%%%%%%%%%%%%%%%%%%%%%%%%%%%%%%%%%%%%%%%%%%%%%%%%%%%%%%%%%%%%%%%

\medskip

% AIM
However, due to the fact that ABS is primarily used for scientific research, producing often break-through scientific results, besides on converging both on standards for testing the robustness of implementations and on its tools, ABS more importantly needs to be \textit{free of bugs}, \textit{verified against their specification}, \textit{validated against hypotheses} and ultimately be \textit{reproducible} \cite{axelrod_chapter_2006}. Further, a special issue with ABS is that the emergent behaviour of the system is generally not known in advance and researchers look for some \textit{unique} emergent pattern in the dynamics. Whether the emergent pattern is then truly due to the system working correctly, or a bug in disguise is often not obvious and becomes increasingly difficult to assess with increasing system complexity. 

Based on the reports of failure and the special requirements of validity on ABS implementations, this thesis claims that the established object-oriented techniques have inherent difficulties with these issues as they are inherently built on \textit{unrestricted side effects}. In general, this makes reasoning about the correctness and validity of an implementation difficult as all possible states of the program need to be understood, which can be too much to comprehend when unrestricted side effects are allowed.

As a potential remedy to these issues, this thesis explores ways of approaching ABS through the \textit{pure} functional programming paradigm using the language Haskell. The focus throughout this thesis is on \textit{purity}. It identifies the guaranteed lack of unrestricted side effects, thus achieving \textit{referential transparency}, where computations do not depend on the history or context of the system, consequently leading to same results when run repeatedly with same inputs. There is also \textit{impure} functional programming, which supports all kind of unrestricted side effects like user input, reading from a file and other interactions with the real world and is the opposite of purity: computations \textit{do} depend on the history of the system \textit{and} the real world and \textit{might} result in different outcomes when executed repeatedly.

Purity has the effect that the state space of the implementation is dramatically reduced up to a point where it becomes feasible to understand all possible states, ultimately leading to an implementation which is more likely to be correct. This opens up the direction for parallelisation of ABS implementations, which has always been difficult with unrestricted side effects, but becomes easier in pure functional programming. Further, the use of pure functional programming with Haskell opens up the opportunity to explore randomised property-based testing for ABS as an alternative approach for automated code testing because the operational unit testing, as used in the established object-oriented techniques, is inherently not very well suited to test stochastic ABS. This is due to the fact that in unit testing each test needs to be constructed manually, potentially resulting in hundreds of tests to cover all cases. Also, unit testing is by definition completely deterministic, where test data is fixed in advance and the expected test results must not vary accross repeated runs. Randomised property-based  testing on the other hand  generates test cases automatically based on specifications expressed in code, therefore being able to deal with the stochastic nature of ABS.

To the best of the author's knowledge, this thesis is the first one to explore these ideas on a \textit{systematic} level, developing a foundation by presenting fundamental concepts and advanced features to show how to leverage the paradigm's benefits \cite{hudak_history_2007} to make them available when implementing ABS functionally. By doing this, the thesis shows \textit{how} to implement ABS in a pure functional way and \textit{why} it is beneficial to do so, what the drawbacks are, and when a pure functional approach should \textit{not} be used. The thesis does this by answering the following questions:

%RESEARCH QUESTIONs
\begin{enumerate}
	\item How can ABS be implemented in a pure functional way and what are the benefits and drawbacks in doing so?
	\item How can pure functional programming be used for parallel and concurrent programming? 
	\item How can pure functional programming be used for testing ABS implementations?
\end{enumerate}

\medskip

%HYPOTHESIS and SELL
This thesis' hypothesises that using pure functional programming for implementing ABS is indeed possible due to the pure computational character of most ABS models, because they do not rely on unrestricted side effects of asynchronous, impure Input/Output (IO) and direct user interaction. This should lead to simulations which are easier to test and verify, guaranteed to be reproducible already at compile time, with fewer potential sources of bugs and consequently can raise the level of confidence in the correctness of an implementation to a new level. Further, it should be easier to add parallelism and concurrency. Addressing the issue of easy parallelisation and concurrency of ABS is of tremendous importance, given the rise of multicore CPU architectures and cloud computing infrastructures in the last decade. By providing concepts which make parallelisation and concurrency in ABS easier and more likely to be correct, this enables implementers to fully exploit the available resources much more easily, with more confidence in the correctness of their implementations. Finally, this thesis hypothesises that by using pure functional programming, ABS becomes applicable to randomised property-based testing, which should be a natural fit to the stochastic nature of ABS. By employing property-based testing, it should finally become possible to address the issue of code testing in ABS to a satisfactory level, further increasing the level of confidence in the validity of an implementation. The main drawbacks are hypothesised to be low performance and agent interactions. Functional programming in general is assumed to have lower performance in comparison to imperative languages due to its higher level of abstraction, immutable data and lazy evaluation, which makes reasoning about performance difficult. Agent interactions on the other hand are trivial in object-oriented techniques due to unrestricted side effects in objects which are not available in functional programming.

The usefulness and potential impact of this thesis research is further underlined by the fact that it offers directions for a few of the research challenges for the future of ABS, called out by the ABS survey and review paper \cite{macal_everything_2016}. The challenge \textit{Large-scale ABMS} which focuses on efficient modelling and simulating large-scale ABS, is directly addressed by this thesis work on parallelism and concurrency in ABS. The challenges \emph{H2: Development of ABMS as an independent discipline with a common language that extends across domains} and \emph{H4: Requirement of complete descriptions of the simulation so others can independently replicate the results} are not directly related to this thesis work, but the declarative nature of pure functional programming is of fundamental importance here. It is well known that functional programming helps in structuring computation in a very clear and precise way, leading to a deeper understanding about the problem to implement. Thus, in this thesis the functional approach is also regarded as a way to think and explore ABS in a more rigorous way: as a tool for developing abstractions and especially to develop a deeper and more complete understanding of the computational structure underlying ABS. By implementing and generalising use cases, implicit knowledge is extracted and made explicit, thus potentially supporting the aforementioned challenges. Another relevant challenge \emph{H5: Requirement that all models be completely validated} is directly addressed by this thesis' pure functional programming approach as it is well established that this paradigm shines in program verification and validation \cite{hudak_history_2007, hutton_tutorial_1999}. Moreover, property-based testing offers a viable direction for this challenge as well, offering a much more natural fit for code testing than unit testing. Further, purity is a fundamentally important concept for this subject as it eases validation tremendously. Finally, the challenge \emph{H6: Developing applications of statistical and non-statistical validation techniques specifically for ABMS} is addressed by the use of randomised property-based testing which might prove to be a huge step forward in this matter.

\medskip

% SCOPE
In conclusion, this thesis claims that the ABS community needs functional programming because of its \textit{scientific computing} nature, where results need to be reproducible and correct, while simulations can massively scale up as well. However, as pointed out above, the established object-oriented approach the ABS community is using, needs a considerable level of effort and might even fail to deliver these objectives due to its conceptually different approach to computing with unrestricted side effects. It is the authors hope that this undertaking into \textit{pure} functional ABS is to the whole benefit of the ABS discipline and will feed back into its established object-oriented implementation techniques.

\section{Contributions}
\begin{enumerate}
	\item To the best knowledge of the author, this thesis is the first to \textit{systematically} investigate the use of the pure functional programming paradigm with Haskell, to ABS, laying out in-depth technical foundations and identifying its benefits and drawbacks. Additionally, the use of pure functional programming, which focuses on explicit data-flow representation, is a strong match with scientific computing, which is data centric as well. Consequently, due to the increased interest in functional concepts added to object-oriented languages in recent years (Lambdas, Method References and Streams in Java 8, rise of functional frameworks in JavaScript, Pythons functional features, etc.), because of its established benefits in concurrent programming, testing and software development in general, presenting such foundational research gives this thesis significant impact. What is more, a pure functional approach leads directly to fewer bugs and guaranteed reproducibility of repeated runs at compile time. This results in implementations which are more likely to be correct, which is of fundamental importance in all kind of scientific computing in general, giving this thesis considerable impact.
	
	\item To the best knowledge of the author, this thesis is the first to show the use of Software Transactional Memory to implement concurrent ABS and its potential benefit over lock-based approaches. The use of Software Transactional Memory is particularly compelling in functional programming because it can guarantee, at compile time, that retry semantics exclude non-repeatable persistent side effects. By employing Software Transactional Memory it is possible to implement a simulation which potentially allows massively large-scale ABS, but without the low level difficulties of concurrent programming. Consequently, it becomes easier and quicker to develop working and correct concurrent ABS models. Further, the use of Software Transactional Memory allows us still to approach concurrency as a data-flow approach, without cluttering model code with concurrency semantics. Although purity is lost when using Software Transactional Memory, it is still possible to retain certain guarantees about reproducibility, making it a highly attractive approach to concurrent scientific computing. Moreover, due to the increasing need for massively large-scale ABS in recent years \cite{lysenko_framework_2008}, showing this within a pure functional approach as well, gives this thesis substantial impact.
	
	\item To the best of the authors knowledge, this thesis is the first to present the use of property-based testing in ABS, which allows declarative specification testing of the implemented ABS directly in code with \textit{automated} random test case generation. This approach is an addition and also applicable to the established Test Driven Development process, where tests are written before the functionality under test. Further, it is also a complementary approach to unit testing, ultimately giving the developers an additional, powerful tool to test the implementation on a more conceptual level. More specifically, the thesis shows how to encode full agent specifications and model invariants and complete validation and verification including hypothesis testing with property-based testing. This should lead to simulation software which is more likely to be correct, thus making this a highly significant contribution with valuable impact.
\end{enumerate}

\section{Thesis structure}
The thesis is divided into three parts which act as the thematic narrative throughout the text followed by an Appendix. 

\paragraph{Part \ref{part:introduction}} begins the thesis by laying out the necessary prerequisites necessary to understand the ideas and motivation in the rest of the thesis.
\medskip

Chapter \ref{ch:motivation} introduces the problem and presents the motivation, aim and hypotheses.

\medskip

Chapter \ref{ch:background} presents related research and discusses the background necessary to understand the rest of the thesis. It presents a definition of ABS and gives an introduction to functional programming with advanced topics necessary to understand the concepts in this thesis. Further, it discusses an architectural categorisation on how to implement ABS from a language-agnostic point of view.

\medskip

Chapter \ref{ch:methodology} contains the methodology.

\medskip

\paragraph{Part \ref{part:research}} presents the main body of research.

\medskip

Chapter \ref{ch:timedriven} derives a time-driven ABS implementation for an agent-based SIR model. Because it is the first chapter discussing how to implement ABS in a pure functional way, it goes quite into detail in order to lay out the basic concepts. This chapter is rich in technical detail and aimed at the functional programming research community.

\medskip

Chapter \ref{ch:eventdriven} presents an event-driven approach to ABS using an event-driven agent-based SIR and the highly complex Sugarscape model. It builds on the concepts derived in Chapter \ref{ch:timedriven}, generalises them and pushes them forward towards a more general solution. Further, it also gives a brief outline how to transform the time-driven SIR implementation into an event-driven one. This chapter is rich in technical detail and aimed at the functional programming research community.

\medskip

Chapter \ref{ch:parallel_abs} establishes parallelism and concurrency to set the scene for the following two chapters on parallel ABS in general. This chapter is aimed at the ABS research community.

\medskip

Chapter \ref{ch:parallelism_ABS} shows how to achieve deterministic and pure parallelism in pure functional ABS. This chapter is aimed at the ABS research community.

\medskip

Chapter \ref{ch:concurrent_abs} presents an in-depth discussion on how to implement concurrent ABS using Software Transactional Memory. This chapter is aimed at the ABS research community.

\medskip

Chapter \ref{ch:property} introduces property-based testing for ABS and presents relevant concepts of property-based testing in general and the QuickCheck library in particular. This chapter is aimed at the ABS research community.

\medskip

Chapter \ref{ch:agentspec} shows how to use property-based testing to implement a full agent-specification test of the event- and time-driven SIR model in code and run it as property tests. This chapter is aimed at the ABS research community.

\medskip

Chapter \ref{ch:sir_invariants} shows how to derive and encode invariants the simulations dynamics must uphold in property tests. It also shows how to compare the dynamics of two implementations of the same underlying model, namely the time- and event-driven SIR implementations. Moreover, it shows how to put model specifications into code and check them with property tests by comparing the System Dynamics simulation to the agent-based one. This chapter is aimed at the ABS community.

\medskip

\paragraph{Part \ref{part:discussion}} is the closing part which discusses and concludes the thesis. 
\medskip

Chapter \ref{ch:discussion} revisits and discusses the initial motivation, aim and hypotheses. Additionally, it presents the drawbacks of the pure functional approach, discusses the Gintis case described above and answers the question of whether agents map naturally to objects or not. This chapter is aimed at the ABS research community.

\medskip

Chapter \ref{ch:conclusion} concludes and presents further research. This chapter is aimed at both the functional programming and ABS research community.

\paragraph{Appendices} contain additional material which relates to the overall research of this thesis but would be out of context in the respective chapters.
\medskip

Appendix \ref{app:validating_sugarscape} contains a brief overview over the validation process we went through when trying to get our Sugarscape implementation from Chapter \ref{ch:eventdriven} in line with the results from the original specification. In addition, we show how we can use property-based testing in an exploratory model to formulate and test hypotheses. This chapter is aimed at the ABS research community.

\medskip

Appendix \ref{app:sd_simulation} shows a pure functional implementation of a System Dynamics SIR simulation using Functional Reactive Programming. It shows that it is possible to directly encode System Dynamics specification in pure functional code with very high guarantees in correctness. This implementation is used in Chapter \ref{ch:sir_invariants} where the agent-based SIR dynamics are tested against the System Dynamics ones. This chapter is aimed at the ABS research community.