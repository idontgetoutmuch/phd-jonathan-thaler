\chapter{Methodology}
\label{ch:methodology}
% PEER: Research Methodology (or Research Design) should be a separate chapter. It should include aim / objectives. It describes how you are planning to do the research.

%- Methodology is the justification for using a particular research method.
%- make clear what our method is (Method is simply a research tool, a component of research – say for example, a qualitative method such as interviews) and then justify it => this is then the methodology 
%- discussion of methodology is missing: what is the scientific approach we used in our thesis to address the aims and answer hypotheses? Basically we perform use-cases and discuss them

% type of research: 
%basic AND applied: Basic research aims to develop knowledge, theories and predictions, while applied research aims to develop techniques, products and procedures. Do you want to expand scientific understanding or solve a practical problem?
% Exploratory vs explanatory: Exploratory research aims to explore the main aspects of an under-researched problem, while explanatory research aims to explain the causes and consequences of a well-defined problem. How much is already known about your research problem? Are you conducting initial research on a newly-identified issue, or seeking precise conclusions about an established issue?
% Inductive vs deductive: Inductive research aims to develop a theory, while deductive research aims to test a theory. Is there already some theory on your research problem that you can use to develop hypotheses, or do you want to propose new theories based on your findings?

% Why is this the most suitable approach to answering your research questions?
% Is this a standard methodology in your field or does it require justification?
% What are the criteria for validity and reliability in this type of research?
% The scientific method means starting with a hypothesis and then collecting evidence to support or deny it. 

%- discussion of methodology is missing: what is the scientific approach we used in our thesis to address the aims and answer hypotheses? Basically we perform use-cases and discuss them

In this section we briefly motivate and justify our methods, point out the scientific approach used in this thesis to address the aims and answer hypotheses put forward in Chapter \ref{ch:motivation}. Generally, all concepts we derive are driven by the hypotheses and aims from Chapter \ref{ch:motivation}. We continuously refer back to these hypotheses, especially in the discussion sections of the respective chapters and the final conclusion and discussion chapters. By taking this methodical approach, we are able to qualitatively assess whether the thesis has achieved the initial aims and answered the hypotheses in a satisfactory way.

Fundamentally, the work in this thesis constitutes basic, methodological, exploratory and deductive research. The aim is to develop new, basic knowledge of how to implement ABS pure functionally. Additionally, we develop new techniques and concepts and show how to apply them to be able to implement ABS pure functionally. The thesis is exploratory as this problem is under-researched, therefore this thesis constitutes initial research, laying out a new field. Consequently, this thesis is deductive research, because it is testing certain hypotheses about pure functional ABS.

\medskip

The first part of our method is dedicated to answering the question of how to implement ABS in a pure functional way, following a time-driven approach in Chapter \ref{ch:timedriven} and an event-driven approach in Chapter \ref{ch:eventdriven}. The reason for including these two techniques is that both are equally important in ABS. Additionally, the concepts of event-driven ABS build on the ones developed in the preceding time-driven approach.

Generally, in both approaches, the aim is to develop a robust, maintainable and extensible implementation of the use-case models through which we develop concepts which can be adopted to ABS in general. The overall goal is a clear representation of agents with their local (immutable) state, a way for the agents to interact with an (active) environment and one-directional and synchronous interactions between agents. To really push a pure functional approach with a complex model, we also undertake a full and validated implementation of the Sugarscape model.

Of fundamental importance is keeping the agent behaviour free of the \texttt{IO} Monad. If side effects are required, we restrict ourselves to use only pure, deterministic ones like \texttt{Reader}, \texttt{Writer}, \texttt{State}, \texttt{Maybe}.

The newly developed concepts and the pure functional SIR and Sugarscape implementations should then provide enough evidence to answer the first research question and support or reject our initial hypothesis that using a pure functional language to implement non-trivial ABS models is possible in a robust and maintainable way.

\medskip

The second part of our method is dedicated to showing the benefits of using the previously developed pure functional approach to ABS. It is split into two sub-parts. In the first we investigate the hypothesis that pure functional programming makes it easy to apply parallel computation using parallelism and concurrency to ABS. We follow a rather applied approach by taking the concepts developed in the previous chapters and explore techniques how to apply parallelism and concurrency to them. The results are then subjected to various performance experiments to test their potential, with the performance benchmarks providing hard, quantitative facts.

The second subpart answers another central hypothesis, namely that randomised property-based testing is a good match to test stochastic ABS implementations. For this we took the pure functional implementations of the time and event-driven SIR and Sugarscape model and looked into how property-based testing can be used to both verify and validate them. More specifically, we want to verify individual, isolated agent behaviour both in time and event-driven ABS. Further, we aim at connecting underlying formal specifications like the System Dynamics model to an actual agent-based implementation like the SIR model. Finally, we aim at testing hypotheses in the exploratory Sugarscape model. The resulting property-based tests should provide evidence to support or reject our hypothesis.