\chapter{Pure Functional ABS}
\label{ch:pfABS}

Is the main chapter of the thesis to discuss \textit{how} can we do pure functional ABS. Roughly 75\% exists already.


TODO: can we derive an agent-monad?

TODO: what about comonads? read essence dataflow paper \cite{uustalu_essence_2006}: monads not capable of stream-based programming and arrows too general therefor comonads, we are using msfs for abs therefore streambased so maybe applicable to our approach/agents=comonads. comonads structure notions of context-dependent computation or streams, which ABS can be seen as of. this paper says that monads are not capable of doing stream functions, maybe this is the reason why i fail in my attempt of defining an ABS in idris because i always tried to implement a monad family. TODO: stopped at comonad section, continue from there. TODO understand comonads: https://www.schoolofhaskell.com/user/edwardk/cellular-automata and https://kukuruku.co/post/cellular-automata-using-comonads/

\section{Time-Driven ABS}
Following pure functional ABS, also discuss how can we implement the 4 update-strategies of the art-iterating paper in our pure functional approach.

\subsection{A hybrid approach}
TODO: discuss using sdhaskell paper

\subsection{Super-Sampling}
discuss using FrABS report

\section{Event-Driven ABS}
Following towards papers SugarScape implementation
\subsection{Synchronised Agent-Interactions}
Following towards papers SugarScape implementation

% direct MSF call, Problem is recursive nature. maybe try it with gintis Implementation. I just learned that what i want to achieve is actually: https://en.wikipedia.org/wiki/This_(computer_programming)#Open_recursion AND OPEN RECURSION IS PRETTY BAD

%Although there are similarities to the work of \cite{botta_time_2010} (the use of messages and the problem of when to advance time in models with arbitrary number synchronised agent-interactions), we approach our agents differently. First in our approach an agent is only a single MSF and thus can not be directly queried for its internal state / its id or outgoing messages, instead of taking a list of messages, our agents take a single event/message and can produce an arbitrary number of outgoing messages together with an observable state - note that this would allow to query the agent for its id and its state as well by simply sending a corresponding message to the agents MSF and requiring the agent to implement message handling for it. Also the state of our agents is \textit{completely} localised and there is no means of accessing the state from outside the agent, they are thus "fully encapsulated agents" \cite{botta_time_2010}. Note that the authors of \cite{botta_time_2010} define their agents with a polymorphic agent-state type \textit{s}, which implies that without knowledge of the specific type of \textit{s} there would be no way of accessing the state, rendering it in fact also fully encapsulated. The problem of advancing time in our approach is solved not exactly the same but conceptually it is the same: after sending a tick message to each agent (in random order), we process all agents until they are idle: there are no more enqueued messages / events in the queue.