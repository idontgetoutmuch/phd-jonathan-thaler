\chapter*{}
In this part of the thesis we had an extensive look into the usefulness of randomised property-based testing in the development, specification and testing of pure functional ABS. We found property-based testing particularly well suited for ABS firstly due to ABS stochastic nature and second because we can formulate specifications, meaning we describe \textit{what} to test instead of \textit{how} to test. Also the deductive nature of falsification in property-based testing suits very well the constructive and often exploratory nature of ABS. 

Indeed, we can see property-based testing not only as a post-implementation testing tool but as an extension to the development process, where the developer engages in an interactive cycle of implementing agent and simulation behaviour and then immediately putting specifications into property tests and running them. This approach of expressing specifications instead of special cases like in unit tests is arguably a much more natural approach to test-driven development in ABS development than relying only on unit tests.

In this part we only focused on the explanatory SIR model and ignored the exploratory Sugarcape. It is important to understand that testing of exploratory models is also possible through hypothesis testing. We discuss this approach in Appendix \ref{app:validating_sugarscape} in the context of validating our Sugarscape implementation.

We have only touched the tip of the iceberg and expect tremendous potential from applying property-based testing to different kind of models and in implementing ABS in general. Indeed, although property-based testing has its origins in Haskell, similar libraries have been developed for other languages like Java, Python and C++ as well and we hope that our research will spark an interest in applying property-based testing to the established object-oriented languages in ABS as well.

%For further insight into testing FRP we can directly leverage on the work of \cite{perez_testing_2017}. Another unique benefit of pure functional programming, \textit{equational reasoning} was not investigated in this thesis as it was beyond the focus of this Ph.D. We expect that this technique is applicable to parts of our approach as well but leave this for further research.