\chapter{Testing Agent specifications}
\label{ch:agentspec}

In this chapter we are showing how to use QuickCheck to encode full agent-specifications directly in code as property-tests. These tests serve then both as formal specification and tests at the same time - a fundamental strength of property-based testing, not possible with unit-testing in this strong expressive form. Besides the high expressivity, QuickCheck also allows us to state statistical coverage for certain cases, which allows to express statistical properties of the agents behaviour, something not directly possible with unit-testing. This is a very strong indication that property-based testing is a natural fit to test agent-based simulation. We discuss both time- and event-driven  implementations of the agent-based SIR model as introduced in Chapter \ref{sec:sir_model}.

\section{Event-driven specification}
In this section we present how QuickCheck can be used to test event-driven agents by expressing their \textit{specification} as property tests in the case of the event-driven SIR implementation from chapter \ref{sec:eventdriven_basics}.

In general, testing event-driven agents is fundamentally different and more complex than testing time-driven agents, as their interface surface is generally much larger. Events form the input to the agents to which they react with new events where the dependencies between those can be quite complex and deep. Using property-based testing we can encode the invariants and end up with an actual specification of their behaviour, acting as documentation, regression test within a TDD and property tests.

Note that the concepts presented here are applicable with slight adjustments to the Sugarscape implementation as well but we focused on the SIR one as its specification is shorter and does not require as much in-depth details. % - after all we are interested in deriving concepts, not dealing with specific technicalities.

With event-driven ABS, a good starting point in specifying and testing the system, is simply relating the input events to expected output events. In the SIR implementation we have only three events, making it feasible to give a full formal specification. The Sugarscape implementation has more than 16 events, which makes it much harder to test it with sufficient coverage, giving a good reason to primarily focus on the SIR implementation for explanatory purposes.

\subsection{Deriving the specification}
We start by giving the full \textit{specification} of the susceptible, infected and recovered agent by stating the input-to-output event relations. The susceptible agent is specified as follows:

\begin{enumerate}
	\item \texttt{MakeContact} - if the agent receives this event it will output $\beta$ \texttt{Contact ai Susceptible} events, where \texttt{ai} is the agents own id. The events have to be scheduled immediately without delay, thus having the current time as scheduling timestamp. The receivers of the events are uniformly randomly chosen from the agent population. The agent doesn't change its state, stays \texttt{Susceptible} and does not schedule any other events than the ones mentioned.
	
	\item \texttt{Contact \_ Infected} - if the agent receives this event there is a chance of uniform probability $\gamma$ (infectivity) that the agent becomes \texttt{Infected}. If this happens, the agent will schedule a \texttt{Recover} event to itself into the future, where the time is drawn randomly from the exponential distribution with $\lambda = \delta$ (illness duration). If the agent does not become infected, it will not change its state, stays \texttt{Susceptible} and does not schedule any events.
	
	\item \texttt{Contact \_ \_} or \texttt{Recover} - if the agent receives any of these other events it will not change its state, stays \texttt{Susceptible} and does not schedule any events.
\end{enumerate}

This specification implicitly covers that a susceptible agent can never transition from a \texttt{Susceptible} to a \texttt{Recovered} state within a single event as it can only make the transition to \texttt{Infected} or stay \texttt{Susceptible}. The infected agent is specified as follows:

\begin{enumerate}
	\item \texttt{Recover} - if the agent receives this, it will not schedule any events and make the transition to the \texttt{Recovered} state.
	
	\item \texttt{Contact sender Susceptible} - if the agent receives this, it will reply immediately with \texttt{Contact ai Infected} to \textit{sender}, where \texttt{ai} is the infected agents' id and the scheduling timestamp is the current time. It will not schedule any events and stays \texttt{Infected}.
	
	\item In case of any other event, the agent will not schedule any events and stays \texttt{Infected}.
\end{enumerate}

This specification implicitly covers that an infected agent never goes back to the \texttt{Susceptible} state as it can only make the transition to \texttt{Recovered} or stay \texttt{Infected}. From the specification of the susceptible agent it becomes clear that a susceptible agent who became infected, will always recover as the transition to \texttt{Infected} includes the scheduling of \texttt{Recovered} to itself. 

\medskip

The \textit{recovered} agent specification is very simple. It stays \texttt{Recovered} forever and does not schedule any events.

\medskip

The question is now how to put these into a property test with QuickCheck. We focus on the susceptible agent, as it it the most complex one, which concepts can then be easily applied to the other two. Generally speaking, we create a random \textit{susceptible} agent and a random event, feed it to the agent to get the output and check the invariants accordingly to input and output. % In the specification there are stated three probabilities regarding $\beta$ (contact rate), $\gamma$ (infectivity) and $\delta$ (illness duration). We will only check one, $\gamma$ (infectivity) using the coverage features of QuickCheck and write additional property tests for the other two. The reason for that is, that checking $\gamma$ is natural with the invariant checking whereas the others need a slightly different approach and are more obviously stated in separate property tests.

\subsection{Encoding invariants}
We start by encoding the invariants of the susceptible agent directly into Haskell, implementing a function which takes all necessary parameters and returns a \texttt{Bool} indicating whether the invariants hold or not. The encoding is straightforward when using pattern matching and it nearly reads like a formal specification due to the declarative nature of functional programming.

\begin{HaskellCode}
susceptibleProps :: SIREvent              -- ^ Random event sent to agent
                 -> SIRState              -- ^ Output state of the agent
                 -> [QueueItem SIREvent]  -- ^ Events the agent scheduled
                 -> AgentId               -- ^ Agent id of the agent
                 -> Bool
-- received Recover => stay Susceptible, no event scheduled
susceptibleProps Recover Susceptible es _ = null es
-- received MakeContact => stay Susceptible, check events
susceptibleProps MakeContact Susceptible es ai
  = checkMakeContactInvariants ai es cor 
-- received Contact _ Recovered => stay Susceptible, no event scheduled
susceptibleProps (Contact _ Recovered) Susceptible es _ = null es
-- received Contact _ Susceptible => stay Susceptible, no event scheduled
susceptibleProps (Contact _ Susceptible) Susceptible es _  = null es
-- received Contact _ Infected, didn't get Infected, no event scheduled
susceptibleProps (Contact _ Infected) Susceptible es _ = null es
-- received Contact _ Infected AND got infected, check events
susceptibleProps (Contact _ Infected) Infected es ai
  = checkInfectedInvariants ai es
-- all other cases are invalid and result in a failed test case
susceptibleProps _ _ _ _ = False
\end{HaskellCode}

Next, we give the implementation for the \texttt{checkMakeContactInvariants} and \texttt{checkInfectedInvariants} functions. The function \texttt{checkMakeContactInvariants} encodes the invariants which have to hold when the susceptible agent receives a \texttt{MakeContact} event. The \texttt{checkInfectedInvariants} function encodes the invariants which have to hold when the susceptible agent got \texttt{Infected}. Both implementations read like a formal specification, again thanks to the declarative nature of functional programming and pattern matching:

\begin{HaskellCode}
checkInfectedInvariants :: AgentId              -- ^ Agent id of the agent 
                        -> [QueueItem SIREvent] -- ^ Events the agent scheduled
                        -> Bool
checkInfectedInvariants sender 
  -- expect exactly one Recovery event
  [QueueItem receiver (Event Recover) t'] 
  -- receiver is sender (self) and scheduled into the future
  = sender == receiver && t' >= t 
-- all other cases are invalid
checkInfectedInvariants _ _ = False
\end{HaskellCode}

The \texttt{checkMakeContactInvariants} is a bit more complex but reads as a formal specification as well:

\begin{HaskellCode}
checkMakeContactInvariants :: AgentId              -- ^ Agent id of the agent 
                           -> [QueueItem SIREvent] -- ^ Events the agent scheduled
                           -> Int                  -- ^ Contact Rate
                           -> Bool
checkMakeContactInvariants sender es contactRate
    -- make sure there has to be exactly one MakeContact event and
    -- exactly contactRate Contact events
    = invOK && hasMakeCont && numCont == contactRate
  where
    (invOK, hasMakeCont, numCont) 
      = foldr checkMakeContactInvariantsAux (True, False, 0) es

    checkMakeContactInvariantsAux :: QueueItem SIREvent 
                                  -> (Bool, Bool, Int)
                                  -> (Bool, Bool, Int)
    checkMakeContactInvariantsAux 
        (QueueItem (Contact sender' Susceptible) receiver t') (b, mkb, n)
      = (b && sender == sender'   -- sender in Contact must be self
           && receiver `elem` ais -- receiver of Contact must be in agent ids
           && t == t', mkb, n+1)  -- Contact event is scheduled immediately
    checkMakeContactInvariantsAux 
        (QueueItem MakeContact receiver t') (b, mkb, n) 
      = (b && receiver == sender  -- receiver of MakeContact is agent itself
           && t' == t + 1         -- MakeContact scheduled 1 timeunit into future
           &&  not mkb, True, n)  -- there can only be one MakeContact event
    checkMakeContactInvariantsAux _ (_, _, _) 
      = (False, False, 0)         -- other patterns are invalid
\end{HaskellCode}

What is left is to actually write a property test using QuickCheck. We are making heavy use of random parameters to express that the properties have to hold invariant of the model parameters. We make use of additional data generator modifiers: \texttt{Positive} ensures that the value generated is positive; \texttt{NonEmptyList} ensures that the randomly generated list is not empty.

\begin{HaskellCode}
prop_susceptible_invariants :: Positive Int         -- ^ Contact rate (beta)
                            -> Probability          -- ^ Infectivity (gamma)
                            -> Positive Double      -- ^ Illness duration (delta)
                            -> Positive Double      -- ^ Current simulation time
                            -> NonEmptyList AgentId -- ^ population agent ids
                            -> Gen Property
prop_susceptible_invariants 
  (Positive beta) (P gamma) (Positive delta) (Positive t) (NonEmpty ais) = do
  -- generate random event, requires the population agent ids
  evt <- genEvent ais
  -- run susceptible random agent with given parameters
  (ai, ao, es) <- genRunSusceptibleAgent beta gamma delta t ais evt
  -- check properties
  return $ property $ susceptibleProps evt ao es ai
\end{HaskellCode}

When running this property test all 100 test cases pass. Due to the large random sampling space with 5 parameters, we increase the number of test cases to generate to 100,000 - still all test cases pass.

\subsection{Encoding transition probabilities}
In the specifications above there are probabilistic state transitions, for example an infected agent \textit{will} recover after a given time, which is randomly distributed with the exponential distribution. The susceptible agent \textit{might} become infected, depending on the events it receives and the infectivity ($\gamma$) parameter. We look now into how we can encode these probabilistic properties using the powerful \texttt{cover} and \texttt{checkCoverage} feature of QuickCheck.

\subsubsection{Susceptible agent}
We follow the same approach as in encoding the invariants of the susceptible agent but instead of checking the invariants, we compute the probability for each case. In this property test we cannot randomise the model parameters because this would lead to random coverage. This might seem like a disadvantage but we do not really have a choice here, still the model parameters can be adjusted arbitrarily and the property must hold. %Note that we do not provide the details of computing the probabilities of each input-to-output case as it is quite technical and of not much importance - it is only a matter of multiplication and divisions amongst the event-frequencies and model parameters.
We make use of the \texttt{cover} function together with \texttt{checkCoverage}, which ensures that we get a statistical robust estimate whether the expected percentages can be reached or not. Implementing this property test is then simply a matter of computing the probabilities and of case analysis over the random input event and the agents output.

\begin{HaskellCode}
...
case evt of 
  Recover -> 
    cover recoverPerc True 
     ("Susceptible receives Recover, expected " ++ show recoverPerc) True
...
\end{HaskellCode}

Note the usage pattern of \texttt{cover} where we unconditionally include the test case into the coverage class so all test cases pass. The reason for this is that we are just interested in testing the coverage, which is in fact the property we want to test. We could have combined this test into the previous one but then we couldn't have used randomised model parameters. For this reason, and to keep the concerns separated we opted for two different tests, which makes them also much more readable.

%\begin{HaskellCode}
%prop_susceptible_proabilities :: Positive Double      -- ^ Current simulation time
%                              -> NonEmptyList AgentId -- ^ Agent ids of the population
%                              -> Property
%prop_susceptible_proabilities (Positive t) (NonEmpty ais) = checkCoverage (do
%  -- fixed model parameters, otherwise random coverage
%  let cor = 5
%      inf = 0.05
%      ild = 15.0
%
%   -- compute distributions for all cases
%  let recoverPerc       = ...
%      makeContPerc      = ...
%      contactRecPerc    = ...
%      contactSusPerc    = ...
%      contactInfSusPerc = ...
%      contactInfInfPerc = ...
%
%  -- generate a random event
%  evt <- genEvent ais
%  -- run susceptible random agent with given parameters
%  (_, ao, _) <- genRunSusceptibleAgent cor inf ild t ais evt
%
%  -- encode expected distributions
%  return $ property $
%    case evt of 
%      Recover -> 
%        cover recoverPerc True 
%          ("Susceptible receives Recover, expected " ++ 
%           show recoverPerc) True
%      MakeContact -> 
%        cover makeContPerc True 
%          ("Susceptible receives MakeContact, expected " ++ 
%           show makeContPerc) True
%      (Contact _ Recovered) -> 
%        cover contactRecPerc True 
%          ("Susceptible receives Contact * Recovered, expected " ++ 
%           show contactRecPerc) True
%      (Contact _ Susceptible) -> 
%        cover contactSusPerc True 
%          ("Susceptible receives Contact * Susceptible, expected " ++ 
%           show contactSusPerc) True
%      (Contact _ Infected) -> 
%        case ao of
%          Susceptible ->
%            cover contactInfSusPerc True 
%              ("Susceptible receives Contact * Infected, stays Susceptible " ++
%               ", expected " ++ show contactInfSusPerc) True
%          Infected ->
%            cover contactInfInfPerc True 
%              ("Susceptible receives Contact * Infected, becomes Infected, " ++
%               ", expected " ++ show contactInfInfPerc) True
%          _ ->
%            cover 0 True "Impossible Case, expected 0" True
%\end{HaskellCode}

When running the property test we get the following output:

\begin{footnotesize}
\begin{verbatim}
+++ OK, passed 819200 tests:
33.3582% Susceptible receives MakeContact, expected 33.33%
33.2578% Susceptible receives Recover, expected 33.33%
11.1643% Susceptible receives Contact * Recovered, expected 11.11%
11.1096% Susceptible receives Contact * Susceptible, expected 11.11%
10.5616% Susceptible receives Contact * Infected, stays Susceptible, expected 10.56%
 0.5485% Susceptible receives Contact * Infected, becomes Infected, expected 0.56%
\end{verbatim}
\end{footnotesize}

After 819,200 (!) test cases QuickCheck comes to the conclusion that the distributions generated by the test cases reflect the expected distributions and passes the property test. We see that the values do not match exactly in some cases but by using sequential statistical hypothesis testing, QuickCheck is able to conclude that the coverage are statistically equal.

\subsubsection{Infected agent}
We want to write a property test which checks whether the transition from \texttt{Infected} to \texttt{Recovered} actually follows the exponential distribution with a fixed $\delta$ (illness duration). The idea is to compute the expected probability for agents having an illness duration of less or equal $\delta$. This probability is given by the Cumulative Density Function (CDF) of the exponential distribution. The question is how to get the infected illness duration. This is simply achieved by infecting a susceptible agent and taking the scheduling time of the \texttt{Recover} event. We have written a custom data generator for this:

\begin{HaskellCode}
getInfectedAgentDuration :: Double -> Gen (SIRState, Double)
getInfectedAgentDuration ild = do
  -- with these parameters the susceptible agent WILL become infected
  (_, ao, es) <- genRunSusceptibleAgent 1 1 ild 0 [0] (Contact 0 Infected)
  return (ao, recoveryTime es)
  where
    -- expect exactly one event: Recover
    recoveryTime :: [QueueItem SIREvent] -> Double
    recoveryTime [QueueItem Recover _ t]  = t
    recoveryTime _ = 0
\end{HaskellCode}

Encoding the probability check into a property test is straightforward:

\begin{HaskellCode}
prop_infected_duration :: Property
prop_infected_duration = checkCoverage (do
  -- fixed model parameter, otherwise random coverage
  let ild  = 15
  -- compute probability drawing a random value less or equal
  -- ild from the exponential distribution (follows the CDF)
  let prob = 100 * expCDF (1 / ild) ild

  -- run random susceptible agent to become infected and
  -- return agents state and recovery time
  (ao, dur) <- getInfectedAgentDuration ild

  return (cover prob (dur <= ild) 
            ("Infected agent recovery time is less or equals " ++ show ild ++ 
             ", expected at least " ++ show prob) 
            (ao == Infected)) -- final state has to Infected
\end{HaskellCode}

When running the property test we get the following output.

\begin{footnotesize}
\begin{verbatim}
+++ OK, passed 3200 tests 
    (63.62% Infected agent recovery time is less or equals 15.0, 
     expected at least 63.21%).
\end{verbatim}
\end{footnotesize}

QuickCheck is able to determine after only 3,200 test cases that the expected coverage is met and passes the property test.

\section{Time-Driven Specification}
\label{sec:timedriven_specification}
The time-driven SIR agents have a very small interface as they only receive the agent population from the previous step and output their state in the current step. We can also assume an implicit forward flow of time, statically guaranteed by Yampas Arrowized FRP. Thus, a specification of a time-driven approach is given in terms of probabilities and timeouts, rather than in events as in the event-driven testing presented before.

\begin{itemize}
	\item Susceptible agent - makes \textit{on average} contact with $\beta$ (contact rate) agents per time unit. The distribution follows the exponential distribution with $\lambda = \frac{1}{\beta}$. If a susceptible agent gets into contact with an infected agent, it will become infected with a uniform probability of $\gamma$ (infectivity).
	
	\item Infected agent - \textit{will} recover \textit{on average} after $\delta$ (illness duration) time units. The distribution follows the exponential distribution with $\lambda = \delta$.

	\item Recovered agent - stays recovered \textit{forever}.
\end{itemize}

\subsection{Specifications of the Susceptible Agent}
We cannot directly observe that a susceptible agent contacts other agents like we can in the event-driven approach, but only indirectly through its change of state. The change of state says that a susceptible agent \textit{might} become infected if there are infected agents in the population.
Consequently, when we run a susceptible agent for some time, we have three possible outcomes of the agents output stream: 1. the agent did not get infected and thus all elements of the stream are \texttt{Susceptible}, 2. the agent got infected and up to a given index in the stream all elements are \texttt{Susceptible} and change to \texttt{Infected} after, 3. the agent got \texttt{Infected} and then \texttt{Recovered} thus the stream is the same as in infected, but there is a second index after where all elements change to \texttt{Recovered}. Encoding them in code is straightforward:

\begin{HaskellCode}
susceptibleInv :: [SIRState] -- output stream of the susceptible agent 
               -> Bool       -- population contains an infected agent
               -> Bool       -- True in case the invariant holds
susceptibleInv aos infInPop
    -- Susceptible became Infected and then Recovered
    | isJust recIdxMay 
      = infIdx < recIdx &&  -- agent has to become infected before recovering
        all (==Susceptible) (take infIdx aos) && 
        all (==Infected) (take (recIdx - infIdx) (drop infIdx aos)) && 
        all (==Recovered) (drop recIdx aos) &&
        infInPop  -- can only happen if there are infected in the population

    -- Susceptible became Infected
    | isJust infIdxMay 
      = all (==Susceptible) (take infIdx aos) &&
        all (==Infected) (drop infIdx aos) &&
        infInPop -- can only happen if there are infected in the population

    -- Susceptible stayed Susceptible
    | otherwise = all (==Susceptible) aos
  where
    -- look for the first element when agent became Infected
    infIdxMay = elemIndex Infected aos
    -- look for the first element when agent became Recovered
    recIdxMay = elemIndex Recovered aos
    -- extract index
    infIdx = fromJust infIdxMay
    recIdx = fromJust recIdxMay
\end{HaskellCode}

Putting this into a property test is also straightforward. We generate a random population, run a random susceptible agent with a sampling rate of $\Delta t = 0.01$, and check the invariants on its output stream. These invariants all have to hold independently of the positive duration we run the random susceptible agent for. Consequently, we run the agent for a random amount of time units. The invariants also have to hold for arbitrary positive beta (contact rate), gamma (infectivity), and delta (illness duration). At the same time, we want to get an idea of the percentage of agents which stayed susceptible, became infected, or made the transition to recovered, thus we \texttt{label} all our test cases accordingly.

\begin{HaskellCode}
prop_susceptible_inv :: Positive Double -- beta, contact rate
                     -> Probability     -- gamma, infectivity within (0,1)
                     -> Positive Double -- delta, illness duration
                     -> TimeRange       -- simulation duration, within (0,50)
                     -> [SIRState]      -- random population
                     -> Property
prop_susceptible_inv
      (Positive beta) (P gamma) (Positive delta) (T t) as = property (do  
    -- population contains an infected agent True/False
    let infInPop = Infected `elem` as
    -- run a random susceptible agent for random time units with 
    -- sampling rate dt 0.01 and return its stream of output
    aos <- genSusceptible beta gamma delta as t 0.01
    -- construct property
    return 
        -- label all test cases
        label (labelTestCase aos) 
        -- check invariants on output stream
        (property (susceptibleInv aos infInPop))
  where
    labelTestCase :: [SIRState] -> String
    labelTestCase aos
      | Recovered `elem` aos = "Susceptible -> Infected -> Recovered"
      | Infected `elem` aos  = "Susceptible -> Infected"
      | otherwise            = "Susceptible"
\end{HaskellCode}

Due to the high dimensionality of the random sampling space, we run 10,000 tests. All succeed as expected.

\begin{verbatim}
SIR Agent Specifications Tests
  Susceptible agents invariants: OK (12.72s)
    +++ OK, passed 10000 tests:
    55.78% Susceptible -> Infected -> Recovered
    37.19% Susceptible -> Infected
     7.03% Susceptible
\end{verbatim}

This test has not stated anything so far about the probability of a susceptible agent getting infected. The probability for it is bimodal (see Chapter \ref{ch:sir_invariants}) due to the combined probabilities of the exponential distribution of the contact rate $\beta$ and the uniform distribution of the infectivity $\gamma$. Unfortunately, the bimodality makes it impossible to compute a coverage percentage of infected in this case, as we did in the event-driven test. The reason for this is because the bimodal distribution can only be described in terms of a distribution and not a single probability. This was possible in the even-driven approach because we decoupled the production of the \texttt{Contact \_ Infected} event from the infection. Both were uniformly distributed, thus we could compute a coverage percentage. Therefore, we see that different approaches also allow different explicitness of testing.

\subsection{Probabilities of the Infected Agent}
An infected agent \textit{will} recover after a \textit{finite} amount of time, thus we assume that an index exists in the output stream, where the elements will change to \texttt{Recovered}. From the index we can compute the time of recovery, knowing the fixed sampling rate $\Delta t$.

\begin{HaskellCode}
infectedInvariant :: [SIRState]   -- stream of outputs from infected agent
                  -> Double       -- Sampling rate dt
                  -> Maybe Double -- Just recovery time, Nothing if no recovery
infectedInvariant aos dt  = do
  -- search for the index of the first Recovery element
  recIdx <- elemIndex Recovered aos
  -- all elements up to the index need to be Infected,
  -- because the agent cannot go back to Susceptible
  if all (==Infected) (take recIdx aos)
    then Just (dt * recIdx)
    else Nothing
\end{HaskellCode}

To put this into a property test, we follow a similar approach as in the event-driven case of the infected agents' invariants. We employ the CDF of the exponential distribution to get the probability of an agent recovering within $\delta$ (illness duration) time steps. We then run a random infected agent for an \textit{unlimited} time with a sampling rate o f $\Delta t = 0.01$. Next, we search in its potentially infinite output stream for the first occurrence of an \texttt{Infected} element to compute the recovery time, as shown in the invariant above. The code is conceptually exactly the same as in the event-driven case, so we will not repeat the property test here.

%\begin{HaskellCode}
%prop_infected_invariants :: [SIRState] -> Property
%prop_infected_invariants as = checkCoverage (do
%  -- delta, illnes duration
%  let illnessDuration = 15.0
%  -- compute perc of agents which recover in less or equal 
%  -- illnessDuration time units. Follows the exponential distribution
%  -- thus we use the CDF to compute the probability.
%  let prob = 100 * expCDF (1 / illnessDuration) illnessDuration
%  -- fixed sampling rate
%  let dt = 0.01
%
%  -- run a random infected agent without time-limit (0) and sampling rate
%  -- of 0.01 and return its infinite output stream 
%  aos <- genInfected illnessDuration as 0 dt
%
%  -- compute the recovery time
%  let dur = infectedInvariant aos dt
%
%  return (cover prob (fromJust dur <= illnessDuration)
%            ("infected agents have an illness duration of " ++ show illnessDuration ++
%             " or less, expected " ++ show prob) (isJust dur)
%\end{HaskellCode}

When running the test we get the following output indicating that QuickCheck finds the coverage to be satisfied after 3,200 test cases:

\begin{verbatim}
+++ OK, passed 3200 tests (62.28% infected agents have an illness 
    duration of 15.0 or less, expected 63.21).
\end{verbatim}

The fact that we run the random infected agent explicitly without time limit expresses the invariant that an infected agent \textit{will} recover in \textit{finite} time steps. A correct implementation will produce a stream, which contains an index after which all elements are \texttt{Infected}, thus resulting in \texttt{Just} recovery time. This is also a direct expression of the fact that the CDF of the exponential distribution reaches 1 at infinity. An approach that would guarantee the termination would be to limit the time to run the infected agent to $\delta$ (illness duration) and always evaluate the property to \texttt{True}. This approach guarantees termination but removes an important part of the specification. We decided to stick to the initial approach to make the specification really clear and in practice it has turned out to terminate within a very short time (see below).

\subsection{The Non-Computability of the Recovered Agent Test}
The property test for the recovered agent is trivial. We run a random recovered agent for a random number of time units with $\Delta t = 0.01$ and require that all elements in the output stream are \texttt{Recovered}. Of course, this is no proof that the recovered agent stays recovered \textit{forever} as this would take \textit{forever} to test and is thus not computable. Here we are hitting the limits of what is possible with random black-box testing. Without looking at the actual implementation it is not possible to prove that the recovered agent is really behaving as specified. We made this fact very clear at the beginning of Chapter \ref{ch:property} that property-based testing is not proof for correctness, but is only a support for raising the confidence in correctness by constructing cases that show that the behaviour is not incorrect.

To be really sure that the recovered agent behaves as specified we need to employ white-box verification and look at the actual implementation. It is immediately obvious that the implementation follows the specification and actually \textit{is} the specification. We can even regard it as a very concise proof that it will stay recovered \textit{forever}:

\begin{HaskellCode}
recoveredAgent :: SIRAgent
recoveredAgent = constant Recovered
\end{HaskellCode}

The signal function \texttt{constant} is the \texttt{const} function lifted into an arrow: \texttt{constant b = arr (const b)}. This should be proof enough that a recovered agent will stay recovered \textit{forever}.

\section{Discussion}
TODO: comparison of time- vs. event-driven testing
- time-driven is rather straight forward: just feed the population and increment the time, the agents are much more a black-box than in the event-driven approach, where there is much revealed about their inner workings through the events, thus we have to be much more explicit in event-driven. Note that we were not able to state a coverage for the infection of susceptible agents because the distribution is bimodal due to the combined probabliities of the exponential and uniform distriubtion. In event-driven we were able to state a coverage because we decoupled the process of making contact from infection: the makeContact events followed a uniform distribution as well, generated by oureselves, so we could state an expected coverage.

Funnily the implementation of all the specifications and property-tests is longer than the original implementation. Though thats not the point here: we showed how to implement a full specification of an ABS model as a property-based test and we succeeded! This is definitely a strong indication that our hypothesis that randomised property-based testing is a suitable tool for testing ABS is valid. With unit tests we would be quite lost here: even for the SIR model, it is hard to enumerate all possible interactions and cases but by stating invariants as properties and generating random test-cases we make sure they are checked.

We have not looked into more complex testing patterns like the synchronous agent-interactions of Sugarscape. We didn't look into testing full agent and interacting agent behaviour using property-tests due to its complexity which would justify a whole paper alone. Due to its inherent stateful nature with complex dependencies between valid states and agents actions we need a more sophisticated approach as outlined in \cite{de_vries_-depth_2019}, where the authors show how to build a meta-model and commands which allow to specify properties and valid state-transitions which can be generated automatically. We leave this for further research.

What is particularly powerful is that one has complete control and insight over the changed state before and after e.g. a function was called on an agent: thus it is very easy to check if the function just tested has changed the agent-state itself or the environment: the new environment is returned after running the agent and can be checked for equality of the initial one - if the environments are not the same, one simply lets the test fail. This behaviour is very hard to emulate in OOP because one can not exclude side-effect at compile time, which means that some implicit data-change might slip away unnoticed. In FP we get this for free.

TODO: with lazy evaluation we scratch on what is conveniently possible in established approaches to ABS: we can let the simulation run potentially forever as in the case of the infected agent and rely on the correctness of the implementation to terminate in finite step when consuming the potentially infinite stream.