\section{Property-Based Testing}
\label{sec:proptesting}

Property-based testing allows to formulate \textit{functional specifications} in code which then a property-based testing library tries to falsify by \textit{automatically} generating test-data, covering as much cases as possible. When a case is found for which the property fails, the library then reduces the test-data to its simplest form for which the test still fails e.g. shrinking a list to a smaller size. It is clear to see that this kind of testing is especially suited to ABS, because we can formulate specifications, meaning we describe \textit{what} to test instead of \textit{how} to test. Also the deductive nature of falsification in property-based testing suits very well the constructive and exploratory nature of ABS. Further, the automatic test-generation can make testing of large scenarios in ABS feasible because it does not require the programmer to specify all test-cases by hand, as is required in e.g. traditional unit tests.

Property-based testing was introduced in \cite{claessen_quickcheck_2000,claessen_testing_2002} where they authors present the QuickCheck library in Haskell, which tries to falsify the specifications by \textit{randomly} sampling the test space. We argue, that the stochastic sampling nature of this approach is particularly well suited to ABS, because it is itself almost always driven by stochastic events and randomness in the agents behaviour, thus this correlation should make it straight-forward to map ABS to property-testing. The main challenge when using QuickCheck, as will be shown later, is to write \textit{custom} test-data generators for agents and the environment which cover the space sufficiently enough to not miss out on important test-cases. According to the authors of QuickCheck \textit{"The major limitation is that there is no measurement of test coverage."} \cite{claessen_quickcheck_2000}. QuickCheck provides help to report the distribution of test-cases but still it could be the case that simple test-cases which would fail are never tested because of the stochastic nature of QuickCheck.

To give a rough idea on how property-based testing works in Haskell, we give a few examples of property-tests on lists, which are directly expressed as functions in Haskell. Such a function has to return a \textit{Bool} which indicates \textit{True} in case the test succeeds or \textit{False} if not and can take input arguments which data is automatically generated by QuickCheck. %Note that the first line of each function defines its name, its inputs (\textit{[Int]} is a list of integers) and the output which is the last type (\textit{Bool}). Note that the \textit{(++)} operator concatenates two lists, \textit{reverse} simply reverses a list.

\begin{HaskellCode}
-- concatenation operator (++) is associative
append_associative :: [Int] -> [Int] -> [Int] -> Bool
append_associative xs ys zs = (xs ++ ys) ++ zs == xs ++ (ys ++ zs)

-- reverse is distributive over concatenation (++)
-- xs and ys need to be swapped on the right-hand side!
reverse_distributive :: [Int] -> [Int] -> Bool
reverse_distributive xs ys = reverse (xs ++ ys) == reverse ys ++ reverse ys

-- the reverse of a reversed list is the original list
reverse_reverse :: [Int] -> Bool
reverse_reverse xs = reverse (reverse xs) == xs
\end{HaskellCode}

% POTENTIAL FOR SHORTENING
As a remedy for the potential sampling difficulties of QuickCheck, there exists also a deterministic property-testing library called SmallCheck \cite{runciman_smallcheck_2008}, which instead of randomly sampling the test-space, enumerates test-cases exhaustively up to some depth. It is based on two observations, derived from model-checking, that (1) \textit{"If a program fails to meet its specification in some cases, it almost always fails in some simple case"} and (2) \textit{"If a program does not fail in any simple case, it hardly ever fails in any case} \cite{runciman_smallcheck_2008}. This non-stochastic approach to property-based testing might be a complementary addition in some cases where the tests are of non-stochastic nature with a search-space  too large to test manually by unit tests but small enough to enumerate exhaustively. The main difficulty and weakness of using SmallCheck is to reduce the dimensionality of the test-case depth search to prevent combinatorial explosion, which would lead to exponential number of cases. Thus one can see QuickCheck and SmallCheck as complementary instead of in opposition to each other.
% POTENTIAL FOR SHORTENING
Note that in this thesis we only use QuickCheck due to the match of ABS stochastic nature and the random test generation. We refer to SmallCheck in cases where appropriate. Also note that we regard property-based testing as \textit{complementary} to unit-tests and not in opposition - we see it as an addition in the TDD process of developing an ABS.