\chapter{Stateful testing}
\label{ch:stateful_testing}

So far, our tests were stateless: only one computational step of an agent was considered by feeding a single event and ignoring the agent continuation. Also the events didn't contain any notion of time as they would carry within the queue. Feeding follow-up events into the continuation would make testing inherently stateful as we introduce history into the system. Such tests would allow to test the full life-cycle of one  agent or a full population.

In this chapter we will discuss how we can encode properties and specifications which require stateful testing. We define stateful testing here as: evolving a simulation state consisting of one or more agents over multiple events. Note that this also includes running the whole simulation.

\section{Agent invariants over time}
TODO

\section{SIR invariants over time}
By informally reasoning about the agent-specification and by realising that they are in fact a state-machine with a on-directional flow from Susceptible to Infected to Recovered, we can come up with a few invariants which have to hold for any SIR simulation run independent of the random-number stream and the population:

\begin{enumerate}
	\item Simulation time is monotonic increasing. For each event an output of the current number of S,I and R agents is generated together with the time when the events occurred. This event time can stay the same between steps, will eventually increase but must never decrease. Obviously this invariant is a fundamental assumption in most simulations: time advances into the future and does not flow backwards.
	
	\item The number of total agents $N$ stays constant in SIR. The SIR model does not specify the dynamic creation or removal of agents during simulation. This is in contrast to the Sugarscape where, depending on the model parameters, this can be very well the case.
	
	\item The number of \textit{Susceptible} agents $S$ is monotonic decreasing. Susceptile agents \textit{might} become infected, reducing the total number of susceptible agents but they can never increase because neither an infected nor recovered agent can go back to susceptible.
	
	\item The number of \textit{Recovered} agents $R$ is monotonic increasing. This is due to infected agents \textit{will} recover, leading to an increase of recovered agents but once the recovered state is reached, there is no escape from it.
	
	\item The number of \textit{Infected} agents respects the invariant of the equation $I = N - (S + R)$ for every step. This follows directly from the first property which says $N = S + I + R$.
\end{enumerate}

All of those properties are easily expressed directly in code (TODO: add a note in the benefits how extremely well this is encoded in functional programming).

\begin{HaskellCode}
sirInvariants :: Int -> [(Time, (Int, Int, Int))] -> Bool
sirInvariants n aos = timeInc && aConst && susDec && recInc && infInv
  where
    (ts, sirs)  = unzip aos
    (ss, _, rs) = unzip3 sirs

    -- 1. time is monotonic increasing
    timeInc = mono (<=)  ts
    -- 2. number of agents N stays constant in each step
    aConst = all agentCountInv sirs
    -- 3. number of susceptible S is monotonic decreasing
    susDec = mono (>=) ss
    -- 4. number of recovered R is monotonic increasing
    recInc = mono (<=)  rs
    -- 5. number of infected I = N - (S + R)
    infInv = all infectedInv sirs

    agentCountInv :: (Int, Int, Int) -> Bool
    agentCountInv (s,i,r) = s + i + r == n

    infectedInv :: (Int, Int, Int) -> Bool
    infectedInv (s,i,r) = i == n - (s + r)

    mono :: (Ord a, Num a) => (a -> a -> Bool) -> [a] -> Bool
    mono f xs = all (uncurry f) (pairs xs)

    pairs :: [a] -> [(a,a)]
    pairs xs = zip xs (tail xs)
\end{HaskellCode}

Putting this property into a QuickCheck test is straightforward:

\begin{HaskellCode}
prop_sir_simulation_invariants :: Property
prop_sir_simulation_invariants = property (do
  let cor = 5    -- beta, contact rate
      inf = 0.05 -- gamma, infectivitry
      ild = 15   -- delta, illness duration

  -- generate population with size of up to 1000
  ss <- resize 1000 (listOf genSIRState)
  -- total agent count
  let n = length ss

  -- run the SIR simulation with a new RNG for up to t = 150 
  ret <- genSimulationSIR ss cor inf ild (-1) 150
  -- check invariants and return result
  return (sirInvariants n ret)
\end{HaskellCode}

Unsurprisingly all 100 tests pass. Note that we put a time-limit of 150 on the simulations to run, meaning that if a simulation does not terminate before that limit, it will be terminated at $t=150$. This is actually not necessary because we can reason that the SIR simulation \textit{will always} reach an equilibrium in finite steps thus not requiring an actual time-limit - we discuss this more in-depth in Chapter \ref{ch:equilibrium_totality}.

\paragraph{Random Event Sampling}
An interesting question is whether or not these properties depend on correct interdependencies of events the agents send to each other in reaction to events they receive. Put in other words: do these invariants also hold under \textit{random event sampling}? To test this, instead of using the actual SIR implementation, which inserts the events generated by the agents into the event-queue, we wrote a new SIR kernel. It completely ignores the events generated by the agents and instead makes use of an infinite stream of random queue-elements from which it executes a given number, 100,000 in our case. Note that queue-elements contain a time-stamp, the receiver agent id and the actual event: the time-stamp is ensured to be increasing, to hold up the monotonic time property, the receiver agent id is drawn randomly from the constant list of all agents in the simulation and the actual event is generated completely randomly. As it turns out, the implementation of the agents ensure that the SIR properties are also invariant under \textit{random event sampling} - all tests pass.
