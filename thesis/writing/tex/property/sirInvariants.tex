\chapter{Testing model invariants}
\label{ch:sir_invariants}

The tests of the event-driven implementation in the previous chapter were stateless: only one computational step of an agent was considered by feeding a single event and ignoring the agent continuation. Also the events didn't contain any notion of time as they would carry within the queue. Feeding follow-up events into the continuation would make testing inherently stateful as we introduce history into the system. Such tests would allow to test the full life-cycle of one  agent or a full population.

In this chapter we will discuss how we can encode properties and specifications which require stateful testing. We define stateful testing here as: evolving a simulation state consisting of one or more agents over multiple events. Note that this also includes running the whole simulation. Note that we primarily focus on the event-driven implementation here unless noted otherwise.

We first show how we can encode actual laws of the underlying SIR model into properties and write property-tests in QuickCheck for them. We then employ random event-sampling to check whether these invariants also hold when ignoring the event-interdependencies between agents. Further, we compare both the event- and time-driven implementations with each other, giving an excellent use case for property-based testing in ABS. Finally we show how to verify both the time- and event-driven implementations against the original SD specification.

\section{Invariants in simulation dynamics}
\label{sec:prop_invariants_dynamics}
By informally reasoning about the agent specification and by realising that they are in fact a state-machine with a one-directional flow of \textit{Susceptible} $\rightarrow$ \textit{Infected} $\rightarrow$ \textit{Recovered}, we can come up with a few invariants which have to hold for any SIR simulation run, independent of the random-number stream and the population:

\begin{enumerate}
	\item Simulation time is monotonic increasing. %For each event an output of the current number of S,I and R agents is generated together with the time when the events occurred. 
	Each event carries a timestamp when it is scheduled. This timestamp may stay constant between multiple events but will eventually increase and must never decrease. Obviously this invariant is a fundamental assumption in most simulations: time advances into the future and does not flow backwards.
	
	\item The number of total agents $N$ stays constant. The SIR model does not specify the dynamic creation or removal of agents during simulation. This is in contrast to the Sugarscape where, depending on the model parameters, this can be very well the case.
	
	\item The number of \textit{Susceptible} agents $S$ is monotonic decreasing. Susceptile agents \textit{might} become infected, reducing the total number of susceptible agents but they can never increase because neither an infected nor recovered agent can go back to \textit{Susceptible}.
	
	\item The number of \textit{Recovered} agents $R$ is monotonic increasing. This is because infected agents \textit{will} recover, leading to an increase of recovered agents but once the recovered state is reached, there is no escape from it.
	
	\item The number of \textit{Infected} agents respects the invariant of the equation $I = N - (S + R)$ for every step. This follows directly from the first property which says $N = S + I + R$.
\end{enumerate}

\subsection{Encoding the invariants}
All of those properties are easily expressed directly in code and read like a formal specification due to the declarative nature of functional programming:

\begin{HaskellCode}
sirInvariants :: Int                    -- ^ N total number of agents
              -> [(Time,(Int,Int,Int))] -- ^ simulation output for each step/event: (Time, (S,I,R))
              -> Bool
sirInvariants n aos = timeInc && aConst && susDec && recInc && infInv
  where
    (ts, sirs)  = unzip aos
    (ss, _, rs) = unzip3 sirs

    -- 1. time is monotonic increasing
    timeInc = allPairs (<=) ts
    -- 2. number of agents N stays constant in each step
    aConst = all agentCountInv sirs
    -- 3. number of susceptible S is monotonic decreasing
    susDec = allPairs (>=) ss
    -- 4. number of recovered R is monotonic increasing
    recInc = allPairs (<=) rs
    -- 5. number of infected I = N - (S + R)
    infInv = all infectedInv sirs

    agentCountInv :: (Int,Int,Int) -> Bool
    agentCountInv (s,i,r) = s + i + r == n

    infectedInv :: (Int,Int,Int) -> Bool
    infectedInv (s,i,r) = i == n - (s + r)

    allPairs :: (Ord a, Num a) => (a -> a -> Bool) -> [a] -> Bool
    allPairs f xs = all (uncurry f) (pairs xs)

    pairs :: [a] -> [(a,a)]
    pairs xs = zip xs (tail xs)
\end{HaskellCode}

Putting this property into a QuickCheck test is straightforward. Note that we randomise the model parameters $\beta$ (contact rate), $\gamma$ (infectivity) and $\delta$ (illness duration) because the properties have to hold for all positive, finite model parameters.

\begin{HaskellCode}
prop_sir_invariants :: Positive Int    -- ^ beta, contact rate
                    -> Probability     -- ^ gamma, infectivity in range (0,1)
                    -> Positive Double -- ^ delta, illness duration
                    -> TimeRange       -- ^ random duration in range (0, 50)
                    -> [SIRState]      -- ^ population
                    -> Property
prop_sir_invariants 
    (Positive cor) (P inf) (Positive ild) (T t) as  = property (do
  -- total agent count
  let n = length ss
  -- run the SIR simulation with a new RNG 
  ret <- genSimulationSIR ss cor inf ild t
  -- check invariants and return result
  return (sirInvariants n ret)
\end{HaskellCode}

Due to the large sampling space, we increase the number of test cases to run to 10,000 and unsurprisingly all tests pass. Note that we put a random time-limit of (0,50) on the simulations to run, meaning that if a simulation does not terminate before that limit, it will be terminated at that random $t$. This is actually not necessary because we can reason that the SIR simulation \textit{will always} reach an equilibrium in finite steps thus not requiring an actual time-limit - we discuss this more in-depth in Appendix \ref{app:equilibrium_totality}.

\subsection{Random event sampling invariants}
An interesting question is whether or not these properties depend on correct interdependencies of events the agents send to each other in reaction to events they receive. Put in other words: do these invariants also hold under \textit{random event sampling}? To test this, instead of using the actual SIR implementation, which inserts the events generated by the agents into the event-queue, we wrote a new SIR kernel. It completely ignores the events generated by the agents and instead makes use of an infinite stream of random queue-elements from which it executes a given number, 100,000 in our case. Note that queue-elements contain a timestamp, the receiving agent id and the actual event: the timestamp is ensured to be increasing, to hold up the monotonic time property, the receiving agent id is drawn randomly from the constant list of all agents in the simulation and the actual event is randomly generated. As it turns out, the implementation of the agents ensure that the SIR properties are also invariant under \textit{random event sampling} - all tests pass.

\subsection{Time-driven invariants}
We can expect that the invariants above also hold for the time-driven implementation. The property-test is exactly the same, with the time-driven implementation running instead of the even-driven one. A big difference is that is not necessary to check the property of monotonic increasing time, as it is an invariant statically guaranteed by arrowized FRP through the Yampa implementation. Due to the fact that the flow of time is always implicitly forward and no time-variable is explicitly made accessible within the code, it is not possible to violate the forward flow of time.

When we ran the property-test we got a big surprise though: after a few test cases the property-test failed due to a violation of the invariants! After a little bit of investigation it became clear that the invariant \textit{(3) number of susceptible agents is monotonic decreasing} was violated: in the failing test case the number of susceptible agents is monotonic decreasing with the exception of one step where it \textit{increases} by 1 just to decrease by 1 in the next step. A coverage test reveals that this happens in about 66\% of 1,000 test cases.

The technicalities of the problem are highly involved and not provided in depth here. The source of the problem are the semantics of \textit{switch} and \textit{dpSwitch} which could lead to a delayed output of the agent state, leading to inconsistencies when feeding it back as environment in the next step. The solution is to delay the output of the susceptible agent by one step using \textit{iPre} as already shown in the original time-driven implementation of \ref{sec:timedriven_firststep}. This solves the problem and the property-test passes. 

%\medskip
%
%The source of the problem is the use of \textit{dpSwitch} in the implementation of the \textit{stepSimulation} function as shown in Chapter \ref{sec:timedriven_firststep}. The d in \textit{dpSwitch} stands for delayed observation, which means that the output of the switch at time of switching is the output of the \textit{old} signal functions \cite{courtney_yampa_2003}. Speaking more technically: \textit{dpSwitch} is non-strict in its switching event, which ultimately results in old signal functions being run on new output which were but produced by those old signal functions: the output of time-step t is only visible in the time-step t=t+dt but the signal functions at time-step t are actually one step ahead. This is particularly visible at $t = 0$ and $t = \Delta t$, where the outputs are the same but the signal functions are not: the one at $t = \Delta t$ has changed already.
%
%This has the desired effect that in the case of our SIR implementation, the population the agents see at time $t = t + \Delta t$ is the one from the previous step t, generated with the signal functions at time $t$. Due to the semantics of \textit{dpSwitch}, in the next switching event those signal functions from time $t$ are run again with the new input to produce the next output \textit{and} the new signal functions - in each step output is produced but due to the delay of \textit{dpSwitch} and the use of \textit{notYet}, we get this alternating behaviour.
%
%This leads to trouble if the very rare case happens when a susceptible agent makes the transition from susceptible to recovered within one time-step. This is indeed possible due to the semantics of \textit{switch}, which is employed to make the state-transitions. In case a switching event occurs, \textit{switch} runs the signal function into which was switched immediately, which makes it highly unlikely but possible, that the susceptible agent, which has just switched into infected  recovers immediately by making the \textit{switch} to recovered. Why does this violate the property then?
%
%\medskip
%
%Lets assume that at $t = t + \Delta t$ the agents receive as input the population from time $t$ which contains, say 42, susceptible agents. A susceptible agent then makes the highly unlikely transition to recovered, reducing the number of susceptible agents by 1 to 41. In the next step the old signal function is run again but with the new input, which is slightly different, thus leading to slightly different probabilities. A susceptible agent has become recovered, which reduces probabilities of a susceptible agent becoming infected. When now the old signal function is run, it leads to a different output due to different probabilities: the susceptible agent stays susceptible instead of becoming infected (and then recovering). 
%
%One way to solve this problem would be to use \textit{pSwitch}. It is the non-delayed, stricht version of \textit{dpSwitch}, which output at time of switching is the output of the \textit{new} signal functions. Using \textit{pSwitch} instead of \textit{dpSwitch} solves the problem because it will use the new signal functions in the second run because it is strict. Indeed, when using this, the property-test passes. This comes though at a high cost: due to \textit{pSwitch} strict semantics, which runs all the signal functions before and at time of switching, all agents are run twice in each step! This is clearly an unacceptable solution, especially because the time-driven approach already suffers severe performance problems. A more performant solution is to delay the susceptible agents output, as we have done already in \ref{sec:timedriven_firststep}. This solves the problem as well and the property-test passes. 

\section{Comparing time- and event-driven implementations}
%% THESE ARE NOTES TAKEN DURING TESTING, DON'T REMOVE, THEY EXPLAIN HOW WE ARRIVE AT THESE RESULTS
%---------------------------------------------------------------------------------------------------------------------------
%ON AVERAGE CONTACT RATE
%NOTE: it seems that with random parameters and normal population but t = 1.0 we reach a coverage around 35\% 
%NOTE: it seems that with random parameters and normal population and random t = 0-10 we reach a coverage around 57\% 
%
%FIXED CONTACT RATE
%NOTE: it seems that with random parameters and normal population but t = 1.0 we reach a coverage around 58\% 
%NOTE: it seems that with random parameters and normal population and random t = 0-10 we reach a coverage around 64\%
%---------------------------------------------------------------------------------------------------------------------------
%NOTE: it seems that fixed contact rate works better than average contact rate => macal was right after all...
%with t between 0 and 50 reaching up to 87\% !!
%---------------------------------------------------------------------------------------------------------------------------

Having two conceptually different implementations of the same model, an obvious questions we want to answer is whether they are producing the same dynamics or not. To be more precise, we need to answer the question whether both simulations produce the same distributions under random model parameters and simulation time. This is a perfect use case for QuickCheck as well and easily encoded into a property-test.

We generate random values for $\beta$ (contact rate), $\gamma$ (infectivity) and $\delta$ (illness duration) as well as a random population and a random duration to run the simulations for. Note that we restrict $\gamma$ to be drawn from the (0,1) range as it represents a probability; the random duration is drawn from the (0, 50) range to reduce the run-time duration to a reasonable amount without taking away the randomness.

Both simulation types are run with the same random parameters for 100 replications, collecting the output of the final step. The samples of these replications are then compared using a Mann-Whitney test with a 95\% confidence (p-value of 0.05). The reason for choosing this statistical test over a two sample t-test is that the Mann-Whitney test does not require the samples to be normally distributed. We know both from experimental observations and discussions in \cite{macal_agent-based_2010} that both implementations produce a bimodal distribution, thus we have to use a non-parametric test like Mann-Whitney to compare them.

%TODO: more discussion about this! this is the place where we can really introduce that. also macal only discusses event-driven and not time-driven, why can we then assume that time-driven also produces a bi-modal distribution? show some histograms of failing test cases with t-tests

We expect a high coverage of at least 90\%, which makes our assumption explicit that we expect both simulations to produce highly similar distributions despite their different underlying implementations.

\begin{HaskellCode}
prop_event_time_equal :: Positive Int    -- ^ beta, contact rate
                      -> Probability     -- ^ gamma, infectivity, within (0,1) range
                      -> Positive Double -- ^ delta, illness duration
                      -> TimeRange       -- ^ time to run within (0, 50) range
                      -> [SIRState]      -- ^ population 
                      -> Property
prop_event_time_equal
    (Positive cor) (P inf) (Positive ild) (T t) as = checkCoverage (do
  -- run 100 replications for time- and event-driven simulation
  (ssTime, isTime, rsTime)    <- unzip3 <$> genTimeSIRRepls 100 as cor inf ild t
  (ssEvent, isEvent, rsEvent) <- unzip3 <$> genEventSIRRepls 100 as cor inf ild t
  
  -- confidence of 95 for Mann Whitney test
  let p = 0.05
  -- perform statistical tests
  let ssTest = mannWhitneyTwoSample ssTime ssEvent p
      isTest = mannWhitneyTwoSample isTime isEvent p
      rsTest = mannWhitneyTwoSample rsTime rsEvent p

  let allPass = ssTest && isTest && rsTest 

  -- add the test to the coverage tests only if it passes.
  return 
    (cover 90 allPass "SIR implementations produce equal distributions" True)
\end{HaskellCode}

Indeed when running this test, enforcing QuickCheck to perform sequential statistical hypothesis testing with \textit{checkCoverage}, after 800 tests QuickCheck passes the test.

\begin{verbatim}
+++ OK, passed 800 tests 
    (90.4% SIR event- and time-driven produce equal distributions).
\end{verbatim}

This result shows that both implementations produce highly similar distributions although they are not exactly the same as the 10\% of failure shows. We will discuss this issue in a broader context in the next section.

\section{Testing the SIR model specification}
\label{sec:prop_sirspecs}
In the previous chapters and sections we have established the correctness of our event- and time-driven implementation up to our informal specification, we derived from the formal SD specification from Chapter \ref{sec:sir_model}. What we are lacking is a verification whether the implementations also match the formal SD specification or not. We aim at connecting the agent-based implementation to the SD specification, by formalising it into properties within a property-test. The SD specification can be given through the differential equations shown in Chapter \ref{sec:sir_model}, which we repeat here:

\begin{equation}
\begin{split}
\frac{\mathrm d S}{\mathrm d t} = -infectionRate \\
\frac{\mathrm d I}{\mathrm d t} = infectionRate - recoveryRate \\
\frac{\mathrm d R}{\mathrm d t} = recoveryRate 
\end{split}
\quad
\begin{split}
infectionRate = \frac{I \beta S \gamma}{N} \\
recoveryRate = \frac{I}{\delta} 
\end{split}
\end{equation}
\label{eq:sir_delta_rates}

Solving these equations is done by integrating over time. In the SD terminology, the integrals are called \textit{Stocks} and the values over which is integrated over time are called \textit{Flows}. At $t = 0$ a single agent is infected because if there wouldn't be any infected agents, the system would immediately reach equilibrium - this is also the formal definition of the steady state of the system: as soon as $I(t) = 0$ the system will not change any more.

\begin{align}
S(t) &= N - I(0) + \int_0^t -infectionRate\, \mathrm{d}t \\
I(0) &= 1 \\
I(t) &= \int_0^t infectionRate - recoveryRate\, \mathrm{d}t \\
R(t) &= \int_0^t recoveryRate\, \mathrm{d}t
\end{align}

\subsection{Deriving a property}
The goal is now to derive a property which connects those equations with our implementation. We have to be careful and realise a fundamental difference between the SD and ABS implementations: SD is deterministic and continuous, ABS is stochastic and discrete. Thus we cannot compare single runs but we can only compare averages: stated informally, the property we want to implement is that the ABS dynamics matches the SD ones \textit{on average}, independent of the finite population size, model parameters $\beta$ (contact rate), $\gamma$ (infectivity) and $\delta$ (illness duration) and duration of the simulation. To be able to compare averages, we run 100 replications of the ABS simulation with same parameters except a different random-number generator in each replication and collect the output of the final steps. We then run a two-sided t-test on the replication values with the expected values generated by an SD simulation.

\begin{HaskellCode}
compareSDToABS :: Int    -- ^ Initial number of susceptibles
               -> Int    -- ^ Initial number of infected
               -> Int    -- ^ Initial number of recovered
               -> [Int]  -- ^ Final number of susceptibles in replications
               -> [Int]  -- ^ Final number of infected in replications
               -> [Int]  -- ^ Final number of recovered in replications
               -> Int    -- ^ beta (contact rate)
               -> Double -- ^ gamma (infectivity)
               -> Double -- ^ delta (illness duration)
               -> Time   -- ^ duration of simulation
               -> Bool
compareSDToABS s0 r0 i0
               ss is rs
               beta gamma delta t = sTest && iTest && rTest
  where
    -- run SD simulation to get expected averages
    (s, i, r) = simulateSD s0 i0 r0 beta gamma delta t
    
    confidence = 0.95
    sTest = tTestSamples TwoTail s (1 - confidence) ss
    iTest = tTestSamples TwoTail i (1 - confidence) is
    rTest = tTestSamples TwoTail r (1 - confidence) rs
\end{HaskellCode}

The implementation of \textit{simulateSD} is discussed in-depth in Appendix \ref{app:sd_simulation}. We are very well aware that comparing the output against an SD simulation is dangerous: after all, why should be trust the SD implementation? As outlined in the Appendix \ref{app:sd_simulation}, great care has been taken to ensure the correctness: the formulas from the SIR specification are directly put into code, allowed by Yampas arrowized FRP which guarantees that at least that translation step is correct - we then only rely on a small enough sampling rate and the correctness of the Yampa library. The former one is very well in our reach and we pick a sufficiently small samply rate; the latter one is beyond our reach but we expect the library to me mature enough to be correct for our purposes.

\subsection{Implementing the test}
Implementing a property-test is straightforward. Here we give the implementation for the time-driven SIR implementation, the implementation for the event-driven SIR implementation is exactly the same with the exception of \textit{genTimeSIRRepls}. We again make use of the \textit{checkCoverage} feature of QuickCheck to get statistical robust results and expect that in 75\% of all test cases the SD and ABS dynamics match \textit{on average} - we discuss below why we chose to use a 75\% coverage. QuickCheck will run as many tests as necessary to reach a statistically robust result which either allows to reject or accept this hypothesis.

\begin{HaskellCode}
prop_sir_time_spec :: Positive Int    -- ^ beta, contact rate
                   -> Propability     -- ^ gamma, infectivity, within (0,1) range
                   -> Positive Double -- ^ delta, illness duration
                   -> TimeRange       -- ^ time to run within (0, 50) range
                   -> [SIRState]      -- ^ population
                   -> Property
prop_sir_time_spec 
    (Positive cor) (P inf) (Positive ild) (T t) as = checkCoverage (do
  -- get initial agent numbers
  let (s0,i0,r0) = aggregateSIRStates as
  -- run 100 replications of time-driven SIR implementation
  (ss, is, rs) <- unzip3 <$> genTimeSIRRepls 100 as cor inf ild t
  let prop = compareSDToABS s0 i0 r0 ss is rs cor inf ild t
  return $ cover 75 prop "SIR time-driven passes t-test with simulated SD" True
\end{HaskellCode}

\subsection{Running the test}
When running the tests for the time- and event-driven implementation, QuickCheck reports the following:

\begin{verbatim}
+++ OK, passed 400 tests 
    (85.2% SIR time-driven passes t-test with simulated SD).

+++ OK, passed 3200 tests 
    (74.84% SIR event-driven passes t-test with simulated SD).
\end{verbatim}

The results clearly show that in both cases we reach the expected 75\% of coverage: the distributions of the time- and event-driven implementations match the simulated SD dynamics to at least 75\%, in case of time-driven this is even substantially higher. Still, this result raises a few questions:

\begin{enumerate}
	\item Why does the performance of the time-driven implementation surpasses the event-driven one by more than 10\%?
	
	\item Why are we not reaching a far higher coverage beyond 90\% and why have we chosen 75\% in the first place? After all our initial assumption was that the time- and event-driven implementations are simply agent-based implementations of the SD model and thus their dynamics should generate the same distributions as the SD ones.	
\end{enumerate}

First of all, the results are a very strong indication that although both implementation techniques try to implement the same underlying model, they generate different distributions and are thus not \textit{statistically} equal. This was already established in Chapter \ref{ch:sir_invariants}, where we have compared the distributions of both simulations and found that although we reach 90\% similarity this means that they are still different in some cases. The results of this property-test reflect that as well and we argue that this is also the reason why we see different performance of each when compared to SD. 

An explanation why the time-driven approach seems to be closer to the SD dynamics is that in the event-driven approach we are dealing with discrete events, jumping from time to time instead of moving forward in time continuously as it happens conceptually in the time-driven approach. Time is also continuous in SD, thus it seems intuitively clear that a time-driven approach is closer to the SD implementation than the event-driven one - it seems valid to call our time-driven approach a continuous agent-based simulation approach. The implication is that depending on our intention, picking a time-driven or an event-driven implementation can and will make a statistical difference. If one is transferring an SD to an ABS model, one might consider to follow the time-driven approach as it seems to come much closer to the SD dynamics than the event-driven approach.

\medskip

The reason that we are not reaching a coverage level up to and beyond 90\% is rooted in the fundamental difference between SD and ABS: due to ABS' stochastic nature, its dynamics cannot match an SD exactly because it generates a \textit{distribution} whereas the SD is deterministic. This enables ABS to explore and reveal paths which are not possible in deterministic SD. In the case of the SIR model, such an alternative path would be the immediate recovery of the single infected agent at the beginning without infecting any other agent. This is not possible in the SD case: in case there is 1 infected agent, the whole epidemic will unfold.

The difficulty of comparing dynamics between SD and ABS and the impracticality to compare them \textit{exactly} was shown by \cite{macal_agent-based_2010} in the case of the SIR model, where the authors showed that it generates a bimodal distribution. Further, the authors report that a 70\% envelope contains both the results of the SD and ABS implementation which is the reason why we chose a a 75\% coverage as our initial guess, which has turned out to work well and is in accordance with the results of \cite{macal_agent-based_2010}. %Indeed, that is also supported by our observations. When looking at the samples of failed t-tests by plotting them in a histogram, it shows clearly that the values exhibit strong outliers, arriving at a skewed / fat tailed / bimodal histogram. 

\medskip

The question which remains is whether it actually even makes sense to compare the approaches to SD or even amongst each other - after all they can be seen as fundamentally different approaches. We can argue that they are qualitatively equal as \cite{figueredo_comparing_2014} has already emphasised in a different study on comparing ABS and SD: although dynamics of ABS models are statistically different from SD ones, they look similar. The main difference is that ABS can contribute additional insight through revealing extra patterns due to its stochasticity, something not possible with SD. Thus in the end we simply have to accept that the respective coverage ratios are probably the closest we can get and that this is also the closest we can get in terms of validating our implementations against the original SD specification.

%% THESE ARE NOTES TAKEN DURING TESTING, DON'T REMOVE, THEY EXPLAIN HOW WE ARRIVE AT THESE RESULTS
%---------------------------------------------------------------------------------------------------------------------------
%ON AVERAGE CONTACT RATE
%NOTE: it seems that with random parameters and normal population but t = 1.0 we reach a coverage around 27\% for EVENT-driven
%NOTE: it seems that with random parameters and normal population AND random time 0-10 we reach a coverage around 38\% for EVENT-driven
%
%FIXED CONTACT RATE
%NOTE: it seems that with random parameters and normal population but t = 1.0 we reach a coverage around 38\% for EVENT-driven
%NOTE: it seems that with random parameters and normal population AND random time 0-10 we reach a coverage around 48\% for EVENT-driven
%
%TIME-DRIVEN
%NOTE: it seems that with random parameters and normal population but t = 1.0 we reach a coverage around 70\% for TIME-driven
%NOTE: it seems that with random parameters and normal population AND random time 0-10 we reach a coverage up to 85\% for TIME-driven
%---------------------------------------------------------------------------------------------------------------------------
%NOTE: it seems that fixed contact rate works better than average contact rate in event-driven => macal was right after all...
%with t between 0 and 50 we reach 73\%
%---------------------------------------------------------------------------------------------------------------------------
% NO LONGER TRUE
%The event-driven approach clearly fails to reach a similar percentage: the differences between the the dynamics generated by the event-driven and the SD seem to be statistically significant. This result is consistent with the one from Chapter \ref{ch:sir_invariants}, where showed that the distribution of the dynamics generated by the event- and time-driven approach vary from each other considerably, it would have been surprising if they both show a similar performance when compared with the SD approach. 
%In our implementation we follow the idea as presented in \cite{macal_agent-based_2010}. Their specification states that in each time-step of 1 time-unit the susceptible make contact with a fixed number (contact rate) of other agents. This is a valid ABS model but it deviates to come closer to the SD as it fixes the number of contacts instead of making contact \textit{on average}. In SD and the time-driven approach, these are all averages and thus we are drawing from the exponential distribution in the time-driven case. This is not the case in our event-drive implementation, following \cite{macal_agent-based_2010}. If we change this behaviour to incorporate the exponential distribution we get a higher match to the SD dynamics of 37\% which is still a far cry form the time-driven approach but it at least indicates that this is a viable option to push the results further towards the SD dynamics. 

\section{Discussion}
In this chapter we have shown how to encode properties about simulation dynamics, generated by executing agents over time. This allowed to encode actual laws of the underlying SIR model in code and check them under random model parameters.

In the case of the time-driven implementation we saw that our initial assumption, that the invariants will hold for this implementation as well was wrong: QuickCheck revealed a \textit{very} subtle bug in our implementation. Although the probability of this bug is very low, QuickCheck found it due to its random testing nature. This is another \textit{strong} evidence, that random property-based testing is an \textit{excellent} approach for testing Agent-Based Simulations. On the other hand, this bug revealed the difficulties in getting the subtle semantics of FRP right to implement pure functional ABS. This is a strong case that in general an event-driven approach should be preferred, which is also much faster and also not subject to the sampling issues discussed in Chapter \ref{sec:timedriven_firststep}.

Further, we showed that property-based testing also allows to compare two conceptually different implementations of the same underlying model with each other. This is indeed a perfect use case for property-based testing as it compares whole distributions and not only single runs using unit tests, making this another strong case for the use of property-based testing in ABS.

Finally, after having shown in previous chapters, that individual agent behaviour is correct up to some specification, in this chapter we also focused on validating the dynamics of the simulation with the original SD specification. By using QuickCheck, we showed how to connect both ABS implementations to the SD specification by deriving a property, based on the SD specification. This property is directly expressed in code and tested through generating random test cases with random agent populations and random model parameters. 

Although our initial idea of matching the ABS implementation to the SD specifications has not worked out in an exact way, we still showed a way of formalizing and expressing these relations in code and testing them using QuickCheck. The results showed that the ABS implementation comes close to the original SD specification but does not match it exactly - it is indeed richer in its dynamics as \cite{figueredo_comparing_2014,macal_agent-based_2010} have already shown. Our approach might work out better for a different model, which has a better behaved underlying specification than the bimodal SIR.
