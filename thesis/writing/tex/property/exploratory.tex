\chapter{Verifying an exploratory model: \\ Hypotheses in Sugarscape}
\label{ch:prop_exploratory}
In this chapter we look at how property-based testing can be made of use to verify the \textit{exploratory} Sugarscape model \cite{epstein_growing_1996} as already introduced in Chapter \ref{sec:sugarscape}. Whereas in the previous chapter on testing the explanatory SIR case-study we had an analytical solution, the fundamental difference in the exploratory Sugarscape model is that none such analytical solutions exist. This raises the question, which properties we can actually test in such a mode.

The answer lies in the very nature of exploratory models: they exist to explore and understand phenomena of the real world. Researchers come up with a model to explain the phenomena and then (hopefully) come up with a few questions and  \textit{hypotheses} about the emergent properties. The actual simulation is then used to test and refine the hypotheses. Indeed, descriptions, assumptions and hypotheses of varying formal degree are abound in the Sugarscape model. Examples are: \textit{the carrying capacity becomes stable after 100 steps; when agents trade with each other, after 1000 steps the standard deviation of trading prices is less than 0.05; when there are cultures, after 2700 steps either one culture dominates the other or both are equally present}. 

We show how to use property-testing to formalise and check such hypotheses. For this purpose we undertook a full \textit{verification} of our implementation \footnote{The code can be accessed freely from \url{https://github.com/thalerjonathan/phd/tree/master/public/towards/SugarScape/sequential}} from Chapter \ref{sec:sugarscape}. We validated it against the book \cite{epstein_growing_1996} and a NetLogo implementation \cite{weaver_replicating_2009} \footnote{\url{https://www2.le.ac.uk/departments/interdisciplinary-science/research/replicating-sugarscape}}.

The property we test for is whether \textit{the emergent property / hypothesis under test is stable under varying random-number seeds} or not. Put another way, we let QuickCheck generate random number streams and require that the tests all pass with arbitrary random number streams. It is very likely that we run into the same problem as in the previous chapter on SIR testing: most of the tests might pass but some might fail for same reasons. By using the \textit{maxFailPercentage} argument, we can formally specify how strict we want the hypothesis to hold and still succeed in cases where there is a low percentage of failure.

%Unfortunately this revealed that this property doesn't hold for all emergent properties. The problem is that QuickCheck generates by default 100 test-cases for each property-test where all need to pass for the whole property-test to pass - this wasn't the case, where most of the 100 test-cases passed but unfortunately not all. Thus in this case a different approach is required: instead of requiring \textit{every} test to pass we require that \textit{most} tests pass, which can be achieved using a t-test with a confidence interval of e.g. 95\%. This means we won't use QuickCheck anymore and resort to a normal unit-test where we run the simulation 100 times with different random number streams each time and then performing a t-test with a 95\% confidence interval. Note that we are now technically speaking of a unit-test but conceptually it is still a property-test.

We have implemented property-tests for the following hypotheses

\begin{enumerate}
	\item Cultural Dynamics -
	\item Disease Dynamics All Recover -
	\item Disease Dynamics Minority Recover -
	\item Inheritance Gini -
	\item Trading Dynamics -
	\item Terracing -
	\item Carrying Capacity -
	\item Wealth Distribution -
\end{enumerate}
