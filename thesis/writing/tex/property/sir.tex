\section{Case Study I: SIR}
\label{sec:case_SIR}
As first use-case we discuss property-based testing for the \textit{explanatory} agent-based SIR model as introduced in Chapter \ref{sec:sir_model}. 

Our goal is to connect the agent-based implementation to the SD equations, which we repeat here:

\begin{equation}
\begin{split}
\frac{\mathrm d S}{\mathrm d t} = -infectionRate \\
\frac{\mathrm d I}{\mathrm d t} = infectionRate - recoveryRate \\
\frac{\mathrm d R}{\mathrm d t} = recoveryRate 
\end{split}
\quad
\begin{split}
infectionRate = \frac{I \beta S \gamma}{N} \\
recoveryRate = \frac{I}{\delta} 
\end{split}
\end{equation}

The idea is to generate random population mixes and compute the theoretical infection- and recovery rate. Then the simulation is run for 1 time unit to get the numbers of S, I and R. We compare the differences to the expected infection and recovery rate. The distribution should average at the expected theoretical means. TODO: use Rand approach to guarantee uncorrelated RNGS. Note that we need to run the whole test multiple times with different rngs because it is a  

can we really do it? is it really possible to compare SD and agent-based simulation? 

TODO: a bit more than 1000 seconds for 100 tests

TODO: can use this test to derive an optimally small dt: the lowest dt for which all tests go through. start with 1.0 and always half it until all tests go through

\subsection{TODO: Stateful testing of event-driven SIR}