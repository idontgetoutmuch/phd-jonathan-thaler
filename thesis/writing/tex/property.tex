\chapter*{} %Property-Based Testing in pure functional ABS
\label{ch:property}

TODO WORDING: don't use replications in terms of QuickCheck! we test properties with random test-cases
TODO IMPROVE EXPLANATION: more discussion of general concepts, otherwise it reads too much like a technical report.

When implementing an Agent-Based Simulation (ABS) it is of fundamental importance that the implementation is correct up to some specification and that this specification matches the real world in some way. This process is called verification and validation (V\&V), where \textit{validation} is the process of ensuring that a model or specification is sufficiently accurate for the purpose at hand whereas \textit{verification} is the process of ensuring that the model design has been transformed into a computer model with sufficient accuracy \cite{robinson_simulation:_2014}. In other words, validation determines if we are we building the \textit{right model}, and verification if we are building the \textit{model right} up to some specification \cite{balci_verification_1998}.

% there is no general validity, an approach is TDD: V&V particularly difficult in ABS
One can argue that ABS should require more rigorous programming standards than other computer simulations \cite{polhill_ghost_2005}. Because researchers in ABS look for an emergent behaviour in the dynamics of the simulation, they are always tempted to look for some surprising behaviour and expect something unexpected from their simulation. 
Also, due to ABS' \textit{constructive / exploratory} nature \cite{epstein_chapter_2006, epstein_generative_2012}, there exists some uncertainty about the dynamics the simulation will produce before running it. The authors \cite{ormerod_validation_2006} see the current process of building ABS as a discovery process where models of an ABS often lack an analytical solution in general, which makes verification much harder if there is no such solution. Thus it is often very difficult to judge whether an unexpected outcome can be attributed to the model or has in fact its roots in a subtle programming error \cite{galan_errors_2009}.

In general this implies that it is not possible to prove that a model is valid in general but that the best we can do is to \textit{raise the confidence} in the correctness of the simulation. Therefore, the process of V\&V is not the proof that a model is correct but the \textit{process} of trying to show that the model is \textit{not incorrect}. The more checks one carries out which show that it is not incorrect, the more confidence we can place on the models validity. To tackle such a problem in software, software engineers have developed the concept of test-driven development (TDD).

Test-Driven Development (TDD) was popularised in the early 00s by Kent Beck \cite{beck_test_2002} as a way to a more agile approach to software-engineering, where instead of doing each step (requirements, implementation, testing,...) as separated from each other, all of them are combined in shorter cycles. Put shortly, in TDD tests are written for each feature before actually implementing it, then the feature is fully implemented and the tests for it should pass. This cycle is repeated until the implementation of all requirements has finished. Traditionally TDD relies on so called unit-tests which can be understood as a piece of code which when run isolated, tests some functionality of an implementation. Thus we can say that test-driven development in general and unit testing together with code-coverage in particular, guarantee the correctness of an implementation to some informal degree, which has been proven to be sufficiently enough through years of practice in the software industry all over the world. 

\medskip

The work \cite{collier_test-driven_2013} was the first to discusses how to apply TDD to ABS, using unit testing to verify the correctness of the implementation up to a certain level. They show how to implement unit-tests within the RePast Framework \cite{north_complex_2013} and make the important point that such a software needs to be designed to be sufficiently modular otherwise testing becomes too cumbersome and involves too many parts. The paper \cite{asta_investigation_2014} discusses a similar approach to DES in the AnyLogic software toolkit. 

The paper \cite{onggo_test-driven_2016} proposes Test Driven Simulation Modelling (TDSM) which combines techniques from TDD to simulation modelling. The authors present a case study for maritime search-operations where they employ ABS. They emphasise that simulation modelling is an iterative process, where changes are made to existing parts, making a TDD approach to simulation modelling a good match. They present how to validate their model against analytical solutions from theory using unit-tests by running the whole simulation within a unit-test and then perform a statistical comparison against a formal specification. %This approach will become of importance later on in our SIR and Sugarscape case studies.

The paper \cite{gurcan_generic_2013} gives an in-depth and detailed overview over verification, validation and testing of agent-based models and simulations and proposes a generic framework for it. The authors present a generic UML class model for their framework which they then implement in the two ABS frameworks RePast and MASON. Both of them are implemented in Java and the authors provide a detailed description how their generic testing framework architecture works and how it utilises unit testing with JUnit to run automated tests. To demonstrate their framework they provide also a case study of an agent-base simulation of synaptic connectivity where they provide an in-depth explanation of their levels of test together with code.

\medskip

% the gap
Although it would be interesting to see how we can apply unit testing to our approach, it is straight forward, nothing new and does not constitute unique research. Thus, in this chapter we introduce an additional technique for TDD: \textit{property-based testing}, which can be seen as complementary to unit testing. Property-based testing has its origins \cite{claessen_quickcheck_2000,claessen_testing_2002,runciman_smallcheck_2008} in Haskell, where it was first conceived and implemented. It has been successfully used for testing Haskell code for years and also been proven to be useful in the industry \cite{hughes_quickcheck_2007}. We show and discuss how this technique can be applied to test pure functional ABS implementations. To our best knowledge property-based testing has never been looked at in the context of ABS and this thesis is the first one to do so.

\medskip

The main idea of property-based testing is to express model-specifications and laws directly in code and test them through \textit{automated} and \textit{randomised} test-data generation. Thus one hypothesis of this thesis is that due to ABS \textit{stochastic} and \textit{exploratory / generative / constructive } nature, property-based testing is a natural fit for testing ABS in general and pure functional ABS implementations in particular. It thus should pose a valuable addition to the already existing testing methods in this field, worth exploring. To substantiate and test our hypothesis, we present two case-studies. First, the agent-based SIR model as introduced in Chapter \ref{sec:sir_model}, which is of explanatory nature, where we show how to express formal model-specifications in property-tests. Second, the SugarScape model as introduced in Chapter \ref{sec:sugarscape}, which is of exploratory nature, where we show how to express hypotheses in property-tests and how to property-test agent functionality. Note that we only focus on ABS specific testing in the form of model verification, individual agent behaviour and agent interaction testing. We don't look into obvious applications of QuickCheck like testing boundary-checks of the environment, helper functions of agents,... as although they are used within ABS, there is nothing new in testing them with QuickCheck.
%Again, we emphasise that we see it as an addition to TDD, where it works in combination with unit testing to verify and validate a simulation to increase the confidence in its correctness. % and is a useful tool for expressing regression tests.

\medskip

Note that property-based testing has a close connection to model-checking \cite{mcmillan_symbolic_1993}, where properties of a system are proved in a formal way. The important difference is that the checking happens directly on code and not on the abstract, formal model, thus one can say that it combines model-checking and unit testing, embedding it directly in the software-development and TDD process without an intermediary step. We hypothesise that adding it to the already existing testing methods in the field of ABS is of substantial value as it allows to cover a much wider range of test-cases due to automatic data generation. This can be used in two ways: to verify an implementation against a formal specification and to test hypotheses about an implemented simulation. This puts property-based testing on the same level as agent- and system testing, where not technical implementation details of e.g. agents are checked like in unit-tests but their individual complete behaviour and the system behaviour as a whole.

The work \cite{onggo_test-driven_2016} explicitly mentions the problem of test coverage which would often require to write a large number of tests manually to cover the parameter ranges sufficiently enough - property-based testing addresses exactly this problem by \textit{automating} the test-data generation. Note that this is closely related to data-generators \cite{gurcan_generic_2013}, load generators and random testing \cite{burnstein_practical_2010}. Property-based testing though goes one step further by integrating this into a specification language directly into code, emphasising a declarative approach and pushing the generators behind the scenes, making them transparent and focusing on the specification rather than on the data-generation. 

\section*{Property-Based Testing}
\label{sec:proptesting}

Property-based testing allows to formulate \textit{functional specifications} in code which then a property-based testing library tries to falsify by \textit{automatically} generating test-data, covering as much cases as possible. When a case is found for which the property fails, the library then reduces the test-data to its simplest form for which the test still fails e.g. shrinking a list to a smaller size. It is clear to see that this kind of testing is especially suited to ABS, because we can formulate specifications, meaning we describe \textit{what} to test instead of \textit{how} to test. Also the deductive nature of falsification in property-based testing suits very well the constructive and exploratory nature of ABS. Further, the automatic test-generation can make testing of large scenarios in ABS feasible because it does not require the programmer to specify all test-cases by hand, as is required in e.g. traditional unit tests.

Property-based testing was introduced in \cite{claessen_quickcheck_2000,claessen_testing_2002} where the authors present the QuickCheck library in Haskell, which tries to falsify the specifications by \textit{randomly} sampling the test space. We argue, that the stochastic sampling nature of this approach is particularly well suited to ABS, because it is itself almost always driven by stochastic events and randomness in the agents behaviour, thus this correlation should make it straightforward to map ABS to property-testing.
%The main challenge when using QuickCheck, as will be shown later, is to write \textit{custom} test-data generators for agents and the environment which cover the space sufficiently enough to not miss out on important test-cases.
According to the authors of QuickCheck \textit{"The major limitation is that there is no measurement of test coverage."} \cite{claessen_quickcheck_2000}. Although QuickCheck provides help to report the distribution of test-cases it is not able to measure the coverage of tests in general. This could lead to the case that test-cases which would fail are never tested because of the stochastic nature of QuickCheck. Fortunately, the library provides mechanisms for the developer to measure coverage in specific test-cases where the data and its (expected) distribution is known to the developer. This is a powerful tool for testing randomness in ABS as will be shown in subsequent chapters.

\medskip

As a remedy for the potential coverage problems of QuickCheck, there exists also a deterministic property-testing library called SmallCheck \cite{runciman_smallcheck_2008}, which instead of randomly sampling the test-space, enumerates test-cases exhaustively up to some depth. It is based on two observations, derived from model-checking, that (1) \textit{"If a program fails to meet its specification in some cases, it almost always fails in some simple case"} and (2) \textit{"If a program does not fail in any simple case, it hardly ever fails in any case} \cite{runciman_smallcheck_2008}. This non-stochastic approach to property-based testing might be a complementary addition in some cases where the tests are of non-stochastic nature with a search-space  too large to test manually by unit tests but small enough to enumerate exhaustively. The main difficulty and weakness of using SmallCheck is to reduce the dimensionality of the test-case depth search to prevent combinatorial explosion, which would lead to exponential number of cases. Thus one can see QuickCheck and SmallCheck as complementary instead of in opposition to each other.

\subsection*{A brief overview of QuickCheck}
To give a rough idea on how property-based testing works in Haskell, we give a few examples of property-tests on lists, which are directly expressed as functions in Haskell. Such a function has to return a \textit{Bool} which indicates \textit{True} in case the test succeeds or \textit{False} if not and can take input arguments which data is automatically generated by QuickCheck.

\begin{HaskellCode}
-- concatenation operator (++) is associative
append_associative :: [Int] -> [Int] -> [Int] -> Bool
append_associative xs ys zs = (xs ++ ys) ++ zs == xs ++ (ys ++ zs)

-- the reverse of a reversed list is the original list
reverse_reverse :: [Int] -> Bool
reverse_reverse xs = reverse (reverse xs) == xs

-- reverse is distributive over concatenation (++)
-- this test fails for explanatory reasons, for a correct 
-- property xs and ys need to be swapped on the right-hand side!
reverse_distributive :: [Int] -> [Int] -> Bool
reverse_distributive xs ys = reverse (xs ++ ys) == reverse xs ++ reverse ys

-- running the tests
main :: IO ()
main = do
  quickCheck append_associative
  quickCheck reverse_reverse
  quickCheck reverse_distributive
\end{HaskellCode}

When we run the tests using \textit{main}, we get the following output:

\begin{verbatim}
+++ OK, passed 100 tests.
+++ OK, passed 100 tests.
*** Failed! Falsifiable (after 5 tests and 6 shrinks):    
[0]
[1]
\end{verbatim}

We see that QuickCheck generates 100 test-cases for each property-test and it does this by generating random data for the input arguments. Note that we have not specified any data for our input arguments; QuickCheck is able to provide a suitable data-generator through type-inference: for lists and all the existing Haskell types like Int there exist custom generators already.

QuickCheck generates 100 test-cases by default and requires all to pass - if there is a test-case which fails, the overall property-test fails and QuickCheck shrinks the input to a minimal size for which the case still fails and reports it as a counter example. This is the case in the last property-test \textit{reverse\_distributive} which is wrong as \textit{xs} and \textit{ys} need to be swapped on the right-hand side. In this run, QuickCheck found a counter-example to the property after 5 tests and applied 6 shrinks to find the minimal failing example of \textit{xs = [0]} and \textit{ys = [1]}. If we swap \textit{xs} and \textit{ys}, the property-test passes 100 test-cases just like the other two did. Note that it is possible to configure QuickCheck to generate more or less random test-cases, which can be used to increase the coverage if the sampling space is quite large - this will become useful later.

\subsubsection*{Properties and Generators}
TODO: property
TODO: label
TODO: ==>
TODO: generators

\subsubsection*{Coverage}
TODO: cover with checkCoverage


\section*{Testing ABS implementations}
\label{sec:testingABS}
TODO: this is a rather weak section, polish it by writing more general about it and how we can use property-based testing for it. if i cannot come up with something more substantial simply scrap it

Generally we need to distinguish between two types of testing / verification in ABS.

\begin{enumerate}
	\item Testing / verification of models for which we have real-world data or an analytical solution which can act as a ground-truth - examples for such models are the SIR model, stock-market simulations, social simulations of all kind.
	\item Testing / verification of models which are of exploratory nature, inspired by real-world phenomena but for which no ground-truth per se exists - examples for such models is the Sugarscape \cite{epstein_growing_1996} or Agent\_Zero model \cite{epstein_agent_zero:_2014}.
\end{enumerate}

The baseline is that either one has an analytical model as the foundation of an agent-based model or one does not. In the former case, e.g. the SIR model, one can very easily validate the dynamics generated by the simulation to the one generated by the analytical solution through System Dynamics. In the latter case one has basically no idea or description of the emergent behaviour of the system prior to its execution e.g. SugarScape. In this case it is important to have some hypothesis about the emergent property / dynamics. The question is how verification / validation works in this setting as there is no formal description of the expected behaviour: we don't have a ground-truth against which we can compare our simulation dynamics.

One distinguishes between black-box and white-box verification where in white-box verification one looks directly at code and reasons about it whereas in black-box verification one generally feeds input to the software / functions / methods and compares it to expected output. Black-box verification is our primary concern in this chapter as property-based testing is an instance of black-box verification. In the case of ABS we have the following levels of black-box tests \cite{nguyen_testing_2011}: %%Although the work on TDD is scarce in ABS, there exists quite some research on applying TDD and unit testing to multi-agent systems (MAS). Although MAS is a different discipline than ABS, the latter one has derived many technical concepts from the former one, thus testing concepts applied to MAS might also be applicable to ABS. The paper \cite{nguyen_testing_2011} is a survey of testing in MAS. It distinguishes between unit-tests of parts that make up an agent, agent tests which test the combined functionality of parts that make up an agent, integration tests which test the interaction of agents within an environment and observe emergent behaviour, system tests which test the MAS as a system running at the target environment and acceptance test in which stakeholders verify that the software meets their goal. Although not all ABS simulations need acceptance and system tests, still this classification gives a good direction and can be directly transferred to ABS. 
\begin{enumerate}
	\item Isolated and interacting agent behaviour parts - test the individual parts which make up the agent behaviour under given inputs. Also test if interaction between agents are correct. For this we can use traditional unit-tests as shown by \cite{collier_test-driven_2013} and also property-based testing as we will show in the case studies.
	\item Simulation dynamics - compare emergent dynamics of the ABS as a whole under given inputs to an analytical solution or real-world dynamics in case there exists some, using statistical tests. We see this type of tests conceptually as property-tests as well because we are testing properties of the model / simulation as we will see in the case-studies. Technically speaking we can use both unit and property-based tests to implement them - conceptually they are property-tests.
	\item Hypotheses - test whether hypotheses about the model are valid or invalid. This is very similar to the previous point but without comparing it to analytical solutions or real-world dynamics but only to some hypothetical values.
\end{enumerate}

TODO: make it clear that we focus on the event-driven SIR implemention only. why? because of its more interesting nature and because its concepts are also applicable to the sugarscape as well. the time-driven implementation will be shortly discussed in releveant sections.

There is a strong relation between property-based tests and dependent types: in property-based testing we express specifications / properties / laws in code and test their invariance at run-time by random sampling the space. In dependent-types it is possible to express such properties already statically in types. This is the subject of the next part of the thesis which tries to move towards dependent types in ABS.

We found property-based testing particularly well suited for ABS firstly due to ABS stochastic nature and second because we can formulate specifications, meaning we describe \textit{what} to test instead of \textit{how} to test. Also the deductive nature of falsification in property-based testing suits very well the constructive and often exploratory nature of ABS. 

Although property-based testing has its origins in Haskell, similar libraries have been developed for other languages e.g. Java, Pyhton, C++ as well and we hope that our research has sparked an interest in applying property-based testing to the established object-oriented languages in ABS as well.

%\chapter{Testing time-driven agents}

TODO: consider removing it because it is not very interesting and does not make a good point of property-based testing because not much randomness involved. Just mention it in the tests of duration / make contact

TODO WEAK ARGUMENT: discussion of time-delta is completely missing, but its a very important concept here! investigate it!

In this chapter we present how QuickCheck can be used to test time-driven agents from the time-driven SIR implementation (TODO ref to chapter). More specifically, we show how to test isolated agent behaviour by deriving invariant properties, encoding them directly in code and running them. 
In general, testing isolated agent behaviour in the time-driven approach is quiet straightforward as the interface surface is quite small, leading to not much options what you can do with them. Basically the only options we have is to run a given agent for a given number of steps with a given time-delta, population and model parameters. Lets start with the \textit{susceptible} agent.

\section{Susceptible agent}
The main concept behind the susceptible agent is, that it might become infected, depending on its contact rate, the infectivity of the disease and the number of infected agents in the population. Let's derive a formula for the probability of a \textit{susceptible} agent to become infected given the above mentioned parameters. This probability will then act as an invariant for a QuickCheck test into which we encode this property. Note that we are talking here about probabilities - the challenge will be to encode them into a hard PASS/FAIL property-test.

\subsection{Deriving a property}
We assume the following parameters:

\begin{enumerate}
	\item The contact rate $\beta$. Per time-unit a \textit{susceptible} agent makes \textit{on average} contact with $\beta$ agents from the population. The distribution follows the exponential distribution with $\lambda = \frac{1}{\beta}$.
	\item The infectivity of the disease $\gamma$. If a \textit{susceptible} agents get into contact with an \textit{infected} agent, it will become infected with a uniform probability of $\gamma$. 
	\item The population size $N$.
	\item The number of \textit{infected} agents $I$ in the population.
\end{enumerate}

TODO UNCLEAR ARGUMENT: it is unclear why does CDF works here? it seems to be arbitrary! CDF returns the probability to draw a random variable X with a given value x or less, from the distribution. for example if we have an exponential distribution with a given lambda, then the CDF(x) gives the probability X of drawing values less or equal x.

First we need to know the probability of an agent making contact with $\beta$ agents per time-unit. Note that this is \textit{not} the mean of the exponential distribution but we need to use the cumulative distribution (CDF) function to get this value. We define this value as $p_{cont} = CDF_{e{\frac{1}{\beta}}} (\beta)$ which is the probability of the CDF of the exponential distribution with $\lambda = \frac{1}{\beta}$ and $x = \beta$.
Next we need to incorporate the probability of an agent getting into contact with an infected agent which is simply the ratio of the number of infected agents to total population size: $ir = \frac{I}{N}$. Finally to arrive at the total probability we multiply these probabilities including the infectivity $\gamma$, arriving at the following formula for a susceptible agent to become infected \textit{on average}: $Sus_{inf} = p_{cont} * \frac{I}{N} * \gamma$.

\subsection{Encoding the property}
Now lets encode this into a property-test. The encoding of the probabilities is straightforward: we assume fixed parameters for $\beta$, $\gamma$ and the population and simply encode the formulas into Haskell code. Then we run a random \textit{susceptible} agent for 1 time-unit with the given parameters using a \textit{Generator} which returns true in case the agent got infected. 

The question is now: how do we check the agent got infected with the computed probability \textit{on average}? We use the coverage features of QuickCheck. The library offers the \textit{cover :: Double $\rightarrow$ Bool $\rightarrow$ String $\rightarrow$ property} function, which works as follows: The first parameter specifies the expected percentage of test-cases belonging to the class which have to succeed, the second parameter indicates if a test-case belongs to the class, the third parameter is a label for the class and the fourth parameter is the test indicating success or failure. We pass the computed probability as the expected percentage and make the test-case belong to the class if the agent got infected. We provide a label and we make \textit{all} tests succeed. The later might be weird, after all, why should all tests succeed? We want to have some measure of failure. QuickCheck provides the \textit{checkCoverage} function, which checks the coverage requirements defined with a \textit{cover} and uses a statistically sound test to check if the encoded cover assumption holds or not - in other words: instead of fixing the number of test-cases, when using \textit{checkCoverage}, QuickCheck runs as many test-cases as necessary until it is convinced that coverage holds, in which case the property-test passes, or fails if this becomes statistically not possible.

\begin{HaskellCode}
prop_susceptible_infected :: Property
prop_susceptible_infected = checkCoverage (do  
  -- define the parameters
  let contactRate = 5     -- beta
      infectivity = 0.05  -- gamma
      population  = [Susceptible, Infected, Recovered]
      n           = length population
      i           = length (filter (==Infected) population)
      	  
  -- compute the probabilities
  let pcont = 100 * expCDF (1 / contactRate) contactRate
      ir    = i / n
    
  -- compute the expected probability
  let susInf = pcont * ir * infectivity

  -- run a random susceptible agent for 1.0 time-unit return true if infected
  infected <- genSusceptibleInfected contactRate infectivity population

  -- expect given percentage of susceptible agents to have become infected
  return (cover susInf infected
          ("susceptible agents became infected, expected at least " ++ show susInf)) True)
\end{HaskellCode}

Note that all the parameters can be varied arbitrarily and the test will still work. Also note that we have to fix the population instead of taking a random one, otherwise percentage of coverage would change in every test-case, which would not make sense and will make the property-test fail.
TODO: why cant we randomize the population?

\subsection{Testing the property}
When running the property-test repeatedly we see that the encoded property holds:

\begin{verbatim}
Susceptible agents become infected:   OK (0.09s)
  +++ OK, passed 6400 tests (1.72% susceptible agents became infected, 
      expected at least 1.05).
Susceptible agents become infected:   OK (0.17s)
  +++ OK, passed 12800 tests (1.539% susceptible agents became infected, 
      expected at least 1.05).
Susceptible agents become infected:   OK (0.04s)
  +++ OK, passed 3200 tests (2.12% susceptible agents became infected, 
      expected at least 1.05).
Susceptible agents become infected:   OK (0.17s)
  +++ OK, passed 12800 tests (1.617% susceptible agents became infected,
      expected at least 1.05).    
\end{verbatim}

It is arguably questionable that we accept a higher percentage than we expected but we argue that we are close enough. TODO: can we really argue like that? also if we decrease the dt to e.g. 0.01 we get higher percentage of recovery... something seems to be not right here...

\section{Infected agent}
The main concept behind the \textit{infected} agent is that it \textit{will} recover \textit{on average} after the illness duration, specified as $\delta$ in the model.

\subsection{Deriving the property}
The idea is to run an infected agent until it recovers and measure the time it has stayed infected, resulting in the illness duration. The time until recovery follows an exponential distribution with $\lambda = \delta$. To make this a property, we need to average the illness duration of multiple random infected agents which we will then compare to an expected value $\delta$, using of a two-tailed t-test with confidence of $\alpha = 0.05$.
 
\subsection{Encoding the property}
We follow the exact same approach as in the susceptible using \textit{cover} and \textit{checkCoverage} but the decision whether the test belongs to the class is now made depending on whether the t-test fails or succeeds. Instead of computing the expected coverage we expect 95\% of all agents recovering within the given time.

\begin{HaskellCode}
prop_infected_meanIllnessDuration :: Property
prop_infected_meanIllnessDuration = checkCoverage (do
  let illnessDuration = 15.0 -- delta
  
  -- run 100 random infected agents until they recover and return the 
  -- duration they were ill
  ids <- vectorOf 100 (genInfectedIllnessDuration illnessDuration)
    
  let confidence = 0.95
      idsTTest   = tTestSamples TwoTail illnessDuration (1 - confidence) ids

  return (cover 95 idsTTest
         ("infected agents have a mean illness duration of " ++ show illnessDuration) True))
\end{HaskellCode}

\subsection{Testing the property}

When running the property-test repeatedly we see that the encoded property holds:

\begin{verbatim}
Infected agent mean illness duration: OK (4.85s)
  +++ OK, passed 800 tests (94.4% infected agents have a 
      mean illness duration of 15.0).
Infected agent mean illness duration: OK (4.79s)
  +++ OK, passed 800 tests (95.0% infected agents have a 
      mean illness duration of 15.0).
Infected agent mean illness duration: OK (4.84s)
  +++ OK, passed 800 tests (93.2% infected agents have a 
      mean illness duration of 15.0).
Infected agent mean illness duration: OK (4.79s)
  +++ OK, passed 800 tests (94.8% infected agents have a 
      mean illness duration of 15.0).
\end{verbatim}

Although we don't hit the 95\% in every case, QuickChecks statistical tests are convinced enough that the coverage of 95\% is sound and thus deems the property-test to be successful.

\subsection{An alternative approach}
TODO: using CDF because illness duration follows the exponential distribution

\section{Discussion}
Testing time-driven agents seems to be quite forward if one can derive properties for them. A crucial thing to get right in time-driven ABS is $\Delta t$ - property-tests can be of help here by stating properties which have to hold and then refining $\Delta t$ until the property holds.

\chapter{Stateless agent testing}

TODO IMPROVE EXPLANATION: more explanations, its too flat, code straight in the face, prepare it gently as it is not directly clear what the intention is. discuss it more generally, otherwise it reads like an advanced technical report and not like a thesis.

In this chapter we present how QuickCheck can be used to test event-driven agents by expressing their \textit{specification} as property-tests in the case of the event-driven SIR implementation from chapter \ref{sec:eventdriven_sir}.

In general, testing event-driven agents is fundamentally different and more complex than testing time-driven agents, as their interface surface is generally much larger: events form the input to the agents to which they react with new events - the dependencies between those can be quite complex and deep. Using property-based tests we can encode the invariants and end up with an actual specification of their behaviour, acting as documentation, regression test within a TDD and property tests.

Note that the concepts presented here are applicable with slight adjustments to the Sugarscape implementation as well but we focused on the SIR one as its specification is shorter and does not require as much in-depth details - after all we are interested in deriving concepts, not dealing with specific technicalities.

With event-driven ABS a good starting point in specifying and then testing the system is simply relating the input events to expected output events. In the SIR implementation we have only 3 events, making it feasible to give a full formal specification - note that the Sugarscape implementation has more than 16 events, which makes it much harder to test it with sufficient coverage, giving a good reason to primarily focus on the SIR implementation. 

\section{Agent specification}
We start by giving the full \textit{specification} of the \textit{susceptible}, \textit{infected} and \textit{recovered} agent by stating the input-to-output event relations. The \textit{susceptible} agent is specified as follows:

\begin{enumerate}
	\item \textit{MakeContact} - If the agent receives this event it will output a random number of \textit{Contact ai Susceptible} events, where ai is the agents self id and the random number follows an exponential distribution with $\lambda = \beta$ (contact rate). The events have to be scheduled immediately without delay, thus having the current time as scheduling time-stamp. The receivers of the events are uniformly randomly chosen from the agent population. The agent doesn't change its state, stays \textit{Susceptible} and does not schedule any other events than the ones mentioned.
	
	\item \textit{Contact * Infected} - If the agent receives this event there is a chance of uniform probability $\gamma$ (infectivity) that the agent becomes \textit{Infected}. If this happens, the agent will schedule a \textit{Recover} event to itself into the future, where the time is drawn randomly from the exponential distribution with $\lambda = \delta$ (illness duration). If the agent does not become infected, it won't change its state, stays \textit{Susceptible} and does not schedule any events.
	
	\item \textit{Contact * *} or \textit{Recover}  - If the agent receives any of these (other) events it won't change its state, stays \textit{Susceptible} and does not schedule any events.
\end{enumerate}

This specification implicitly covers that a \textit{susceptible} agent can never transition from a \textit{Susceptible} to a \textit{Recovered} state within a single event - it can only make the transition to \textit{Infected} or stays \textit{Susceptible}. The \textit{infected} agents are specified as follows:

\begin{enumerate}
	\item \textit{Recover} - If the agent receives this, it will not schedule any events and make the transition to the \textit{Recovered} state.
	
	\item \textit{Contact sender Susceptible} - If the agent receives this, it will reply immediately with \textit{Contact ai Infected} to the sender, where \textit{ai} is the infected agents id and the scheduling time-stamp is the current time. It will not schedule any events and stays \textit{Infected}.
	
	\item In case of any other event, the agent will not schedule any events and stays \textit{Infected}.
\end{enumerate}

This specification implicitly covers that an \textit{infected} agent never goes back to the \textit{Susceptible} state - it can only make the transition to \textit{Recovered} or stay \textit{Infected}. From the specification of the \textit{susceptible} agent it becomes clear that a susceptible agent who became infected, will always recover as the transition to \textit{Infected} includes the scheduling of \textit{Recovered} to itself. 

The \textit{recovered} agents specification is very simple. It stays \textit{Recovered} forever and does not schedule any events.

The question is now how to put these into a property-test with QuickCheck. We focus on the \textit{susceptible} agent, as it it the most complex one, which concepts can then be easily applied to the other two. Generally speaking, we create a random \textit{susceptible} agent and a random event, feed it to the agent to get the output and check the invariants accordingly to input and output. In the specification there are stated three probabilities regarding $\beta$ (contact rate), $\gamma$ (infectivity) and $\delta$ (illness duration). We will only check one, $\gamma$ (infectivity) using the coverage features of QuickCheck and write additional property-tests for the other two. The reason for that is, that checking $\gamma$ is natural with the invariant checking whereas the others need a slightly different approach and are more obviously stated in separate property-tests.

\section{Encoding invariants}
We start by giving our property a name and use \textit{checkCoverage} to ask QuickCheck to enforce statistical testing to ensure soundness of our coverage which we will encode within this property. Also we define default values for the model parameters $\beta$, $\gamma$ and $\delta$.

\begin{HaskellCode}
prop_susceptible_invariants :: Property
prop_susceptible_invariants = checkCoverage (do
  let contactRate     = 5     -- beta
      infectivity     = 0.05  -- gamma
      illnessDuration = 15.0  -- delta
\end{HaskellCode}

Next we generate random attributes for the agent we want to create

\begin{HaskellCode}
-- need a random number generator
g <- genStdGen
-- generate non-empty list of agent ids, we have at least one agent
-- the susceptible agent itself
ais <- genNonEmptyAgentIds
-- generate positive time
(Positive t) <- arbitrary
-- the susceptible agents id is picked randomly from all empty agent ids
ai <- elements ais 
\end{HaskellCode}

Next we need to generate a random event out of \textit{MakeContact}, \textit{Recover} and \textit{Contact Int SIRState}. For this we have written a custom \textit{Generator}, which allows to specify frequencies and the agent ids to draw the contact ids from:

\begin{HaskellCode}
genEventFreq :: Int -> Int -> Int -> (Int, Int, Int) -> [AgentId] -> Gen SIREvent
genEventFreq mcf _ rcf _ []  
  -- no agent ids, will not generate Contact event
  = frequency [ (mcf, return MakeContact), (rcf, return Recover)]
genEventFreq mcf cof rcf (s,i,r) ais
  = frequency [ (mcf, return MakeContact)
              , (cof, do
                  -- draw SIRState with frequency 
                  ss <- frequency [ (s, return Susceptible)
                                  , (i, return Infected)
                                  , (r, return Recovered)]
                  -- draw random element from agent ids
                  ai <- elements ais
                  return (Contact ai ss))
              , (rcf, return Recover)]
\end{HaskellCode}

It is important to understand that together with the $\gamma$ (infectivity) parameter, the frequency of the \textit{Contact * Infected} event determines the probability of a susceptible agent to become infected. Thus we explicitly state the frequencies so we can compute the probabilities for the coverage feature.

\begin{HaskellCode}
let mkEvtFreq = 1
    -- will never happen as Recover will never be sent to a Susceptible
    recEvtFreq = 0
    contEvtFreq = 5
    contSusEvtFreq = 1
    contInfEvtFreq = 3
    -- will never happen, as a Recovered agent does not send any event,
    contRecEvtFreq = 0
    sirFreq    = (contSusEvtFreq, contInfEvtFreq, contRecEvtFreq)
    sirFreqSum = contSusEvtFreq + contInfEvtFreq + contRecEvtFreq
    evtFreqSum = mkEvtFreq + recEvtFreq + contEvtFreq
    
-- generate a random event with given frequencies
evt <- genEventFreq mkEvtFreq contEvtFreq recEvtFreq sirFreq ais
\end{HaskellCode}

Then we simply create and run the random susceptible agent and collect its output.

\begin{HaskellCode}
-- create susceptible agent with agen id
let a = susceptibleAgent ai contactRate infectivity illnessDuration
-- run agent with given event and configuration and collect its output state ao
-- the events es it has scheduled
let (_, _, ao, es) = runAgent g a evt t ais
\end{HaskellCode}

Having generated the output of the random susceptible agent, we can now start encoding the invariants. In case the random generated event was \textit{Recover} we encode that the agent stays Susceptible and does not schedule any events. Further we compute its coverage given the frequencies of the events.

\begin{HaskellCode}
case evt of
  Recover -> do
    -- compute coverage
    let cp = 100 * (recEvtFreq / evtFreqSum)
    return (cover cp True "Susceptible receives Recover"
           -- must stay Susceptible and not schedule any events
           (null es && ao == Susceptible))
\end{HaskellCode}

Next is the invariant of the \textit{MakeContact} event. This is quite complex as it has a lot of small invariants encoded. We use a helper function to iterate over all events generated by the agent in which most of the invariants are encoded.

\begin{HaskellCode}
MakeContact -> do
  -- compute coverage
  let cp  = 100 * (mkEvtFreq / evtFreqSum)
  -- check invariants
  let ret = checkMakeContactInvariants ai ais t es
  return (cover cp True "Susceptible receives MakeContact" 
         -- must stay Susceptible 
         (ret && ao == Susceptible))
         
checkMakeContactInvariants :: AgentId -> [AgentId] -> Time -> [QueueItem SIREvent] -> Bool
checkMakeContactInvariants sender ais t es 
    -- make sure there was exactly one MakeContact event
    = uncurry (&&) ret
  where
    -- start out with all OK and no MakeContact event found
    ret = foldr checkMakeContactInvariantsAux (True, False) es

    checkMakeContactInvariantsAux :: QueueItem SIREvent -> (Bool, Bool) -> (Bool, Bool)
    checkMakeContactInvariantsAux 
        (QueueItem receiver (Event (Contact sender' Susceptible)) t') (b, mkb)
      = (b && sender == sender'    -- the sender in Contact must be the Susceptible agent
           && receiver `elem` ais  -- the receiver of Contact must be in the agent ids
           && t == t', mkb)        -- the Contact event is scheduled immediately
    checkMakeContactInvariantsAux 
        (QueueItem receiver (Event MakeContact) t') (b, mkb) 
      = (b && receiver == sender  -- the receiver of MakeContact is the Susceptible agent itself
           && t' == t + 1.0       -- the MakeContact event is scheduled 1 time-unit into the future
           && not mkb, True)      -- there can only be one MakeContact event
    checkMakeContactInvariantsAux evt (_, mkb) = error ("failure " ++ show evt) (False, mkb)
\end{HaskellCode}

What is left is the \textit{Contact} event. We have to differentiate between the \textit{SIRState} it carries. We start by looking into \textit{Recovered}. In this case the agent stays \textit{Susceptible} and schedules no events. 

\begin{HaskellCode}
Contact _ s -> 
  case s of
    Recovered -> do
      -- compute coverage
      let cp = contEvtSplitProb * (contRecEvtFreq / sirFreqSum)
      return (cover cp True "Susceptible receives Contact * Recovered"
              -- stays Susceptible and schedules no events
              (null es && ao == Susceptible))
\end{HaskellCode}

The behaviour in case of \textit{Susceptible} within the \textit{Contact} event is the same and thus not repeated here. What is left is the handling of a \textit{Contact} event with \textit{Infected}. In case the \textit{susceptible} agent didn't get infected nothing happens and the agent stays \textit{Susceptible}. On the other hand, in case of infection, the invariants of scheduled events must be checked.

\begin{HaskellCode}
Infected -> do
  -- compute coverage
  let contInfProb = contEvtSplitProb * (contInfEvtFreq / sirFreqSum)
  -- compute infection coverage
  let infProb = contInfProb * infectivity

  if ao /= Infected
    -- not infected, nothing happens
    then return
          cover (contInfProb - infProb) True "Susceptible receives Contact * Infected, stays Susceptible"
            -- stays Susceptible and does not schedule any events 
            (null es && ao == Susceptible) 
    -- infected, check invariants
    else return 
          cover infProb True ("Susceptible receives Contact * Infected, becomes Infected with prob " ++ show infProb)
            (checkInfectedInvariants ai t es)
            
checkInfectedInvariants :: AgentId -> Time -> [QueueItem SIREvent] -> Bool
checkInfectedInvariants sender t 
    -- pattern match on exactly one Recovery event
    [QueueItem receiver (Event Recover) t'] 
  = sender == receiver && t' >= t           -- receiver is sender (self) and scheduled into the future
checkInfectedInvariants _ _ _ = False       -- no other event allowed
\end{HaskellCode}

The specifications and encodings of the infected and recovered agent follow same patterns and are not repeated here.

\section{Encoding probabilities}
In the invariant we only checked the probability that a susceptible agent can become infected. What is missing is the property that the contact rate $\beta$, as well as the illness duration $\delta$ follow an exponential distribution. To test this we again make use of the coverage features of QuickCheck and encoding the property that the number of \textit{Contact} events scheduled follows the exponential distribution. Technically this can be checked by using the cummulative densitiy functions (CDF) which gives the probability that a random variable $X$ has a value less or equal than a given $x$. Using this, we can state that percentage of the test-cases where the susceptible agent generates less or equal $\beta$ \textit{Contact} events has to be probability of the given CDF parametrised by $\beta$.

\begin{HaskellCode}
prop_susceptible_meancontactrate :: Property
prop_susceptible_meancontactrate = checkCoverage (do
  let contactRate = 5 -- beta
  
  -- run a random susceptible agent making contactRate contacts
  -- on average and get number c of Contact events
  c <- genSusceptibleAgentMakeContact contactRate

  -- compute probability that c is less than or equal contactRate
  -- which follows the CDF of the exponential distribution
  let prob = 100 * expCDF (1 / contactRate) contactRate

  return
    -- the test-case belongs to the class if the number of Contact events
    -- scheduled is less or equal the contactRate
    cover prob (c <= contactRate) 
      ("susceptibles have mean contact rate of up " ++ show contactRate ++ 
        ", expected " ++ printf "\%.2f" prob)  True
\end{HaskellCode}

The encoding of the illness duration probability follows the exactly same pattern and is not repeated here.

TODO: shortly discuss that time-driven testing follows exactly the same approach only without events and only by advancing time. can be used to get the right time-delta

\section{Running the tests}
When running the tests we see QuickChecks coverage features at work. It will generate as many test-cases as necessary to ensure the soundness and robustness of the coverage properties and indeed all go through.

\begin{verbatim}
  Susceptible invariants:               OK (29.00s)
    +++ OK, passed 204800 tests:
    59.4336% Susceptible receives Contact * Infected, stays Susceptible
    20.8301% Susceptible receives Contact * Susceptible
    16.6772% Susceptible receives MakeContact
     3.0591% Susceptible receives Contact * Infected, becomes Infected with prob 3.13
     
  Susceptible agent mean contact rate:  OK (0.19s)
    +++ OK, passed 1600 tests (66.31% susceptibles have mean contact rate of up 5, expected 63.21).
    
  Infected agent mean illness duration: OK (0.81s)
    +++ OK, passed 3200 tests (63.59% infected have an illness duration of up to 15.0, expected 63.21).
\end{verbatim}

We see that the mean contact rate lies 3\% above the expected value, which is normal due to ABS stochastic nature, which makes the actual values vary. In repeated runs of the tests we get slightly different percentages due to different random number streams and random behaviour - still our properties hold.

\section{Discussion}
Funnily the implementation of all the specifications and property-tests is longer than the original implementation. Though thats not the point here: we showed how to implement a full specification of an ABS model as a property-based test and we succeeded! This is definitely a strong indication that our hypothesis that randomised property-based testing is a suitable tool for testing ABS is valid. With unit tests we would be quite lost here: even for the SIR model, it is hard to enumerate all possible interactions and cases but by stating invariants as properties and generating random test-cases we make sure they are checked.

We have not looked into more complex testing patterns like the synchronous agent-interactions of Sugarscape. We didn't look into testing full agent and interacting agent behaviour using property-tests due to its complexity which would justify a whole paper alone. Due to its inherent stateful nature with complex dependencies between valid states and agents actions we need a more sophisticated approach as outlined in \cite{de_vries_-depth_2019}, where the authors show how to build a meta-model and commands which allow to specify properties and valid state-transitions which can be generated automatically. We leave this for further research.

What is particularly powerful is that one has complete control and insight over the changed state before and after e.g. a function was called on an agent: thus it is very easy to check if the function just tested has changed the agent-state itself or the environment: the new environment is returned after running the agent and can be checked for equality of the initial one - if the environments are not the same, one simply lets the test fail. This behaviour is very hard to emulate in OOP because one can not exclude side-effect at compile time, which means that some implicit data-change might slip away unnoticed. In FP we get this for free.

\chapter{Stateful agent testing}
So far, our tests were stateless: only one computational step of an agent was considered by feeding a single event and ignoring the agent continuation. Also the events didn't contain any notion of time as they would carry within the queue. Feeding follow-up events into the continuation would make testing inherently stateful as we introduce history into the system. Such tests would allow to test the full life-cycle of an agent, which we were not able to model properly with QuickCheck so far due to the libraries stateless nature.

Fortunately there exists the \textit{quickcheck-state-machine} library \footnote{See \url{https://hackage.haskell.org/package/quickcheck-state-machine}}, which allows to let actions depend on the history of the system by using a state-machine approach. This is an ideal match for property-testing ABS as agents almost always have some notion of inner state and very often agents are actually implemented in terms of state-machines. The agents in the event-driven SIR are no exception to this: although their state is not explicit as for the agents in the Sugarscape model, they are state-machines, transitioning from the \textit{Susceptible} to the \textit{Recovered} state, depending on incoming events generated by actions of other agents.

To build a model for property-tests with quickcheck-state-machine we need the following ingredients:

\begin{itemize}
	\item Underlying state - This is the very state of the stateful testing, upon which the tests operate. In our case the state consists of a single agent continuation, the random-number generator, the current simulation time and the list of all agent ids. Note that we are interested in testing a single agent over a longer history, not the interaction of a whole bunch of agents, that is why we only hold a single agent in the state.
	
	\item An API to test against - This is the interface to manipulate the underlying state and present valid actions the test can carry out, generally consisting of a bunch of functions to test. In the stateful approach it is convenient to represent the API as an ADT where the constructors represent API calls. This allows easy enumeration and generation of random API calls with generate arguments following some constraints and interdependencies. In our case of even-driven agents the API is the agent MSF with its monadic context, receiving events, which can already be seen as API calls represented through ADT - the agent simply interprets them. Thus this step is very naturally expressed in event-driven ABS testing and does not require any complicated translation of API calls. Technically it is a bit more subtle: because we talk about history now, instead of representing the API as plain SIREvents, we use QueueItem which holds the receiver, the scheduling time and the SIREvent.
	
	\item A shrinker - reduces the sequence of API calls but which still generate the same error - in other words a shrinker reduces a failing sequence of API calls to the most minimal one in which the test-case still fails. A shrinker depends highly on the state and the API. In our case for example a shrinker can safely remove \textit{Recover} events from the sequence of API calls in case the agent is \textit{Susceptible} or \textit{Recovered} as those events have no effect in these states.
	
	\item Test-cases - We need to come up with some test-cases we want to validate. In state-machine testing the intention is often a verification of a 
\end{itemize}


TODO: can we also implement the existing tests in terms of the state-machine? yes we can but we leave this for further research. Also we don't think it is very useful but disguises things more and hides the detail that the existing tests are stateless.


\chapter{Verification \& Correctness}
Testing of functional ABS paper
10\%

- correctness \& verification
	-> static type system eliminates a large number run-time bugs: if we decide to rule out IO then we can guarantee 


\chapter{The Equilibrium-Totality Correspondence}
\label{ch:equilibrium_totality}

The idea is to implement a total agent-based SIR simulation, where the termination does NOT depend on time (is not terminated after a finite number of time-steps, which would be trivial).  We argue that the underlying SIR model actually has a steady state.

The dynamics of the System Dynamics SIR model are in equilibrium (won't change any more) when the infected stock is 0. This might be shown formally but intuitively it is clear because only infected agents can lead to infections of susceptible agents which then make the transition to recovered after having gone through the infection phase. 

Thus an agent-based implementation of the SIR simulation has to terminate if it is implemented correctly because all infected agents will recover after a finite number of steps after then the dynamics will be in equilibrium. Thus we have the following conditions for totality:
\begin{enumerate}
	\item The simulation shall terminated when there are no more infected agents.
	\item All infected agents will recover after a finite number of time, which means that the simulation will eventually run out of infected agents. 
	
	Unfortunately this criterion alone does not suffice because when we look at the SIR+S model, which adds a cycle from Recovered back to Susceptible, we have the same termination criterion, but we cannot guarantee that it will run out of infected. We need an additional criteria.
	\item The source of infected agents is the pool of susceptible agents which is monotonous decreasing (not strictly though!) because recovered agents do NOT turn back into susceptibles.
\end{enumerate}

Thus we can conclude that a SIR model must enter a steady state after finite steps / in finite time. %\footnote{Note that there exists a SIR+S model, which adds a cycle back from Recovered to Susceptible - if we find a total implementation of the SIR model and add this transition then the simulation should become non-total, checked by the compiler.}.

By this reasoning, a non-total, correctly implemented agent-based simulations of the SIR model will eventually terminate (note that this is independent of which environment is used and which parameters are selected). Still this does not formally proof that the agent-based approach itself will terminate and so far no formal proof of the totality of it was given.

Dependent Types and Idris' ability for totality- and termination-checking should theoretically allow us to proof that an agent-based SIR implementation terminates after finite time: if an implementation of the agent-based SIR model in Idris is total it is a formal proof by construction. Note that such an implementation should not run for a limited virtual time but run unrestricted of the time and the simulation should terminate as soon as there are no more infected agents, returning the termination time as an output. Also if we find a total implementation of the SIR model and extend it to the SIR+S model, which adds a cycle from Recovered back to Susceptible, then the simulation should become again non-total as reasoned above.

The HOTT book \cite{program_homotopy_2013} states that lists, trees,... are inductive types/inductively defined structures where each of them is characterized by a corresponding \textit{induction principle}. Thus, for a constructive proof of the totality of the agent-based SIR model we need to find the induction principle of it. This leaves us with the question of what the inductive, defining structure of the agent-based SIR model is? Is it a tree where a path through the tree is one way through the simulation or is it something else? It seems that such a tree would grow and then shrink again e.g. infected agents. Can we then apply this further to (agent-based) simulation in general?

%TODO: \url{https://stackoverflow.com/questions/19642921/assisting-agdas-termination-checker/39591118}

%We hypothesize that it should be possible due to the nature of the state transitions where there are no cycles and that all infected agents will eventually reach the recovered state. 
%
%-- TODO: express in the types
%-- SUSCEPTIBLE: MAY become infected when making contact with another agent
%-- INFECTED:    WILL recover after a finite number of time-steps
%-- RECOVERED:   STAYS recovered all the time
%
%-- SIMULATION:  advanced in steps, time represented as Nat, as real numbers are not constructive and we want to be total
%--              terminates when there are no more INFECTED agents

\section{Dependent Types}
TODO: THIS BELONGS PROBABLY BETTER IN CONCLUSIONS, FUTURE RESEARCH (together with linear types)

TODO: connect to property-based testing and put emphasise on the constructive nature and hypothesis testing: this is a popperian approach.

TODO: FP is the first step towards a more structural understanding of ABS implementations where dependent types should allow us to develop this even further. we leave this for further research and outline only broadly the ideas we want to follow.

%My work is all nice and good but it solves problems the ABS community and implementations never really had. My FRP/MSF approach is quite complex and can be equally difficult to get right. Even worse, the bugs were not primarily those I am solving with FP but the REAL problem in ABS is translating the model into code. Can FP help us here? Can my pure FP approach help here? expressing invariants in FP code? can we express them in types? 

The pure functional implementation techniques have a number of technical benefits but don't help as much in closing the gap between specification and implementation as one is used from functional programming in general. Therefore we take a step back and abstract from these highly complex implementation techniques and move towards dependent types. Follow \cite{botta_time_2010} and \cite{botta_functional_2011}.

Conceptually discuss how dependent types can be made of use in ABS without going into lot of technical detail because: 1. i didn't do enough research on it and 2. dependent types seem to be nearly out of focus of the thesis.

%Linear and Dependent Types with Idris 2: more general ideas / hints / research on how it is applicable to ABS

%dependent types in ABS paper, explore totality - equilibrium correspondence idea
%About 20\% finished.

After having established the concepts of dependent types, we want to briefly discuss ideas where and how they could be made of use in ABS. We expect that dependent types will help ruling out even more classes of bugs at compile-time and encode even more invariants. Additionally by constructively implementing model specifications on the type level could allow the ABS community to reason about a model directly in code as it narrows the gap between model specification and implementation.

By definition, ABS is of constructive nature, as described by Epstein \cite{epstein_chapter_2006}: "If you can't grow it, you can't explain it" - thus an agent-based model and the simulated dynamics of it is itself a constructive proof which explain a real-world phenomenon sufficiently well. Although Epstein certainly wasn't talking about a constructive proof in any mathematical sense in this context (he was using the word \textit{generative}), dependent types \textit{might} be a perfect match and correspondence between the constructive nature of ABS and programs as proofs.

When we talk about dependently typed programs to be proofs, then we also must attribute the same to dependently typed agent-based simulations, which are then constructive proofs as well. The question is then: a constructive proof of what? It is not entirely clear \textit{what we are proving} when we are constructing dependently typed agent-based simulations. Probably the answer might be that a dependently typed agent-based simulation is then indeed a constructive proof in a mathematical sense, explaining a real-world phenomenon sufficiently well - we have closed the gap between a rather informal constructivism as mentioned above when citing Epstein who certainly didn't mean it in a constructive mathematical sense, and a formal constructivism, made possible by the use of dependent types.

In the following subsections we will discuss related work in this field (\ref{sub:dep_abs_relwork}), discuss general concepts where dependent types might be of benefit in ABS (\ref{sub:dep_abs_generalconcepts}), present a dependently typed implementation of a 2D discrete environment (\ref{sub:dep_abs_2denv}) and finally discuss potential use of dependent types in the SIR model (\ref{sub:dep_abs_sir}) and SugarScape model (\ref{sub:dep_abs_sugarscape}).

Dependent types are a very powerful addition to functional programming as they allow us to express even stronger guarantees about the correctness of programs \textit{already at compile-time}. They go as far as allowing to formulate programs and types as constructive proofs which must be \textit{total} by definition \cite{thompson_type_1991, mckinna_why_2006, altenkirch_pi_2010}. 

So far no research using dependent types in agent-based simulation exists at all. We have already started to explore this for the first time and ask more specifically how we can add dependent types to our functional approach, which conceptual implications this has for ABS and what we gain from doing so. We are using Idris \cite{brady_idris_2013} as the language of choice as it is very close to Haskell with focus on real-world application and running programs as opposed to other languages with dependent types e.g. Agda and Coq which serve primarily as proof assistants.

We hypothesise, that  dependent types will allow us to push the correctness of agent-based simulations to a new, unprecedented level by narrowing the gap between model specification and implementation. The investigation of dependent types in ABS will be the main unique contribution to knowledge of my Ph.D.

In the following section \ref{sec:dep_background}, we give an introduction of the concepts behind dependent types and what they can do. Further we give a very brief overview of the foundational and philosophical concepts behind dependent types. In Section \ref{sec:dep_absconcepts} we briefly discuss ideas of how the concepts of dependent types could be applied to agent-based simulation and in Section \ref{sec:dep_vav_deptypes} we very shortly discuss the connection between Verification \& Validation and dependent types.

There exist a number of excellent introduction to dependent types which we use as main ressources for this section: \cite{thompson_type_1991, program_homotopy_2013, stump_verified_2016, brady_type-driven_2017, pierce_programming_2018}.

Generally, dependent types add the following concepts to pure functional programming:

\begin{enumerate}
	\item Types are first-class citizen - In dependently types languages, types can depend on any \textit{values}, and can be \textit{computed} at compile-time which makes them first-class citizen. This becomes apparent in Section \ref{sub:dep_vector} where we compute the return type of a function depending on its input values.

	\item Totality and termination - A total function is defined in \cite{brady_type-driven_2017} as: it terminates with a well-typed result or produces a non-empty finite prefix of a well-typed infinite result in finite time. This makes run-time overhead obsolete, as one does not need to drag around additional type-information as everything can be resolved at compile-time. Idris is turing-complete but is able to check the totality of a function under some circumstances but not in general as it would imply that it can solve the halting problem. Other dependently typed languages like Agda or Coq restrict recursion to ensure totality of all their functions - this makes them non turing-complete. All functions in Section \ref{sub:dep_vector} are total, they terminate under all inputs in finite steps.

	\item Types as \textit{constructive} proofs - Because types can depend on any values and can be computed at compile-time, they can be used as constructive proofs (see \ref{sub:dep_foundations}) which must terminate, this means a well-typed program (which is itself a proof) is always terminating which in turn means that it must consist out of total functions. Note that Idris does not restrict us to total functions but we can enforce it through compiler flags. We implement a constructive proof of showing whether two natural numbers are decidable equal in the Section \ref{sub:dep_equality}.
\end{enumerate}

%The authors of \cite{ionescu_dependently-typed_2012} discuss how to use dependent types to specify fundamental theorems of economics, unfortunately they are not computable and thus not constructive, thus leaving it more to a theoretical, specification side.
%Ionesus talk on dependently typed programming in scientific computing
%https://www.pik-potsdam.de/members/ionescu/cezar-ifl2012-slides.pdf
%Ionescus talk on Increasingly Correct Scientific Computing
%%https://www.cicm-conference.org/2012/slides/CezarIonescu.pdf
%Ionescus talk on Economic Equilibria in Type Theory
%https://www.pik-potsdam.de/members/ionescu/cezar-types11-slides.pdf
%Ionescus talk on Dependently-Typed Programming in Economic Modelling
%https://www.pik-potsdam.de/members/ionescu/ee-tt.pdf

\paragraph{State-Machines}
Often, Agent-Based Models define their agents in terms of state-machines. It is easy to make wrong state-transitions e.g. in the SIR model when an infected agent should recover, nothing prevents one from making the transition back to susceptible. 

Using dependent types it might be possible to encode invariants and state-machines on the type level which can prevent such invalid transitions already at compile-time. This would be a huge benefit for ABS because of the popularity of state-machines in agent-based models.

\paragraph{Flow Of Time}
State-Machines often have timed transitions e.g. in the SIR model, an infected agent recovers after a given time. Nothing prevents us from introducing a bug and \textit{never} doing the transition at all.

With dependent types we might be able to encode the passing of time in the types and guarantee on a type level that an infected agent has to recover after a finite number of time steps. Also can dependent types be used to express the flow of time and that it is strongly monotonic increasing?
	
\paragraph{Existence Of Agents}
In more sophisticated models agents interact in more complex ways with each other e.g. through message exchange using agent IDs to identify target agents. The existence of an agent is not guaranteed and depends on the simulation time because agents can be created or terminated at any point during simulation. 

Dependent types could be used to implement agent IDs as a proof that an agent with the given id exists \textit{at the current time-step}. This also implies that such a proof cannot be used in the future, which is prevented by the type system as it is not safe to assume that the agent will still exist in the next step. %So it is a proof of the existence of an agent: holds always only for the current time-step or for all time, depending on the model. e.g. in the SIR model no agents are removed from / added to the system thus a proof holds for all time. 
\\

\section{Equilibrium and Totality}
For some agent-based simulations there exists equilibria, which means that from that point the dynamics won't change any more e.g. when a given type of agents vanishes from the simulation or resources are consumed. This means that at that point the dynamics won't change any more, thus one can safely terminate the simulation. Very often, despite such a global termination criterion exists, such simulations are stepped for a fixed number of time-steps or events or the termination criterion is checked at run-time in the feedback-loop. 
	
Using dependent types it might be possible to encode equilibria properties in the types in a way that the simulation automatically terminates when they are reached. This results then in a \textit{total} simulation, creating a \textit{correspondence between the equilibrium of a simulation and the totality of its implementation}. Of course this is only possible for models in which we know about their equilibria a priori or in which we can reason somehow that an equilibrium exists.

A central question in tackling this is whether to follow a model- or an agent-centric approach. The former one looks at the model and its specifications as a whole and encodes them e.g. one tries to directly find a total implementation of an agent-based model. The latter one looks only at the agent level and encodes that as dependently typed as possible and hopes that model guarantees emerge on a meta-level - put otherwise: does the totality of an implementation emerge when we follow an agent-centric approach?

\section{Hypotheses}
Models like the Sugarscape are exploratory in nature and don't have a formal ground truth where one could derive equilibria or dynamics from and validate with. In such models the researchers work with informal hypotheses which they express before running the model and then compare them informally against the resulting dynamics.

It would be of interest if dependent types could be made of use in encoding hypotheses on a more constructive and formal level directly into the implementation code. So far we have no idea how this could be done but it might be a very interesting application as it allows for a more formal and automatic testable approach to hypothesis checking.

\subsection{Philosophical Foundations: Constructivism}
\label{sub:dep_foundations}

The main theoretical and philosophical underpinnings of dependent types as in Idris are the works of Martin-L\"of intuitionistic type theory. The view of dependently typed programs to be proofs is rooted in a deep philosophical discussion on the foundations of mathematics, which revolve around the existence of mathematical objects, with two conflicting positions known as classic vs. constructive \footnote{We follow the excellent introduction on constructive mathematics \cite{thompson_type_1991}, chapter 3.}. In general, the constructive position has been identified with realism and empirical computational content where the classical one with idealism and pragmatism.

In the classical view, the position is that to prove $\exists x. P(x)$ it is sufficient to prove that $\forall x. \neg P(x)$ leads to a contradiction. The constructive view would claim that only the contradiction is established but that a proof of existence has to supply an evidence of an $x$ and show that $P(x)$ is provable. In the end this boils down whether to use proof by contradiction or not, which is sanctioned by the law of the excluded middle which says that $A \lor \neg A$ must hold. The classic position accepts that it does and such proofs of existential statements as above, which follow directly out of the law of the excluded middle, abound in mathematics \footnote{Polynomial of degree n has n complex roots; continuous functions which change sign over a compact real interval have a zero in that interval,...}. The constructive view rejects the law of the excluded middle and thus the position that every statement is seen as true or false, independently of any evidence either way. \cite{thompson_type_1991} (p. 61): \textit{The constructive view of logic concentrates on what it means to prove or to demonstrate convincingly the validity of a statement, rather than concentrating on the abstract truth conditions which constitute the semantic foundation of classical logic}.

To prove a conjunction $A \land B$ we need prove both $A$ and $B$, to prove $A \lor B$ we need to prove one of $A, B$ and know which we have proved. This shows that the law of the excluded middle can not hold in a constructive approach because we have no means of going from a proof to its negation. Implication $A \Rightarrow B$ in constructive position is a transformation of a proof $A$ into a proof $B$: it is a function which transforms proofs of $A$ into proofs of $B$. The constructive approach also forces us to rethink negation, which is now an implication from some proof to an absurd proposition (bottom): $A \Rightarrow \perp$. Thus a negated formula has no computational context and the classical tautology $\neg \neg A \Rightarrow A$ is then obviously no longer valid.  Constructively solving this would require us to be able to effectively compute / decide whether a proposition is true or false - which amounts to solving the halting problem, which is not possible in the general case.

A very important concept in constructivism is that of finitary representation / description. Objects which are infinite e.g. infinite sets as in classic mathematics, fail to have computational computation, they are not computable. This leads to a fundamental tenet in constructive mathematics: \cite{thompson_type_1991} (p. 62): \textit{Every object in constructive mathematics is either finite [..] or has a finitary description}

Concluding, we can say that constructive mathematics is based on principles quite different from classical mathematics, with the idealistic aspects of the latter replaced by a finitary system with computational content. Objects like functions are given by rules, and the validity of an assertion is guaranteed by a proof from which we can extract relevant computational information, rather than on idealist semantic principles. 

All this is directly reflected in dependently typed programs as we introduced above: functions need to be total (finitary) and produce proofs like in \textit{checkEqNat} which allows the compiler to extract additional relevant computational information. Also the way we described the (infinite) natural numbers was in an finitary way. In the case of decidable equality, the case where it is not equal, we need to provide an actual proof of contradiction, with the type of Void which is Idris representation of $\perp$. 

\subsection{Verification, Validation and Dependent Types}
\label{sec:dep_vav_deptypes}
Dependent types allow to encode specifications on an unprecedented level, narrowing the gap between specification and implementation - ideally the code becomes the specification, making it correct-by-construction. The question is ultimately how far we can formulate model specifications in types - how far we can close the gap in the domain of ABS. Unless we cannot close that gap completely, to arrive at a sufficiently confidence in correctness, we still need to test all properties at run-time which we cannot encode at compile-time in types.

Nonetheless, dependent types should allow to substantially reduce the amount of testing which is of immense benefit when testing is costly. Especially in simulations, testing and validating a simulation can often take many hours - thus guaranteeing properties and correctness already at compile time can reduce that bottleneck substantially by reducing the number of test-runs to make.

Ultimately this leads to a very different development process than in the established object-oriented approaches, which follow a test-driven process. There one defines the necessary interface of an object with empty implementations for a given use-case first, then writes tests which cover all possible cases for the given use-case. Obviously all tests should fail because the functionality behind it was not implemented yet. Then one starts to implement the functionality behind it  step-by-step until no test-case fails. This means that one runs all tests repeatedly to both check if the test-case one is working on is not failing anymore and to make sure that old test-cases are not broken by new code. The resulting software is then trusted to be correct because no counter examples through test hypotheses, could be found. The problem is: we could forget / not think of cases, which is the easier the more complex the software becomes (and simulations are quite complex beasts). Thus in the end this is a deductive approach.

With pure functional programming and dependent types the process is now mostly constructive, type-driven (see \cite{brady_type-driven_2017}). In that approach one defines types first and is then guided by these types and the compiler in an interactive fashion towards a correct implementation, ensured at compile-time. As already noted, the ABS methodology is constructive in nature but the established object-oriented test-driven implementation approach not as much, creating an impedance mismatch. We expect that a type-driven approach using dependent types reduces that mismatch by a substantial amount.

Note that \textit{validation} is a different matter here: independent of our implementation approach we still need to validate the simulation against the real-world / ground-truth. This obviously requires to run the full simulation which could take up hours in either programming paradigm, making them absolutely equal in this respect. Also the comparison of the output to the real-world / ground-truth is completely independent to the paradigm. The fundamental difference happens in case of changes made to the code during validation: in case of the established test-driven object-oriented approach for every minor change one (should) re-run all tests, which could take up a substantial amount of additional time. Using a constructive, type-driven approach this is dramatically reduced and can often be completely omitted because the correctness of the change can be either guaranteed in the type or by informally reasoning about the code.

%-------------------------
%TODO: not sure where to put this
%ABS as a constructive / generative science, follows Poperian approach of falsification: we try to construct a model which explains a real-world (empirical) phenomenon - if validation shows that the generated dynamics match the ones of the real-world sufficiently enough, we say that we have found \textit{a} hypothesis (the model) which emergent properties explains the real-world phenomenon sufficiently enough. This is not a proof but only one possible explanation which holds for now and might be falsified in the future.
%
%When we implement our simulation things change a bit as we add another layer: the conceptual model, describing the phenomenon, which is an abstraction of reality. This description can be of many forms but can be regarded on a line between completely formal (economic models) to informal (sociology) but the implementation will follow that description. The fundamental difference here is that in this case we want our implementation to be exactly the same as the conceptual model. Contrary to the real-world, where it is not possible to find a \textit{true} model (as was argued by Popper), on this level we actually can construct an implementation which matches the conceptual model exactly because we have a description of the conceptual model. In the end we transform the conceptual model description in code, which is itself a formal description. In this translation process (speak: implementation / programming), one can make an endless number of mistakes. Generally we can distinguish between two classes of mistakes: 
%1) conceptual mistakes - wrong translation of the model specifications into code due to various reasons e.g. imprecise description, human error. The more precise an unambiguous a model description is, the less probable conceptual mistakes will be.
%2) internal mistakes - normal programming mistakes e.g. access of arrays out of bounds, ... also using correlated Random Number generators.
%
%Level 0: Real-World phenomenon
%Level 1: Conceptual model of the real-world phenomenon
%Level 2: Implementation of the conceptual model
%
%Note that we must speak of falsification and constructiveness on two different levels:
%- validation level: do the results of the conceptual model match the real-world phenomenon? the conceptual model is the hypothesis which says that its mechanics are sufficient to generate / construct the real-world phenomenon. At this level we are not interested in the implementation level anymore - the implemented model \textit{is} (seen as) the conceptual model, and one only compares its output to the real-world. If the dynamics match, then we got a valid hypothesis which works for now. If the dynamics do NOT match, then the hypothesis (the model) is falsified and one needs to adjust / change the hypothesis (model). The validation will happen by tests, there is no other way, we have no formal specification of the real-world, we can only observe empirically the phenomena, so we run tests which try to falsify the outputs of the model: assuming it will generate phenomena of the real-world and test if it does.
%- implementation \& verficiation level: in this step we are matching the code to the conceptual model. Here we are not only restricted to a test-driven approach because we have a more or less formal description of the conceptual model which we directly encode in our programming language. If the language allows to express model specifications already at compile-time then this means that the implementation narrows the gap between model specification and implementation which means it does not need to be tested at run-time because it is guaranteed for all inputs for all time. 
%
%The constructiveness of ABS and impendance mismatch: ABS methodology is constructive but the established implementation approach not too much, creating an impedance mismatch. this is especially visible in the test-driven development dependent types constructive nature could close this mismatch.
%