\chapter*{} %Property-Based Testing in pure functional ABS
\label{ch:property}

When implementing an Agent-Based Simulation (ABS) it is of fundamental importance that the implementation is correct up to some specification and that this specification matches the real world in some way. This process is called verification and validation (V\&V), where \textit{validation} is the process of ensuring that a model or specification is sufficiently accurate for the purpose at hand whereas \textit{verification} is the process of ensuring that the model design has been transformed into a computer model with sufficient accuracy \cite{robinson_simulation:_2014}. In other words, validation determines if we are we building the \textit{right model}, and verification if we are building the \textit{model right} up to some specification \cite{balci_verification_1998}.

% there is no general validity, an approach is TDD: V&V particularly difficult in ABS
One can argue that ABS should require more rigorous programming standards than other computer simulations \cite{polhill_ghost_2005}. Because researchers in ABS look for an emergent behaviour in the dynamics of the simulation, they are always tempted to look for some surprising behaviour and expect something unexpected from their simulation. 
Also, due to ABS' \textit{constructive / exploratory} nature \cite{epstein_chapter_2006, epstein_generative_2012}, there exists some uncertainty about the dynamics the simulation will produce before running it. The authors \cite{ormerod_validation_2006} see the current process of building ABS as a discovery process where models of an ABS often lack an analytical solution in general, which makes verification much harder if there is no such solution. Thus it is often very difficult to judge whether an unexpected outcome can be attributed to the model or has in fact its roots in a subtle programming error \cite{galan_errors_2009}.

In general this implies that it is not possible to prove that a model is valid in general but that the best we can do is to \textit{raise the confidence} in the correctness of the simulation. Therefore, the process of V\&V is not the proof that a model is correct but the \textit{process} of trying to show that the model is \textit{not incorrect}. The more checks one carries out which show that it is not incorrect, the more confidence we can place on the models validity. To tackle such a problem in software, software engineers have developed the concept of test-driven development (TDD).

Test-Driven Development (TDD) was popularised in the early 00s by Kent Beck \cite{beck_test_2002} as a way to a more agile approach to software-engineering, where instead of doing each step (requirements, implementation, testing,...) as separated from each other, all of them are combined in shorter cycles. Put shortly, in TDD tests are written for each feature before actually implementing it, then the feature is fully implemented and the tests for it should pass. This cycle is repeated until the implementation of all requirements has finished. Traditionally TDD relies on so called unit-tests which can be understood as a piece of code which when run isolated, tests some functionality of an implementation. Thus we can say that test-driven development in general and unit testing together with code-coverage in particular, guarantee the correctness of an implementation to some informal degree, which has been proven to be sufficiently enough through years of practice in the software industry all over the world. 

\medskip

The work \cite{collier_test-driven_2013} was the first to discusses how to apply TDD to ABS, using unit testing to verify the correctness of the implementation up to a certain level. They show how to implement unit-tests within the RePast Framework \cite{north_complex_2013} and make the important point that such a software needs to be designed to be sufficiently modular otherwise testing becomes too cumbersome and involves too many parts. The paper \cite{asta_investigation_2014} discusses a similar approach to DES in the AnyLogic software toolkit. 

The paper \cite{onggo_test-driven_2016} proposes Test Driven Simulation Modelling (TDSM) which combines techniques from TDD to simulation modelling. The authors present a case study for maritime search-operations where they employ ABS. They emphasise that simulation modelling is an iterative process, where changes are made to existing parts, making a TDD approach to simulation modelling a good match. They present how to validate their model against analytical solutions from theory using unit-tests by running the whole simulation within a unit-test and then perform a statistical comparison against a formal specification. %This approach will become of importance later on in our SIR and Sugarscape case studies.

The paper \cite{gurcan_generic_2013} gives an in-depth and detailed overview over verification, validation and testing of agent-based models and simulations and proposes a generic framework for it. The authors present a generic UML class model for their framework which they then implement in the two ABS frameworks RePast and MASON. Both of them are implemented in Java and the authors provide a detailed description how their generic testing framework architecture works and how it utilises unit testing with JUnit to run automated tests. To demonstrate their framework they provide also a case study of an agent-base simulation of synaptic connectivity where they provide an in-depth explanation of their levels of test together with code.

\medskip

% the gap
Although it would be interesting to see how we can apply unit testing to our approach, it is straight forward, nothing new and does not constitute unique research. Thus, in this chapter we introduce an additional technique for TDD: \textit{property-based testing}, which can be seen as complementary to unit testing. Property-based testing has its origins \cite{claessen_quickcheck_2000,claessen_testing_2002,runciman_smallcheck_2008} in Haskell, where it was first conceived and implemented. It has been successfully used for testing Haskell code for years and also been proven to be useful in the industry \cite{hughes_quickcheck_2007}. We show and discuss how this technique can be applied to test pure functional ABS implementations. To our best knowledge property-based testing has never been looked at in the context of ABS and this thesis is the first one to do so.

\medskip

The main idea of property-based testing is to express model-specifications and laws directly in code and test them through \textit{automated} and \textit{randomised} test-data generation. Thus one hypothesis of this thesis is that due to ABS \textit{stochastic} and \textit{exploratory / generative / constructive } nature, property-based testing is a natural fit for testing ABS in general and pure functional ABS implementations in particular. It thus should pose a valuable addition to the already existing testing methods in this field, worth exploring. To substantiate and test our hypothesis, we present two case-studies. First, the agent-based SIR model as introduced in Chapter \ref{sec:sir_model}, which is of explanatory nature, where we show how to express formal model-specifications in property-tests. Second, the SugarScape model as introduced in Chapter \ref{sec:sugarscape}, which is of exploratory nature, where we show how to express hypotheses in property-tests and how to property-test agent functionality. 
%Again, we emphasise that we see it as an addition to TDD, where it works in combination with unit testing to verify and validate a simulation to increase the confidence in its correctness. % and is a useful tool for expressing regression tests.

\medskip

Note that property-based testing has a close connection to model-checking \cite{mcmillan_symbolic_1993}, where properties of a system are proved in a formal way. The important difference is that the checking happens directly on code and not on the abstract, formal model, thus one can say that it combines model-checking and unit testing, embedding it directly in the software-development and TDD process without an intermediary step. We hypothesise that adding it to the already existing testing methods in the field of ABS is of substantial value as it allows to cover a much wider range of test-cases due to automatic data generation. This can be used in two ways: to verify an implementation against a formal specification and to test hypotheses about an implemented simulation. This puts property-based testing on the same level as agent- and system testing, where not technical implementation details of e.g. agents are checked like in unit-tests but their individual complete behaviour and the system behaviour as a whole.

The work \cite{onggo_test-driven_2016} explicitly mentions the problem of test coverage which would often require to write a large number of tests manually to cover the parameter ranges sufficiently enough - property-based testing addresses exactly this problem by \textit{automating} the test-data generation. Note that this is closely related to data-generators \cite{gurcan_generic_2013}, load generators and random testing \cite{burnstein_practical_2010}. Property-based testing though goes one step further by integrating this into a specification language directly into code, emphasising a declarative approach and pushing the generators behind the scenes, making them transparent and focusing on the specification rather than on the data-generation. 

\section*{Property-Based Testing}
\label{sec:proptesting}

Property-based testing allows to formulate \textit{functional specifications} in code which then a property-based testing library tries to falsify by \textit{automatically} generating test-data, covering as much cases as possible. When a case is found for which the property fails, the library then reduces the test-data to its simplest form for which the test still fails e.g. shrinking a list to a smaller size. It is clear to see that this kind of testing is especially suited to ABS, because we can formulate specifications, meaning we describe \textit{what} to test instead of \textit{how} to test. Also the deductive nature of falsification in property-based testing suits very well the constructive and exploratory nature of ABS. Further, the automatic test-generation can make testing of large scenarios in ABS feasible because it does not require the programmer to specify all test-cases by hand, as is required in e.g. traditional unit tests.

Property-based testing was introduced in \cite{claessen_quickcheck_2000,claessen_testing_2002} where the authors present the QuickCheck library in Haskell, which tries to falsify the specifications by \textit{randomly} sampling the test space. We argue, that the stochastic sampling nature of this approach is particularly well suited to ABS, because it is itself almost always driven by stochastic events and randomness in the agents behaviour, thus this correlation should make it straightforward to map ABS to property-testing.
%The main challenge when using QuickCheck, as will be shown later, is to write \textit{custom} test-data generators for agents and the environment which cover the space sufficiently enough to not miss out on important test-cases.
According to the authors of QuickCheck \textit{"The major limitation is that there is no measurement of test coverage."} \cite{claessen_quickcheck_2000}. Although QuickCheck provides help to report the distribution of test-cases it is not able to measure the coverage of tests in general. This could lead to the case that test-cases which would fail are never tested because of the stochastic nature of QuickCheck. Fortunately, the library provides mechanisms for the developer to measure coverage in specific test-cases where the data and its (expected) distribution is known to the developer. This is a powerful tool for testing randomness in ABS as will be shown in subsequent chapters.

\medskip

As a remedy for the potential coverage problems of QuickCheck, there exists also a deterministic property-testing library called SmallCheck \cite{runciman_smallcheck_2008}, which instead of randomly sampling the test-space, enumerates test-cases exhaustively up to some depth. It is based on two observations, derived from model-checking, that (1) \textit{"If a program fails to meet its specification in some cases, it almost always fails in some simple case"} and (2) \textit{"If a program does not fail in any simple case, it hardly ever fails in any case} \cite{runciman_smallcheck_2008}. This non-stochastic approach to property-based testing might be a complementary addition in some cases where the tests are of non-stochastic nature with a search-space  too large to test manually by unit tests but small enough to enumerate exhaustively. The main difficulty and weakness of using SmallCheck is to reduce the dimensionality of the test-case depth search to prevent combinatorial explosion, which would lead to exponential number of cases. Thus one can see QuickCheck and SmallCheck as complementary instead of in opposition to each other.

\subsection*{A brief overview of QuickCheck}
To give a rough idea on how property-based testing works in Haskell, we give a few examples of property-tests on lists, which are directly expressed as functions in Haskell. Such a function has to return a \textit{Bool} which indicates \textit{True} in case the test succeeds or \textit{False} if not and can take input arguments which data is automatically generated by QuickCheck.

\begin{HaskellCode}
-- concatenation operator (++) is associative
append_associative :: [Int] -> [Int] -> [Int] -> Bool
append_associative xs ys zs = (xs ++ ys) ++ zs == xs ++ (ys ++ zs)

-- the reverse of a reversed list is the original list
reverse_reverse :: [Int] -> Bool
reverse_reverse xs = reverse (reverse xs) == xs

-- reverse is distributive over concatenation (++)
-- this test fails for explanatory reasons, for a correct 
-- property xs and ys need to be swapped on the right-hand side!
reverse_distributive :: [Int] -> [Int] -> Bool
reverse_distributive xs ys = reverse (xs ++ ys) == reverse xs ++ reverse ys

-- running the tests
main :: IO ()
main = do
  quickCheck append_associative
  quickCheck reverse_reverse
  quickCheck reverse_distributive
\end{HaskellCode}

When we run the tests using \textit{main}, we get the following output:

\begin{verbatim}
+++ OK, passed 100 tests.
+++ OK, passed 100 tests.
*** Failed! Falsifiable (after 5 tests and 6 shrinks):    
[0]
[1]
\end{verbatim}

We see that QuickCheck generates 100 test-cases for each property-test and it does this by generating random data for the input arguments. Note that we have not specified any data for our input arguments; QuickCheck is able to provide a suitable data-generator through type-inference: for lists and all the existing Haskell types like Int there exist custom generators already.

QuickCheck generates 100 test-cases by default and requires all to pass - if there is a test-case which fails, the overall property-test fails and QuickCheck shrinks the input to a minimal size for which the case still fails and reports it as a counter example. This is the case in the last property-test \textit{reverse\_distributive} which is wrong as \textit{xs} and \textit{ys} need to be swapped on the right-hand side. In this run, QuickCheck found a counter-example to the property after 5 tests and applied 6 shrinks to find the minimal failing example of \textit{xs = [0]} and \textit{ys = [1]}. If we swap \textit{xs} and \textit{ys}, the property-test passes 100 test-cases just like the other two did. Note that it is possible to configure QuickCheck to generate more or less random test-cases, which can be used to increase the coverage if the sampling space is quite large - this will become useful later.

\subsubsection*{Properties and Generators}
TODO: property
TODO: label
TODO: ==>
TODO: generators

\subsubsection*{Coverage}
TODO: cover with checkCoverage


\section*{Testing ABS implementations}
\label{sec:testingABS}
TODO: this is a rather weak section, polish it by writing more general about it and how we can use property-based testing for it. if i cannot come up with something more substantial simply scrap it

Generally we need to distinguish between two types of testing / verification in ABS.

\begin{enumerate}
	\item Testing / verification of models for which we have real-world data or an analytical solution which can act as a ground-truth - examples for such models are the SIR model, stock-market simulations, social simulations of all kind.
	\item Testing / verification of models which are of exploratory nature, inspired by real-world phenomena but for which no ground-truth per se exists - examples for such models is the Sugarscape \cite{epstein_growing_1996} or Agent\_Zero model \cite{epstein_agent_zero:_2014}.
\end{enumerate}

The baseline is that either one has an analytical model as the foundation of an agent-based model or one does not. In the former case, e.g. the SIR model, one can very easily validate the dynamics generated by the simulation to the one generated by the analytical solution through System Dynamics. In the latter case one has basically no idea or description of the emergent behaviour of the system prior to its execution e.g. SugarScape. In this case it is important to have some hypothesis about the emergent property / dynamics. The question is how verification / validation works in this setting as there is no formal description of the expected behaviour: we don't have a ground-truth against which we can compare our simulation dynamics.

One distinguishes between black-box and white-box verification where in white-box verification one looks directly at code and reasons about it whereas in black-box verification one generally feeds input to the software / functions / methods and compares it to expected output. Black-box verification is our primary concern in this chapter as property-based testing is an instance of black-box verification. In the case of ABS we have the following levels of black-box tests \cite{nguyen_testing_2011}: %%Although the work on TDD is scarce in ABS, there exists quite some research on applying TDD and unit testing to multi-agent systems (MAS). Although MAS is a different discipline than ABS, the latter one has derived many technical concepts from the former one, thus testing concepts applied to MAS might also be applicable to ABS. The paper \cite{nguyen_testing_2011} is a survey of testing in MAS. It distinguishes between unit-tests of parts that make up an agent, agent tests which test the combined functionality of parts that make up an agent, integration tests which test the interaction of agents within an environment and observe emergent behaviour, system tests which test the MAS as a system running at the target environment and acceptance test in which stakeholders verify that the software meets their goal. Although not all ABS simulations need acceptance and system tests, still this classification gives a good direction and can be directly transferred to ABS. 
\begin{enumerate}
	\item Isolated and interacting agent behaviour parts - test the individual parts which make up the agent behaviour under given inputs. Also test if interaction between agents are correct. For this we can use traditional unit-tests as shown by \cite{collier_test-driven_2013} and also property-based testing as we will show in the case studies.
	\item Simulation dynamics - compare emergent dynamics of the ABS as a whole under given inputs to an analytical solution or real-world dynamics in case there exists some, using statistical tests. We see this type of tests conceptually as property-tests as well because we are testing properties of the model / simulation as we will see in the case-studies. Technically speaking we can use both unit and property-based tests to implement them - conceptually they are property-tests.
	\item Hypotheses - test whether hypotheses about the model are valid or invalid. This is very similar to the previous point but without comparing it to analytical solutions or real-world dynamics but only to some hypothetical values.
\end{enumerate}

TODO: make it clear that we focus on the event-driven SIR implemention only. why? because of its more interesting nature and because its concepts are also applicable to the sugarscape as well. the time-driven implementation will be shortly discussed in releveant sections.

There is a strong relation between property-based tests and dependent types: in property-based testing we express specifications / properties / laws in code and test their invariance at run-time by random sampling the space. In dependent-types it is possible to express such properties already statically in types. This is the subject of the next part of the thesis which tries to move towards dependent types in ABS.

We found property-based testing particularly well suited for ABS firstly due to ABS stochastic nature and second because we can formulate specifications, meaning we describe \textit{what} to test instead of \textit{how} to test. Also the deductive nature of falsification in property-based testing suits very well the constructive and often exploratory nature of ABS. 

Although property-based testing has its origins in Haskell, similar libraries have been developed for other languages e.g. Java, Pyhton, C++ as well and we hope that our research has sparked an interest in applying property-based testing to the established object-oriented languages in ABS as well.

\chapter{Testing Agents}
\label{ch:prop_agents}

% TODO read

% https://fsharpforfunandprofit.com/posts/property-based-testing-2/
% https://iohk.io/blog/an-in-depth-look-at-quickcheck-state-machine/
% https://hackage.haskell.org/package/quickcheck-state-machine

% OK
% https://begriffs.com/posts/2017-01-14-design-use-quickcheck.html
% https://blog.ploeh.dk/2019/03/11/an-example-of-state-based-testing-in-haskell/
% https://jaspervdj.be/posts/2015-03-13-practical-testing-in-haskell.html#the-action-trick
% https://wickstrom.tech/programming/2019/03/24/property-based-testing-in-a-screencast-editor-case-study-1.html

Agent behaviour - obviously full agent behaviour could be tested with property-tests, using randomly generated agents (with random values in their properties). As it turned out to be quite difficult to derive properties for, in this paper we restricted ourselves to test parts of agent behaviour and also left out testing of agent interactions.

We didn't look into testing full agent and interacting agent behaviour using property-tests due to its complexity which would justify a whole paper alone. Due to its inherent stateful nature with complex dependencies between valid states and agents actions we need a more sophisticated approach as outlined in \cite{de_vries_-depth_2019}, where the authors show how to build a meta-model and commands which allow to specify properties and valid state-transitions which can be generated automatically. We leave this for further research.

What is particularly powerful is that one has complete control and insight over the changed state before and after e.g. a function was called on an agent: thus it is very easy to check if the function just tested has changed the agent-state itself or the environment: the new environment is returned after running the agent and can be checked for equality of the initial one - if the environments are not the same, one simply lets the test fail. This behaviour is very hard to emulate in OOP because one can not exclude side-effect at compile time, which means that some implicit data-change might slip away unnoticed. In FP we get this for free.

\section{Individual Agent behaviour}
We implemented a number of property-tests for agent functions which just cover a part of an agents behaviour: checks whether an agent has died of age or starved to death, the metabolism, immunisation step, check if an agent is a potential borrower or fertile, lookout, trading transactions. We provided custom data-generators for the agents and let QuickCheck generate the random data and us running the agent with the provided data, checking for the properties. 

As an example, provided in listing \ref{alg:prop_test_agent}, we give the property-test of an agent dying from age, which happens when the agents age is greater or equal its maximum age. It might look trivial but property-based testing helps us here to clearly state the invariants (properties) and relieves us from constructing all possible edge-cases because we rely on QuickChecks abilities to cover them for us.

\begin{algorithm}
\SetKwInOut{Input}{input}\SetKwInOut{Output}{output}
\Input{Random agent \textit{ag} generated by QuickCheck}
\Input{Random age \textit{randAge} within a specified range generated by QuickCheck}
set age of agent ag to randAge\;
ageLimit = get agents age limit\; 
died = agentDieOfAge ag;\

\eIf{died == (randAge >= ageLimit}{
  PASS\;
} {
  FAIL\;
}
\caption{Property-based test for agent dying of age.}
\end{algorithm}
\label{alg:prop_test_agent}


TODO: sir individual Agent testing: randomly generate events and feed them into agents,susceptible must can never become recovered, infected never susceptible and recovered never susceptible or infected. test propabilities e.g. infected agent recovering schedules event with average illnessduration. susceptible schedules on average contactrate events, recovered schedules no events, an infected replies to each contact

TODO: refine existing Sugarscape quickcheck: look into monadic testing

TODO: sugarscape agent testing: avoid error with ErrorEvent, which is handled individually by the kernel. this makes such cases testable

TODO: look at coverage of the sir testing usinh stack test --coverage

TODO: agent testing: look into statechart quickcheck to get idea and understanding by the example of event-driven SIR, then adopt it to the Sugarscape

\section{Agent interaction}

TODO: sugarscape: individual agent testing works analog to SIR, can we do random testing of synchronous agent-interactions? generate valid sequences e.g. test trading and mating - but what are the properties?

\chapter{Other}

Environment behaviour - the Sugarscape environment has its own behaviour which boils down to regrowing of resources. The correct working can be tested using property-tests by generating random environments and checking laws governing the regrowth.

\section{Environment behaviour}
The environment in the Sugarscape model has some very simple behaviour: each site has a sugar level and when harvested by an agent, it regrows back to the full level over time. Depending on the configuration of the model it either grows back immediately within 1 tick or over multiple ticks. We can construct simple property-tests for these behaviours. In the case the sugar grows back immediately we let QuickCheck generate a random environment and then run the environment behaviour for 1 tick and then check the property that all sites have to be back to their maximum sugar level. In the case of regrow over multiple ticks, we also use QuickCheck to generate a random environment but additionally a random \textit{positive} rate (which is a floating point number) which we then use to calculate the number of ticks until full regrowth. After running the random environment for the given number of ticks all sites have to be back to full sugar level - see \ref{alg:prop_test_rateregwroth} for an algorithm for this case.

Note that QuickCheck initially doesn't know how to generate a random environment because each site consists of a custom data-structure for which QuickCheck is not able to generate random instances by default. This problem is solved by writing a custom data-generator, for which existing QuickCheck functions can be used e.g. picking the current sugar level of a site from a random range.

\begin{algorithm}
\SetKwInOut{Input}{input}\SetKwInOut{Output}{output}
\Input{Random environment \textit{env} generated by QuickCheck}
\Input{Regrowth rate \textit{randRate} (positive floating point) generated by QuickCheck}
maxTicks = maxSugarCapacityOnSites / randRate\;
env' = runEnvironmentTicks maxTicks env\;
sites = getEnvironmentSites env'\;

\eIf{all sites maxSugarLevel}{
  PASS\;
} {
  FAIL\;
}
\caption{Property-based test for rate-based regrow of sugar on all sites.}
\end{algorithm}
\label{alg:prop_test_rateregwroth}

The Sugarscape environment is a torus where the coordinates wrap around in both dimensions. To check whether the implementation of the wrapping-calculation is correct we used both unit- and property-tests. With the unit-tests we carefully constructed all possible cases we could think of and came up with 13 test-cases. With the property-based test we simply defined a single test-case where we expressed the property that after wrapping \textit{any} coordinates, supplied by QuickCheck, the wrapped coordinates have to be within bounds. See the algorithm in \ref{alg:prop_test_wrapcoords}.

\begin{algorithm}
\SetKwInOut{Input}{input}\SetKwInOut{Output}{output}
\Input{Random 2d discrete coordinate \textit{randCoord} generated by QuickCheck}
(x, y) = wrapCoordinates randCoord\;

\eIf{(x $\geq$ 0 and x $\leq$ environmentDimX) and (y $\geq$ 0 and y $\leq$ environmentDimY)}{
  PASS\;
} {
  FAIL\;
}
\caption{Property-based test for wrap-coordinates functionality.}
\end{algorithm}
\label{alg:prop_test_wrapcoords}

\chapter{Verifying an exploratory model: \\ Hypotheses in Sugarscape}
\label{ch:prop_exploratory}

In this chapter we look at how property-based testing can be made of use to verify the \textit{exploratory} Sugarscape model \cite{epstein_growing_1996} as already introduced in Chapter \ref{sec:sugarscape}. Whereas in the previous chapter on testing the explanatory SIR case-study we had an analytical solution, the fundamental difference in the exploratory Sugarscape model is that none such analytical solutions exist. This raises the question, which properties we can actually test in such a mode.

The answer lies in the very nature of exploratory models: they exist to explore and understand phenomena of the real world. Researchers come up with a model to explain the phenomena and then (hopefully) come up with a few questions and  \textit{hypotheses} about the emergent properties. The actual simulation is then used to test and refine the hypotheses. Indeed, descriptions, assumptions and hypotheses of varying formal degree are abound in the Sugarscape model. Examples are: \textit{the carrying capacity becomes stable after 100 steps; when agents trade with each other, after 1000 steps the standard deviation of trading prices is less than 0.05; when there are cultures, after 2700 steps either one culture dominates the other or both are equally present}. 

We show how to use property-testing to formalise and check such hypotheses. For this purpose we undertook a full \textit{verification} of our implementation \footnote{The code can be accessed freely from \url{https://github.com/thalerjonathan/phd/tree/master/public/towards/SugarScape/sequential}} from Chapter \ref{sec:sugarscape}. We validated it against the book \cite{epstein_growing_1996} and a NetLogo implementation \cite{weaver_replicating_2009} \footnote{\url{https://www2.le.ac.uk/departments/interdisciplinary-science/research/replicating-sugarscape}}. A longer report on the details of this validation process is attached as Appendix \ref{app:validating_sugarscape}, in this section we focus on QuickChecks role in this process.

The property we test for is whether \textit{the emergent property / hypothesis under test is stable under replicated runs} or not. To put it more technical, we use QuickCheck to run multiple replications with the same configuration but with different random-number streams and require that the tests all pass. During the verification process described in Appendix \ref{app:validating_sugarscape} we have derived and implemented property-tests for the following hypotheses:

\begin{enumerate}
	\item Disease Dynamics all recover - When disease are turned on, if the number of initial diseases is 10, then the population is  able to rid itself completely from all disease within 100 ticks. 
	
	\item Disease Dynamics minority recover - When disease are turned on, if the number of initial diseases is 25, the population is not able to rid itself completely from all diseases within 1,000 ticks.
	
	\item Trading Dynamics - When trading is enabled, the trading prices stabilise after 1,000 ticks with the standard deviation of the prices having dropped below 0.05.
	
	\item Cultural Dynamics - When having two cultures, red and green, after 2,700 ticks, either the red or the blue culture dominates or both are equally strong. If they dominate they make up 95\% of all agents, if they are equally strong they are both within 45\% - 55\%.
	
	\item Inheritance Gini Coefficient - When agents reproduce and can die of age then inheritance of their wealth leads to an unequal wealth distribution measured using the Gini Coefficient \textit{averaging} at 0.35. 

	\item Carrying Capacity - When agents don't mate nor can die from age (chapter II), due to the environment, there is an \textit{average} maximum carrying capacity of agents the environment can sustain. The capacity should be reached after 100 ticks and should be stable from then on.
		
	\item Terracing - When resources regrow immediately, after a few steps the simulation becomes static. Agents will stay on their terraces and will not move any more because they have found the best spot due to their behaviour. About 45\% will be on terraces and 95\% - 100\% are static and not moving any more.
\end{enumerate}

\section{Implementation}
To implement this, we make use of QuickChecks \textit{Generator} concept. A \textit{Generator} defines how specific (random) data can be generated and is implemented using the \textit{Gen} monad where all of QuickChecks random distribution functionality is available. Thus it is only natural that we implement a \textit{Generator} to produce output from a Sugarscape simulation. The generator takes the number of ticks and the scenario with which to run the simulation and returns a list of outputs, one for each tick.

\begin{HaskellCode}
sugarscapeUntil :: Int -> SugarScapeScenario -> Gen [SimStepOut]
sugarscapeUntil ticks params = do
  -- draw a seed for the random-number generator from the full range of Int
  seed <- choose (minBound, maxBound)
  -- create a random-number generators
  let g = mkStdGen seed
  -- initialise the simulation state with the given random-number generator
  -- and the parameters
  let (simState, _, _) = initSimulationRng g params
  -- run the simulation with the given state for number of ticks
  return (simulateUntil ticks simState)
\end{HaskellCode}

Using this generator, we can very conveniently produce sugarscape data within a property. Depending on the problem, we can generate only a single run or multiple replications, in case the hypothesis is assuming \textit{averages}. To see its use, we show the encoding of the \textit{Disease Dynamics (1)} hypothesis. Its type is \textit{Property}, which is required by QuickChecks top-level testing function. To generate a property, the \textit{property} function is used which takes a \textit{Gen Bool} computation or a simple \textit{Bool} function as predicate to indicate success (True) or failure (False). In this case we use a \textit{Gen Bool} to be able to run the sugarscape data generator.

\begin{HaskellCode}
prop_disease_allrecover :: Property
prop_disease_allrecover = property (do
  -- after 100 ticks...
  let ticks  = 100
  -- ... given Animation V-1 parameter configuration ...
  let params = mkParamsAnimationV_1

  -- ... from 1 sugarscape simulation ...
  aos <- sugarscapeLast ticks params
  -- ... counting all infected agents ...
  let infected = length (filter (==False)) map (null . sugObsDiseases . snd) aos
  -- ... should result in all agents to be recovered
  return (infected == 0))
\end{HaskellCode}

QuickCheck runs multiple replications of a property when testing it, where the number is 100 by default. From the implementation it becomes clear, that this hypothesis states that the property has to hold \textit{for all} replications. The \textit{Inheritance Gini Coefficient (5)} hypothesis on the other hand assumes that the Gini Coefficient \textit{averages} at 0.35. We cannot average over replicated runs of the same property thus we have no other option to do the process of averaging within the property. We force this property to be run only \textit{once} but we generate multiple replications of the sugarscape data within the property and employ a two-sided t-test to test the hypothesis. Here is how we encode this into a property-test:

\begin{HaskellCode}
prop_gini :: Int -> Double -> Property
prop_gini repls confidence = once (do
  -- after 1000 ticks...
  let ticks = 1000
  -- ... the gini coefficient should average at 0.35 ...
  let expGini = 0.35
  -- ... given the Figure III-7 parameter configuration ...
  let params = mkParamsFigureIII_7
  
  -- ... from repls replciations ... 
  gini <- vectorOf repls (genGiniCoeff ticks params)

  -- on a two-tailed t-test with given confidence
  let giniTTest = tTestSamples TwoTail expGini (1 - confidence) gini

  return giniTTest)
  
genGiniCoeff :: Int -> SugarScapeScenario -> Gen Double
genGiniCoeff ticks params = do
  -- generate sugarscape data
  aos <- sugarscapeUntil ticks params
  -- extract wealth of the agents in the last step
  let agentWealths = map (sugObsSugLvl . snd) (last aos)
  -- compute gini coefficient and return it
  return (giniCoeff agentWealths)
\end{HaskellCode}

\section{Running the tests}
As already pointed out, QuickChec by default tries to run up 100 replications of a \textit{Property}. If for all 100 replications the predicate evaluates to \textit{True} the property-test succeeds. On the other hand, QuickCheck will stop at the first predicate which evalutes to \textit{False} and marks this property-test as failed. When running all property-tests, this leaves us with the question: \textit{Will they go through?}.

Due to the long time even 1,000 ticks can take to compute, we reduce the number of maximum successful replications required to 10 and indeed when we run it, we are happy: \textit{All went through!}. 

It is important to understand that QuickCheck is always initialised with a new random-number seed when run, thus we might have just been lucky. Indeed,when we run it again, we are not so lucky, and the following hypotheses fail:

Trading Dynamics
Cultural Dynamics

The hypotheses which went through were:
Disease Dynamics all recover
Disease Dynamics no recover

\subsection{Allowing failure}
It is arguably the case that the binary approach of QuickCheck, where the whole property-test fails when a single replication fails, is too strict for testing ABS in general and our hypotheses in particular. The reason for that is, that due to ABS stochastic nature, the hypotheses might hold for a large number of replications but not strictly for all.

As a remedy, we can use \textit{maxFailPercent} \footnote{As of the time of writing this thesis (2nd April 2019), this only exists as a pull request \url{https://github.com/nick8325/quickcheck/pull/239} and has not been merged into the main branch of QuickCheck. Thus we use the QuickCheck from \url{https://github.com/stevana/quickcheck/tree/feat/max-failed-percent} who has provided the implementation of \textit{maxFailPercent}.} as a configuration argument to QuickCheck, which allows the failure of a given percentage of replications. The argument behaves in a way that it tries to run up to the 100 default (or whatever the configuration is) successful replications but fails the overall property-test if the percentage of failed replications is reached. By switching from a binary PASS/FAIL to a more probabilistic measure, reflecting reliability, we have now a more appropriate tool for testing the suitability of our hypotheses. 

We run the tests again with 10 replications each but now allowing 100\% of failure in each case to see how reliable each hypothesis is. In one specific run we get the following result:

\begin{enumerate}
	\item Disease Dynamics all recover: \textit{+++ OK, passed 10 tests.}

	\item Disease Dynamics minority recover: \textit{+++ OK, passed 10 tests.}
		
	\item Trading Dynamics: \textit{+++ OK, passed 10 tests; 2 failed (16\%).} \\ In total 12 tests (replications) were run, out of which 2 failed, which is a 16\% failure rate.
	
	\item Cultural Dynamics: \textit{+++ OK, passed 10 tests; 3 failed (23\%).} \\ In total 13 tests (replications) were run, out of which 3 failed, which is a 23\% failure rate.

	\item Inheritance Gini Coefficient: 
	
	\item Carrying Capacity:
		
	\item Terracing:
\end{enumerate}

How to deal with the failure of the hypotheses is obviously highly model specific. A first approach is to increase the number of replications to run to 100 to get a more robust estimate of the failure rate. If the failure rate stays within reasonable ranges then one can arguably assume that the hypothesis is valid for sufficiently enough cases. On the other hand, if the failure rate escalates, then it is reasonable to deem the hypothesis invalid and refine it or even abandon it altogether. In the hypotheses we presented here we accept the failure rate and even though there the hypotheses fail in some cases, we deem them sufficiently valid for the task at hand.

Note that we disabled shrinking for hypothesis testing as it has no meaning here. The only thing which would be shrunk would be the seed of the random-number generator, which has no intrinsic meaning.

\section{Discussion}
In this chapter we showed how to use QuickCheck to formalise and check hypotheses about an \textit{exploratory} agent-based model, in which no ground truth exists. Due to ABS stochastic nature in general it became obvious that to get a good measure of a hypotheses validity we need to allow failure using the \textit{maxFailPercent} argument of QuickCheck. This allowed us to show that the hypotheses we have presented are sufficiently valid for the task at hand and can indeed be used for expressing and formalising emergent properties of the model and also as regression tests within a TDD cycle.

\chapter{Verifying an explanatory model: \\ The SIR model specification}
\label{ch:prop_explanatory}
As first use-case we discuss how we can use property-based testing to verify the \textit{explanatory} agent-based SIR model as introduced in Chapter \ref{sec:sir_model}. To verify it, we need to make sure it is correct up to some specification. We aim at connecting the agent-based implementation to the SD specification, by formalising it into properties within a property-test. The SD specification can be given through the differential equations shown in Chapter \ref{sec:sir_model}, which we repeat here:

\begin{equation}
\begin{split}
\frac{\mathrm d S}{\mathrm d t} = -infectionRate \\
\frac{\mathrm d I}{\mathrm d t} = infectionRate - recoveryRate \\
\frac{\mathrm d R}{\mathrm d t} = recoveryRate 
\end{split}
\quad
\begin{split}
infectionRate = \frac{I \beta S \gamma}{N} \\
recoveryRate = \frac{I}{\delta} 
\end{split}
\end{equation}
\label{eq:sir_delta_rates}

Solving these equations is done by integrating over time. In the SD terminology, the integrals are called \textit{Stocks} and the values over which is integrated over time are called \textit{Flows}. At $t = 0$ a single agent is infected because if there wouldn't be any infected agents, the system would immediately reach equilibrium - this is also the formal definition of the steady state of the system: as soon as $I(t) = 0$ the system won't change any more.

\begin{align}
S(t) &= N - I(0) + \int_0^t -infectionRate\, \mathrm{d}t \\
I(0) &= 1 \\
I(t) &= \int_0^t infectionRate - recoveryRate\, \mathrm{d}t \\
R(t) &= \int_0^t recoveryRate\, \mathrm{d}t
\end{align}

\section{Deriving the properties}
The key to encode these specifications into a property is to understand that the stocks of \textit{S}, \textit{I} and \textit{R} change \textit{per time-unit} by the given rates. This means that if we run an SD simulation for 1 time-unit, the differences between the S, I and R stocks are the values specified in equations \ref{eq:sir_delta_rates}. %This property has to hold for \textit{any} initial value for S, I and R.

Translating this into a property of our ABS implementation is analogous. We count the number of initially S, I and R agents and run the simulation for 1 time-unit to get new S, I and R numbers. The differences should \textit{average} at the values specified in equations \ref{eq:sir_delta_rates} as well. This property has to hold for \textit{any} agent population. Note that due to ABS stochastic nature it is not enough to run only one replication of the simulation for 1 time-unit but we actually need multiple replications ($> 100$) to get statistically robust results. We then use two-tailed t-tests to compare the expected averages to the actual averages. If all 3 tests pass the whole property-test passes.

\section{Implementing a property-test}
We start by defining the type of our property, which takes a list of \textit{SIRStates} and returns a \textit{Bool}.

\begin{HaskellCode}
prop_sd_rates :: [SIRState] -> Bool
\end{HaskellCode}

Properties in QuickCheck are required to return \textit{Bool} to indicate success or failure - the arguments required for the function are then randomly generated and provided by QuickCheck. For QuickCheck to be able to generate random values of \textit{SIRState} we need to implement an instance of the \textit{Arbitrary} typeclass for \textit{SIRState}. This is straight forward: we \textit{uniformly} pick one out of the 3 possible values.

\begin{HaskellCode}
instance Arbitrary SIRState where
  -- arbitrary :: Gen SIRState
  -- Uniformly pick one of the 3 elements.
  arbitrary = elements [Susceptible, Infected, Recovered]
\end{HaskellCode}

If we want to have a different distribution of the Susceptible, Infected and Recovered states we can also provide a different \textit{Arbitrary} implementation:

\begin{HaskellCode}
instance Arbitrary SIRState where
  -- arbitrary :: Gen SIRState
  -- Susceptible are picked 3 times, Infected 2 times more often than Recovered
  arbitrary = frequency [ (3, return Susceptible)
                        , (2, return Infected)
                        , (1, return Recovered) ]
\end{HaskellCode}

Because we are running replications, we need random seeds to create random-number generators for each replication. We don't make them an argument to the property-test because we don't want to let QuickCheck handle them for us. The reason for that is that if we make it a parameter to the function, QuickCheck tries to vary it as well, meaning that we have less variance and test-cases over the initial population. Thus we generate the seeds manually but relying on QuickChecks random functionality.

\begin{HaskellCode}
prop_sd_rates :: [SIRState] -> Gen Bool
prop_sd_rates as = do
    -- Draw a list of replications random Ints over the full Int range to  
    -- reach maximum variance (reduce probability of drawing identical seeds)
    seeds <- vectorOf replications (choose (minBound, maxBound))
    return (prop_sd_ratesAux seeds)
  where
    prop_sd_ratesAux :: [Int] -> Bool
    prop_sd_ratesAux seeds = allTTestsPass -- see below
\end{HaskellCode}

Next we encode the SD specification as explained above into code.

% NOTE: we omited fromIntegral to make it more readable
\begin{HaskellCode}
-- initial values of S,I and R and total number of agents N
s0 = length (filter (==Susceptible) as)
i0 = length (filter (==Infected) as)
r0 = length (filter (==Recovered) as)
n  = s0 + i0 + r0

-- explicit re-naming
beta  = contactRate
gamma = infectivity
delta = illnessDuration

-- infection-rate
ir = if n == 0 then 0 else (i0 * beta * s0 * gamma) / n
-- recovery-rate 
rr = i0 / delta

-- S value after 1 time-unit 
s = s0 - ir
-- I value after 1 time-unit
i = i0 + (ir - rr)
-- R value after 1 time-unit
r = r0 + rr
\end{HaskellCode}

Then we run replications (100) of the simulation for 1.0 time-unit with same $\Delta t = 0.1$ as in Chapter \ref{sec:timedriven_firststep} to get lists of new S, I and R values.

\begin{HaskellCode}
-- run for 1 time-unit
dur = 1.0
-- same dt as in time-driven chapter implementation
dt = 0.1
-- generate random-number generator for each replication
rngs = map mkStdGen seeds
-- compute simulated values for s, i and r
(ss, is, rs) = unzip (map (last . runSIR dur dt as beta gamma delta) rngs)
\end{HaskellCode}

Finally we run two-tailed t-tests with confidence of 0.95 ($\alpha = 0.05$) for all 3 lists of the new S,I and R values.

% NOTE: tTestSamples return a Maybe Bool but we dont care about that detail here
\begin{HaskellCode}
confidence = 0.95
sTest = tTestSamples TwoTail s (1 - confidence) ss
iTest = tTestSamples TwoTail i (1 - confidence) is
rTest = tTestSamples TwoTail r (1 - confidence) rs

-- property-test passes if all 3 t-tests pass
allTTestsPass = sTest && iTest && rTest
\end{HaskellCode}

\section{Test results}
When testing the property with QuickCheck, by default 100 random test-cases will be generated and \textit{all} have to pass so that the whole property-test passes. Unfortunately, the whole property-test fails - not all 100 random test-cases go through even if we run the whole property-test repeatedly. QuickCheck prints out the random population it generated for the failing random test-case to give a counter-example for the assumption we encoded. Repeated runs show that the counter-examples seem to be lacking any regular pattern like Susceptibles only or 0 Infected - we seem to be failing randomly.

Indeed the fact that we are failing randomly reveals the fundamental difference between SD and ABS: due to ABS' stochastic nature, an ABS cannot match an SD exactly because it is much richer in its dynamics. This enables ABS to explore and reveal paths which are not possible in deterministic SD. In the case of the SIR model, such an alternative path would be the immediate recovery of the single infected agent at the beginning without infecting any other agent. This is not possible in the SD case: in case there is 1 infected agent, the whole epidemic will unfold.

The difficulty of comparing dynamics between SD and ABS and the impracticality to compare them \textit{exactly} was shown by \cite{macal_agent-based_2010} in the case of the SIR model, where the author shows that it is bimodal. Indeed, that is also supported by our observations. When looking at the samples of failed t-tests by plotting them in a histogram, it shows clearly that the values exhibit strong outliers, arriving at a skewed / fat tailed / bimodal histogram. %This means that they are not normally distributed, which is a base assumption and a necessity for t-tests.
The authors \cite{figueredo_comparing_2014} approach the problem of comparing ABS to SD more generally and propose different statistical techniques of how to approach the problem. 
We don't go into further statistical analysis of this problem here as it is not the aim of this thesis but we rather want to see what options we have in pushing the existing approach closer to the SD dynamics, giving us some measure of approximation.

\section{Accepting failure}
The different nature of ABS and SD has the implication that the binary approach of QuickCheck, where the whole property-test fails when a single random test-case fails, is too strict for testing ABS in general and our problem in particular. As a remedy, we can use \textit{maxFailPercent} \footnote{As of the time of writing this thesis (2nd April 2019), this only exists as a pull request \url{https://github.com/nick8325/quickcheck/pull/239} and has not been merged into the main branch of QuickCheck. Thus we use the QuickCheck from \url{https://github.com/stevana/quickcheck/tree/feat/max-failed-percent} who has provided the implementation of \textit{maxFailPercent}.} as a configuration argument to QuickCheck which allows the failure of a given percentage of random-tests cases. The argument behaves in a way that it tries to run up to the 100 default successful random test-cases but fails the overall property-test if the percentage of failed random test-cases is reached. By switching from a binary PASS/FAIL to a more probabilistic measure, reflecting reliability, we have now a tool we can use for measuring the gap between the SD specification and our ABS implementation. Note that a single run is not enough to create robust estimates about failure because QuickCheck always starts out with new seeds. Thus in our experiments, to get a more robust estimate we average 10 runs and report the average with the standard deviation. 

As a first test we run the same property-test again but allow a failure of 100\% to see how many tests will actually pass and how many will fail. In a first estimate run we get: \textit{*** Failed! Passed only 88 tests; 100 failed (53\%) tests}. This means QuickCheck ran 88 successful random test-cases for the property-test before reaching 100\% of failed tests, meaning that out of a total of 173 random test-cases 53\% were failed ones. When averaging 10 runs, 86.1 (4.3) pass with 100 failed tests amounting to a failure percentage of 53.3\% (1.05).
%1. *** Failed! Passed only 88 tests; 100 failed (53%) tests
%2. *** Failed! Passed only 83 tests; 100 failed (54%) tests.
%3. *** Failed! Passed only 86 tests; 100 failed (53%) tests.
%4. *** Failed! Passed only 82 tests; 100 failed (54%) tests.
%5. *** Failed! Passed only 94 tests; 100 failed (51%) tests.
%6. *** Failed! Passed only 88 tests; 100 failed (53%) tests.
%7. *** Failed! Passed only 82 tests; 100 failed (54%) tests.
%8. *** Failed! Passed only 92 tests; 100 failed (52%) tests.
%9. *** Failed! Passed only 82 tests; 100 failed (54%) tests.
%10. *** Failed! Passed only 84 tests; 100 failed (54%) tests.
%[88,83,86,82,94,88,82,92,82,84]
%[53,54,53,54,51,54,54,52,54,54]

\paragraph{Comparison to noise}
To make sure that our approach is going in the right direction at all, we replaced the SIR simulation with a rather simple noise-generator: it generates random values for S, I and R which have to sum up to the size of the population we are comparing against. Indeed, it performs considerably worse than our initial property-test: on average only 19.7 (2.7) tests pass with 100 failed, and a failure percentage of 83\% (1.7). 

%1. *** Failed! Passed only 15 tests; 100 failed (86%) tests.
%2. *** Failed! Passed only 20 tests; 100 failed (83%) tests.
%3. *** Failed! Passed only 23 tests; 100 failed (81%) tests.
%4. *** Failed! Passed only 20 tests; 100 failed (83%) tests.
%5. *** Failed! Passed only 23 tests; 100 failed (81%) tests.
%6. *** Failed! Passed only 20 tests; 100 failed (83%) tests.
%7. *** Failed! Passed only 23 tests; 100 failed (81%) tests.
%8. *** Failed! Passed only 18 tests; 100 failed (84%) tests.
%9. *** Failed! Passed only 17 tests; 100 failed (85%) tests.
%10. *** Failed! Passed only 18 tests; 100 failed (84%) tests.
%
%[15,20,23,20,23,20,23,18,17,18]
%[86,83,81,83,81,83,81,84,85,84]

\paragraph{Optimising $\Delta t$}
As already pointed out in \ref{sub:timedriven_results}, the selection of a sufficiently small $\Delta t$ is crucial and it might be very well the case that the original $\Delta t = 0.1$ we used in our implementation and in Chapter \ref{sub:timedriven_results} is not small enough. 

Indeed, when we half it to $\Delta t = 0.05$, 100 tests pass with 54 (10.3) failing on average, resulting in a failure percentage of 34.4\% (4.2) on average.
%1. +++ OK, passed 100 tests; 39 failed (28%).
%2. +++ OK, passed 100 tests; 48 failed (32%).
%3. +++ OK, passed 100 tests; 49 failed (32%).
%4. +++ OK, passed 100 tests; 65 failed (39%).
%5. +++ OK, passed 100 tests; 68 failed (40%).
%6. +++ OK, passed 100 tests; 46 failed (31%).
%7. +++ OK, passed 100 tests; 55 failed (35%).
%8. +++ OK, passed 100 tests; 55 failed (35%).
%9. +++ OK, passed 100 tests; 69 failed (40%).
%10. +++ OK, passed 100 tests; 46 failed (31%).
%[39,48,49,65,68,46,55,55,69,46]
%[28,32,32,39,40,31,35,35,40,31]

Lowering it to $\Delta t = 0.01$ decreases the failure percentage further with 100 tests passing and failed tests averaging at 15.9 (5.2) with 13.4\% (3.8).
%
%1. +++ OK, passed 100 tests; 25 failed (20%).
%2. +++ OK, passed 100 tests; 16 failed (13%).
%3. +++ OK, passed 100 tests; 25 failed (20%).
%4. +++ OK, passed 100 tests; 10 failed (9%).
%5. +++ OK, passed 100 tests; 10 failed (9%).
%6. +++ OK, passed 100 tests; 15 failed (13%).
%7. +++ OK, passed 100 tests; 14 failed (12%).
%8. +++ OK, passed 100 tests; 14 failed (12%).
%9. +++ OK, passed 100 tests; 15 failed (13%).
%10.+++ OK, passed 100 tests; 15 failed (13%).
%
%[25,16,25,10,10,15,14,14,15,15]
%[20,13,20,9,9,13,12,12,13,13]

Using such a property-based test can be used to find an optimally low $\Delta t$. The optimal $\Delta t$ is the lowest for which sufficiently enough tests go through.

\paragraph{Fixing random population size}
The size of the random population is random itself and we observed coverage of ranges from 0 up to 90. Due to ABS' discrete nature, an increased population size \textit{might} lead to a closer approximation to SD dynamics, which are continuous. When fixing the size of the random population to 100 ($\Delta t = 0.01$) we arrive at 100 passing tests and 35 (6) failed on average with 27.4\% (5.8) failure percentage.

%1. +++ OK, passed 100 tests; 30 failed (23%).
%2. +++ OK, passed 100 tests; 45 failed (31%)
%3. +++ OK, passed 100 tests; 32 failed (24%).
%4. +++ OK, passed 100 tests; 31 failed (23%).
%5. +++ OK, passed 100 tests; 36 failed (26%).
%
%[30,45,32,32,36]
%[23,31,24,23,36]

Fixing the population size leads to more failed tests. The reason for that might be that it is less likely to generate test-cases where the dynamics match exactly e.g. 0 agents of either Susceptible, Infected or Recovered. Still this property-test is not as general as varying also the size of the population.

\paragraph{Increasing confidence}
Another approach would be to increase the confidence in our t-tests e.g. to 99\% ($\alpha = 0.01$). When doing so ($\Delta t = 0.01$), 100 tests will pass and an average of 5.6 (2.9) fail with 4.6\% (2.9) of failure.

%
%1. +++ OK, passed 100 tests; 4 failed (3%).
%2. +++ OK, passed 100 tests; 5 failed (4%).
%3. +++ OK, passed 100 tests; 9 failed (8%).
%4. +++ OK, passed 100 tests; 3 failed (2%).
%5. +++ OK, passed 100 tests; 5 failed (4%).
%6. +++ OK, passed 100 tests; 10 failed (9%).
%7. +++ OK, passed 100 tests; 8 failed (7%).
%8. +++ OK, passed 100 tests; 2 failed (1%).
%9. +++ OK, passed 100 tests; 8 failed (7%).
%10 +++ OK, passed 100 tests; 2 failed (1%).
%
%[4,5,9,3,5,10,8,2,8,2]
%[3,4,8,2,4,9,7,1,7,1]

Although it seems that we have finally found a test configuration with a sufficiently low percentage of failure, changing the confidence can be problematic though. By increasing it we lower the risk of rejecting a test which matches the SD dynamics (type I error). On the other hand increasing the confidence also increases the risk of not rejecting tests which do not match the SD dynamics (type II error). 

\paragraph{Comparison to event-driven implementation}
We also implemented this property-test for our event-driven SIR implementation from Chapter \ref{sec:eventdriven_sir}. It has the main advantage that it does not suffer from the sampling issues of the time-driven approach and thus does not require the selection of an optimal $\Delta t$.
The results match on average the ones from the time-driven approach. Running it normally roughly matches the dynamics of $\Delta t = 0.01$, a fixed population size of 100 also matches the time-driven approach with a random population of size 100 and $\Delta t = 0.01$. Further we also got the same results when increasing confidence to 99\% ($\alpha = 0.01$) as in the time-driven approach.

%1. +++ OK, passed 100 tests; 25 failed (20%).
%2. +++ OK, passed 100 tests; 22 failed (18%).
%3. +++ OK, passed 100 tests; 26 failed (20%).
%4. +++ OK, passed 100 tests; 15 failed (13%).
%5. +++ OK, passed 100 tests; 29 failed (22%).
%6. +++ OK, passed 100 tests; 12 failed (10%).
%7. +++ OK, passed 100 tests; 17 failed (14%).
%8. +++ OK, passed 100 tests; 16 failed (13%).
%9. +++ OK, passed 100 tests; 13 failed (11%).
%10. +++ OK, passed 100 tests; 14 failed (12%).
%
%[25,22,26,15,29,12,17,16,13,14]
%>> mean (x)
%ans =  18.900
%>> std (x)
%ans =  6.0818

%TODO: what about using a fixed size random population instead of random size random population? this should increase the smoothness when always using e.g. 1000 or 10.000
%1. +++ OK, passed 100 tests; 30 failed (23%).
%2. +++ OK, passed 100 tests; 29 failed (22%).
%3. +++ OK, passed 100 tests; 44 failed (30%).
%4. +++ OK, passed 100 tests; 40 failed (28%).
%5. +++ OK, passed 100 tests; 41 failed (29%).
%6. +++ OK, passed 100 tests; 41 failed (29%).
%7. +++ OK, passed 100 tests; 38 failed (27%).
%8. +++ OK, passed 100 tests; 36 failed (26%).
%9. +++ OK, passed 100 tests; 30 failed (23%).
%10. +++ OK, passed 100 tests; 46 failed (31%).
%
%[30,29,44,40,41,41,38,36,30,46]
%>> mean (x)
%ans =  37.500
%>> std (x)
%ans =  6.0782

%TODO: confidence 0.99
%1. +++ OK, passed 100 tests; 6 failed (5%).
%2. +++ OK, passed 100 tests; 5 failed (4%).
%3. +++ OK, passed 100 tests; 7 failed (6%).
%4. +++ OK, passed 100 tests; 1 failed (0%).
%5. +++ OK, passed 100 tests; 5 failed (4%).
%6. +++ OK, passed 100 tests; 6 failed (5%).
%7. +++ OK, passed 100 tests; 6 failed (5%).
%8. +++ OK, passed 100 tests; 5 failed (4%).
%9. +++ OK, passed 100 tests; 10 failed (9%).
%10. +++ OK, passed 100 tests; 6 failed (5%).
%
%[6,5,7,1,5,6,6,5,10,6]
%>> mean (x)
%ans =  5.7000
%>> std (x)
%ans =  2.2136

This is a \textit{strong} indication, that although their underlying implementation technique is different, both implementations produce qualitatively same dynamics. It would be interesting to compare both implementations on using property-based testing, using an unpaired two-tailed t-test. We leave this for further research.

\section{Discussion}
By using QuickCheck, we showed how to connect the ABS implementation to the SD specification by deriving a property, based on the SD specification. This property is directly expressed in code and tested through generating random test-cases with random agent populations. We assumed that the underlying SIR implementation, more specific, that all agent behaviour, is correct - we explore testing of individual agent behaviour in the later chapters.

Although our initial idea of matching the ABS implementation to the SD specifications has not worked out in an exact way, we still showed a way of formalizing and expressing these relations in code and testing them using QuickCheck. By allowing failure in our tests using the \textit{maxFailPercent} parameter, we confirmed the importance of selecting an optimal $\Delta t$ as already pointed out in Chapter \ref{sub:timedriven_results}. By having measure of failure, we can use a property-test also as a way of systematically searching for the smallest optimal $\Delta t$, relieving us from making conservative guesses. For the event-driven implementation, there is no such issue and we have verified that it produces qualitatively the same results.

The results showed that the ABS implementation comes close to the original SD specification but does not match it exactly - it is indeed richer in its dynamics as \cite{macal_agent-based_2010, figueredo_comparing_2014} have already shown. Our approach might work out better for a different model, which has a better behaved underlying specification than the bimodal SIR. % for which a proper statistical analysis is not the aim and focus of this thesis is left for other researchers to dwell upon.