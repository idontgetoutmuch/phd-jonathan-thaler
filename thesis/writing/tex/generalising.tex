\chapter{The structure of ABS computation}
generalising the structure of agent computation - with our case studies we explore them in a more practical / applied way and in this chapter we extract and distil the general concepts and abstractions behind agent computation: how can ABS, which is pure computation, can be seen structurally? This gives the ABS field for the first time a deeper understanding of the deeper structure of the computations behind agent-based simulation, which has so far always been more ad-hoc without a proper, more rigorous formulation. 

Note that agent-based simulation is almost always entirely pure computation without the need for direct, synchronous user-interaction or impure IO. When IO is really needed we can keep purity by creating IO actions and pass them to the simulation kernel which executes them and communicates the result back if needed - in this case only the simulation kernel needs to run in IO monad but not the agents and the environment computations.

agentout as monoid with writer: solves the Problem of iteratively constructing it the output during an event.

BUT: isnt our approach similar to the early days IO of Haskell with continuations? if this is the case we should be able to get the direct method style by writing an agent monad?

TODO: this is still research which needs to be done by reading the papers below and reflecting  and understanding on co-monads and my implementations in general.

TODO: can we derive an agent-monad?

TODO: what about comonads? read essence dataflow paper \cite{uustalu_essence_2006}: monads not capable of stream-based programming and arrows too general therefor comonads, we are using msfs for abs therefore streambased so maybe applicable to our approach/agents=comonads. comonads structure notions of context-dependent computation or streams, which ABS can be seen as of. this paper says that monads are not capable of doing stream functions, maybe this is the reason why i fail in my attempt of defining an ABS in idris because i always tried to implement a monad family. TODO: stopped at comonad section, continue from there. TODO understand comonads: https://www.schoolofhaskell.com/user/edwardk/cellular-automata and https://kukuruku.co/post/cellular-automata-using-comonads/

independent of time-driven or event-driven, our agents are MSFs.

in fact i am deriving pure functional objects

TODO: i have the feeling that co-algebras might be an underlying structure, which in CS come up in infinite streams - ABS can be seen as this where the agents are such streams with their output and potentially running for an infinite time, depending on the model. Ionescus thesis might reveal more information / might be an additional source on that.

In general it is easy to see why agents can not be represented by pure functions: they change over time. This is precisely what pure functions cannot do: they can't rely on some surrounding context / or on history - everything what they do is determined by their input arguments and their output. In general we have two ways of approaching this: we either have the agents changing data and behaviour internalised as we did in the previous chapters or we externalise it e.g. in the simulation kernel and provide all necessary information through arguments which was the case in the sugarscape environment.

TODO: FREE MONADS % http://www.haskellforall.com/2012/07/purify-code-using-free-monads.html http://comonad.com/reader/2011/free-monads-for-less/, https://stackoverflow.com/questions/13352205/what-are-free-monads
