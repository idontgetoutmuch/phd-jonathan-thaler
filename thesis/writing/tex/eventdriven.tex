\chapter{Pure Functional Event-Driven ABS}
\label{ch:eventdriven}

In this chapter we build on the previous discussion of update-strategies in Chapter \ref{ch:impl_abs} and the implementation techniques presented in the time-driven approach of Chapter \ref{ch:timedriven} to develop concepts for event-driven ABS in a pure functional way. 

In event-driven ABS \cite{meyer_event-driven_2014}, the simulation is advanced through events: agents and the environment schedule events into the future and react to incoming events scheduled by themselves, other agents, the environment or the simulation kernel. Time is discrete in this approach: it advances step-wise from event to event, where each event has an associated time-stamp, indicating the virtual simulation time when it is scheduled. This implies that time could stay constant e.g. when an event is scheduled with a time-delay of 0 the virtual simulation time does not advance. Because agents can adopt and change their state and behaviour when processing an event, this means that even if time does not advance, agents can change. This non-signal behaviour is the fundamental difference to the time-driven approach in Chapter \ref{ch:timedriven}. Further, we exploit this mechanism to implement direct agent-interactions pure functionally as discussed in the implementation discussion below.

The event-driven approach makes the simulation kernel technically closely related to a Discrete Event Simulation (DES) \cite{zeigler_theory_2000}. Due to the necessity of imposing a correct ordering of events in this type of ABS, we need to step it event by event, with the \textit{sequential} update-strategy, as introduced in Chapter \ref{sec:seq_strategy}, being the only feasible one for this type of ABS. Note that there exists also Parallel DES (PDES) \cite{fujimoto_parallel_1990}, which processes events in parallel and deals with inconsistencies by reverting to consistent states - we hypothesize that a pure functional approach could be beneficial in such an approach due to persistent data-structures and explicit handling of side-effects but we leave this for further research.

\medskip

First we start using an event-driven agent-based SIR model, inspired by \cite{macal_agent-based_2010} to introduce the concepts of agent identity and also event scheduling utilising both \textit{Reader} and \textit{Writer} Monads which we then generalise to a \textit{tagless final} approach. Note that in event-driven ABS, due to the fact that agents are not signals any more, we abandon time-aware signal functions from the previous chapter and focus solely on Monadic Stream Functions (MSFs).

We then use the highly complex Sugarscape model as introduced in Chapter \ref{sec:sugarscape}, to develop more advanced features of ABS in a pure functional context: dynamic creation and removal of agents during simulation, reading/writing from a shared mutable environment, local mutable agent state and synchronised direct agent-interactions. 
Note that the Sugarscape model is not a real event-driven model like the event-driven SIR one is: in it the agents do schedule events but they don't do this into the future - events in Sugarscape don't have associated time-stamps.

\section{Implementation Approaches}
5 Pages

This is now very programming-language specific

\begin{itemize}
	\item Mapping the strategies to 3 programming-languages: Java, Scala with Actors, Haskell
	\item Comparing the programming languages in regard of their suitability to implement each of these strategies
	\item Screen-shots of results of the same simulation-model with all the strategies
\end{itemize}

\section{Event-Driven SIR}
\label{sec:eventdriven_sir}
This short section shows how to implement the SIR model, as introduced in Chapter \ref{sec:sir_model}, with an event-driven approach. This is in stark contrast to the time-driven implementation in Chapter \ref{sec:timedriven_firststep}. The solutions are quantitatively equal as they produce the same class of dynamics. Qualitatively they fundamentally differ though in terms of expressivity and performance as we will see below.

To keep this section simple, we reduce code-examples as far as possible and focus on the most fundamental differences to the approach used in Sugarscape. An agent in the event-driven SIR has no output (that is: the return-type of the MSF is the empty tuple ()), because the SIR model is much less dynamic than the Sugarscape one: agents don't spawn other agents and agents can't die. Further there is no environment (see Chapter \ref{sec:timedriven_firststep} how to add an environment to the SIR model) and observable dynamics happen not through agent-output but through side-effects.

The very heart of this implementation is the simulation state, which holds (amongst others) a priority queue and a tuple with the number of susceptible, infected and recovered agents. Agents schedule events with a time-stamp and receiver agent id, using this priority queue, which the simulation kernel processes then in order of the time-stamps to run the next agent. This is conceptionally very close to the Sugarscape implementation but events have now an additional time-stamp, which indicates the time when they are about to be scheduled - the priority queue is sorted according to the time-stamps and the simulation kernel simply processes them in order. When agents change their state they also increment / decrement the number of susceptible / infected / recovered agents, depending on which transition they make.

This makes the structure of the whole implementation much smaller than the one of Sugarscape: there is no local monad transformer stack to the agent, only a global one, which holds the simulation state as described above and the random-number monad. To get a feeling on the different approach between the Sugarscape and the SIR we show the initial function of the SIR agent. It is called by the simulation kernel to schedule initial events, adjust the simulation state and get the agents MSF.

\begin{HaskellCode}
-- | A sir agent is in one of three states
sirAgent :: RandomGen g 
         => SIRState    -- ^ the initial state of the agent
         -> SIRAgent g  -- ^ the continuation
sirAgent Susceptible aid = do
    modifyDomainState incSus -- increment number of susceptible agents
    scheduleEvent aid MakeContact makeContactInterval -- schedule make contact event to self
    return (susceptibleAgent aid) -- return susceptible MSF
sirAgent Infected aid = do
    modifyDomainState incInf -- increment number of infected agents
    dt <- lift (randomExpM (1 / illnessDuration)) -- draw random illness duration
    scheduleEvent aid Recover dt -- schedule recovery to self
    return (infectedAgent aid) -- returns infected MSF 
sirAgent Recovered _ = do
    modifyDomainState incRec -- increment number of recovered agents
    return recoveredAgent -- return recovered MSF
\end{HaskellCode}

In the next sections we have a quick look at how we translate the time-driven susceptible, infected and recovered behaviours of into event-driven behaviours.

\subsection{Susceptible Agent}
We use the same switch mechanism of making the transition from a susceptible to an infected agent in case of an infection but how a susceptible agent gets infected works now different:

\begin{itemize}
	\item A susceptible agent initially schedules a \textit{MakeContact} event with $\Delta t = 1$ to itself.
	\item When receiving \textit{MakeContact}, the agent sends a \textit{Contact} event to 5 random other agents with $\Delta t = 0$. This will result in these events to be scheduled immediately. Further the agent schedules \textit{MakeContact} with $\Delta t = 1$ to itself.
	\item When the agent receives a \textit{Contact} event, it checks if it is from an infected agent. If the event is not from an infected agent, it ignores it. Otherwise it becomes infected with a given probability. In case of infection the agent decrements the number of susceptible and increments the number of infected agents.
\end{itemize}

\subsection{Infected Agent}
We use the same switch mechanism of making the transition from an infected to a recovered agent after the illness duration. The main difference is that agents send \textit{Contact} events to each other to indicate that contacts have happened. Thus susceptible and infected agents need to react to incoming \textit{Contact} events.

\begin{itemize}
	\item An infected agent initially schedules a \textit{Recover} event with a random $\Delta t$ (following exponential distribution) to itself.
	\item When the agent receives a \textit{Contact} event, it checks if it is from a susceptible agent. If the event is not from a susceptible agent, it ignores it. Otherwise it simply replies to this susceptible agent with a \textit{Contact} event with $\Delta t = 0$.
\end{itemize}

\subsection{Recovered Agent}
The recovered agent does not change any more, reacts to no incoming events and schedules no events - it stays constant forever and thus outputs the empty tuple forever.

\subsection{Reflections}
Transforming a time-driven into an event-driven approach should always be possible because the ability to schedule events with time-stamps allows to map all features of time-driven ABS to an event-driven one - the discussion above should give a good direction of how this process works. Still for some models one can argue that the time-driven approach is much more expressive than an event-driven one, and we think this is certainly the case for the SIR model. The event-driven approach leads to much more fragmented logical flow and agent behaviour.

The event-driven implementation from this Chapter is around 60 - 70\% faster than the time-driven implementation from Chapter \ref{sec:timedriven_firststep}, which is non-monadic and uses the FRP library Yampa. For the monadic time-driven approach of Chapter \ref{sec:adding_env} the difference is much more dramatic: it is about 700 - 800\% slower. These results dramatically highlight the problem of time-driven ABS: its performance cannot compete with an even-driven approach. This is exaggerated even more so when making use of MSFs as in Chapter \ref{sec:adding_env}. In this case, a time-driven approach becomes extremely expensive in terms of performance and one should consider an event-driven approach. In case the model is specified in a time-driven way, a transformation into an event-driven approach should always be possible as outlined above.

\section{Discussion}
Although there are similarities to the work of \cite{botta_time_2010} (the use of messages and the problem of when to advance time in models with arbitrary number synchronised agent-interactions), we approach our agents differently. First in our approach an agent is only a single MSF and thus can not be directly queried for its internal state / its id or outgoing messages, instead of taking a list of messages, our agents take a single event/message and can produce an arbitrary number of outgoing messages together with an observable state - note that this would allow to query the agent for its id and its state as well by simply sending a corresponding message to the agents MSF and requiring the agent to implement message handling for it. Also the state of our agents is \textit{completely} localised and there is no means of accessing the state from outside the agent, they are thus "fully encapsulated agents" \cite{botta_time_2010}. Note that the authors of \cite{botta_time_2010} define their agents with a polymorphic agent-state type \textit{s}, which implies that without knowledge of the specific type of \textit{s} there would be no way of accessing the state, rendering it in fact also fully encapsulated. The problem of advancing time in our approach is solved not exactly the same but conceptually it is the same: after sending a tick message to each agent (in random order), we process all agents until they are idle: there are no more enqueued messages / events in the queue.

our eventdriven approach makes heavy use of 2 state monads, thus one might ask what the benefits are, after all we seem to fall back into stateful, imperative style programming. we agree that our approach is just one way of implementing abs in fp but we think we have come a long way thus making our approach quite valuable even if there might be other approaches like shallow EDSLs. on the other hand even our stateful programming is highly restricted to only those 2 local datatypes which makes it much more manageable than unrestricted data mutation

quote carmack (\url{http://www.gamasutra.com/view/news/169296/Indepth_Functional_programming_in_C.php}): the main difficulty as a developer in software programming is to keep track of the states a program can be in and reason about them and their Validity

TODO: report LoC and compare it with other implementations we found on the internet