\section{Pure functional programming}
\label{sec:background_fp}
Functional programming is called \textit{functional} because it makes functions the main concept of programming, promoting them to first-class citizens: functions can be assigned to variables, they can be passed as arguments to other functions and they can be constructed as return values from functions. The roots of functional programming lie in the Lambda Calculus which was first described by Alonzo Church \cite{church_unsolvable_1936}. This is a fundamentally different approach to computing than imperative programming (including established object-orientation)  which roots lie in the Turing Machine \cite{turing_computable_1937}. Rather than describing \textit{how} something is computed as in the more operational approach of the Turing Machine, due to the more \textit{declarative} nature of the Lambda Calculus, code in functional programming describes \textit{what} is computed.

In \cite{maclennan_functional_1990} the author defines functional programming as a methodology attributing the following properties to it: programming without the assignment-operator; allowing for higher levels of abstraction; allowing to develop executable specifications and prototype implementations; connected to computer science theory; allowing to do algebraic reasoning. Further the author makes the subtle distinction between \textit{applicative} and \textit{functional} programming. Applicative programming can be understood as applying values to functions where one deals with pure expressions. In those expressions the value is independent from the evaluation order, also known as referential transparency. This means that such functions have no side effects and thus the outcome of their execution does not depend on the history or context of the system. Further, inputs and effects to an operation are obvious from the written form.

Note that applicative programming is not necessarily unique to the functional programming paradigm but can be emulated in an imperative language e.g. C as well. Functional programming is then defined by \cite{maclennan_functional_1990} as applicative programming with \textit{higher-order} functions. These are functions which operate themselves on functions: they can take functions as arguments, construct new functions and return them as values. This is in stark contrast to the \textit{first-order} functions as used in applicative or imperative programming which just operate on data alone. Higher-order functions allow to capture frequently recurring patterns in functional programming in the same way like imperative languages captured patterns like GOTO, while-do, if-then-else, for. Common patterns in functional programming are (amongst others) the map, fold, zip, operators. So functional programming is not really possible in this way in classic imperative languages e.g. C as you cannot construct new functions and return them as results from functions. Object-Oriented languages like Java provide mechanisms, allowing to partially work around this limitation but are still far from \textit{pure} functional programming.

The equivalence in functional programming to the \textit{;} operator of imperative programming, which allows to compose imperative statements, is function composition. Function composition has no side effects as opposed to the imperative ; operator, which simply composes destructive assignment statements executed after another resulting in side effects.
At the heart of modern functional programming is monadic programming which is polymorphic function composition: one can implement a user-defined function composition by allowing to run some code in-between function composition - this code of course depends on the type of the Monad one runs in. This allows to emulate all kind of effectful programming in an imperative style within a pure functional language (see Section \ref{sec:purity_sideeffects} below). Although it might seem strange following an imperative style in a pure functional language, some problems are inherently imperative in the way that computations need to be executed in a given sequence with some effects. Also a pure functional language needs to have some way to deal with effects otherwise it would never be able to interact with the outside-world and would be practically useless. The real benefit of monadic programming is that it is explicit about side effects and allows only effects which are fixed by the type of the monad - the side effects which are possible are determined statically during compile time by the type system. Some general patterns can be extracted e.g. a map, zip, fold over monads which results in polymorphic behaviour - this is the meaning when one says that a language is polymorphic in its side effects.

\subsection{Language of choice}
In our research we are using the \textit{pure} functional programming language Haskell. The paper of \cite{hudak_history_2007} gives a comprehensive overview over the history of the language, how it developed and its features and is very interesting to read and get accustomed to the background of the language. The reasons for choosing Haskell are:

\begin{itemize}
	\item Rich Feature-Set - it has all fundamental concepts of the pure functional programming paradigm included, of which we explain the most important ones below. Further, Haskell has influenced a large number of languages, underlining its importance and influence in programming language design.
	
	\item Real-World applications - the strength of Haskell has been proven through a vast amount of highly diverse real-world applications \cite{hudak_haskell_1994, hudak_history_2007}, is applicable to a number of real-world problems \cite{osullivan_real_2008} and has a large number of libraries available \footnote{\url{https://wiki.haskell.org/Applications_and_libraries}}.
	
	\item Modern - Haskell is constantly evolving through its community and adapting to keep up with the fast changing field of computer science. Further, the community is the main source of high-quality libraries.
	
	\item Highly advance type system - Haskell has a strong static type system, which catches all type errors already at compile time and does not allow to bypass the type system (unless \textit{coerce} or other cheating functions like \textit{unsafePerformIO} are used). Further, Haskell is a \textit{pure} functional language and in our research it is absolutely paramount, that we focus on \textit{pure} functional ABS, which avoids any IO type under all circumstances. This property is enabled by the advanced type system and its strong static nature.
\end{itemize}

A highly compelling example motivating the benefits of pure functional programming is the report \cite{hudak_haskell_1994}, where in a prototyping contest of DARPA the Haskell prototype was by far the shortest with 85 lines of code (LoC) as compared to the C++ solution with 1105 LoC. The remarkable thing is that the Jury mistook the Haskell code as specification because its approach was to implement a small embedded domain specific language (EDSL) to solve the problem - this is a perfect proof how close an EDSL can get to a specification. When implementing an EDSL one develops and programs primitives e.g. types and functions in a host language (embed) in a way that they can be combined. The combination of these primitives then looks like a language specific to a given domain. The ease of development of EDSLs in pure functional programming is also a proof of the superior extensibility and composability of pure functional languages over object-orientation and is definitely one of its major strength. The classic paper \cite{henderson_functional_1982} gives a wonderful way of constructing an EDSL to denotationally construct a picture reminiscent of the works of Escher. A major strength of developing an EDSL is that one can reason about and do formal verification. A nice introduction how to do reasoning in Haskell is given in \cite{hutton_tutorial_1999}.

% SHORTENING
It may seem that one runs into efficiency-problems in Haskell when using algorithms which are implemented in imperative languages through mutable data which allows in-place update of memory. The seminal work of \cite{okasaki_purely_1999} showed that when approaching this problem with a functional mind-set this does not necessarily be the case. The author presents functional data structures which are asymptotically as efficient as the best imperative implementations and discusses the estimation of the complexity of lazy programs.

% SHORTENING
For an excellent and widely used introduction to programming in Haskell I refer to \cite{hutton_programming_2016}. Other, more exhaustive books on learning Haskell are \cite{allen_haskell_2016, lipovaca_learn_2011}. For an introduction to programming with the Lambda-Calculus I refer to \cite{michaelson_introduction_2011}. For more general discussion of functional programming I refer to \cite{hudak_history_2007,hughes_why_1989,maclennan_functional_1990}.

\subsection{Purity and side effects}
\label{sec:purity_sideeffects}
Consider the factorial function in Haskell:
\begin{HaskellCode}
factorial :: Integer -> Integer
factorial 0 = 1
factorial n = n * factorial (n-1)
\end{HaskellCode}

When looking at this function, the following can be identified: 
\begin{enumerate}
	\item Declarative - describe \textit{what} the factorial function is rather than how to compute it. This is supported by \textit{pattern matching} which allows to give multiple equations for the same function, matching on its input. 
	
	\item Immutable data - in functional programming there are no mutable variables - after a variable is assigned, it cannot change its contents. This also means that there is no destructive assignment operator which can re-assign values to a variable. To change values, recursion is employed.
	
	\item Recursion - the function calls itself with a structurally smaller argument and will eventually reach the base case of 0. Recursion is the very meat of functional programming because it is the only way to implement loops in this paradigm due to immutable data.
	
	\item Static Types - the first line indicates the name and the type of the function. In this case the function takes one Integer as input and returns an Integer as output. Types are static in Haskell which means that there can be no type errors at run-time e.g. when one tries to implicitly cast one type into another because this is not supported by this kind of type system.

	\item Explicit input and output - all data which are required and produced by the function have to be explicitly passed in and out of it. There exists no global mutable data whatsoever and data-flow is always explicit.
	
	\item Referential transparency - calling this function with the same argument will \textit{always} lead to the same result, meaning one can replace this function by its value. This means that when implementing this function one can not read from a file or open a connection to a server. This is also known as \textit{purity} and is indicated in Haskell in the types which means that it is also guaranteed by the compiler.
\end{enumerate}

One of the fundamental strengths of Haskell is its way of dealing with side effects in functions. A function with side effects has observable interactions with some state outside of its explicit scope. This means that its behaviour depends on the history of the system and that it loses its referential transparency character, which makes understanding and debugging much harder. Examples for side effects are (amongst others): modifying state, await an input from the keyboard, read or write to a file, open a connection to a server, drawing random-numbers,...

Obviously, to write real-world programs which interact with the outside world needs side effects. Haskell allows to indicate in the \textit{type} of a function that it does or does \textit{not} have side effects. Further, there are a broad range of different effect types available, to restrict the possible effects a function can have to only the required type. This is then ensured by the compiler, which means that when trying to read from a file in a function which only allows drawing random-numbers, will fail to compile. The most common side effect types are: \textit{IO} allows all kind of I/O related side effects: reading/writing a file, creating threads, write to the standard output, read from the keyboard, opening network-connections, mutable references; \textit{Rand}  allows drawing random-numbers; \textit{Reader / Writer / State} allows to read / write / both from / to an environment.

A function without any side effect type is called \textit{pure}, and the \textit{factorial} function discussed above is indeed pure. Below we give an example of a function which is not pure. The \textit{queryUser} function \textit{constructs} a computation which, when executed, asks the user for its user-name and compares it with a given user-configuration. In case the user-name matches it returns True, and False otherwise after printing a corresponding message. 

\begin{HaskellCode}
queryUser :: String -> IO Bool
queryUser username = do
  -- print text to console
  putStr "Type in user-name: "
  -- wait for user-input
  str <- getLine
  -- check if input matches user-name
  if str == username
    then do
      putStrLn "Welcome!"			
      return True
    else do
      putStrLn "Wrong user-name!"
      return False
\end{HaskellCode}

The \textit{IO} in the first line indicates that the function runs in the IO effect and can thus (amongst others) print to the console and read input from it. What seems striking is that this looks very much like imperative code - this is no accident and intended. When we are dealing with side effects, ordering becomes important, thus Haskell introduced the so-called \textit{do} notation which emulates an imperative style of programming. Whereas in imperative programming languages like C, commands are chained or composed together using the ; operator, in functional programming this is done using function composition: feeding the output of a function directly into the next function. The machinery behind the \textit{do} notation does exactly this and desugars this imperative-style code into function compositions which run custom code between each line, depending on the type of effect the computation runs in. This approach of function composition with custom code in between each function allows to emulate a broad range of imperative-style effects, including the above mentioned ones. For a technical, in-depth discussion of the concept of side effects and how they are implemented in Haskell using Monads, we refer to the following papers: \cite{jones_tackling_2002,moggi_computational_1989,wadler_essence_1992,wadler_monads_1995,wadler_how_1997}.

\medskip

Although it might seem very restrictive at first, we get a number of benefits from making the type of effects we can use in the function explicit. First, we can restrict the side effects a function can have to a very specific type which is guaranteed at compile time. This means we can have much stronger guarantees about our program and the absence of potential errors already at compile time. Second, because running effects themselves is \textit{pure}, we can execute effectful functions in a very controlled way by making the effect-context explicit in the parameters to the effect execution. This allows a much easier approach to isolated testing because the history of the system is made explicit. 

A note on effect execution: it is important to understand that the code fragments of effectful computations are in fact  made up of enclosing lambda expressions, with the \textit{do} notation being a syntactic sugared version. Thus functions which have an effect in their type can be seen as \textit{pure} functions, which are referentially transparent and return such a fragment. This fragment, also often called \textit{action}, results in an effect and a result when executed. We have to distinguish between the execution of pure effects like Rand,Read,Write,State and the impure effect of IO. Pure effects are executed using special runner functions. They take an action together with initial values defining the history / context of the effect e.g. an initial value for the State or the read-only value of the Reader, and run the action and return their result value. Thus, these pure effects can be executed in a referential transparent and completely controlled way. The impure IO effect works different: there exists no dedicated IO execution function but it can only be executed from within the root IO action, which emanates from the \textit{main :: IO ()} function of each Haskell program. Thus IO actions can only be run within an enclosing IO action, with the main IO action ultimately being executed by the Haskell Runtime which is linked against the executable. The reason for that is that if we would have a way of executing IO actions within pure code we would lose all guarantees about referential transparency. There exists indeed the function \textit{unsafePerformIO :: IO a $\rightarrow$ a}, which allows to execute an IO action within a pure function but its use is very limited and highly discouraged. Throughout this thesis and in all our code we have avoided the use of this function under all costs and it is not used anywhere.

\subsubsection{Stacking effects using Monad Transformers}
Often it is necessary to have multiple effects available, for example we want to manipulate a global state, write to some logging mechanism and need to be able to draw random numbers. The way this is achieved in Haskell is by using Monad Transformers \cite{jones_functional_1995}, for which Haskell provides the two libraries \textit{mtl} and \textit{transformers}, which basically achieve the same things with slightly different philosophies. In our approach we primarily use \textit{mtl} as it allows to overload functions with monadic typeclasses as explained below.

\medskip

Although Monads share a common interface and properties, it is not possible to compose Monads in a general way as each monad has different internals and thus it always depends on the Monad how to compose it into another arbitrary Monad \footnote{The technical details are quite involved but we don't go into that here and refer to the respective literature and tutorials on Monad Transformers}. Thus, the \textit{mtl} library provides so called transformer implementations of each of the standard Monads. A transformer has an additional type parameter in its type-constructor which has to be a Monad or another Transformer. This allows to stack multiple Monads/Transformers on top of each other. The stack is closed by using a normal Monad. Note that in \textit{mtl} all non-transformer Monads are actually transformers with the Identity Monad as type parameter e.g. \textit{State Int} is \textit{StateT Int Identity}. Access to the various layers of the stack is achieved with the \textit{lift} function. Lets look at how we can define the type of a function which has multiple effects available:

\begin{HaskellCode}
data SimState = SimState 
  { simStateAgents :: [SimAgent]
  ...
  }

simulationCore :: RandomGen g 
               => Time
               -> StateT SimState (WriterT [String] (Rand g)) SimOut
simulationCore t = do
  -- get the agents from the simulation state 
  -- encapsulated in StateT SimState
  as <- gets simStateAgents
  
  -- writing a logging output to the WriterT [String]
  -- here we need 1 lift 
  lift (tell ["Next step " ++ show t])
  
  -- shuffle agents by running the MonadRandom action using the
  -- Rand Monad, need 2 lifts as it is the outermost monad
  asShuf <- lift $ lift $ randomShuffle as
  
  let simOut = ...
  
  return simOut
  
randomShuffle :: MonadRandom m => [a] -> m [a]
\end{HaskellCode}

The Monad stack consists of three effects: the \textit{StateT} with \textit{SimState} as its internal state as the innermost Monad. Note that although in the types it is the outermost, in terms of the transformer stack it is the innermost, requiring no \textit{lift} to access. \textit{WriterT} with \textit{[String]} as the logging facility is a parameter to the \textit{StateT} transformer, making it the second effect in the stack, thus requiring one \textit{lift}. The stack is closed using the \textit{Rand} Monad, which is the outermost effect, requiring two \textit{lifts} to access it.

Note the use of \textit{randomShuffle} and its type. It is an overloaded function, which has the \textit{MonadRandom} typeclass in its type-constraints. This indicates that it is a monadic action where \textit{m} is of type \textit{MonadRandom}, which supports the same functionality as \textit{Rand}. This is the major benefit \textit{mtl} provides, which often results in much cleaner function types and does not require to fix the order of the Monads in the stacks. Another benefit is that we do not need lifts any more; the drawback is that we cannot have multiple Monads of the same type which would be still possible in a fully qualified Monad stack. The benefits becomes particularly clear when more than one effect is required, for example we could have written the type of \textit{simulationCore}

\begin{HaskellCode}
simulationCore :: (MonadState SimState m, MonadWriter [String] m, MonadRandom m) 
               => Time -> m SimOut
\end{HaskellCode}

A final note on commutativity on Monad Transformers: because we are stacking effects on top of each other, subsequent effects can change the final outcome, depending on their position within the stack - this is called commutativity of Monads. All the Monads in the example above commute, which means it does not matter where they are positioned in the stack, the outcome will be the same. An exception to this is the \textit{MaybeT} transformer, which introduces failure as an effect within a stack, thus when failure occurs, subsequent effects will not be applied any more, making \textit{MaybeT} non-commutative. 

\subsection{Functional Reactive Programming}
\label{sec:back_frp}
Functional Reactive Programming (FRP) is a way to implement systems with continuous and discrete time-semantics in pure functional languages. There are many different approaches and implementations but in this thesis \textit{Arrowized} FRP \cite{hughes_generalising_2000, hughes_programming_2005} as implemented in the library Yampa \cite{courtney_yampa_2003,hudak_arrows_2003,nilsson_functional_2002} is used.

The central concept in Arrowized FRP is the Signal Function (SF), which can be understood as a \textit{process over time} which maps an input- to an output-signal. A signal can be understood as a value which varies over time. Thus, signal functions have an awareness of the passing of time by having access to $\Delta t$ which are positive time steps, the system is sampled with. 

\begin{flalign*}
Signal \, \alpha \approx Time \rightarrow \alpha \\
SF \, \alpha \, \beta \approx Signal \, \alpha \rightarrow Signal \, \beta 
\end{flalign*}

Yampa provides a number of combinators for expressing time-semantics, events and state-changes of the system. They allow to change system behaviour in case of events, run signal functions and generate stochastic events and random-number streams. Below, the relevant combinators and concepts used throughout the thesis are discuss briefly. For a more in-depth discussion I refer to \cite{courtney_yampa_2003, hudak_arrows_2003, nilsson_functional_2002}.

\paragraph{Event}
An event in FRP is an occurrence at a specific point in time, which has no duration e.g. the recovery of an infected agent. Yampa represents events through the \textit{Event} type, which is programmatically equivalent to the \textit{Maybe} type. 

\paragraph{Dynamic behaviour}
To change the behaviour of a signal function at an occurrence of an event during run-time, (amongst others) the combinator \textit{switch :: SF a (b, Event c) $\rightarrow$ (c $\rightarrow$ SF a b) $\rightarrow$ SF a b} is provided. It takes a signal function, which is run until it generates an event. When this event occurs, the function in the second argument is evaluated, which receives the data of the event and has to return the new signal function, which will then replace the previous one. Note that the semantics of \textit{switch} are that the signal function, into which is switched, is also executed at the time of switching.

\paragraph{Randomness}
In ABS, often there is the need to generate stochastic events, which occur based on e.g. an exponential distribution. Yampa provides the combinator \textit{occasionally :: RandomGen g $\Rightarrow$ g $\rightarrow$ Time $\rightarrow$ b $\rightarrow$ SF a (Event b)} for this. It takes a random-number generator, a rate and a value the stochastic event will carry. It generates events on average with the given rate. Note that at most one event will be generated and no 'backlog' is kept. This means that when this function is not sampled with a sufficiently high frequency, depending on the rate, it will lose events.

Yampa also provides the combinator \textit{noise :: (RandomGen g, Random b) $\Rightarrow$ g $\rightarrow$ SF a b}, which generates a stream of noise by returning a random number in the default range for the type \textit{b}.

\paragraph{Running signal functions}
To \textit{purely} run a signal function Yampa provides the function \textit{embed :: SF a b $\rightarrow$ (a, [(DTime, Maybe a)]) $\rightarrow$ [b]}, which allows to run an SF for a given number of steps where in each step one provides the $\Delta t$ and an input \textit{a}. The function then returns the output of the signal function for each step. Note that the input is optional, indicated by \textit{Maybe}. In the first step at $t = 0$, the initial \textit{a} is applied and whenever the input is \textit{Nothing} in subsequent steps, the last \textit{a} which was not \textit{Nothing} is re-used.

\subsection{Arrowized programming}
Yampa's signal functions are arrows, requiring us to program with arrows. Arrows are a generalisation of monads, which in addition to the already familiar parameterisation over the output type, allow parameterisation over their input type as well \cite{hughes_generalising_2000, hughes_programming_2005}.

In general, arrows can be understood to be computations that represent processes, which have an input of a specific type, process it and output a new type. This is the reason why Yampa is using arrows to represent their signal functions: the concept of processes, which signal functions are, maps naturally to arrows.

There exists a number of arrow combinators, which allow arrowized programing in a point-free style but due to lack of space we will not discuss them here. Instead we make use of Paterson's \textit{do} notation for arrows \cite{paterson_new_2001}, which makes code more readable as it allows us to program with points.

To show how arrowized programming works, we implement a simple signal function, which calculates the acceleration of a falling mass on its vertical axis as an example \cite{perez_testing_2017}.

\begin{HaskellCode}
fallingMass :: Double -> Double -> SF () Double
fallingMass p0 v0 = proc _ -> do
  v <- arr (+v0) <<< integral -< (-9.8)
  p <- arr (+p0) <<< integral -< v
  returnA -< p
\end{HaskellCode}

To create an arrow, the \textit{proc} keyword is used, which binds a variable after which the \textit{do} of Patersons \textit{do} notation \cite{paterson_new_2001} follows. Using the signal function \textit{integral :: SF v v} of Yampa, which integrates the input value over time using the rectangle rule, we calculate the current velocity and the position based on the initial position \textit{p0} and velocity \textit{v0}. The $<<<$ is one of the arrow combinators, which composes two arrow computations and \textit{arr} simply lifts a pure function into an arrow. To pass an input to an arrow, $-<$ is used and $<-$ to bind the result of an arrow computation to a variable. Finally to return a value from an arrow, \textit{returnA} is used.

\subsection{Monadic Stream Functions}
\label{sec:back_msf}

Monadic Stream Functions (MSF) are a generalisation of Yampa's signal functions with additional combinators to control and stack side effects. An MSF is a polymorphic type and an evaluation function, which applies an MSF to an input and returns an output and a continuation, both in a monadic context \cite{perez_extensible_2017,perez_functional_2016}:
\begin{HaskellCode}
newtype MSF m a b = MSF {unMSF :: MSF m a b -> a -> m (b, MSF m a b)}
\end{HaskellCode}

MSFs are also arrows, which means we can apply arrowized programming with Patersons \textit{do} notation as well. MSFs are implemented in Dunai, which is available on Hackage. Dunai allows us to apply monadic transformations to every sample by means of combinators like \textit{arrM :: Monad m $\Rightarrow$ (a $\rightarrow$ m b) $\rightarrow$ MSF m a b} and \textit{arrM\_ :: Monad m $\Rightarrow$ m b $\rightarrow$ MSF m a b}. A part of the library Dunai is BearRiver, a wrapper, which re-implements Yampa on top of Dunai, which enables one to run arbitrary monadic computations in a signal function. BearRiver simply adds a monadic parameter \textit{m} to each SF, which indicates the monadic context this signal function runs in.

To show how arrowized programming with MSFs works, we extend the falling mass example from above to incorporate monads. In this (artificial) example we assume that in each step we want to accelerate our velocity \textit{v} not by the gravity constant any more but by a random number in the range of 0 to 9.81. Further we want to count the number of steps it takes us to hit the floor, that is when position \textit{p} is less than 0. Also when hitting the floor we want to print a debug message to the console with the velocity by which the mass has hit the floor and how many steps it took.

We define a corresponding Monad stack with \textit{IO} as the outermost Monad, followed by a \textit{RandT} transformer for drawing random-numbers and finally a \textit{StateT} transformer as innermost Monad, to count the number of steps we compute. We can access the monadic functions using \textit{arrM} in case we need to pass an argument and \textit{\_arrM} in case no argument to the monadic function is needed:

\begin{HaskellCode}
type FallingMassStack g = StateT Int (RandT g IO)
type FallingMassMSF g   = SF (FallingMassStack g) () Double

fallingMassMSF :: RandomGen g => Double -> Double -> FallingMassMSF g
fallingMassMSF v0 p0 = proc _ -> do
  -- drawing random number for our gravity range
  r <- arrM_ (lift $ lift $ getRandomR (0, 9.81)) -< ()
  v <- arr (+v0) <<< integral -< (-r)
  p <- arr (+p0) <<< integral -< v
  -- count steps
  arrM_ (lift (modify (+1))) -< ()
  if p > 0
    then returnA -< p
    -- we have hit the floor
    else do
      -- get number of steps
      s <- arrM_ (lift get) -< ()
      -- write to console
      arrM (liftIO . putStrLn) -< "hit floor with v " ++ show v ++ 
                                  " after " ++ show s ++ " steps"
      returnA -< p
\end{HaskellCode}

To run the \textit{fallingMassMSF} function until it hits the floor we proceed as follows:

\begin{HaskellCode}
runMSF :: RandomGen g => g -> Int -> FallingMassMSF g -> IO ()
runMSF g s msf = do
  let msfReaderT = unMSF msf ()
      msfStateT  = runReaderT msfReaderT 0.1
      msfRand    = runStateT msfStateT s
      msfIO      = runRandT msfRand g
  (((p, msf'), s'), g') <- msfIO
  when (p > 0) (runMSF g' s' msf')
\end{HaskellCode}

Dunai does not know about time in MSFs, which is exactly what BearRiver builds on top of MSFs. It does so by adding a \textit{ReaderT Double}, which carries the $\Delta t$. This is the reason why we need one extra lift for accessing \textit{StateT} and \textit{RandT}. Thus \textit{unMSF} returns a computation in the \textit{ReaderT Double} Monad, which we need to peel away using \textit{runReaderT}. This then results in a \textit{StateT Int} computation, which we evaluate by using \textit{runStateT} and the current number of steps as state. This then results in another monadic computation of \textit{RandT} Monad, which we evaluate using \textit{runRandT}. This finally returns an \textit{IO} computation, which we simply evaluate to arrive at the final result.