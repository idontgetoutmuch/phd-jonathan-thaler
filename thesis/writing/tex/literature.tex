\chapter{Related research and literature}
\label{ch:literature}

The amount of research on using pure functional programming with Haskell in the field of ABS has been moderate so far. Most of the papers are related to the field of Multi Agent Systems (MAS) and look into how agents can be specified using the belief-desire-intention paradigm \cite{de_jong_suitability_2014,sulzmann_specifying_2007,jankovic_functional_2007}.

A multi-method simulation library in Haskell called \textit{Aivika 3} is described in the technical report \cite{sorokin_aivika_2015}. It supports implementing Discrete Event Simulations (DES), System Dynamics and comes with basic features for event-driven ABS which is realised using DES under the hood. Further it provides functionality for adding GPSS to models and supports parallel and distributed simulations. It runs within the IO effect type for realising parallel and distributed simulation but also discusses generalising their approach to avoid running in IO.

In his master thesis \cite{bezirgiannis_improving_2013} the author investigates Haskells' parallel and concurrency features to implement (amongst others) \textit{HLogo}, a Haskell clone of the NetLogo \cite{wilensky_introduction_2015} simulation package, focusing on using STM for a limited form of agent-interactions. \textit{HLogo} is basically a re-implementation of NetLogos API in Haskell where agents run within an unrestricted context (known as \textit{IO}) and thus can also make use of STM functionality. The benchmarks show that this approach does indeed result in a speed-up especially under larger agent-populations. The authors' thesis can be seen as one of the first works on ABS using Haskell. Despite the concurrency and parallel aspect our work share, our approach is rather different: we avoid IO within the agents under all costs and explore the use of STM more on a conceptual level rather than implementing a ABS library and compare our case-studies with lock-based and imperative implementations.

There exists some research \cite{di_stefano_using_2005, varela_modelling_2004, sher_agent-based_2013} using the functional programming language Erlang \cite{armstrong_erlang_2010} to implement concurrent ABS. The language is inspired by the actor model \cite{agha_actors:_1986} and was created in 1986 by Joe Armstrong for Eriksson for developing distributed high reliability software in telecommunications. The actor model can be seen as quite influential to the development of the concept of agents in ABS, which borrowed it from Multi Agent Systems \cite{wooldridge_introduction_2009}. It emphasises message-passing concurrency with share-nothing semantics (no shared state between agents), which maps nicely to functional programming concepts. The mentioned papers investigate how the actor model can be used to close the conceptual gap between agent-specifications, which focus on message-passing and their implementation. Further they show that using this kind of concurrency allows to overcome some problems of low level concurrent programming as well.
Also \cite{bezirgiannis_improving_2013} ported NetLogos API to Erlang mapping agents to concurrently running processes, which interact with each other by message-passing. With some restrictions on the agent-interactions this model worked, which shows that using concurrent message-passing for parallel ABS is at least \textit{conceptually} feasible. Despite the natural mapping of ABS concepts to such an actor language, it leads to simulations, which despite same initial starting conditions, might result in different dynamics each time due to concurrency.

The work \cite{lysenko_framework_2008} discusses a framework, which allows to map Agent-Based Simulations to Graphics Processing Units (GPU). Amongst others they use the SugarScape model \cite{epstein_growing_1996} and scale it up to millions of agents on very large environment grids. They reported an impressive speed-up of a factor of 9,000. Although their work is conceptually very different we can draw inspiration from their work in terms of performance measurement and comparison of the SugarScape model.

% THIS IS MY OWN REASEARCH, DON'T CITE IT HERE
%In \cite{thaler_pure_2019} the authors showed how to implement a spatial SIR model in pure Haskell using Functional Reactive Programming \cite{hudak_arrows_2003}. They report quite low performance but mention that STM may be a way to considerably speed up the simulation. We follow their approach in implementation technique, using Functional Reactive Programming and Monadic Stream Functions \cite{perez_functional_2016} (we don't go into implementation details as it is out of the scope of this paper) and use the spatial SIR model as the first case-study.

Using functional programming for DES was discussed in \cite{jankovic_functional_2007} where the authors explicitly mention the paradigm of Functional Reactive Programming (FRP) to be very suitable to DES.

A domain-specific language for developing functional reactive agent-based simulations was presented in \cite{schneider_towards_2012,vendrov_frabjous_2014}. This language called FRABJOUS is human readable and easily understandable by domain-experts. It is not directly implemented in FRP/Haskell but is compiled to Haskell code which they claim is also readable. This supports that FRP is a suitable approach to implement ABS in Haskell. Unfortunately, the authors do not discuss their mapping of ABS to FRP on a technical level, which would be of most interest to functional programmers.

Object-oriented programming and simulation have a long history together as the former one emereged out of Simula 67 \cite{dahl_birth_2002} which was created for simulation purposes. Simula 67 already supported Discrete Event Simulation and was highly influential for today's object-oriented languages. Although the language was important and influential, in our research we look into different approaches, orthogonal to the existing object-oriented concepts.

Lustre is a formally defined, declarative and synchronous dataflow programming language for programming reactive systems \cite{halbwachs_synchronous_1991}. While it has solved some issues related to implementing ABS in Haskell it still lacks a few important features necessary for ABS. We don't see any way of implementing an environment in Lustre as we do in Chapters  \ref{ch:timedriven} and \ref{ch:eventdriven}. Also the language seems not to come with stochastic functions, which are but the very building blocks of ABS. Finally, Lustre does only support static networks, which is clearly a drawback in ABS in general where agents can be created and terminated dynamically during simulation.

The authors of \cite{botta_time_2010} discuss the problem of advancing time in message-driven agent-based socio-economic models. They formulate purely functional definitions for agents and their interactions through messages. Our architecture for synchronous agent-interaction as discussed in Chapter TODO was not directly inspired by their work but has some similarities: the use of messages and the problem of when to advance time in models with arbitrary number synchronised agent-interactions.

The authors of \cite{botta_functional_2011} are using functional programming as a specification for an agent-based model of exchange markets but leave the implementation for further research where they claim that it requires dependent types. This paper is the closest usage of dependent types in agent-based simulation we could find in the existing literature and to our best knowledge there exists no work on general concepts of implementing pure functional agent-based simulations with dependent types. As a remedy to having no related work to build on, we looked into works which apply dependent types to solve real world problems from which we then can draw inspiration from.

In his talk \cite{sweeney_next_2006}, Tim Sweeney CTO of Epic Games discussed programming languages in the development of game engines and scripting of game logic. Although the fields of games and ABS seem to be very different, Gregory \cite{gregory_game_2018} defines computer-games as \textit{"[..] soft real-time interactive agent-based computer simulations"} (p. 9) and in the end they have also very important similarities: both are simulations which perform numerical computations and update objects in a loop either concurrently or sequential. In games these objects are called \textit{game-objects} and in ABS they are called \textit{agents} but they are conceptually the same thing. The two main points Sweeney made were that dependent types could solve most of the run-time failures and that parallelism is the future for performance improvement in games. He distinguishes between pure functional algorithms which can be parallelized easily in a pure functional language and updating game-objects concurrently using software transactional memory (STM).


