\section{Event-Driven}
Due to the necessity of imposing a correct ordering of events in this type of ABS, we are only left with one way to step such an implementation: sequentially, event by event. In this case the ABS is technically quite close to a DES. Note that there exists also Parallel DES (PDES) \cite{fujimoto_parallel_1990}, which deals with processing events in parallel and dealing with inconsistencies in ordering in different ways - we leave this in the context for pure functional ABS for further research.

\subsection{Sugarscape}

\subsection{Synchronised Agent-Interactions}
Following towards papers SugarScape implementation

% direct MSF call, Problem is recursive nature. maybe try it with gintis Implementation. I just learned that what i want to achieve is actually: https://en.wikipedia.org/wiki/This_(computer_programming)#Open_recursion AND OPEN RECURSION IS PRETTY BAD

%Although there are similarities to the work of \cite{botta_time_2010} (the use of messages and the problem of when to advance time in models with arbitrary number synchronised agent-interactions), we approach our agents differently. First in our approach an agent is only a single MSF and thus can not be directly queried for its internal state / its id or outgoing messages, instead of taking a list of messages, our agents take a single event/message and can produce an arbitrary number of outgoing messages together with an observable state - note that this would allow to query the agent for its id and its state as well by simply sending a corresponding message to the agents MSF and requiring the agent to implement message handling for it. Also the state of our agents is \textit{completely} localised and there is no means of accessing the state from outside the agent, they are thus "fully encapsulated agents" \cite{botta_time_2010}. Note that the authors of \cite{botta_time_2010} define their agents with a polymorphic agent-state type \textit{s}, which implies that without knowledge of the specific type of \textit{s} there would be no way of accessing the state, rendering it in fact also fully encapsulated. The problem of advancing time in our approach is solved not exactly the same but conceptually it is the same: after sending a tick message to each agent (in random order), we process all agents until they are idle: there are no more enqueued messages / events in the queue.