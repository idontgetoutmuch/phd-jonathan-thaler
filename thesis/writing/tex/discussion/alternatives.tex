\section{Alternatives}
\label{sec:alternatives}

Shortly discuss alternative implementation directions which we didn't / couldn't follow because of not enough time / not enough experience / developed experience too late. 

Freer Monads: \url{https://reasonablypolymorphic.com/blog/freer-monads/}. They aim to separate Definition from implementation by writing a domain-specific language using GADTs which are then interpreted. This allows to strictly separate implementation from specification, composes very well and thus is easier to test as parts can be easily mocked. Also Freer Monads free one from the order of effects imposed through Monad Transformers. In general Freer Monads seem to aim for the same abstraction what modern interface-based oop does.
Problem: Yes, freer monads are today somewhere around 30x slower than the equivalent mtl code. because its $O(n^2)$. ABS are not IO bound, so raw computation is all what counts and this is undoubtly worse with Freer monads. Given that we are already having problematic performance, we can't sacrifice even more. There seem to be a better encoding possible, which is about 2x slower than MTL: \url{https://reasonablypolymorphic.com/blog/too-fast-too-free/}. Still it might prove to be useful in other terms like proving correctness and then translating it, but how could we do it?  
ContT is Not an Algebraic Effect so it seems to be difficult to implement continuations in freer monads. Unfortunately this is what we really need as shown in Event driven ABS and generalising structure chapters.
Other criticism of Freer Monads are: \url{https://medium.com/barely-functional/freer-doesnt-come-for-free-c9fade793501} are: boilerplate code which though can be generated automatically by some libraries, performance when not IO based because the program is bascially a data-structure which is interpreted, concurrency seems to be tricky, 

TODO: it seems that Final Tagless is another alternative. Look into it