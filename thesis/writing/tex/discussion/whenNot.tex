\section{Maturity}
%Thus it is not that implementing ABS with OO is wrong - it works reasonably well as a large number of industry strength libraries and frameworks demonstrate. It is more the \textit{missed potential} of a (pure) functional, data-centric approach: strong static type system with explicit controlled side-effects; parallel computation to speed up the simulation with very few changes but retaining static guarantees at compile time; STM to implement concurrent data-flow problems as actual data-flow problems without the need to resort to synchronisation primitives and cluttering the program logic with semantics for synchronisation and concurrency; Property-based testing for verification and validation of a data-centric approach which is central to all simulations; actor model concurrency in the case of Cloud Haskell and Erlang for agent-interaction centric models with a functional, data-centric core. 

We are very well aware that our approach has yet to reach maturity and prove itself over time in real simulation studies, especially as it is missing a mature and stable library yet. The fact that there exist a number of industry strength toolkits and libraries (Repast, NetLogo, AnyLogic) and the widespread use and knowledge of object-oriented programming, makes object-oriented programming still the highly compelling approach to implement ABS. This allows for a quick and cheap implementation of low-impact and straightforward models where the need for correctness, reproducibility, verification and validation is not of primary concern. As outlined, performance in functional programming is still nowhere near object-oriented programming although that argument might get diminished by further research and the potential of using actor based concurrency like in Erlang to implement ABS. %Another benefit is that object-oriented programming as a modelling tool to a problem is still highly useful in the case of UML. %TODO:  discusses if and how peers object-oriented agent-based modelling framework can be applied to our pure functional approach. TODO: i need to re-read peers framework specifications / paper from the simulation bible book. Although peers framework uses UML and OO techniques to create an agent-based model, we realised from a short case-study with him that most of the framework can be directly applied to our pure functional approach as well, which is not a huge surprise, after all the framework is more a modelling guide than an implementation one. E.g. a class diagram identifies the main datastructures, their operations and relations, which can be expressed equally in our approach - though not that directly as in an oo language but at least the class diagram gives already a good outline and understanding of the required datafields and operations of the respective entities (e.g. agents, environment, actors,...). A state diagram expresses internal states of e.g. an agent, which we discussed how to do in both our time- and even-driven approach. A sequence diagram e.g. expresses the (synchronous) interactions between agents or with their environment, something for which we developed techniques in our event-driven approach and we discuss in depth there. 