\section{Agents As Objects}
After having undertaken this long journey on how to implement ABS pure functionally, what the general computational structures are in ABS and what benefits and drawbacks there are, at the very end of our discussion I want to return to the claim that \textit{agents map naturally to objects} \cite{north_managing_2007}.

My approach of doing ABS and representing agents in pure FP can be interpret as trying to emulate objects in a purely functional way. In this case we have to say: yes agents map naturally to objects. The question is then: are there other, better mechanisms, more in FP I missed / didnt think of ... to implement ABS in FP? I hypothesize that this is probably NOT the case and that every approach in pure FP follows a roughly similar direction with only a few differences. Obviously it is apparent that both OOP and FP are not silverbullets to ABS and both come with their benefits and drawbacks and both have their existence. Thus I hypothesize that we might see the emergence of different computation paradigms in the future which might fit better to ABS than either one.

yes agents map naturally to objects, but what kind of objects? they differ in implementation details and in this thesis proposed a pure functional approach to a notion of objects. objects in Java work different, as well as in Smalltalk and objective c. processes in the functional language Erlang can be seen as objects too as they fullfill all criteria. also cite alan kay