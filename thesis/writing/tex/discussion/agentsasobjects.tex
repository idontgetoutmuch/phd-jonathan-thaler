\chapter{Agents As Objects}
When we go back to the beginning of this thesis and revisit the viewpoints that object-oriented programming (OO) is a particularly natural development environment for ABS \cite{epstein_growing_1996} and \textit{"agents map naturally to objects"} \cite{north_managing_2007} in the light of the full thesis, the obvious question which comes to mind is whether this thesis implies that we have been doing implementing ABS wrong ever since Epstein advocated OO for it in 1996 - shall Haskell be the new way to got? Obviously that is not the case but as we have shown, with (pure) functional programming comes a lot of potential. Lets elaborate on that.

\medskip

It is a fact that simulations are about consuming, processing and producing data. ABS being simulation methodology is no exception to that fact. Unfortunately, due to OO lack of rigour theoretical foundations, OO as it is used today is \textit{not} very good at representing and manipulating pure data and its data-flow because of mutable shared state and its unfortunate coupling with code. 

%TODO: i REALLY need to find proper literature / research / evidence which shows the problematic nature of modern OO: mutable shared state which is tied to code. Inheritance and open recursion gives the rest. the problem is that deeply linking \textit{shared mutable} state to its code is the path to failure: abstraction breaks, concurrency and parallelism becomes hard and breaks abstraction, data-driven programming becomes difficult (although that got addressed by adding functional features). NOTE: my approach and erlang have state and behaviour as well but in our case the state is shared nothing and immutable (yes in Haskell we update the agents state but that happens ultimately through closures and continuation in a referential transparent way and still no state is shared between agents. the environment is an exception to some extent as agents can access it globally: this works but requires a specific ordering either through sequential access or STM. this is no different than in an erlang implementation of sugarscape: there needs to be some arbitration of concurrent access). TODO: isnt there some fundamental research on that issue out there?
% TODO: maybe these act as a starting point?
% https://www.reddit.com/r/programming/comments/41jf45/objectoriented_programming_is_bad_brian_will/
% https://www.yegor256.com/2016/08/15/what-is-wrong-object-oriented-programming.html
% https://dl.acm.org/citation.cfm?id=1806847
% https://web.cs.ucdavis.edu/~filkov/papers/lang_github.pdf "Most notably, it does appear that strong typing is modestly better than weak typing, and among functional languages, static typing is also somewhat better than dynamic typing" "We also find that functional languages are somewhat better than procedural languages" but modest effects
% READ extension problem paper
% READ Ted Kaminskis thesis

%The rise of functional concepts in OO languages in the last years are a strong indication that OO is lacking features which have existed in FP for decades:
%
%\begin{itemize}
%	\item Java 8 added lambda expressions and functional style programming using map, fold, reduce, filter, which together with lambdas allow a data-flow oriented approach to computing.
%	 
%	\item Python, which surges in popularity within the OO family of languages, allows very data-flow centric and functional style of programming through lambda functions, list comprehensions and other functional features as it does not require programmers to stick to the OO paradigm.
%	
%	\item Popularisation of JavaScript frameworks like React, Elm and Purescript, which emphasise a functional, data-flow driven approach of web-programming.
%\end{itemize}

This was by no means clear in the early-to-mid 1990s where the OO paradigm was seen as a silver bullet to the problems of programming: a whole software industry had to re-learn best practices, patterns \cite{gamma_design_1994} and how to avoid pitfalls and bad code \cite{fowler_refactoring:_2012}. Thus we cannot blame \cite{epstein_growing_1996} for advertising OO as the ways to implement ABS, at that time it seemed indeed to be the right thing to do. 

The question is then why not use toolkits like Matlab or R - after all they are completely data-centric? This would be the other extreme, just like OO is and we would run into difficulties as well. The point is that ABS is not purely data-centric either and is indeed richer: agents can interact with each other and with an environment. So we have a tension here: ABS is data-centric on the one hand, and interaction-centric on the other - can we combine both worlds?

The combination of both was exactly the sales pitch of OO for the last 20+ years. Unfortunately this combination leads to nasty bugs due to shared mutable state, deeply complex object hierarchies due to inheritance overuse which also fix behaviour at compile time, open recursion which in the end costs the potential for higher degree of correctness, ease of parallelism and concurrency and the use of property-based testing. Thus we need to separate both: what we need is immutable, shared-nothing state allowing for a data-centric approach \textit{and} an interaction mechanism which allows agents to communicate with each other.

\medskip

This thesis is \textit{one} way of showing how to separate both and reap the benefits. A time-driven ABS like SIR or an ACE with simple agents not interaction with each other like ZI traders is heavily data-centric and very low on agent-interaction. Such data-driven ABS models are quite well expressed in a purely functional approach with the advantage that one can reap the benefits of reproducibility at compile time, using STM for concurrency and property-based testing for verification and validation. An event-driven simulation with complex agent state and agent-interactions like social simulations like Sugarscape or Chemical or Biology simulation with cell interactions are also possible in a pure functional setting as we have shown in the case of the Sugarscape model. Although we were able to give a good solution to complex agent state and synchronous, direct Agent-interactions in our event-driven SIR and Sugarscape and they \textit{do} work in Haskell, they are cumbersome to get right without library support (see further research below) and we cannot reap the benefits of STM in their case. 

This has lead to the fundamental conclusion that simulations which implement complex agent state and agent-communication centric models should rather be implemented in an functional language with actor based concurrency messaging. There are basically two options Cloud Haskell and Erlang \footnote{There is also Elixir, an Erlang dialect. Also there is Scala with the Akka / Actor Library but this is not purely functional and violates the shared nothing semantics, making it less data-centric}.
They are all message orientation will allow you to easily express the complex agent-interactions whilst having the potential to run them in parallel to gain a potentially substantial speed up. Despite its focus on messages, all are (pure) functional languages, which puts you into the data-centric approach: messages are pure data with \textit{shared nothing semantics}. This makes testing easier and also opens the way for property-based testing which is available in Erlang as well where it even allows to detect race conditions \cite{claessen_finding_2009}. Thus we can conclude that agents do \textit{not} map naturally to objects, agents map naturally to actors. 

% TODO: cite armstrongs blog about shared nothing semantics, should be somewhere in my 1st or 2nd year report \\

\medskip

Thus it is not that implementing ABS with OO is wrong - it works reasonably well as a large number of industry strength libraries and frameworks demonstrate. It is more the \textit{missed potential} of a (pure) functional, data-centric approach: strong static type system with explicit controlled side-effects; parallel computation to speed up the simulation with very few changes but retaining static guarantees at compile time; STM to implement concurrent data-flow problems as actual data-flow problems without the need to resort to synchronisation primitives and cluttering the program logic with semantics for synchronisation and concurrency; Property-based testing for verification and validation of a data-centric approach which is central to all simulations; actor model concurrency in the case of Cloud Haskell and Erlang for agent-interaction centric models with a functional, data-centric core. 

Still, there are reasons to stick with OO and avoid FP. There exist a bunch the industry strength toolkits and libraries (Repast, NetLogo, AnyLogic) and the widespread use and knowledge of OO which makes abs implementers readily available. This allows for a quick and cheap implementation of low-impact and straightforward models where the need for correctness, reproducibility, verification and validation is not of primary concern. Also, performance in FP is still a far cry from OO although that argument might get diminished by the potential of using actor based concurrency like in Erlang to implement ABS. Another benefit is that OO as a modelling tool to a problem is still highly useful in the case of UML. %TODO:  discusses if and how peers object-oriented agent-based modelling framework can be applied to our pure functional approach. TODO: i need to re-read peers framework specifications / paper from the simulation bible book. Although peers framework uses UML and OO techniques to create an agent-based model, we realised from a short case-study with him that most of the framework can be directly applied to our pure functional approach as well, which is not a huge surprise, after all the framework is more a modelling guide than an implementation one. E.g. a class diagram identifies the main datastructures, their operations and relations, which can be expressed equally in our approach - though not that directly as in an oo language but at least the class diagram gives already a good outline and understanding of the required datafields and operations of the respective entities (e.g. agents, environment, actors,...). A state diagram expresses internal states of e.g. an agent, which we discussed how to do in both our time- and even-driven approach. A sequence diagram e.g. expresses the (synchronous) interactions between agents or with their environment, something for which we developed techniques in our event-driven approach and we discuss in depth there. 

\medskip

After having undertaken this long journey on how to implement ABS pure functionally, what the general computational structures are in ABS and what benefits and drawbacks there are, at the very end of our discussion I want to return to the claim that \textit{agents map naturally to objects} \cite{north_managing_2007}.

My approach of doing ABS and representing agents in pure FP can be interpret as trying to emulate objects in a purely functional way. In this case we have to say: yes agents map naturally to objects. The question is then: are there other, better mechanisms, more in FP I missed / didnt think of ... to implement ABS in FP? I hypothesize that this is probably NOT the case and that every approach in pure FP follows a roughly similar direction with only a few differences. Obviously it is apparent that both OOP and FP are not silverbullets to ABS and both come with their benefits and drawbacks and both have their existence. Thus I hypothesize that we might see the emergence of different computation paradigms in the future which might fit better to ABS than either one.

yes agents map naturally to objects, but what kind of objects? they differ in implementation details and in this thesis proposed a pure functional approach to a notion of objects. objects in Java work different, as well as in Smalltalk and objective c. processes in the functional language Erlang can be seen as objects too as they fullfill all criteria. also cite alan kay

\section{Haskell}
Freer Monads: \url{https://reasonablypolymorphic.com/blog/freer-monads/}. They aim to separate Definition from implementation by writing a domain-specific language using GADTs which are then interpreted. This allows to strictly separate implementation from specification, composes very well and thus is easier to test as parts can be easily mocked. Also Freer Monads free one from the order of effects imposed through Monad Transformers. In general Freer Monads seem to aim for the same abstraction what modern interface-based oop does.
Problem: Yes, freer monads are today somewhere around 30x slower than the equivalent mtl code. because its $O(n^2)$. ABS are not IO bound, so raw computation is all what counts and this is undoubtly worse with Freer monads. Given that we are already having problematic performance, we can't sacrifice even more. There seem to be a better encoding possible, which is about 2x slower than MTL: \url{https://reasonablypolymorphic.com/blog/too-fast-too-free/}. Still it might prove to be useful in other terms like proving correctness and then translating it, but how could we do it?  
ContT is Not an Algebraic Effect so it seems to be difficult to implement continuations in freer monads. Unfortunately this is what we really need as shown in Event driven ABS and generalising structure chapters.
Other criticism of Freer Monads are: \url{https://medium.com/barely-functional/freer-doesnt-come-for-free-c9fade793501} are: boilerplate code which though can be generated automatically by some libraries, performance when not IO based because the program is bascially a data-structure which is interpreted, concurrency seems to be tricky, 
TODO read: \url{https://reasonablypolymorphic.com/blog/freer-yet-too-costly/}

TODO: it seems that Final Tagless is another alternative. Look into it: \url{https://jproyo.github.io/posts/2019-03-17-tagless-final-haskell.html}

\section{Languages}
We precisely pointed out in the beginning of this thesis why we chose Haskell as language of choice. Obviously Haskell is not the only (pure) functional language and there exist a number of other alternatives which would be equally worth of systematic investigation of their use for ABS. Shortly we can conclude that the use of Haskell moves the nature of the structure of ABS computation into the light, together with compile-time guarantees, and benefits in testing and parallel implementations. Depending on each language though we get a very different direction:

\paragraph{LISP} Being the oldest functional programming language and the 2nd oldest high-level programming language ever created and still used by many people, LISP had to be considered in the beginning of the thesis. The language has the immense powerful feature of homoiconicity: data is code and code is data at the same time. This allows a LISP program to generate data-structures, which resemble valid LSIP code thus mutating its own code at runtime. This would give immense power to create powerful abstractions in terms of ABS. Unfortunately LISP is fully interpreted and has no types and is also impure, which would probably have led to very imperative, traditional approaches to ABS. Still, there exists research \cite{kawabe_nepi2programming_2000} which implements a MAS in LISP.

\paragraph{Erlang} The programming-model of actors \cite{agha_actors:_1986} was the inspiration for the Erlang programming language \cite{armstrong_erlang_2010}, which was created in the 1980's by Joe Armstrong for Eriksson for developing distributed high reliability software in telecommunications. The implication is that, the focus would shift immediately to the use of the actor model for concurrent interaction of agents through messages. The languages type-system is strong and dynamic and thus lacks type-checking at compile-time. Thus the structure of computation plays naturally no role because we cannot look at it from the abstract perspective as we can in Haskell. Purity can not be guaranteed and due to agents being processes concurrency is everywhere, and even though it is very tamed through shared-nothing messaging semantics, this implies that repeated runs with same initial conditions might lead to different results. Obviously we could avoid implementing agents as processes but then we basically sacrifice the very heart and feature of the language.
	
\paragraph{Scala} Scala is a multi-paradigm language, which also comes with an implementation of the actor-model as a library which enables to do actor-programming in the way of Erlang. It was developed in 2004 and became popular in recent years due to the increased availability of multi-core CPUs which emphasised the distributed, parallel and concurrent programming for which the actor-model is highly suited. There exists research on using Scala for ABS \cite{krzywicki_massively_2015, todd_multi-agent_nodate}. The benefit Scala has over Erlang is that it has type-checking at compile-time and is thus more robust, still it is impure due to side-effects and messages can contain references, thus violating the original shared-nothing semantics of Erlang.

\paragraph{F\#} Widely used in finance TODO 

\section{Actors}
\label{sec:actors}
TODO: this seems not to fit into the narrative here, maybe it fits into discussion part or further research

The Actor-Model, a model of concurrency, was initially conceived by Hewitt in 1973 \cite{hewitt_universal_1973} and refined later on \cite{hewitt_what_2007}, \cite{hewitt_actor_2010}. It was a major influence in designing the concept of agents and although there are important differences between actors and agents there are huge similarities thus the idea to use actors to build agent-based simulations comes quite natural. The theory was put on firm semantic grounds first through Irene Greif by defining its operational semantics \cite{grief_semantics_1975} and then Will Clinger by defining denotational semantics \cite{clinger_foundations_1981}. In the seminal work of Agha \cite{agha_actors:_1986} he developed a semantic mode, he termed \textit{actors} which was then developed further \cite{agha_foundation_1997} into an actor language with operational semantics which made connections to process calculi and functional programming languages (see both below). 

An actor is a uniquely addressable entity which can do the following \textit{in response to a message}
\begin{itemize}
	\item Send an arbitrary number (even infinite) of messages to other actors.
	\item Create an arbitrary number of actors.
	\item Define its own behaviour upon reception of the next message.
\end{itemize}

In the actor model theory there is no restriction on the order of the above actions and so an actor can do all the things above in parallel and concurrently at the same time. This property and that actors are reactive and not pro-active is the fundamental difference between actors and agents, so an agent is \textit{not} an actor but conceptually nearly identical and definitely much closer to an agent in comparison to an object. The actor model can be seen as quite influential to the development of the concept of agents in ABS, which borrowed it from Multi Agent Systems \cite{wooldridge_introduction_2009}. Technically, it emphasises message-passing concurrency with share-nothing semantics (no shared state between agents), which maps nicely to functional programming concepts.

There have been a few attempts on implementing the actor model in real programming languages where the most notable ones are Erlang and Scala. Erlang was created in 1986 by Joe Armstrong for Eriksson for developing distributed high reliability software in telecommunications. It implements light-weight processes, which allows to spawn thousands of them without heavy memory overhead. The language saw some use in implementing ABS with notable papers being \cite{di_stefano_using_2005, di_stefano_exat:_2007, varela_modelling_2004, sher_agent-based_2013, bezirgiannis_improving_2013}

Scala is a modern mixed paradigm programming language, which also allows functional programming and also incorporates a library for the actor model. It also saw the use in the implementation of ABS with a notable paper \cite{krzywicki_massively_2015} and ScalABM \footnote{https://github.com/ScalABM} which is alibrary for ABM in economics.

The paper of \cite{jennings_agent-based_2000} gives an excellent overview over the strengths and weaknesses of agent-based software-engineering, which can be directly applied to both Erlang and Scala.

Due to the very different approach and implications the actor model of concurrency implies, we don't explore it further and leave it for further research as it is beyond the focus of the thesis.