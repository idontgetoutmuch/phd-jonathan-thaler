\chapter{Evaluation}
\label{ch:evaluation}
% starting point
This thesis started out by challenging the established views that \textit{"[..] object-oriented programming to be a particularly natural development environment for Sugarscape specifically and artificial societies generally [..]"} \cite{epstein_growing_1996} (p. 179) and that \textit{agents map naturally to objects} \cite{north_managing_2007}. As a highly challenging alternative it proposed a radical different approach to implementing ABS, using the pure functional programming paradigm. As language of choice Haskell was motivated due to its matureness, increasing relevance to real-world applications and cutting edge pure functional programming concepts. 

% motivation
The motivation are the promises pure functional programming in Haskell comes with, which are relevant to ABS as well:
\begin{itemize}
	\item The static strong type system allows to remove substantial number and class of bugs at run-time and if one programs careful one can even guarantee that no bugs as in crashes or exceptions will occur at run-time. This is especially true for purely computational problems (without IO) as ABS almost always are.
	
	\item Explicit handling and control of side-effects delivers even more static guarantees at compile time and allows to deal with deterministic side-effects (random-number streams, read-/write only contexts, state) in a referential transparent way. In combination with strong static type this allows to reduce logical bugs (subject to the domain of the problem) by dramatically reducing available and valid operations on data - after all stateful applications are a fact, the challenge is how to deal with state. As ABS is full of state due to agents and the environment, this seems to be an obvious great thing to have to increase correctness of an implementation. Further, this should allow to produce an implementation which is guaranteed to reproducible (runs with same initial conditions \textit{will} result in same dynamics) at compile time.
	
	\item Parallel and concurrent programming is seen to be lot easier, less painful and less error prone in functional programming in general and in Haskell in particular due to immutable data and the explicit side-effects. The concept of Software Transactional Memory, which offers to formulate a problem as a data-flow problem like there was no concurrency seemed highly promising. Besides, data-parallel programming promises to speed up code without the need for changing any of the logic or types. This seemed to offer a straightforward way of speeding up ABS implementations either through data-parallelism or concurrency. This has always been quite difficult to achieve in traditional ABS due to OO and pure functional programming seems to offer a solution.
	
	\item The data-centric declarative style, referential transparency and immutability of data makes testing substantially easier due to composability: functions can be easily tested in isolation from each other even if they involve side-effects. This opens the door for (randomised) property-based testing which intuitively seemed to be a perfect match to test ABS implementations which are (almost) always stochastic in nature.
\end{itemize}

The relevance of each of these promises to ABS is pointed out in the respective promise and it is quite obvious that these benefits would clearly be of immense value in ABS, with the common baseline that they all support implementations of ABS to be more likely to be correct - something of fundamental value in simulation. The question was then how much of them can we actually reap if we follow a pure functional approach to ABS? To answer this we first needed to actually find out \textit{how} ABS could be done pure functionally, as there didn't exist any research which offered a systematic solution to that problem. This was answered in-depth in part II. Let shortly recap what we have done and put it into perspective:

% the problem
It was unclear how to represent agents: how can we express agent identity, local agents state, changing behaviour and interactions amongst agents and the environment? After all this is straightforward in OO due to mutable shared state encapsulated in objects.

% the answer
The solution was to use arrowized FRP both in the pure implementation of Yampa and the monadic version as in the library Dunai. The central concept behind these approaches are Signal Functions (SFs) (which were generalised in Dunai to Monadic Stream Functions MSFs ) which are implemented using closures and continuations, fundamental building blocks and concepts of pure functional programming. This allows SFs/MSFs to be seen as very simple \textit{immutable} objects with a single method following a \textit{shared nothing} semantics. 

% how: what we have achieved


%we now know how to engineer time- and event-driven ABS with complex state both in the agent and environment, main difficulty is direct agent-interaction (see macal classification into 4 types of ABS), compile-time guaranteed reproducibility, explicit handling of complex state (read only, read/write), concurrency explicit and limited to STM, very promising concurrency but direct agent-interactions main problem (erlang as a rescue?), main drawbacks: everything is explicit, performance

%The case-study strongly hints that our claim that pure FP has indeed its place in ABS is valid but we conclude that it is yet too early to pick up this paradigm for ABS. We think that engineering a proper implementation of a complex ABS model takes substantial effort in pure FP due to different techniques required. We believe that at the moment such an effort pays off only in cases of high-impact and large-scale simulations which results might have far-reaching consequences e.g. influence policy decisions. It is only there where the high requirements for reproducibility, robustness and correctness provided by FP are really needed. Still, we plan on distilling the developed techniques of the case-study into a general purpose ABS library. This should allow implementing models much easier and quicker, making the pure FP approach an attractive alternative for prototyping, opening the direction for a broad use of FP in the field of ABS.
% TODO: emphasise that we pulled of to implement the full sugarscape with a reasonable approach, this is strong evidence that it works

Probably the biggest strength is that we can guarantee reproducibility at compile time: given identical initial conditions, repeated runs of the simulation will lead to same outputs. This is of fundamental importance in simulation and addressed in the Sugarscape model: \textit{"... when the sequence of random numbers is specified ex ante the model is deterministic. Stated yet another way, model output is invariant from run to run when all aspects of the model are kept constant including the stream of random numbers."} (page 28, footnote 16) - we can guarantee that in our pure functional approach already \textit{at compile time}.

Refactoring is very convenient and quickly becomes the norm: guided by types (change / refine them) and relying on the compiler to point out problems, results in very effective and quick changes without danger of bugs showing up at run-time. This is not possible in Python because of its lack of compiler and types, and much less effective in Java due to its dynamic type-system which is only remedied through strong IDE support.

Adding data-parallelism is easy and often requires simply swapping out a data-structure or library function against its parallel version. Concurrency, although still hard, is less painful to address and add in a pure functional setting due to immutable data and explicit side-effects. Further, the benefits of implementing concurrent ABS based on Software Transactional Memory (STM) has been shown \cite{thaler_tale_2018} which underlines the strength of Haskell for concurrent ABS due to its strong guarantees about retry-semantics.

Testing in general allows much more control and checking of invariants due to the explicit handling of effects - together with the strong static type system, the testing-code is in full, explicit control over the functionality it tests. Property-based testing in particular is a perfect match to testing ABS due to the random nature in both and because it supports convenient expressing of specifications. Thus we can conclude that in a pure functional setting, testing is very expressive and powerful and supports working towards an implementation which is very likely to be correct.


\subsection{Verification and Correctness}
General there are the following basic verification \& validation requirements to ABS \cite{robinson_simulation:_2014}, which all can be addressed in our \textit{pure} functional approach as described in the paper in Appendix \ref{app:pfe}:

\begin{itemize}
	%\item Modelling progress of time - achieved using functional reactive programming (FRP)
	%\item Modelling variability - achieved using FRP
	\item Fixing random number streams to allow simulations to be repeated under same conditions - ensured by \textit{pure} functional programming and Random Monads
	\item Rely only on past - guaranteed with \textit{Arrowized} FRP
	\item Bugs due to implicitly mutable state - reduced using pure functional programming
	\item Ruling out external sources of non-determinism / randomness - ensured by \textit{pure} functional programming
	\item Deterministic time-delta - ensured by \textit{pure} functional programming
	\item Repeated runs lead to same dynamics - ensured by \textit{pure} functional programming
\end{itemize}

\begin{enumerate}
	\item Run-Time robustness by compile-time guarantees - by expressing stronger guarantees already at compile-time we can restrict the classes of bugs which occur at run-time by a substantial amount due to Haskell's strong and static type system.  This implies the lack of dynamic types and dynamic casts \footnote{Note that there exist casts between different numerical types but they are all safe and can never lead to errors at run-time.} which removes a substantial source of bugs.  Note that we can still have run-time bugs in Haskell when our functions are partial.
	\item Purity - By being explicit and polymorphic in the types about side-effects and the ability to handle side-effects explicitly in a controlled way allows to rule out non-deterministic side-effects which guarantees reproducibility due to guaranteed same initial conditions and deterministic computation. Also by being explicit about side-effects e.g. Random-Numbers and State makes it easier to verify and test.
	\item Explicit Data-Flow and Immutable Data - All data must be explicitly passed to functions thus we can rule out implicit data-dependencies because we are excluding IO. This makes reasoning of data-dependencies and data-flow much easier as compared to traditional object-oriented approaches which utilize pointers or references.
	\item Declarative - describing \textit{what} a system is, instead of \textit{how} (imperative) it works. In this way it should be easier to reason about a system and its (expected) behaviour because it is more natural to reason about the behaviour of a system instead of thinking of abstract operational details.
	\item Concurrency and parallelism - due to its pure and 'stateless' nature, functional programming is extremely well suited for massively large-scale applications as it allows adding parallelism without any side-effects and provides very powerful and convenient facilities for concurrent programming. The paper of (TODO: cite my own paper on STM) explores the use Haskell for concurrent and parallel ABS in a deeper way.
\end{enumerate}

TODO: haskell-titan
TODO: Testing and Debugging Functional Reactive Programming \cite{perez_testing_2017}

Static type system eliminates a large number run-time bugs.

TODO: can we apply equational reasoning? Can we (informally) reason about various properties e.g. termination?

Follow unit testing of the whole simulation as prototyped for towards paper.

in this we explore something new: property-based testing in ABS

\section{Generalising Research}
We hypothesize that our research can be transferred to other related fields as well, which puts our contributions into a much broader perspective, giving it more impact than restricting it just to the very narrow field of Agent-Based Simulation. Although we don't have the time to back up our claims with in-depth research, we argue that our findings might be applicable to the following fields at least on a conceptual level.

\subsection{Simulation in general}
We already showed in the paper \cite{thaler_pure_2019}, that purity in a simulation leads to repeatability which is of utmost importance in scientific computation. These insights are easily transferable to simulation software in general and might be of huge benefit there. Also my approach to dependent types in ABS might be applicable to simulations in general due to the correspondence between equilibrium \& totality, in use for hypotheses formulation and specifications formulation as pointed out in Chapter \ref{ch:dependent}. 

\subsection{System Dynamics}
\label{sub:generalising_system_dynamics}
we have done that already in the appendix

discuss pure functional system dynamics - correct by construction: benefits: strictly deterministic already at compile time, encode equations directly in code => correct by construction. Can serve as backend implementation of visual SD packages.

\subsection{Discrete Event Simulation}
pure functional DES easily possible with my developed synchronous messaging ABS
DES in FP: we doing part of it in event-driven SIR. this is a potential hint at how to achieve DES in pure FP: arrowized FRP seems to be perfect for it because it allows to express data-flow networks, which is exactly what DES is as well. The problem with pure FP is that it will be more involved to propagate events through the network especially when they don't originate from the source but e.g. after a time-out from a queue. Generally all the techniques are there as discussed in the event-driven chapter but it would be interesting to do research on how to achieve the same in DES. Because data-flow networks of DES generally don't change at runtime, they are fixed already at compile time - it would be interesting to see what dependent types could offer for additional compile-time guarantees.

PDES, should be  conceptually easil possible using STM, optimistic approach should be conceptually easier to implement due to persistent data-structures and controlled side-effects
 
\subsection{Recursive Simulation}
Due to the recursive nature of FP we believe that it is also a natural fit to implement recursive simulations as the one discussed in \cite{gilmer_recursive_2000}. In recursive ABS agents are able to halt time and 'play through' an arbitrary number of actions, compare their outcome and then to resume time and continue with a specifically chosen action e.g. the best performing or the one in which they haven't died. More precisely, an agent has the ability to run the simulation recursively a number of times where the number is not determined initially but can depend on the outcome of the recursive simulation. So recursive ABS gives each Agent the ability to run the simulation locally from its point of view to project its actions into the future and change them in the present. Due to controlled side-effects and referential transparency, combined with the recursive nature of pure FP, we think that implementing a recursive simulation in such a setting should be straight-forward.

Inspired by \cite{gilmer_recursive_2000}, add ideas about recursive simulation described in 1st year report and "paper". functional programming maps naturally here due to its inherently recursive nature and controlled side-effects which makes it easier to construct correct recursive simulations.
recursive simulation should be conceptually easier to implememt and more likely to be correct due to recursive Nature of haskell itself, lack of sideeffeccts and mutable data

\subsection{Multi Agent Systems}
The fields of Multi Agent Systems (MAS) and ABS are closely related where ABS has drawn much inspiration from MAS \cite{wooldridge_introduction_2009}, \cite{weiss_multiagent_2013}. It is important to understand that MAS and ABS are two different fields where in MAS the focus is more on technical details, implementing a system of interacting intelligent agents within a highly complex environment with the focus on solving AI problems.

Because in both fields, the concept of interacting agents is of fundamental importance, we expect our research also to be applicable in parts to the field of MAS. Especially the work on dependent types should be very useful there because MAS is very interested in correctness, verification and formally reasoning about a system and their agents, to show that a system follows a formal specifications.

\section{Drawbacks}
\label{sec:drawbacks}

\subsection{Space-Leaks}
discuss the problem (and potential) of lazy evaluation for ABS: can under some circumstances really increase performance when some stuff is not evaluated (see STM study) but mostly it causes problems by piling up unevaluated thunks leading to crazy memory usage which is a crucial problem in simulation. Using strict pragmas, annotations and data-structures solves the problem but is not trivial and involves carefully studying the code / getting it right from the beginning / and using the haskell profiling tools (which are fucking great at least). TODO: show the stats of memory usage

Haskell is notorious for its memory-leaks due to lazy evaluation: data is only evaluated when required. Even for simple programs one can be hit hard by a serious space-leak where unevaluated code pieces (thunks) build up in memory until they are needed, leading to dramatically increased memory usage. It is no surprise that our highly complex Sugarscape implementation initially suffered severely from space-leaks, piling up about 40 MByte / second. In simulation this is a big issue, threatening the value of the whole implementation despite its other benefits: because simulations might run for a (very) long time or conceptually forever, one must make absolutely sure that the memory usage stays somewhat constant. As a remedy, Haskell allows to add so-called strictness pragmas to code-modules which forces strict evaluation of all data even if it is not used. Carefully adding this conservatively file-by file applying other techniques of forcing evaluation removed most of the memory leaks.

% Another memory leak was caused by selecting the wrong data-structure for our environment, for which we initially used an immutable array. The problem is that in the case of an update the whole array is copied, causing memory leaks AND a performance problem. We replaced it by an IntMap which uses integers as key (mapping 2d coordinates to unique integers is trivial) and is internally implemented as a radix-tree which allows for very fast lookups and inserts because whole sub-trees can be re-used.


\subsection{Efficiency}
ordering of the transformers
when to run a transformer,
lazyness vs. strictness

The main drawback of our approach is performance, which at the moment does not come close to OO implementations. There are two main reasons for it: first, FP is known for being slower due to higher level of abstractions, which are bought by slower code in general and second, updates are the main bottleneck due to immutable data requiring to copy the whole (or subparts) of a data structure in cases of a change. The first one is easily addressable through the use of data-parallelism and concurrency as we have done in our paper on STM \cite{thaler_tale_2018}. The second reason can be addressed by the use of linear types \cite{bernardy_linear_2017}, which allow to annotate a variable with how often it is used within a function. From this a compiler can derive aggressive optimisations, resulting in imperative-style performance but retaining the declarative nature of the code.

\subsection{Productivity and learning curve}
A case study in \cite{hanenberg_experiment_2010} hints that simply by switching to a static typesystem alone does not gain anything and can even be detriment. It also needs to have a certain level of abstractions like Haskell type system does or even dependent types as in Idris. This also mean that there is a substantial learning curve to master when one wants to enter pure functional ABS.

\section{Alternatives}
\subsection{Haskell}
Freer Monads: \url{https://reasonablypolymorphic.com/blog/freer-monads/}. They aim to separate Definition from implementation by writing a domain-specific language using GADTs which are then interpreted. This allows to strictly separate implementation from specification, composes very well and thus is easier to test as parts can be easily mocked. Also Freer Monads free one from the order of effects imposed through Monad Transformers. In general Freer Monads seem to aim for the same abstraction what modern interface-based oop does.
Problem: Yes, freer monads are today somewhere around 30x slower than the equivalent mtl code. because its $O(n^2)$. ABS are not IO bound, so raw computation is all what counts and this is undoubtly worse with Freer monads. Given that we are already having problematic performance, we can't sacrifice even more. There seem to be a better encoding possible, which is about 2x slower than MTL: \url{https://reasonablypolymorphic.com/blog/too-fast-too-free/}. Still it might prove to be useful in other terms like proving correctness and then translating it, but how could we do it?  
ContT is Not an Algebraic Effect so it seems to be difficult to implement continuations in freer monads. Unfortunately this is what we really need as shown in Event driven ABS and generalising structure chapters.
Other criticism of Freer Monads are: \url{https://medium.com/barely-functional/freer-doesnt-come-for-free-c9fade793501} are: boilerplate code which though can be generated automatically by some libraries, performance when not IO based because the program is bascially a data-structure which is interpreted, concurrency seems to be tricky, 
TODO read: \url{https://reasonablypolymorphic.com/blog/freer-yet-too-costly/}

TODO: it seems that Final Tagless is another alternative. Look into it: \url{https://jproyo.github.io/posts/2019-03-17-tagless-final-haskell.html}

\subsection{Languages}
We precisely pointed out in the beginning of this thesis why we chose Haskell as language of choice. Obviously Haskell is not the only (pure) functional language and there exist a number of other alternatives which would be equally worth of systematic investigation of their use for ABS. Shortly we can conclude that the use of Haskell moves the nature of the structure of ABS computation into the light, together with compile-time guarantees, and benefits in testing and parallel implementations. Depending on each language though we get a very different direction:

\paragraph{LISP} Being the oldest functional programming language and the 2nd oldest high-level programming language ever created and still used by many people, LISP had to be considered in the beginning of the thesis. The language has the immense powerful feature of homoiconicity: data is code and code is data at the same time. This allows a LISP program to generate data-structures, which resemble valid LSIP code thus mutating its own code at runtime. This would give immense power to create powerful abstractions in terms of ABS. Unfortunately LISP is fully interpreted and has no types and is also impure, which would probably have led to very imperative, traditional approaches to ABS. Still, there exists research \cite{kawabe_nepi2programming_2000} which implements a MAS in LISP.

\paragraph{Erlang} The programming-model of actors \cite{agha_actors:_1986} was the inspiration for the Erlang programming language \cite{armstrong_erlang_2010}, which was created in the 1980's by Joe Armstrong for Eriksson for developing distributed high reliability software in telecommunications. The implication is that, the focus would shift immediately to the use of the actor model for concurrent interaction of agents through messages. The languages type-system is strong and dynamic and thus lacks type-checking at compile-time. Thus the structure of computation plays naturally no role because we cannot look at it from the abstract perspective as we can in Haskell. Purity can not be guaranteed and due to agents being processes concurrency is everywhere, and even though it is very tamed through shared-nothing messaging semantics, this implies that repeated runs with same initial conditions might lead to different results. Obviously we could avoid implementing agents as processes but then we basically sacrifice the very heart and feature of the language.
	
\paragraph{Scala} Scala is a multi-paradigm language, which also comes with an implementation of the actor-model as a library which enables to do actor-programming in the way of Erlang. It was developed in 2004 and became popular in recent years due to the increased availability of multi-core CPUs which emphasised the distributed, parallel and concurrent programming for which the actor-model is highly suited. There exists research on using Scala for ABS \cite{krzywicki_massively_2015, todd_multi-agent_nodate}. The benefit Scala has over Erlang is that it has type-checking at compile-time and is thus more robust, still it is impure due to side-effects and messages can contain references, thus violating the original shared-nothing semantics of Erlang.

\paragraph{F\#} Widely used in finance TODO 

\section{Actors}
\label{sec:actors}
TODO: this seems not to fit into the narrative here, maybe it fits into discussion part or further research

The Actor-Model, a model of concurrency, was initially conceived by Hewitt in 1973 \cite{hewitt_universal_1973} and refined later on \cite{hewitt_what_2007}, \cite{hewitt_actor_2010}. It was a major influence in designing the concept of agents and although there are important differences between actors and agents there are huge similarities thus the idea to use actors to build agent-based simulations comes quite natural. The theory was put on firm semantic grounds first through Irene Greif by defining its operational semantics \cite{grief_semantics_1975} and then Will Clinger by defining denotational semantics \cite{clinger_foundations_1981}. In the seminal work of Agha \cite{agha_actors:_1986} he developed a semantic mode, he termed \textit{actors} which was then developed further \cite{agha_foundation_1997} into an actor language with operational semantics which made connections to process calculi and functional programming languages (see both below). 

An actor is a uniquely addressable entity which can do the following \textit{in response to a message}
\begin{itemize}
	\item Send an arbitrary number (even infinite) of messages to other actors.
	\item Create an arbitrary number of actors.
	\item Define its own behaviour upon reception of the next message.
\end{itemize}

In the actor model theory there is no restriction on the order of the above actions and so an actor can do all the things above in parallel and concurrently at the same time. This property and that actors are reactive and not pro-active is the fundamental difference between actors and agents, so an agent is \textit{not} an actor but conceptually nearly identical and definitely much closer to an agent in comparison to an object. The actor model can be seen as quite influential to the development of the concept of agents in ABS, which borrowed it from Multi Agent Systems \cite{wooldridge_introduction_2009}. Technically, it emphasises message-passing concurrency with share-nothing semantics (no shared state between agents), which maps nicely to functional programming concepts.

There have been a few attempts on implementing the actor model in real programming languages where the most notable ones are Erlang and Scala. Erlang was created in 1986 by Joe Armstrong for Eriksson for developing distributed high reliability software in telecommunications. It implements light-weight processes, which allows to spawn thousands of them without heavy memory overhead. The language saw some use in implementing ABS with notable papers being \cite{di_stefano_using_2005, di_stefano_exat:_2007, varela_modelling_2004, sher_agent-based_2013, bezirgiannis_improving_2013}

Scala is a modern mixed paradigm programming language, which also allows functional programming and also incorporates a library for the actor model. It also saw the use in the implementation of ABS with a notable paper \cite{krzywicki_massively_2015} and ScalABM \footnote{https://github.com/ScalABM} which is alibrary for ABM in economics.

The paper of \cite{jennings_agent-based_2000} gives an excellent overview over the strengths and weaknesses of agent-based software-engineering, which can be directly applied to both Erlang and Scala.

Due to the very different approach and implications the actor model of concurrency implies, we don't explore it further and leave it for further research as it is beyond the focus of the thesis.

\section{The Gintis case revisited}
\label{sec:gintis_case} 
After having established all the benefits and drawbacks pure functional programming has to offer for ABS, we want to return to the Introduction again, where we mentioned the work of Gintis \cite{gintis_emergence_2006}. To repeat, in this paper Gintis has claimed to have found a mechanism in bilateral decentralized exchange, which resulted in Walrasian General Equilibrium without the neo-classical approach of a tatonement process through a central auctioneer. Due to its high-impact result for economics, researchers \cite{ionescu_dependently-typed_2012} tried to reproduce the results independently but were not able to do so. After Gintis provided the code, it was found that there was a bug in his implementation which led to the unexpected results, which were seriously damaged through this error. The work of \cite{evensen_extensible_2010} investigated the specific nature of Gintis bugs and reported the following (Section \textit{3.1.1  Deviations from the paper}, page 23):

\begin{enumerate}
	\item An agent calculates the optimum inventory (or demand) according to a specific formula, which involves a factor $\lambda$ which gets computed as $\lambda = \frac{\sum_{j}{} p_{ij}x_{ij}}{\sum_{j}{} p_{ij}x_{ij}}$. What the specific computation denotes is not important here, what is important is that this computation is always 1. The authors \cite{evensen_extensible_2010} claim that the correct formula should have been: $\lambda = \frac{\sum_{j}{} p_{ij}x_{ij}}{\sum_{j}{} p_{ij}o_{j}}$ with \textit{o} in the denominator instead of \textit{x}. 
	
	\item There seems to be a discrepancy between the paper and the implementation describing the trading of agents. According to the paper the amount they exchange follows $x_{ig} \equiv \frac{p_{ig}x_{ig}}{p_{ih}}$. In the implementation however, Gintis uses $x_{ig} \equiv \frac{p_{ih}x_{ih}}{p_{ig}}$ but seems to do so inconsistently throughout the his code, leading to wrong calculations.
	
	\item Another bug was due to reversed ordering of events, where agents use old prices vectors due to missing recalculation which only happens in the next step.

	\item There are a number of other subtle bugs, which, according to the authors \cite{evensen_extensible_2010} seemed to have no impact on the outcome of the simulation: slight miscalculation of the standard deviation for consumer and producer prices, crashing the program at runtime when dividing the agents unequally between different production goods due to a negative random number which in turn raises an exception in the Delphi Random function.

\end{enumerate}

After having claimed that pure functional programming helps in implementing simulations which are more likely to be correct, the following questions arise:

\paragraph{Do the techniques introduced in this thesis transfer to this problem and model as well?}
The short answer is, yes they obviously do as Gintis models interacting individual agents which consume and produce and trade with each other. This could be implemented both with our time- and event-driven approach, with a better choice probably being the event-driven one due to agents directly trading with each other. However, for some models, our techniques introduced in Part II are too powerful and a much simpler approach would suffice to implement it. In general too much power should always be avoided (at least in programming and software engineering) because with much power comes much responsibility: more power requires to pay more attention to details and thus there is increased risk to make mistakes. Thus we should always look for the technique with minimal power but maximum abstraction, which solves our problem sufficiently. Gintis model is such an example: it resides in a very different domain of ABS, called Agent-Based Computational Economics (ACE).

ACE is a very important field, which picked up ABS as a method for research in recent years. The field of economics is an immensely vast and complex one with many facets to it, ranging from firms, to financial markets to whole economies of a country \cite{bowles_understanding_2005}. Today its very foundations rest on rational expectations, optimization and the efficient market hypothesis. The idea is that the macroeconomics are explained by the micro foundations \cite{colell_microeconomic_1995} defined through behaviour of individual agents. These agents are characterized by rational expectations, optimizing behaviour, having perfect information, equilibrium \cite{focardi_is_2015}.
This approach to economics has come under heavy criticism in the last years for being not realistic, making impossible assumptions like perfect information, not being able to provide a process under which equilibrium is reached \cite{kirman_complex_2010} and failing to predict crashes like the sub-prime mortgage crisis despite all the promises - the science of economics is perceived to be detached from reality \cite{focardi_is_2015}. 
ACE is a promise to repair the empirical deficit which (neo-classic) economics seem to exhibit by allowing to make more realistic, empirical assumptions about the agents which form the micro foundations. The ACE agents are characterized by bounded rationality, local information, restricted interactions over networks and out-of-equilibrium behaviour \cite{farmer_economy_2009}. 
Works which investigate ACE as a discipline and discuss its methodology are \cite{ballot_agent-based_2015,blume_introduction_2015,tesfatsion_agent-based_2002,richiardi_agent-based_2007}.
Tesfatsion \cite{tesfatsion_agent-based_2017} defines ACE as \textit{[...] computational modelling of economic processes (including whole economies) as open-ended dynamic systems of interacting agents.}. 

It is important to understand, that ACE utilises ABS different than the social sciences do. The latter one focuses more on agent-interactions, where in ACE the rational and non-rational actions of individual agents are more important. Thus in many ACE models, the full power of the techniques introduced in Part II is not required. More specifically, agents of ACE models tend to have much simpler state, behave often in only one specific way, don't use synchronised agent-interactions and are very rarely located in a spatial environment but focus more on network connections \cite{glasserman_contagion_2015,wilhite_economic_2006} or avoid the notion of connectivity altogether. This is certainly the fact in the case of the Gintis model, as implemented by Gintis himself and the Java implementation of \cite{evensen_extensible_2010}: 

\begin{itemize}
	\item Even though it is an agent-based model and there is a clear notion of agents in the code, where they are represented as objects, the agents are very simple in terms of their structure. They are characterised by a few floating-point values with very simple behaviour that does not change over time.

	\item There are only very simple direct agent-interactions, exchanging bids and asks, leading to an exchange.

	\item There is no environment whatsoever and a fully connected network is implicitly assumed because each agent can trade with all other agents.
	
	\item The only side effect necessary in this simulation is to draw random-numbers.
\end{itemize}

Thus a Haskell implementation in this domain can be achieved completely without the full force of our techniques: agents can be represented as simple data structures without SFs / MSFs, interactions handled by the kernel, no need for an environment, scheduling is handled directly by the kernel and side effects are restricted to run in the \textit{Rand} Monad only. Due to the substantial complexity of the Gintis model, we have refrained from attempting a full implementation of it in Haskell. However, to investigate this point more in-depth we implemented a simulation \footnote{Available at \url{https://github.com/thalerjonathan/zerointelligence}} with so called Zero Intelligence traders \cite{gode_allocative_1993}, a well known and fundamental concept found in ACE. Our implementation was inspired by an implementation in Python \footnote{\url{http://people.brandeis.edu/~blebaron/classes/agentfin/GodeSunder.html}} and is a satisfactory proof-of-concept underlining the fact that we can do pure functional ABS also in a robust way without the tremendous power offered by our techniques. However, a full treatment of this is beyond the scope of this thesis and is left for further research.

\paragraph{Would the use of Haskell have prevented the bugs which Gintis made?}
The first and second bugs as reported above are an indication for a problematic use of indexed lists or arrays. Generally, one can say that independent of programming languages, indices into an array or list are \textit{always} problematic because it is very easy to make very subtle mistakes by getting indices wrong. Pure functional programming would suffer the same problem \textit{if} a similar technique would have been used. However, due to the fact that data in Haskell is immutable, an \textit{idiomatic} implementation, following pure functional programming concepts, we would have probably not seen the use of lists but (Generalised) Algebraic Data Types, making this kind of bug much less likely. Further, due to the \textit{declarative} nature of pure functional programming in Haskell, it is more likely that the implementer, reading through the existing code might have spotted the bug: there is much less noise in an idiomatic functional implementation than in imperative code, making code highly expressive and concise, thus less to read and more obvious.

The third bug is a very subtle logical error, regarding the semantics of the simulation. A pure functional alone would not have helped avoiding this mistake. However, we think that due to the declarative nature it would have been possible to spot, also because data is immutable and the fact that we are dealing with an old version of data is much more explicit. 

The last bugs are typical run-time errors, which would and do occur in Haskell as well, so a pure functional approach would not necessarily avoid these kind of bugs. However, such bugs can be avoided by using a dependently typed functional language like Idris \cite{brady_idris_2013} as such a type system allows to ensure in the types that only positive random numbers are drawn, and that the input ranges are strictly positive.

Thus summarizing, we hypothesise that with a clean and \textit{idiomatic} pure functional implementation it would have been very likely that Gintis would have avoided or spot the bugs. Further, we claim that dependent types might have been of substantial benefit in the Gintis case, but we leave this for further research as it is beyond the scope of this thesis (see Appendix \ref{app:equilibrium_totality}).

\paragraph{Would property-based tests have been of any help to prevent the bugs?}
We hypothesise that it is very likely that if Gintis would have applied rigorous unit and property-based testing to his model - which he should have, due to the high impact of his outcome - he would have found the inconsistencies and could have corrected them. The code of \cite{evensen_extensible_2010} contains numerous \textit{checkInvariants} and assertions, which \textit{are} properties expressed in code, thus immediately applicable to property-based testing. Further, due to the mathematical nature of the problem, many properties in the form of formulas can be found in the paper specification, which should be directly expressible using property-based and unit testing.

\chapter{Agents As Objects}
After having undertaken this long journey on how to implement ABS pure functionally, what the general computational structures are in ABS and what benefits and drawbacks there are, at the very end of our discussion I want to return to the claim that \textit{agents map naturally to objects} \cite{north_managing_2007}.

My approach of doing ABS and representing agents in pure FP can be interpret as trying to emulate objects in a purely functional way. In this case we have to say: yes agents map naturally to objects. The question is then: are there other, better mechanisms, more in FP I missed / didnt think of ... to implement ABS in FP? I hypothesize that this is probably NOT the case and that every approach in pure FP follows a roughly similar direction with only a few differences. Obviously it is apparent that both OOP and FP are not silverbullets to ABS and both come with their benefits and drawbacks and both have their existence. Thus I hypothesize that we might see the emergence of different computation paradigms in the future which might fit better to ABS than either one.

yes agents map naturally to objects, but what kind of objects? they differ in implementation details and in this thesis proposed a pure functional approach to a notion of objects. objects in Java work different, as well as in Smalltalk and objective c. processes in the functional language Erlang can be seen as objects too as they fullfill all criteria. also cite alan kay
