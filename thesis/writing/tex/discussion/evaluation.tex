\chapter{Evaluation}
\label{ch:evaluation}
TODO: emphasise that we pulled of to implement the full sugarscape with a reasonable approach, this is strong evidence that it works

%we now know how to engineer time- and event-driven ABS with complex state both in the agent and environment, main difficulty is direct agent-interaction (see macal classification into 4 types of ABS), compile-time guaranteed reproducibility, explicit handling of complex state (read only, read/write), concurrency explicit and limited to STM, very promising concurrency but direct agent-interactions main problem (erlang as a rescue?), main drawbacks: everything is explicit, performance

%The case-study strongly hints that our claim that pure FP has indeed its place in ABS is valid but we conclude that it is yet too early to pick up this paradigm for ABS. We think that engineering a proper implementation of a complex ABS model takes substantial effort in pure FP due to different techniques required. We believe that at the moment such an effort pays off only in cases of high-impact and large-scale simulations which results might have far-reaching consequences e.g. influence policy decisions. It is only there where the high requirements for reproducibility, robustness and correctness provided by FP are really needed. Still, we plan on distilling the developed techniques of the case-study into a general purpose ABS library. This should allow implementing models much easier and quicker, making the pure FP approach an attractive alternative for prototyping, opening the direction for a broad use of FP in the field of ABS.

Probably the biggest strength is that we can guarantee reproducibility at compile time: given identical initial conditions, repeated runs of the simulation will lead to same outputs. This is of fundamental importance in simulation and addressed in the Sugarscape model: \textit{"... when the sequence of random numbers is specified ex ante the model is deterministic. Stated yet another way, model output is invariant from run to run when all aspects of the model are kept constant including the stream of random numbers."} (page 28, footnote 16) - we can guarantee that in our pure functional approach already \textit{at compile time}.

Refactoring is very convenient and quickly becomes the norm: guided by types (change / refine them) and relying on the compiler to point out problems, results in very effective and quick changes without danger of bugs showing up at run-time. This is not possible in Python because of its lack of compiler and types, and much less effective in Java due to its dynamic type-system which is only remedied through strong IDE support.

Adding data-parallelism is easy and often requires simply swapping out a data-structure or library function against its parallel version. Concurrency, although still hard, is less painful to address and add in a pure functional setting due to immutable data and explicit side-effects. Further, the benefits of implementing concurrent ABS based on Software Transactional Memory (STM) has been shown \cite{thaler_tale_2018} which underlines the strength of Haskell for concurrent ABS due to its strong guarantees about retry-semantics.

Testing in general allows much more control and checking of invariants due to the explicit handling of effects - together with the strong static type system, the testing-code is in full, explicit control over the functionality it tests. Property-based testing in particular is a perfect match to testing ABS due to the random nature in both and because it supports convenient expressing of specifications. Thus we can conclude that in a pure functional setting, testing is very expressive and powerful and supports working towards an implementation which is very likely to be correct.


\subsection{Verification and Correctness}
General there are the following basic verification \& validation requirements to ABS \cite{robinson_simulation:_2014}, which all can be addressed in our \textit{pure} functional approach as described in the paper in Appendix \ref{app:pfe}:

\begin{itemize}
	%\item Modelling progress of time - achieved using functional reactive programming (FRP)
	%\item Modelling variability - achieved using FRP
	\item Fixing random number streams to allow simulations to be repeated under same conditions - ensured by \textit{pure} functional programming and Random Monads
	\item Rely only on past - guaranteed with \textit{Arrowized} FRP
	\item Bugs due to implicitly mutable state - reduced using pure functional programming
	\item Ruling out external sources of non-determinism / randomness - ensured by \textit{pure} functional programming
	\item Deterministic time-delta - ensured by \textit{pure} functional programming
	\item Repeated runs lead to same dynamics - ensured by \textit{pure} functional programming
\end{itemize}

\begin{enumerate}
	\item Run-Time robustness by compile-time guarantees - by expressing stronger guarantees already at compile-time we can restrict the classes of bugs which occur at run-time by a substantial amount due to Haskell's strong and static type system.  This implies the lack of dynamic types and dynamic casts \footnote{Note that there exist casts between different numerical types but they are all safe and can never lead to errors at run-time.} which removes a substantial source of bugs.  Note that we can still have run-time bugs in Haskell when our functions are partial.
	\item Purity - By being explicit and polymorphic in the types about side-effects and the ability to handle side-effects explicitly in a controlled way allows to rule out non-deterministic side-effects which guarantees reproducibility due to guaranteed same initial conditions and deterministic computation. Also by being explicit about side-effects e.g. Random-Numbers and State makes it easier to verify and test.
	\item Explicit Data-Flow and Immutable Data - All data must be explicitly passed to functions thus we can rule out implicit data-dependencies because we are excluding IO. This makes reasoning of data-dependencies and data-flow much easier as compared to traditional object-oriented approaches which utilize pointers or references.
	\item Declarative - describing \textit{what} a system is, instead of \textit{how} (imperative) it works. In this way it should be easier to reason about a system and its (expected) behaviour because it is more natural to reason about the behaviour of a system instead of thinking of abstract operational details.
	\item Concurrency and parallelism - due to its pure and 'stateless' nature, functional programming is extremely well suited for massively large-scale applications as it allows adding parallelism without any side-effects and provides very powerful and convenient facilities for concurrent programming. The paper of (TODO: cite my own paper on STM) explores the use Haskell for concurrent and parallel ABS in a deeper way.
\end{enumerate}

TODO: haskell-titan
TODO: Testing and Debugging Functional Reactive Programming \cite{perez_testing_2017}

Static type system eliminates a large number run-time bugs.

TODO: can we apply equational reasoning? Can we (informally) reason about various properties e.g. termination?

Follow unit testing of the whole simulation as prototyped for towards paper.

in this we explore something new: property-based testing in ABS

\section{Generalising Research}
We hypothesize that our research can be transferred to other related fields as well, which puts our contributions into a much broader perspective, giving it more impact than restricting it just to the very narrow field of Agent-Based Simulation. Although we don't have the time to back up our claims with in-depth research, we argue that our findings might be applicable to the following fields at least on a conceptual level.

\subsection{Simulation in general}
We already showed in the paper \cite{thaler_pure_2019}, that purity in a simulation leads to repeatability which is of utmost importance in scientific computation. These insights are easily transferable to simulation software in general and might be of huge benefit there. Also my approach to dependent types in ABS might be applicable to simulations in general due to the correspondence between equilibrium \& totality, in use for hypotheses formulation and specifications formulation as pointed out in Chapter \ref{ch:dependent}. 

\subsection{System Dynamics}
\label{sub:generalising_system_dynamics}
we have done that already in the appendix

discuss pure functional system dynamics - correct by construction: benefits: strictly deterministic already at compile time, encode equations directly in code => correct by construction. Can serve as backend implementation of visual SD packages.

\subsection{Discrete Event Simulation}
pure functional DES easily possible with my developed synchronous messaging ABS
DES in FP: we doing part of it in event-driven SIR. this is a potential hint at how to achieve DES in pure FP: arrowized FRP seems to be perfect for it because it allows to express data-flow networks, which is exactly what DES is as well. The problem with pure FP is that it will be more involved to propagate events through the network especially when they don't originate from the source but e.g. after a time-out from a queue. Generally all the techniques are there as discussed in the event-driven chapter but it would be interesting to do research on how to achieve the same in DES. Because data-flow networks of DES generally don't change at runtime, they are fixed already at compile time - it would be interesting to see what dependent types could offer for additional compile-time guarantees.

PDES, should be  conceptually easil possible using STM, optimistic approach should be conceptually easier to implement due to persistent data-structures and controlled side-effects
 
\subsection{Recursive Simulation}
Due to the recursive nature of FP we believe that it is also a natural fit to implement recursive simulations as the one discussed in \cite{gilmer_recursive_2000}. In recursive ABS agents are able to halt time and 'play through' an arbitrary number of actions, compare their outcome and then to resume time and continue with a specifically chosen action e.g. the best performing or the one in which they haven't died. More precisely, an agent has the ability to run the simulation recursively a number of times where the number is not determined initially but can depend on the outcome of the recursive simulation. So recursive ABS gives each Agent the ability to run the simulation locally from its point of view to project its actions into the future and change them in the present. Due to controlled side-effects and referential transparency, combined with the recursive nature of pure FP, we think that implementing a recursive simulation in such a setting should be straight-forward.

Inspired by \cite{gilmer_recursive_2000}, add ideas about recursive simulation described in 1st year report and "paper". functional programming maps naturally here due to its inherently recursive nature and controlled side-effects which makes it easier to construct correct recursive simulations.
recursive simulation should be conceptually easier to implememt and more likely to be correct due to recursive Nature of haskell itself, lack of sideeffeccts and mutable data

\subsection{Multi Agent Systems}
The fields of Multi Agent Systems (MAS) and ABS are closely related where ABS has drawn much inspiration from MAS \cite{wooldridge_introduction_2009}, \cite{weiss_multiagent_2013}. It is important to understand that MAS and ABS are two different fields where in MAS the focus is more on technical details, implementing a system of interacting intelligent agents within a highly complex environment with the focus on solving AI problems.

Because in both fields, the concept of interacting agents is of fundamental importance, we expect our research also to be applicable in parts to the field of MAS. Especially the work on dependent types should be very useful there because MAS is very interested in correctness, verification and formally reasoning about a system and their agents, to show that a system follows a formal specifications.

\section{Drawbacks}
\label{sec:drawbacks}

\subsection{Space-Leaks}
discuss the problem (and potential) of lazy evaluation for ABS: can under some circumstances really increase performance when some stuff is not evaluated (see STM study) but mostly it causes problems by piling up unevaluated thunks leading to crazy memory usage which is a crucial problem in simulation. Using strict pragmas, annotations and data-structures solves the problem but is not trivial and involves carefully studying the code / getting it right from the beginning / and using the haskell profiling tools (which are fucking great at least). TODO: show the stats of memory usage

Haskell is notorious for its memory-leaks due to lazy evaluation: data is only evaluated when required. Even for simple programs one can be hit hard by a serious space-leak where unevaluated code pieces (thunks) build up in memory until they are needed, leading to dramatically increased memory usage. It is no surprise that our highly complex Sugarscape implementation initially suffered severely from space-leaks, piling up about 40 MByte / second. In simulation this is a big issue, threatening the value of the whole implementation despite its other benefits: because simulations might run for a (very) long time or conceptually forever, one must make absolutely sure that the memory usage stays somewhat constant. As a remedy, Haskell allows to add so-called strictness pragmas to code-modules which forces strict evaluation of all data even if it is not used. Carefully adding this conservatively file-by file applying other techniques of forcing evaluation removed most of the memory leaks.

% Another memory leak was caused by selecting the wrong data-structure for our environment, for which we initially used an immutable array. The problem is that in the case of an update the whole array is copied, causing memory leaks AND a performance problem. We replaced it by an IntMap which uses integers as key (mapping 2d coordinates to unique integers is trivial) and is internally implemented as a radix-tree which allows for very fast lookups and inserts because whole sub-trees can be re-used.


\subsection{Efficiency}
ordering of the transformers
when to run a transformer,
lazyness vs. strictness

The main drawback of our approach is performance, which at the moment does not come close to OO implementations. There are two main reasons for it: first, FP is known for being slower due to higher level of abstractions, which are bought by slower code in general and second, updates are the main bottleneck due to immutable data requiring to copy the whole (or subparts) of a data structure in cases of a change. The first one is easily addressable through the use of data-parallelism and concurrency as we have done in our paper on STM \cite{thaler_tale_2018}. The second reason can be addressed by the use of linear types \cite{bernardy_linear_2017}, which allow to annotate a variable with how often it is used within a function. From this a compiler can derive aggressive optimisations, resulting in imperative-style performance but retaining the declarative nature of the code.

\subsection{Productivity and learning curve}
A case study in \cite{hanenberg_experiment_2010} hints that simply by switching to a static typesystem alone does not gain anything and can even be detriment. It also needs to have a certain level of abstractions like Haskell type system does or even dependent types as in Idris. This also mean that there is a substantial learning curve to master when one wants to enter pure functional ABS.
