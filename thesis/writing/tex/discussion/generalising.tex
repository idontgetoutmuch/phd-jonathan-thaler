\section{Generalising research}
We hypothesise that our research might be applicable to related fields as discussed below, which puts our contributions into a much broader perspective, giving it more impact.

\subsection{Simulation in general}
We showed at length in this thesis that purity in a simulation leads to reproducibility, which is of utmost importance in scientific computation. These insights are easily transferable to simulation software in general and might be of huge benefit there. Also, dependent types in ABS might be applicable to simulations in general due to the correspondence between equilibrium and totality, in use for hypotheses formulation and specifications formulation as pointed out in Appendix \ref{app:equilibrium_totality}.  We think the use of Software Transactional Memory for implementing concurrent simulations is not only beneficial to ABS in particular but potentially applies to a wide range of simulation problems with a data-driven approach. Also, we demonstrated the usefulness of property-based testing in this thesis, which we believe can be a useful tool for simulation in general, to encode specifications, explore dynamics, test hypotheses and perform verification and validation directly in code.

\subsection{System Dynamics}
We have implemented a System Dynamics (SD) simulation in Appendix \ref{app:sd_simulation}. Although it is only an implementation of the SIR model, it shows clearly how an SD \textit{specification} can directly be encoded in pure functional programming using arrowized FRP. The approach is simple enough to be generated automatically, is highly performant and could serve as language of choice for an SD simulation engine backend.

\subsection{Discrete Event Simulation}
We have already implemented basic mechanics of a Discrete Event Simulation (DES) in the case of the event-driven SIR. However, it would be very interesting to see whether we can express a DES model network directly in code, guaranteeing compatibility of elements statically at compile time. This should be possible due to the declarative nature of pure functional programming and the process-oriented nature of arrowized FRP. In the end both DES and FRP model data-flow networks which are fixed at compile time, so it should be a natural fit.

There also exists a parallel version of DES, called Parallel DES \cite{fujimoto_parallel_2017}, which is concerned about the problem of running a DES in parallel, where the challenge is how to deal with sequential dependencies. Optimistic approaches run them in parallel, just like Software Transactional Memory does, and rolls back actions in case sequential orderings are violated. We hypothesise that in a pure functional language with or without Software Transactional Memory it should be conceptually easier to implement such rollback semantics due to immutable persistent data structures and controlled side effects.
 
\subsection{Recursive simulation}
Due to the recursive nature of functional programming we believe that it is also a natural fit to implement recursive simulations as the one discussed in \cite{gilmer_recursive_2000}. In recursive ABS agents are able to halt time and anticipate an arbitrary number of actions, compare their outcome, resume time and continue with a specifically chosen action with the best outcome. More precisely, an agent has the ability to run the simulation recursively a number of times where the number is not determined initially but can depend on the outcome of the recursive simulation. So recursive ABS gives each agent the ability to run the simulation locally from its point of view to project its actions into the future and change them in the present. Due to controlled side effects and referential transparency, combined with the recursive nature of pure functional programming, we think that implementing a recursive simulation in such a setting should be straightforward.

\subsection{Multi Agent Systems}
The fields of Multi Agent Systems (MAS) and ABS are closely related, where ABS has drawn much inspiration from MAS \cite{weiss_multiagent_2013,wooldridge_introduction_2009}. It is important to understand that MAS and ABS are two different fields where in MAS the focus is more on technical details, implementing a system of interacting intelligent agents within a highly complex environment with the focus on solving AI problems.

In both fields, the concept of interacting agents is of fundamental importance, so we expect our research also to be applicable in parts to the field of MAS. Especially the work on static guarantees and property-based testing should be very useful there because MAS is very interested in correctness, verification and formally reasoning about a system and their agents, to show that a system follows a formal specification.