\chapter{Favouring less power}
combine this into the gintis case as gintis simulation IS an ACE! so its a perfect match without being too detached.

For many models, our techniques introduced in Part II are too powerful and a much simpler approach would suffice to implement it. In general too much power should always be avoided (at least in programming and software engineering) because with much power comes much responsibility: more power requires to pay more attention to details and thus there is more potential to make mistakes. Thus we should always look for the technique with minimal power, which solves our problem sufficiently.

A very important field, which picked up ABS in recent years is economics. The field of economics is an immensely vast and complex one with many facets to it, ranging from firms, to financial markets to whole economies of a country \cite{bowles_understanding_2005}. Today its very foundations rest on rational expectations, optimization and the efficient market hypothesis. The idea is that the macroeconomics are explained by the micro foundations \cite{colell_microeconomic_1995} defined through behaviour of individual agents. These agents are characterized by rational expectations, optimizing behaviour, having perfect information, equilibrium \cite{focardi_is_2015}.
This approach to economics has come under heavy critizism in the last years for being not realistic, making impossible assumptions like perfect information, not being able to provide a process under which equilibrium is reached \cite{kirman_complex_2010} and failing to predict crashes like the sub-prime mortgage crisis despite all the promises - the science of economics is perceived to be detached from reality \cite{focardi_is_2015}. 
ACE is a promise to repair the empirical deficit which (neo-classic) economics seem to exhibit by allowing to make more realistic, empirical assumptions about the agents which form the micro foundations. The ACE agents are characterized by bounded rationality, local information, restricted interactions over networks and out-of-equilibrium behaviour \cite{farmer_economy_2009}. 
Works which investigate ACE as a discipline and discuss its methodology are \cite{tesfatsion_agent-based_2002}, \cite{richiardi_agent-based_2007}, \cite{ballot_agent-based_2015}, \cite{blume_introduction_2015}.
%look into computable economics book: \url{http://www.e-elgar.com/shop/computable-economics}
Tesfatsion \cite{tesfatsion_agent-based_2017} defines ACE as \textit{[...] computational modelling of economic processes (including whole economies) as open-ended dynamic systems of interacting agents.}. She gives a broad overview \cite{tesfatsion_agent-based_2006} of ACE, discusses advantages and disadvantages and giving the four primary objectives of it which are:

\begin{enumerate}
	\item Empirical understanding: why have particular global regularities evolved and persisted, despite the absence of centralized planning and control?
	\item Normative understanding: how can agent-based models be used as laboratories for the discovery of good economic designs?
	\item Qualitative insight and theory generation: how can economic systems be more fully understood through a systematic examination of their potential dynamical behaviours under alternatively specified initial conditions?
	\item Methodological advancement: how best to provide ACE researchers with the methods and tools they need to undertake the rigorous study of economic systems through controlled computational experiments?
\end{enumerate}

It is important to understand, that ACE utilises ABS different than the social sciences do. The latter one focuses more on agent-interactions, where in ACE the rational and non-rational actions of individual agents are more important. Thus in many ACE models, the full power of the techniques introduced in Part II is not required. More specifically, agents of ACE models tend to have much simpler state, behave often in only one specific way, don't use synchronised agent-interactions and are very rarely located in a spatial environment but focus more on network connections \cite{wilhite_economic_2006, glasserman_contagion_2015} or avoid the notion of connectivity altogether.

\medskip

To investigate this point more in-depth we implemented \footnote{Freely available at \url{https://github.com/thalerjonathan/zerointelligence}} a simulation with so called Zero Intelligence traders \cite{gode_allocative_1993}, inspired by an implementation in Python \footnote{\url{http://people.brandeis.edu/~blebaron/classes/agentfin/GodeSunder.html}}. We don't go into any technical detail here but the implementation drives the main points home:

\begin{itemize}
	\item Even though it is an agent-based model and there is a clear notion of agents in the Python code, where they are represented as objects, the agents are extremely simple. They are characterised by a single floating-point value, identifying how much value they attribute to an asset. Their behaviour is also very simple and does not change over time: they always bid randomly within their profit range. Thus we do \textit{not} implement agents as MSFs in this case but represent them indeed only through a \textit{Double} value, reducing the complexity of the implementation considerably.

	\item There are no direct agent-interactions. Although agents trade with each other, this happens through a central authority (the simulation kernel), which acts like a market with a limit order book. This reduces the complexity of the implementation considerably because there is no need for the full approach of synchronised direct agent-interactions. We could have implemented it in that way but that would have only increased complexity through the use of a quite powerful technique, which is actually not really needed because the same effect can be achieved in much simpler terms.

	\item There is no environment whatsoever and a fully connected network is implicitly assumed because each agent can trade with all other agents. This implies that the full technique of applying an environment is not necessary, which makes the simulation a lot less complex. Still adding an environment e.g. a network would be quite simple and does not require any monadic code as the network information can be made read-only in the way as we do in Chapter \ref{sec:adding_env}.
	
	\item The only side-effect necessary in this simulation is to draw random-numbers. By fixing the seed, repeated runs of initial conditions will always lead to same output, which is guaranteed at compile time. This was already shown in Part II and is a direct consequence of Haskells type-system and explicit way of dealing with effects.	 Further, we focused on keeping as much code \textit{pure} as possible thus splitting code which does not require random numbers into pure functions and only having the basic structure of the implementation running in the Rand Monad. This makes testing and reasoning considerably easier than running everything in the Rand Monad.
\end{itemize}

We are very well aware that this simple example is only one of many ACE models but even though it implements very simple \textit{zero} intelligence agents, it shows that ABS in Haskell does not need to be as complex as the use-cases in Part II - on the contrary, ABS implementations can be very concise and highly performant in Haskell.