\section{The Gintis Case}
\label{sec:gintis_case}
Discuss my developed techniques to the Gintis paper (and its follow ups: the Ionescu paper \cite{botta_functional_2011} and a Masterthesis \cite{evensen_extensible_2010} on it). Answer the following:

\begin{enumerate}
	\item Do the techniques transfer to this problem and model? 
	
	\item Could pure functional programming have prevented the bugs which Gintis made? 
	
	\item Would property-based tests have been of any help to preven the bugs?
	
	\item Could dependent and / or types have prevented the bugs which Gintis made? 
	
	\item How close is our (dependently typed) implementation to Ionescus functional specification? 
	
	\item When having Cezar Ionescu as external examiner, this chapter will be of great influence as it deals heavily with his work.

\end{enumerate}

Not yet started, need to implement it but there exists code for it already (gintis and java implementations)

%
%after re-reading ionescu paper: too complex and out of scope, but ionescu work more directly applicable in a pure functional implementation than in e.g. c++ (that was what they used).
%
%we base our implementation on the existing gintis code from https://people.umass.edu/gintis/ 
%also we make use of the \cite{evensen_extensible_2010} on gintis work which revealed a few bugs
%
%NOTE: my hypothesis is that just by having used our pure functional approach would NOT have prevented gintis to have made the bugs as reported in the masterthesis \cite{evensen_extensible_2010} because they seemed to be like copy-paste bugs. Only rigorous code-testing (unit- / property-based) would have probably revealed these problems.
%
