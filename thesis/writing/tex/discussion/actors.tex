\section{Actors}
\label{sec:actors}
TODO: this seems not to fit into the narrative here, maybe it fits into discussion part or further research

The Actor-Model, a model of concurrency, was initially conceived by Hewitt in 1973 \cite{hewitt_universal_1973} and refined later on \cite{hewitt_what_2007}, \cite{hewitt_actor_2010}. It was a major influence in designing the concept of agents and although there are important differences between actors and agents there are huge similarities thus the idea to use actors to build agent-based simulations comes quite natural. The theory was put on firm semantic grounds first through Irene Greif by defining its operational semantics \cite{grief_semantics_1975} and then Will Clinger by defining denotational semantics \cite{clinger_foundations_1981}. In the seminal work of Agha \cite{agha_actors:_1986} he developed a semantic mode, he termed \textit{actors} which was then developed further \cite{agha_foundation_1997} into an actor language with operational semantics which made connections to process calculi and functional programming languages (see both below). 

An actor is a uniquely addressable entity which can do the following \textit{in response to a message}
\begin{itemize}
	\item Send an arbitrary number (even infinite) of messages to other actors.
	\item Create an arbitrary number of actors.
	\item Define its own behaviour upon reception of the next message.
\end{itemize}

In the actor model theory there is no restriction on the order of the above actions and so an actor can do all the things above in parallel and concurrently at the same time. This property and that actors are reactive and not pro-active is the fundamental difference between actors and agents, so an agent is \textit{not} an actor but conceptually nearly identical and definitely much closer to an agent in comparison to an object. The actor model can be seen as quite influential to the development of the concept of agents in ABS, which borrowed it from Multi Agent Systems \cite{wooldridge_introduction_2009}. Technically, it emphasises message-passing concurrency with share-nothing semantics (no shared state between agents), which maps nicely to functional programming concepts.

There have been a few attempts on implementing the actor model in real programming languages where the most notable ones are Erlang and Scala. Erlang was created in 1986 by Joe Armstrong for Eriksson for developing distributed high reliability software in telecommunications. It implements light-weight processes, which allows to spawn thousands of them without heavy memory overhead. The language saw some use in implementing ABS with notable papers being \cite{di_stefano_using_2005, di_stefano_exat:_2007, varela_modelling_2004, sher_agent-based_2013, bezirgiannis_improving_2013}

Scala is a modern mixed paradigm programming language, which also allows functional programming and also incorporates a library for the actor model. It also saw the use in the implementation of ABS with a notable paper \cite{krzywicki_massively_2015} and ScalABM \footnote{https://github.com/ScalABM} which is alibrary for ABM in economics.

The paper of \cite{jennings_agent-based_2000} gives an excellent overview over the strengths and weaknesses of agent-based software-engineering, which can be directly applied to both Erlang and Scala.

Due to the very different approach and implications the actor model of concurrency implies, we don't explore it further and leave it for further research as it is beyond the focus of the thesis.