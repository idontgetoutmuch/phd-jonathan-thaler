\chapter{Discussion}
\label{ch:discussion}
% starting point
This thesis started out by challenging the established views that \textit{"[..] object-oriented programming to be a particularly natural development environment for Sugarscape specifically and artificial societies generally [..]"} \cite{epstein_growing_1996} (p. 179) and that \textit{agents map naturally to objects} \cite{north_managing_2007}. As an alternative, a radical different approach to implementing ABS was proposed, using the pure functional programming paradigm. As language of choice, Haskell was motivated due to its  pure functional features, matureness and increasing relevance to real-world applications. 

% motivation
The conjecture was that by using Haskell, one can directly transfer the promises made by pure functional programming to ABS as well, directly gaining a few highly important benefits.  The relevance of each of these promises to ABS was already pointed out in the respective chapters and it is quite obvious that these benefits would clearly be of immense value in ABS. The common baseline is that all those benefits support making implementations of ABS which are more likely to be correct, something of fundamental value in simulation.

\begin{enumerate}
	\item The static strong type system allows to remove a substantial number and classes of bugs at run-time and if one programs careful one can even guarantee that no bugs as in crashes or exceptions will occur at run-time. This is particularly the case for purely computational problems, without IO \footnote{Obviously IO is involved in all ABS, otherwise the results are not observable. By purely computational, we identify lack of IO in the agents, as fundamental part of the model, other than visualisation and exporting to an output file.}, as ABS almost always are. 
	
	\item Explicit handling and control of side-effects delivers even more static guarantees at compile time and allows to deal with deterministic side-effects (random-number streams, read-/write only contexts, state) in a referential transparent way. In combination with strong static typing this allows to reduce logical bugs (subject to the domain of the problem) by dramatically reducing available, valid operations on data - after all stateful applications are a fact, the challenge is how to deal with state. As ABS is an inherently stateful problem due to agents and the environment, this should  theincrease correctness of an ABS implementation as well. Further, this should allow to produce an implementation which is guaranteed to reproducible (runs with same initial conditions \textit{will} result in same dynamics) at compile time.
	
	\item Parallel and concurrent programming is claimed to be lot easier, less painful and less error prone in functional programming in general and in Haskell in particular due to immutable data and the explicit handling of side-effects. The concept of Software Transactional Memory, which allows to express a problem as a data-flow problem is highly promising. Besides, data-parallel programming promises to speed up code without the need for changing any of the logic or types. This seemed to offer a straightforward way of speeding up ABS implementations either through data-parallelism or concurrency. This has always been quite difficult to achieve in traditional object-oriented ABS and pure functional programming seems to offer a solution.
	
	\item The data-centric declarative style, referential transparency and immutability of data makes testing substantially easier due to composability: functions can be easily tested in isolation from each other even if they involve side-effects. This opens the door for (randomised) property-based testing which intuitively seemed to be a perfect match to test ABS implementations which are (almost) always stochastic in nature.
\end{enumerate}

We argue that the thesis has successfully demonstrated in the respective chapters that all promises successfully transfer to pure functional ABS implementations in a reasonable, robust and maintainable way. Below we discuss respective points more in-depth and give a holistic reflection and critique of our pure functional approach in general.

\chapter{Benefits}
TODO

TODO: emphasise that we pulled of to implement the full sugarscape with a reasonable approach, this is strong evidence that it works

Probably the biggest strength is that we can guarantee reproducibility at compile time: given identical initial conditions, repeated runs of the simulation will lead to same outputs. This is of fundamental importance in simulation and addressed in the Sugarscape model: \textit{"... when the sequence of random numbers is specified ex ante the model is deterministic. Stated yet another way, model output is invariant from run to run when all aspects of the model are kept constant including the stream of random numbers."} (page 28, footnote 16) - we can guarantee that in our pure functional approach already \textit{at compile time}.

Refactoring is very convenient and quickly becomes the norm: guided by types (change / refine them) and relying on the compiler to point out problems, results in very effective and quick changes without danger of bugs showing up at run-time. This is not possible in Python because of its lack of compiler and types, and much less effective in Java due to its dynamic type-system which is only remedied through strong IDE support.

Adding data-parallelism is easy and often requires simply swapping out a data-structure or library function against its parallel version. Concurrency, although still hard, is less painful to address and add in a pure functional setting due to immutable data and explicit side-effects. Further, the benefits of implementing concurrent ABS based on Software Transactional Memory (STM) has been shown \cite{thaler_tale_2018} which underlines the strength of Haskell for concurrent ABS due to its strong guarantees about retry-semantics.

Testing in general allows much more control and checking of invariants due to the explicit handling of effects - together with the strong static type system, the testing-code is in full, explicit control over the functionality it tests. Property-based testing in particular is a perfect match to testing ABS due to the random nature in both and because it supports convenient expressing of specifications. Thus we can conclude that in a pure functional setting, testing is very expressive and powerful and supports working towards an implementation which is very likely to be correct.


\subsection{Verification and Correctness}
General there are the following basic verification \& validation requirements to ABS \cite{robinson_simulation:_2014}, which all can be addressed in our \textit{pure} functional approach as described in the paper in Appendix \ref{app:pfe}:

\begin{itemize}
	%\item Modelling progress of time - achieved using functional reactive programming (FRP)
	%\item Modelling variability - achieved using FRP
	\item Fixing random number streams to allow simulations to be repeated under same conditions - ensured by \textit{pure} functional programming and Random Monads
	\item Rely only on past - guaranteed with \textit{Arrowized} FRP
	\item Bugs due to implicitly mutable state - reduced using pure functional programming
	\item Ruling out external sources of non-determinism / randomness - ensured by \textit{pure} functional programming
	\item Deterministic time-delta - ensured by \textit{pure} functional programming
	\item Repeated runs lead to same dynamics - ensured by \textit{pure} functional programming
\end{itemize}

\begin{enumerate}
	\item Run-Time robustness by compile-time guarantees - by expressing stronger guarantees already at compile-time we can restrict the classes of bugs which occur at run-time by a substantial amount due to Haskell's strong and static type system.  This implies the lack of dynamic types and dynamic casts \footnote{Note that there exist casts between different numerical types but they are all safe and can never lead to errors at run-time.} which removes a substantial source of bugs.  Note that we can still have run-time bugs in Haskell when our functions are partial.
	\item Purity - By being explicit and polymorphic in the types about side-effects and the ability to handle side-effects explicitly in a controlled way allows to rule out non-deterministic side-effects which guarantees reproducibility due to guaranteed same initial conditions and deterministic computation. Also by being explicit about side-effects e.g. Random-Numbers and State makes it easier to verify and test.
	\item Explicit Data-Flow and Immutable Data - All data must be explicitly passed to functions thus we can rule out implicit data-dependencies because we are excluding IO. This makes reasoning of data-dependencies and data-flow much easier as compared to traditional object-oriented approaches which utilize pointers or references.
	\item Declarative - describing \textit{what} a system is, instead of \textit{how} (imperative) it works. In this way it should be easier to reason about a system and its (expected) behaviour because it is more natural to reason about the behaviour of a system instead of thinking of abstract operational details.
	\item Concurrency and parallelism - due to its pure and 'stateless' nature, functional programming is extremely well suited for massively large-scale applications as it allows adding parallelism without any side-effects and provides very powerful and convenient facilities for concurrent programming. The paper of (TODO: cite my own paper on STM) explores the use Haskell for concurrent and parallel ABS in a deeper way.
\end{enumerate}

TODO: haskell-titan
TODO: Testing and Debugging Functional Reactive Programming \cite{perez_testing_2017}

Static type system eliminates a large number run-time bugs.

TODO: can we apply equational reasoning? Can we (informally) reason about various properties e.g. termination?

Follow unit testing of the whole simulation as prototyped for towards paper.

in this we explore something new: property-based testing in ABS

\section{Drawbacks}
\label{sec:drawbacks}
Obviously nothing comes without drawbacks, and also our approach suffers from a few. We discuss them here and propose solutions where applicable.

\subsection{Efficiency}
TODO FINISH
- report timings of time- and event-driven java implementation
- performance difference to time- and event-driven java implementation is extreme, up to a point where it completely invalidates our approach. the STM improves on that but lacks behind. this is really really bad and basically a negative result of our phd...
- we dont have much experience improving performance of haskell but directions are: careful usage of MTL: every layer increases run-time, also re-ordering of lifts results in improved performance e.g. instead of mapM (lift ...) do lift (mapM ...).
- read a bit about potential haskell performance optimisation, https://wiki.haskell.org/Performance

Currently, the performance of the system does not come close to imperative implementations. We compared the performance of our pure functional approach as presented in Section \ref{sec:adding_env} to an implementation in Java using the ABS library RePast \cite{north_complex_2013}. We ran the simulation until $t = 100$ on a 51x51 (2,601 agents) with $\Delta t = 0.1$ (unknown in RePast) and averaged 8 runs. The performance results make the lack of speed of our approach quite clear: the pure functional approach needs around 72.5 seconds whereas the Java RePast version just 10.8 seconds on our machine to arrive at $t = 100$. It must be mentioned, that RePast does implement an event-driven approach to ABS, which can be much more performant \cite{meyer_event-driven_2014} than a time-driven one as ours, so the comparison is not completely valid. Still, we have already started investigating speeding up performance through the use of Software Transactional Memory \cite{harris_composable_2005, harris_transactional_2006}, which is quite straight forward when using MSFs. It shows very good results but we have to leave the investigation and optimization of the performance aspect of our approach for further research as it is beyond the scope of this paper.

The main drawback of our approach is performance, which at the moment does not come close to OO implementations. There are two main reasons for it: first, FP is known for being slower due to higher level of abstractions, which are bought by slower code in general and second, updates are the main bottleneck due to immutable data requiring to copy the whole (or subparts) of a data structure in cases of a change. The first one is easily addressable through the use of data-parallelism and concurrency as shown in the respective chapters. The second reason can be addressed by the use of linear types \cite{bernardy_linear_2017}, which allow to annotate a variable with how often it is used within a function. From this a compiler can derive aggressive optimisations, resulting in imperative-style performance but retaining the declarative nature of the code. We leave the latter one for further research.

Also the use of monad transformer stacks has performance implications which can be quite subtle. We point at alternative approaches to effects in the Further research chapter.

\subsection{Space-Leaks}
Haskell is notorious for its memory-leaks due to lazy evaluation: data is only evaluated when required. Even for simple programs one can be hit hard by a serious space-leak where unevaluated code pieces (thunks) build up in memory until they are needed, leading to dramatically increased memory usage. It is no surprise that our highly complex Sugarscape implementation initially suffered severely from space-leaks, piling up about 40 MByte / second. In simulation this is a big issue, threatening the value of the whole implementation despite its other benefits: because simulations might run for a (very) long time or conceptually forever, one must make absolutely sure that the memory usage stays somewhat constant. As a remedy, Haskell allows to add so-called strictness pragmas to code-modules which forces strict evaluation of all data even if it is not used. Carefully adding this conservatively file-by file applying other techniques of forcing evaluation removed most of the memory leaks.

Another memory leak was caused by selecting the wrong data-structure for our environment, for which we initially used an immutable array. The problem is that in the case of an update the whole array is copied, causing memory leaks AND a performance problem. We replaced it by an IntMap which uses integers as key (mapping 2d coordinates to unique integers is trivial) and is internally implemented as a radix-tree which allows for very fast lookups and inserts because whole sub-trees can be re-used.

%discuss the problem (and potential) of lazy evaluation for ABS: can under some circumstances really increase performance when some stuff is not evaluated (see STM study) but mostly it causes problems by piling up unevaluated thunks leading to crazy memory usage which is a crucial problem in simulation. Using strict pragmas, annotations and data-structures solves the problem but is not trivial and involves carefully studying the code / getting it right from the beginning / and using the haskell profiling tools (which are fucking great at least). TODO: show the stats of memory usage

\subsection{Productivity and learning curve}
A case study in \cite{hanenberg_experiment_2010} hints that simply by switching to a static type system alone does not gain anything and can even be detriment. It also needs to have a certain level of abstractions like Haskell type system does or even dependent types as in Idris. Although such case-studies have to be taken with care, there is also some truth in it: working in a statically strong type system prevents the developer from moving quickly and making quick and dirty changes. This can be both a benefit and a drawback: in general it prevents from breaking changes which show only up at compile time as in dynamic languages but at the same time the whole program is much more rigid and a proper structure needs to be thought out and designed often up-front, slowing down the process. However, this is a contribution of this thesis that it outlines exactly these structures within ABS so that implementers who want to use the same approach do not have to reinvent the wheel.

A more severe problem is that pure functional programming, especially Haskell is seen as hard to learn with a steep learning curve. This and the fact that pure functional ABS achieves similar things as existing OO approaches puts a high barrier to implementers picking up a pure functional approach to ABS. Thus the lack of availability of Haskell expertise can be enough to pose a serious drawback even if the approach of this thesis seem to be desirable in a project.

\subsection{Agent interactions}
This thesis is \textit{one} way of showing how to separate both and reap the benefits. A time-driven ABS like SIR or an ACE with simple agents not interaction with each other like ZI traders is heavily data-centric and very low on agent-interaction. Such data-driven ABS models are quite well expressed in a purely functional approach with the advantage that one can reap the benefits of reproducibility at compile time, using STM for concurrency and property-based testing for verification and validation. An event-driven simulation with complex agent state and agent-interactions like social simulations like Sugarscape or Chemical or Biology simulation with cell interactions are also possible in a pure functional setting as we have shown in the case of the Sugarscape model. Although we were able to give a good solution to complex agent state and synchronous, direct Agent-interactions in our event-driven SIR and Sugarscape and they \textit{do} work in Haskell, they are cumbersome to get right without library support and as we pointed out in the STM chapter, it seems that this approach to synchronous bi-directional agent-interactions is not applicable to concurrency with STM. 

This has lead to the fundamental conclusion of this thesis, that although we have shown the benefit of functional programming in ABS, in models which require complex agent-interactions in a potentially concurrent we are hitting the limits of the pure functional approach here. The reason for it is that ultimately agents do not map naturally to objects but agents map naturally to actors. 

The Actor-Model, a model of concurrency, was initially conceived by Hewitt in 1973 \cite{hewitt_universal_1973} and refined later on \cite{hewitt_what_2007}, \cite{hewitt_actor_2010}. It was a major influence in designing the concept of agents and although there are important differences between actors and agents there are huge similarities thus the idea to use actors to build agent-based simulations comes quite natural. The theory was put on firm semantic grounds first through Irene Greif by defining its operational semantics \cite{grief_semantics_1975} and then Will Clinger by defining denotational semantics \cite{clinger_foundations_1981}. In the seminal work of Agha \cite{agha_actors:_1986} he developed a semantic mode, he termed \textit{actors} which was then developed further \cite{agha_foundation_1997} into an actor language with operational semantics which made connections to process calculi and functional programming languages (see both below). 

An actor is a uniquely addressable entity which can do the following \textit{in response to a message}
\begin{itemize}
	\item Send an arbitrary number (even infinite) of messages to other actors.
	\item Create an arbitrary number of actors.
	\item Define its own behaviour upon reception of the next message.
\end{itemize}

In the actor model theory there is no restriction on the order of the above actions and so an actor can do all the things above in parallel and concurrently at the same time. This property and that actors are reactive and not pro-active is the fundamental difference between actors and agents, so an agent is \textit{not} an actor but conceptually nearly identical and definitely much closer to an agent in comparison to an object. The actor model can be seen as quite influential to the development of the concept of agents in ABS, which borrowed it from Multi Agent Systems \cite{wooldridge_introduction_2009}. Technically, it emphasises message-passing concurrency with share-nothing semantics (no shared state between agents), which maps nicely to functional programming concepts.

The programming-model of actors \cite{agha_actors:_1986} was the inspiration for the Erlang programming language \cite{armstrong_erlang_2010}, which was created in the 1980's by Joe Armstrong for Eriksson for developing distributed high reliability software in telecommunications. The implication is that, the focus would shift immediately to the use of the actor model for concurrent interaction of agents through messages. The languages type-system is strong and dynamic and thus lacks type-checking at compile-time. Thus the structure of computation plays naturally no role because we cannot look at it from the abstract perspective as we can in Haskell. Purity can not be guaranteed and due to agents being processes concurrency is everywhere, and even though it is very tamed through shared-nothing messaging semantics, this implies that repeated runs with same initial conditions might lead to different results. Obviously we could avoid implementing agents as processes but then we basically sacrifice the very heart and feature of the language.

Despite its focus on messages, Erlang is a functional languages, which puts you into the data-centric approach: messages are pure data with \textit{shared nothing semantics}. This makes testing easier and also opens the way for property-based testing which is available in Erlang as well where it even allows to detect race conditions \cite{claessen_finding_2009}. 

We hypothesize that an true concurrent actor approach like Erlang is substantially more natural, much more performant and opens up for concurrency. its main drawback is its dynamic type system which reduces guarantees we have about correctness at run-time, also concurrency leads to different dynamics. We have prototyped highly promising concurrent event-driven SIR and Sugarscape implementations in Erlang and they supports our hypothesis. Unfortunately, an in-depth discussion is beyond the scope of this thesis and which we leave this for highly promising further research in the conclusion.

%erlang processes can implement everything objects can but in a referential transparent and pure functional way: encapsulation, polymorphism, identity, message passing, even inheritance (which you wouldnt want to do)

%difference of erlang to objects is that although it encapsulate state it cannot be accessed at the same time but only through the message passing Interface with one message at a time. this means that state is not really shared and protected against mutation - the process is in full control. it is simply a function which captures the full state of the process in an immutable way: to change the state a recursive call needs to be done. so in the end although it seems conceptually related its technically difference is fundamental importance.
%
%my mistake was to confuse the concept of objects with their implementation. i was too focused on the drawbacks of e.g. java objects that i forgot that i was critisising its IMPLEMENTATION. the Original idea of alan kays objects IS a deep and strong idea, though it differes substantially from java objects, erlang comes closest. thus the concept is important and valid but different implementations have different benefits and drawbacks. 

\section{The Gintis case revisited}
\label{sec:gintis_case} 
After having established all the benefits and drawbacks pure functional programming has to offer for ABS, we want to return to the Introduction again, where we mentioned the work of Gintis \cite{gintis_emergence_2006}. To repeat, in this paper Gintis has claimed to have found a mechanism in bilateral decentralized exchange, which resulted in Walrasian General Equilibrium without the neo-classical approach of a tatonement process through a central auctioneer. Due to its high-impact result for economics, researchers \cite{ionescu_dependently-typed_2012} tried to reproduce the results independently but were not able to do so. After Gintis provided the code, it was found that there was a bug in his implementation which led to the unexpected results, which were seriously damaged through this error. The work of \cite{evensen_extensible_2010} investigated the specific nature of Gintis bugs and reported the following (Section \textit{3.1.1  Deviations from the paper}, page 23):

\begin{enumerate}
	\item An agent calculates the optimum inventory (or demand) according to a specific formula, which involves a factor $\lambda$ which gets computed as $\lambda = \frac{\sum_{j}{} p_{ij}x_{ij}}{\sum_{j}{} p_{ij}x_{ij}}$. What the specific computation denotes is not important here, what is important is that this computation is always 1. The authors \cite{evensen_extensible_2010} claim that the correct formula should have been: $\lambda = \frac{\sum_{j}{} p_{ij}x_{ij}}{\sum_{j}{} p_{ij}o_{j}}$ with \textit{o} in the denominator instead of \textit{x}. 
	
	\item There seems to be a discrepancy between the paper and the implementation describing the trading of agents. According to the paper the amount they exchange follows $x_{ig} \equiv \frac{p_{ig}x_{ig}}{p_{ih}}$. In the implementation however, Gintis uses $x_{ig} \equiv \frac{p_{ih}x_{ih}}{p_{ig}}$ but seems to do so inconsistently throughout the his code, leading to wrong calculations.
	
	\item Another bug was due to reversed ordering of events, where agents use old prices vectors due to missing recalculation which only happens in the next step.

	\item There are a number of other subtle bugs, which, according to the authors \cite{evensen_extensible_2010} seemed to have no impact on the outcome of the simulation: slight miscalculation of the standard deviation for consumer and producer prices, crashing the program at runtime when dividing the agents unequally between different production goods due to a negative random number which in turn raises an exception in the Delphi Random function.

\end{enumerate}

After having claimed that pure functional programming helps in implementing simulations which are more likely to be correct, the following questions arise:

\paragraph{Do the techniques introduced in this thesis transfer to this problem and model as well?}
The short answer is, yes they obviously do as Gintis models interacting individual agents which consume and produce and trade with each other. This could be implemented both with our time- and event-driven approach, with a better choice probably being the event-driven one due to agents directly trading with each other. However, for some models, our techniques introduced in Part II are too powerful and a much simpler approach would suffice to implement it. In general too much power should always be avoided (at least in programming and software engineering) because with much power comes much responsibility: more power requires to pay more attention to details and thus there is increased risk to make mistakes. Thus we should always look for the technique with minimal power but maximum abstraction, which solves our problem sufficiently. Gintis model is such an example: it resides in a very different domain of ABS, called Agent-Based Computational Economics (ACE).

ACE is a very important field, which picked up ABS as a method for research in recent years. The field of economics is an immensely vast and complex one with many facets to it, ranging from firms, to financial markets to whole economies of a country \cite{bowles_understanding_2005}. Today its very foundations rest on rational expectations, optimization and the efficient market hypothesis. The idea is that the macroeconomics are explained by the micro foundations \cite{colell_microeconomic_1995} defined through behaviour of individual agents. These agents are characterized by rational expectations, optimizing behaviour, having perfect information, equilibrium \cite{focardi_is_2015}.
This approach to economics has come under heavy criticism in the last years for being not realistic, making impossible assumptions like perfect information, not being able to provide a process under which equilibrium is reached \cite{kirman_complex_2010} and failing to predict crashes like the sub-prime mortgage crisis despite all the promises - the science of economics is perceived to be detached from reality \cite{focardi_is_2015}. 
ACE is a promise to repair the empirical deficit which (neo-classic) economics seem to exhibit by allowing to make more realistic, empirical assumptions about the agents which form the micro foundations. The ACE agents are characterized by bounded rationality, local information, restricted interactions over networks and out-of-equilibrium behaviour \cite{farmer_economy_2009}. 
Works which investigate ACE as a discipline and discuss its methodology are \cite{ballot_agent-based_2015,blume_introduction_2015,tesfatsion_agent-based_2002,richiardi_agent-based_2007}.
Tesfatsion \cite{tesfatsion_agent-based_2017} defines ACE as \textit{[...] computational modelling of economic processes (including whole economies) as open-ended dynamic systems of interacting agents.}. 

It is important to understand, that ACE utilises ABS different than the social sciences do. The latter one focuses more on agent-interactions, where in ACE the rational and non-rational actions of individual agents are more important. Thus in many ACE models, the full power of the techniques introduced in Part II is not required. More specifically, agents of ACE models tend to have much simpler state, behave often in only one specific way, don't use synchronised agent-interactions and are very rarely located in a spatial environment but focus more on network connections \cite{glasserman_contagion_2015,wilhite_economic_2006} or avoid the notion of connectivity altogether. This is certainly the fact in the case of the Gintis model, as implemented by Gintis himself and the Java implementation of \cite{evensen_extensible_2010}: 

\begin{itemize}
	\item Even though it is an agent-based model and there is a clear notion of agents in the code, where they are represented as objects, the agents are very simple in terms of their structure. They are characterised by a few floating-point values with very simple behaviour that does not change over time.

	\item There are only very simple direct agent-interactions, exchanging bids and asks, leading to an exchange.

	\item There is no environment whatsoever and a fully connected network is implicitly assumed because each agent can trade with all other agents.
	
	\item The only side effect necessary in this simulation is to draw random-numbers.
\end{itemize}

Thus a Haskell implementation in this domain can be achieved completely without the full force of our techniques: agents can be represented as simple data structures without SFs / MSFs, interactions handled by the kernel, no need for an environment, scheduling is handled directly by the kernel and side effects are restricted to run in the \textit{Rand} Monad only. Due to the substantial complexity of the Gintis model, we have refrained from attempting a full implementation of it in Haskell. However, to investigate this point more in-depth we implemented a simulation \footnote{Available at \url{https://github.com/thalerjonathan/zerointelligence}} with so called Zero Intelligence traders \cite{gode_allocative_1993}, a well known and fundamental concept found in ACE. Our implementation was inspired by an implementation in Python \footnote{\url{http://people.brandeis.edu/~blebaron/classes/agentfin/GodeSunder.html}} and is a satisfactory proof-of-concept underlining the fact that we can do pure functional ABS also in a robust way without the tremendous power offered by our techniques. However, a full treatment of this is beyond the scope of this thesis and is left for further research.

\paragraph{Would the use of Haskell have prevented the bugs which Gintis made?}
The first and second bugs as reported above are an indication for a problematic use of indexed lists or arrays. Generally, one can say that independent of programming languages, indices into an array or list are \textit{always} problematic because it is very easy to make very subtle mistakes by getting indices wrong. Pure functional programming would suffer the same problem \textit{if} a similar technique would have been used. However, due to the fact that data in Haskell is immutable, an \textit{idiomatic} implementation, following pure functional programming concepts, we would have probably not seen the use of lists but (Generalised) Algebraic Data Types, making this kind of bug much less likely. Further, due to the \textit{declarative} nature of pure functional programming in Haskell, it is more likely that the implementer, reading through the existing code might have spotted the bug: there is much less noise in an idiomatic functional implementation than in imperative code, making code highly expressive and concise, thus less to read and more obvious.

The third bug is a very subtle logical error, regarding the semantics of the simulation. A pure functional alone would not have helped avoiding this mistake. However, we think that due to the declarative nature it would have been possible to spot, also because data is immutable and the fact that we are dealing with an old version of data is much more explicit. 

The last bugs are typical run-time errors, which would and do occur in Haskell as well, so a pure functional approach would not necessarily avoid these kind of bugs. However, such bugs can be avoided by using a dependently typed functional language like Idris \cite{brady_idris_2013} as such a type system allows to ensure in the types that only positive random numbers are drawn, and that the input ranges are strictly positive.

Thus summarizing, we hypothesise that with a clean and \textit{idiomatic} pure functional implementation it would have been very likely that Gintis would have avoided or spot the bugs. Further, we claim that dependent types might have been of substantial benefit in the Gintis case, but we leave this for further research as it is beyond the scope of this thesis (see Appendix \ref{app:equilibrium_totality}).

\paragraph{Would property-based tests have been of any help to prevent the bugs?}
We hypothesise that it is very likely that if Gintis would have applied rigorous unit and property-based testing to his model - which he should have, due to the high impact of his outcome - he would have found the inconsistencies and could have corrected them. The code of \cite{evensen_extensible_2010} contains numerous \textit{checkInvariants} and assertions, which \textit{are} properties expressed in code, thus immediately applicable to property-based testing. Further, due to the mathematical nature of the problem, many properties in the form of formulas can be found in the paper specification, which should be directly expressible using property-based and unit testing.

\chapter{The structure of ABS computation}
\label{ch:structure_abs_computation}

TODO UNFINISHED MANDATORY CHAPTER

The purpose of abstraction is not to be vague, but to create a new semantic level in which one can be absolutely precise. - Dijkstra, EWD340

generalising the structure of agent computation - with our case studies we explore them in a more practical / applied way and in this chapter we extract and distil the general concepts and abstractions behind agent computation: how can ABS, which is pure computation, can be seen structurally? This gives the ABS field for the first time a deeper understanding of the deeper structure of the computations behind agent-based simulation, which has so far always been more ad-hoc without a proper, more rigorous formulation. 

pure functional computation with effects can be seen as computations over some data-structure where the data-structure defines the structure of the computation as well e.g. monoids, applicatives, monads, traversable, foldable

Note that agent-based simulation is almost always entirely pure computation without the need for direct, synchronous user-interaction or impure IO. When IO is really needed we can keep purity by creating IO actions and pass them to the simulation kernel which executes them and communicates the result back if needed - in this case only the simulation kernel needs to run in IO monad but not the agents and the environment computations.

agentout as monoid with writer: solves the Problem of iteratively constructing it the output during an event.

BUT: isnt our approach similar to the early days IO of Haskell with continuations? if this is the case we should be able to get the direct method style by writing an agent monad?

NOTE: "And a closure is just a primitive form of object: the special case of an object with just one method." https://www.tedinski.com/2018/11/20/message-oriented-programming.html

- this is still research which needs to be done by reading the papers below and reflecting  and understanding on co-monads and my implementations in general.

- can we derive an agent-monad?

https://www.javiercasas.com/articles/codata-in-action/

- what about comonads? read essence dataflow paper \cite{uustalu_essence_2006}: monads not capable of stream-based programming and arrows too general therefor comonads, we are using msfs for abs therefore streambased so maybe applicable to our approach/agents=comonads. comonads structure notions of context-dependent computation or streams, which ABS can be seen as of. this paper says that monads are not capable of doing stream functions, maybe this is the reason why i fail in my attempt of defining an ABS in idris because i always tried to implement a monad family. stopped at comonad section, continue from there. understand comonads: https://www.schoolofhaskell.com/user/edwardk/cellular-automata and https://kukuruku.co/post/cellular-automata-using-comonads/ and https://chshersh.github.io/posts/2019-03-25-comonadic-builders

- Conal Elliott has examined a comonadic formulation of functional reactive programming http://conal.net/blog/posts/functional-interactive-behavior

- comonads https://fmapfixreturn.wordpress.com/2008/07/09/comonads-in-everyday-life/

- comonads are objects very important and closely related http://www.haskellforall.com/2013/02/you-could-have-invented-comonads.html

- if conal elliott can make a comonadic formulatin of FRP and comonads are objects, then i guess i am very close to a pure functional representation of objects? pure functional objects?


independent of time-driven or event-driven, our agents are MSFs.

in fact i am deriving pure functional objects

- i have the feeling that co-algebras might be an underlying structure, which in CS come up in infinite streams - ABS can be seen as this where the agents are such streams with their output and potentially running for an infinite time, depending on the model. Ionescus thesis might reveal more information / might be an additional source on that.

In general it is easy to see why agents can not be represented by pure functions: they change over time. This is precisely what pure functions cannot do: they can't rely on some surrounding context / or on history - everything what they do is determined by their input arguments and their output. In general we have two ways of approaching this: we either have the agents changing data and behaviour internalised as we did in the previous chapters or we externalise it e.g. in the simulation kernel and provide all necessary information through arguments which was the case in the sugarscape environment.

- FREE MONADS % http://www.haskellforall.com/2012/07/purify-code-using-free-monads.html http://comonad.com/reader/2011/free-monads-for-less/, https://stackoverflow.com/questions/13352205/what-are-free-monads

\section{A Functional View}
Due to the fundamentally different approaches of FP, an ABS needs to be implemented fundamentally differently, compared to established OOP approaches. We face the following challenges:

\begin{enumerate}
	\item How can we represent an Agent, its local state and its interface?
	\item How can we implement direct agent-to-agent interactions?
	\item How can we implement an environment and agent-to-environment interactions? 
\end{enumerate}

\subsection{Agent representation}
The fundamental building blocks to solve these problems are \textit{recursion} and \textit{continuations}. In recursion a function is defined in terms of itself: in the process of computing the output it \textit{might} call itself with changed input data. Continuations are functions which allow to encapsulate the execution state of a program by capturing local variables (known as closure) and pick up computation from that point later on by returning a new function. As an illustratory example, we implement a continuation in Haskell which sums up integers and stores the sum locally as well as returning it as return value for the current step:

\begin{HaskellCode}
-- define the type of the continuation: it takes an arbitrary type a 
-- and returns a type a with a new continuation
newtype Cont a = Cont (a -> (a, Cont a))

-- an instance of a continuation with type a fixed to Int
-- takes an initial value x and sums up the values passed to it
-- note that it returns adder with the new sum recursively as 
-- the new continuation
adder :: Int -> Cont Int
adder x = Cont (\x' -> (x + x', adder (x + x')))

-- this function runs the given continuation for a given number of steps
-- and always passes 1 as input and prints the continuations output
runCont :: Int -> Cont Int -> IO ()
runCont 0 _ = return () -- finished
runCont n (Cont cont) = do -- pattern match to extract the function
  -- run the continuation with 1 as input, cont' is the new continuation
  let (x, cont') = cont 1
  print x
  -- recursive call, run next step
  runCont (n-1) cont'

-- main entry point of a Haskell program
-- run the continuation adder with initial value of 0 for 100 steps 
main :: IO ()
main = runCont 100 (adder 0)
\end{HaskellCode}

We implement an agent as a continuation: this lets us encapsulate arbitrary complex agent-state which is only visible and accessible from within the continuation - the agent has exclusive access to it. Further, with a continuation it becomes possible to switch behaviour dynamically e.g. switching from one mode of behaviour to another like in a state-machine, simply by returning new functions which encapsulate the new behaviour. If no change in behaviour should occur, the continuation simply recursively returns itself with the new state captured as seen in the example above.

The fact that we design an agent as a function, raises the question of the interface of it: what are the inputs and the output? Note that the type of the function has to stay the same (type \textit{a} in the example above) although we might switch into different continuations - our interface needs to capture all possible cases of behaviour. The way we define the interface is strongly determined by the direct agent-agent interaction. In case of Sugarscape, agents need to be able to conduct two types of direct agent-agent interaction: 1. one-directional, where agent A sends a message to agent B without requiring agent B to synchronously reply to that message e.g. repaying a loan or inheriting money to children; 2. bi-directional, where two agents negotiate over multiple steps e.g. accepting a trade, mating or lending. Thus it seems reasonable to define as input type an enumeration (algebraic data-type in Haskell, see example below) which defines all possible incoming messages the agent can handle. The agents continuation is then called every time the agent receives a message and can process it, update its local state and might change its behaviour.

As output we define a data-structure which allows the agent to communicate to the simulation kernel 1. whether it wants to be removed from the system, 2. a list of new agents it wants to spawn, 3. a list of messages the agent wants to send to other agents. Further because the agents data is completely local, it also returns a data-structure which holds all \textit{observable} information the agent wants to share with the outside world. Together with the continuation this guarantees that the agent is in full control over its local state, no one can mutate or access from outside. This also implies that information can only get out of the agent by actually running its continuation. It also means that the output type of the function has to cover all possible input cases - it cannot change or depend on the input. 

\begin{HaskellCode}
type AgentId    = Int
data Message    = Tick Int | MatingRequest AgentGender ... 
data AgentState = AgentState { agentAge :: Int, ... }             
data Observable = Observable { agentAgeObs :: Int, ... } 
data AgentOut   = AgentOut
  { kill       :: Bool
  , observable :: Observable
  , messages   :: [(AgentId, Message)] -- list of messages with receiver
  }
-- agent continuation has different types for input and output
newtype AgentCont inp out = AgentCont (in -> (out, AgentCont inp out))
-- taking the initial AgentState as input and returns the continuation
sugarscapeAgent :: AgentState -> AgentCont (AgentId, Message) AgentOut
sugarscapeAgent asInit = AgentCont (\ (sender, msg) -> 
  case msg of
    agentCont (sender, Tick t) = ... handle tick
    agentCont (sender, MatingRequest otherGender) = ... handle mating request)
\end{HaskellCode}

\subsection{Stepping the simulation}
The simulation kernel keeps track of the existing agents and the message-queue and processes the queue one element at a time. The new messages of an agent are inserted \textit{at the front} of the queue, ensuring that synchronous bi-directional messages are possible without violating resources constraints. The Sugarscape model specifies that in each tick all agents run in random order, thus to start the agent-behaviour in a new time-step, the core inserts a \textit{Tick} message to each agent in random order which then results in them being executed and emitting new messages. The current time-step has finished when all messages in the queue have been processed. See algorithm \ref{alg:stepping_simulation} for the pseudo-code for the simulation stepping.
%
%\begin{algorithm}
%\SetKwInOut{Input}{input}\SetKwInOut{Output}{output}
%\Input{All agents \textit{as}}
%\Input{List of agent observables}
%shuffle all agents as\;
%messageQueue = schedule Tick to all agents\;
%agentObservables = empty List\;
%\While{messageQueue not empty} {
%  msg = pop message from messageQueue\;
%  a = lookup receiving agent in as\;
%  (out, a') = runAgent a msg\;
%  update agent with continuation a' in as\;
%  add agent observable from out to agentObservables\;
%  add messages of agent at front of messageQueue\;
%}
%return agentObservables\;
%\caption{Stepping the simulation.}
%\end{algorithm}
%\label{alg:stepping_simulation}

\subsection{Environment and agent-environment interaction}
The agents in the Sugarscape are located in a discrete 2d environment where they move around and harvest resources, which means the need to read and write data of environment. This is conveniently implemented by adding a State side-effect type to the agent continuation function. Further we also add a Random effect type because dynamics in most ABS in general and Sugarscapes in particular are driven by random number streams, so our agent needs to have access to one as well. All of this low level continuation plumbing exists already as a high quality library called Dunai, based on research on Functional Reactive Programming  \cite{hudak_arrows_2003} and Monadic Stream Functions \cite{perez_functional_2016,perez_extensible_2017}.

\section{Maturity}
%Thus it is not that implementing ABS with OO is wrong - it works reasonably well as a large number of industry strength libraries and frameworks demonstrate. It is more the \textit{missed potential} of a (pure) functional, data-centric approach: strong static type system with explicit controlled side-effects; parallel computation to speed up the simulation with very few changes but retaining static guarantees at compile time; STM to implement concurrent data-flow problems as actual data-flow problems without the need to resort to synchronisation primitives and cluttering the program logic with semantics for synchronisation and concurrency; Property-based testing for verification and validation of a data-centric approach which is central to all simulations; actor model concurrency in the case of Cloud Haskell and Erlang for agent-interaction centric models with a functional, data-centric core. 

We are very well aware that our approach has yet to reach maturity and prove itself over time in real simulation studies, especially as it is missing a mature and stable library yet. The fact that there exist a number of industrial strength toolkits and libraries (Repast, NetLogo, AnyLogic) and the widespread use and knowledge of object-oriented programming, makes object-oriented programming still the highly compelling approach to implement ABS. This allows for a quick and cheap implementation of low-impact and straightforward models where the need for correctness, reproducibility, verification and validation is not of primary concern. As outlined, performance in functional programming is still nowhere near object-oriented programming although that argument might get diminished by further research and the potential of using actor based concurrency like in Erlang to implement ABS. %Another benefit is that object-oriented programming as a modelling tool to a problem is still highly useful in the case of UML. %TODO:  discusses if and how peers object-oriented agent-based modelling framework can be applied to our pure functional approach. TODO: i need to re-read peers framework specifications / paper from the simulation bible book. Although peers framework uses UML and OO techniques to create an agent-based model, we realised from a short case-study with him that most of the framework can be directly applied to our pure functional approach as well, which is not a huge surprise, after all the framework is more a modelling guide than an implementation one. E.g. a class diagram identifies the main datastructures, their operations and relations, which can be expressed equally in our approach - though not that directly as in an oo language but at least the class diagram gives already a good outline and understanding of the required datafields and operations of the respective entities (e.g. agents, environment, actors,...). A state diagram expresses internal states of e.g. an agent, which we discussed how to do in both our time- and even-driven approach. A sequence diagram e.g. expresses the (synchronous) interactions between agents or with their environment, something for which we developed techniques in our event-driven approach and we discuss in depth there. 