\chapter{Agents}

TODO: general agents: what they are and what they are not


\section{Agent Models}
Agent Models are NOT specific to any programming language implementation but should in theory be implementable in all languages which support the required primitives of the model or which allows the primitives of the model to be mapped to primitives of the language. Of course this says nothing about how well a language is suited to implement a given agent model and how readable and natural the mapping and implementation is. \\

Start from wooldridge 2.6 (look at the original papers which inspired the 2.6 chapter) and weiss book and the original papers those chapter is based upon. Look into denotational semantics of actor model. Look also in functional models of czesar ionescu. why: this is the major contribution of my thesis and is new knowledge. Must find intuitive, original and creative approach. \\
From functional agent models (e.g. Wooldridge) to implementation of ACE in Haskell (this should go into research-proposal introduction)

\subsection{Actors}

\subsection{TODO: is FRP an agent-model?}
It is specific to (pure) functional programming languages and can only be implemented with very cumbersome overhead in object-oriented languages like Java.
So maybe we can do that in Scala and LISP as well?

\section{Implementing Agents}
In the end it all boils down to 1. the agent-model and 2. how the agent-model is implemented in the according language. Of course both points influence each other: functional languages will come up with a different agent-model (e.g. hybrid like yampa) than object-oriented ones (e.g. actors).

The only constants are: see actors