%*******************************************************************************
%*********************************** Agent-based computational Economics *****************************
%*******************************************************************************

\chapter{ACE} 
read ACE introduction papers, summarize in this research-proposal

look into computable economics book: \url{http://www.e-elgar.com/shop/computable-economics}

TODO: the reading should pull out the essence of what types of ACE there are and what features each type has (continuous/discrete time, complex agent communication, equilibriua, networks amongst agents,...)

NOTE: I REALLY need to work out what is special in ACE? what is the unique property of ACE AS compared to other ABM/S? Conjecture: equilibrium of dynamics is the central aspect.
\url{http://www2.econ.iastate.edu/tesfatsi/ace.htm}

\cite{mandel_2015} Agent-based modeling and economic theory: where do we stand? - Ballot, Mandel, Vignes \\
\cite{richiardi_2007} Agent-based Computational Economics. A Short Introduction - Richiardi \\
\cite{tesfatsion_2006} Agent-based computational economics: a constructive approach to economic theory - tesfatsion \\
\cite{kleinberg_easley_2015} Introduction to computer science and economic theory - blume, easley, kleinberg \\
\cite{tesfatsion_2002} agent-based computational economics - tesfatsion 




“[...] computational modeling of economic processes (including whole economies) as open-ended dynamic systems of interacting agents.” Leigh Tesfatsion

The book \cite{KirmanComplex2010} is a critique of classic economics with the triple of rational agents, "average" inidividuum, equilibrium theory. Although it does not mention ACE it  can be seen as an important introduction to the approach of ACE as it introduces many important concepts and views dominant in ACE. Also ACE can be seen as an approach of tackling the problems introduced in this book: \\

page 6: "the view of economy is much closer to that of social insects than to the traditional view of how economies function." \\
page 7: "... main argument that it is the interaction between individuals that is at the heart of the explanation of many macroeconomic phenomena..." \\
page 15: "problem of equilibrium is information" \\
page 21: "the theme of this book will be that the very fact individuals interact with each other causes aggregate behaviour to be different from that of individuals" \\
