%%%%%%%%%%%%%%%%%%%%%%%%%%%%%%%%%%%%%%%%%
% University/School Laboratory Report
% LaTeX Template
% Version 3.1 (25/3/14)
%
% This template has been downloaded from:
% http://www.LaTeXTemplates.com
%
% Original author:
% Linux and Unix Users Group at Virginia Tech Wiki 
% (https://vtluug.org/wiki/Example_LaTeX_chem_lab_report)
%
% License:
% CC BY-NC-SA 3.0 (http://creativecommons.org/licenses/by-nc-sa/3.0/)
%
%%%%%%%%%%%%%%%%%%%%%%%%%%%%%%%%%%%%%%%%%

%----------------------------------------------------------------------------------------
%	PACKAGES AND DOCUMENT CONFIGURATIONS
%----------------------------------------------------------------------------------------

\documentclass{article}

\usepackage[utf8]{inputenc}
\usepackage{graphicx} % Required for the inclusion of images
\usepackage{natbib} % Required to change bibliography style to APA
\usepackage{amsmath} % Required for some math elements 

\setlength\parindent{0pt} % Removes all indentation from paragraphs

%\usepackage{times} % Uncomment to use the Times New Roman font

%----------------------------------------------------------------------------------------
%	DOCUMENT INFORMATION
%----------------------------------------------------------------------------------------

\title{The genesis according to computer science:\\Reality as an agent-based simulation of free will} % Title

\author{Jonathan \textsc{Thaler}} % Author name

\date{\today} % Date for the report

\begin{document}

\maketitle % Insert the title, author and date

% If you wish to include an abstract, uncomment the lines below
\begin{abstract}
TODO: maybe remove "agent-based" and talk more generally about a simulation of free will

All sciences have their genesis-model which are basically explanations of how the world did come into existence, what the reason for existence is and who or what God is. Unfortunately computer science has none so far, so the aim of this paper is to set out to develop such genesis-model from a computer-science perspective. The model is motivated from the perspective that the world and humankind is a simulation to see free will in action in a sand-boxed environment as opposed to the afterlife / afterworld or just the Beyond, which is an outer level of simulation and itself again a simulation as will be shown in subsequent sections. \\
This paper addresses important fundamental questions of belief and religion and tries to explain them using this model - which it does surprisingly well. Thus this paper has a keen aim: it wants to amalgamate religious concepts with concepts of theoretical computer science. It is an attempt to think out-of-the-box, having fun looking at dogmatic things from a total different perspective, breaking down conventional view on old religious things and does not take itself too serious - after all the least thing we need is a new dogma or idealism, the world is full of it.

NOTE: not falsifyable, thus no scientific theory but its a framework within theories/statements which are falsifiable can be formulated

\bigskip

"In the beginning there was nothing, which exploded" - Terry Pratchett
\end{abstract}

\section{Introduction}

The following questions will be addressed and explained in this new context

\begin{itemize}
\item Who or what is God?
\item Who or what is Christ, Buddah, Vishnu, Mohammed,...?
\item What is free will? 
\item What is meditation?
\item What is conciousness?
\item What is life after death?
\item Where do we come from?
\item What is the omega point?
\item What is spiritual enlightenment?
\end{itemize}

Approaching the subject from a more technical view-point:

\begin{itemize}
\item Who or what implemented the simulation?
\item What is outside this simulation?
\item What is free will in this context? Can it be defined formally?
\item On which hardware does this simulation run? Where does the energy come from?
\item What is the computational complexity of this simulation?
\item What are the memory-requirements of this simulation?
\end{itemize}

karma: cause and effect. is a controlling factor in the simulation to prevebt the dynamics to get out of hand

The nothing was something: it was indeed nothingness, a

free will on a machine is a contradiction. the machine works according to very strict rules. free will can be completely unpredictable. or is free will just an imagination? if one confronts a decision maker within short time with too much information then the outcome it is unpredictable 

parallels to matrix

TODO: lucifer, satan, out-of-the-creation, left-handed vs. right-handed path

pro-activity possible through conciousness: the brain produces thoughts and the conciousness can observe these and decide to follow them or not. This is observable on oneself during meditation!

Free will: deliberately ignore thoughts

\end{document}