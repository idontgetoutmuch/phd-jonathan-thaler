\section{Introduction}
Agent-Based Modelling and Simulation (ABM/S) is still a young discipline and the dominant approach to it is object-oriented computation. This thesis goes into the opposite direction and asks how ABM/S can be mapped to and implemented using pure functional computation and what one gains from doing so. To the best knowledge of the author, so far no proper treatment of ABM/S in this field exists but a few papers which only scratch the surface. The author argues that approaching ABM/S from a pure functional direction offers a wealth of new powerful tools and methods. The most obvious one is that when using pure functional computation reasoning about the correctness and about total and partial correctness of the simulation becomes possible. Also pure functional approaches allow the design of an embedded domain specific language (EDSL) in which then the models can be formulated by domain-experts. The strongest point in using EDSL is that ideally the distinction between specification and implementation disappears: the model specification is then already the code of the simulation-program. This allows to rule out a serious class of errors where specification and implementation does not match, which is especially a big problem in scientific computing. The application will be in the field of agent-based computational economics (ACE) where the primary goal will be to implement a pure functional framework with a suitable EDSL in Haskell. This will give computational economists a tool to formulate their models in this EDSL and run them directly in Haskell.

