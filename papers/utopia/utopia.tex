%%%%%%%%%%%%%%%%%%%%%%%%%%%%%%%%%%%%%%%%%
% University/School Laboratory Report
% LaTeX Template
% Version 3.1 (25/3/14)
%
% This template has been downloaded from:
% http://www.LaTeXTemplates.com
%
% Original author:
% Linux and Unix Users Group at Virginia Tech Wiki 
% (https://vtluug.org/wiki/Example_LaTeX_chem_lab_report)
%
% License:
% CC BY-NC-SA 3.0 (http://creativecommons.org/licenses/by-nc-sa/3.0/)
%
%%%%%%%%%%%%%%%%%%%%%%%%%%%%%%%%%%%%%%%%%

%----------------------------------------------------------------------------------------
%	PACKAGES AND DOCUMENT CONFIGURATIONS
%----------------------------------------------------------------------------------------

\documentclass{article}

\usepackage[utf8]{inputenc}
\usepackage{graphicx} % Required for the inclusion of images
\usepackage{natbib} % Required to change bibliography style to APA
\usepackage{amsmath} % Required for some math elements 

\setlength\parindent{0pt} % Removes all indentation from paragraphs

%\usepackage{times} % Uncomment to use the Times New Roman font

%----------------------------------------------------------------------------------------
%	DOCUMENT INFORMATION
%----------------------------------------------------------------------------------------

\title{Utopia: an alternative to the current structure of society, economy and politics} % Title

\author{Jonathan \textsc{Thaler}} % Author name

\date{\today} % Date for the report

\begin{document}

\maketitle % Insert the title, author and date

% If you wish to include an abstract, uncomment the lines below
\begin{abstract}

\end{abstract}

\section{Introduction}


\section{The current system}

current system supports debt which introduces structural inequalities and power of some (minorities) over others (majorities), inflation? it always has to grow, which is completely inrealistic as no system can continuously grow as it will converge to an infinite consumption of energy: time and space are in the end limited. exploiting of impowered majorities by minorities through: debt, labor and its conditions, 

it reinforces and supports structural inequalities by exploiting people and manipulating people to not cooperate and work against each othet

also the economic research tries (still) to keep up the status quo and is holding on to total unrealistic assumptions which helps manifesting the current economic system by giving it "firm scientific foundations" - science serves here an ideological purpose: justification of an unjust, repressive system.

power is a major problem as it corrupts people. all. no exception.

the political system and political class is so far away from the people they should serve. they only serve their own purpose and try to keep up their power - thats juat natural for any system to keep itself alive

\section{A Solution}
dissolve states

cooperative small-scale economies without monetary system organized in small communities. no minority ruling class, no minority controlling surplus, no debt, no individual richness, no individual property

but: communities consists of people. there will be struggle for more power, more influence over others,... how can this be adressed? keep it small enough, no rulers, everyone equal rights, no individual property. different people will contribute different to the community: this WILL result in jealousy as one thinks he has contributed more than another. this needs to be considered VERY careful

what can we do against manipulation? people are easily manipulateable, especially by fear (the small death). when groups and communities are manipulated the dynamics can become terrible with extreme irrationalities

thus we must constitite a faith in the greater good of all => creating an ideology => will lead to MASSIVE destruction and death

is there a way out?

the question is if the non-monetary, cooperative, democrative communities economy works on a global scale or if the whole system would break down if it has reached some critical point. it would be interesting to pursue this question in a simulation but for this a model needs to be developed and research-questions need to be formulated.

\section{The Human Factor}
In the end humans will be the problem with their fear they are susceptible to manipulation and thus the whole thing can be brought down again. What can take fear away? Belief, strong leaders, security,...

%----------------------------------------------------------------------------------------
%	BIBLIOGRAPHY
%----------------------------------------------------------------------------------------

\bibliographystyle{apalike}

\bibliography{./utopia.bib}

\end{document}