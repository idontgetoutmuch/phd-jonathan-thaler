%%%%%%%%%%%%%%%%%%%%%%%%%%%%%%%%%%%%%%%%%
% University/School Laboratory Report
% LaTeX Template
% Version 3.1 (25/3/14)
%
% This template has been downloaded from:
% http://www.LaTeXTemplates.com
%
% Original author:
% Linux and Unix Users Group at Virginia Tech Wiki 
% (https://vtluug.org/wiki/Example_LaTeX_chem_lab_report)
%
% License:
% CC BY-NC-SA 3.0 (http://creativecommons.org/licenses/by-nc-sa/3.0/)
%
%%%%%%%%%%%%%%%%%%%%%%%%%%%%%%%%%%%%%%%%%

%----------------------------------------------------------------------------------------
%	PACKAGES AND DOCUMENT CONFIGURATIONS
%----------------------------------------------------------------------------------------

\documentclass{article}

\usepackage{graphicx} % Required for the inclusion of images
\usepackage{natbib} % Required to change bibliography style to APA
\usepackage{amsmath} % Required for some math elements 
\usepackage{hyperref}

\setlength\parindent{0pt} % Removes all indentation from paragraphs

%\usepackage{times} % Uncomment to use the Times New Roman font

%----------------------------------------------------------------------------------------
%	DOCUMENT INFORMATION
%----------------------------------------------------------------------------------------

\title{ {\normalsize Research Proposal} \\ Functional methods in agent-based modelling \& simulation.} % Title

\author{Jonathan \textsc{Thaler}} % Author name

\date{\today} % Date for the report

\begin{document}

\maketitle % Insert the title, author and date

\begin{center}
\begin{tabular}{l r}
Institution: University of Nottingham \\
Department: School of Computer Science \\
Project: PhD Programme 2016-2019 \\
Supervisor: Dr. Peer-Olaf Siebers \\
Co-Supersvisor: Dr. Thorsten Altenkirch 
\end{tabular}
\end{center}

% If you wish to include an abstract, uncomment the lines below
\begin{abstract}
TODO: Consider a provocative abstract and an interesting title

Agent-Based Modelling and Simulation (ABM/S) is still a young discipline and the dominant approach to it is using object-oriented methods. This thesis goes into the opposite direction and asks how ABM/S can be mapped to and implemented using functional methods and what one gains from doing so. To the best knowledge of the author, so far no proper treatment of ABM/S in this field exists but a few papers which only scratch the surface. The author argues that approaching ABM/S using functional methods offers a wealth of new powerful tools and methods. The most obvious one is that when using pure functional computation reasoning about the correctness and about total and partial correctness of the simulation becomes possible. Also pure functional approaches allow the design of an embedded domain specific language (EDSL) in which then the models can be formulated by domain-experts. The strongest point in using EDSL is that ideally the distinction between specification and implementation disappears: the model specification is then already the code of the simulation-program. This allows to rule out a serious class of errors where specification and implementation does not match, which is especially a big problem in scientific computing thus making functional methods in ABM/S especially suitable for scientific computing. The application will be in the field of agent-based computational economics (ACE) where the primary goal will be to compare functional and non-functional methods for developing ACE simulations and to identify in which scenarios pure functional methods shine and where their limits are.
\end{abstract}

%----------------------------------------------------------------------------------------
%	SECTION 1
%----------------------------------------------------------------------------------------

TODO: Contributions should appear in the conclusion, introduction and abstract


\section{Introduction}
WHY FUNCTIONAL? "because its the ultimate approach to scientific computing": fewer bugs due to mutable state (why? is thos shown obkectively by someone?), shorter (again as above, productivity), more expressive and closer to math, EDSL, EDSL=model=simulation, better parallelising due to referental transparency, reasoning

scientific results need to be reproduced, especially when they have high impact. a more formal approach of specifying the model and the simulation (model=simulation) could lead to easier sharing and easier reporduction without ambigouites

pure functional agent-model \& theory, EDSL framework in Haskell for ACE

\begin{enumerate}
\item Which kind of problem do we have?
\item What aim is there? Solving the problem? 
\item How the aim is achieved by enumerating VERY CLEAR objectives.
\item What the impact one expects (hypothesis) and what it is (after results).
\end{enumerate}

Note: It is not in the interest of the researcher to develop new economic theories but to research the use of functional methods (programming and specification) in agent-based computational economics (ACE).

NOTE: Get the reader’s attention early in the introduction: motivation, significance, originality and novelty.

\subsection{Methods}
Methods need to be selected to implement the simulations. Special emphasis will be put on functional ones which will then be compared to established methods in the field of ABM/S and ACE. 

\subsection{Scenarios}
To apply and test functional methods in ACE, four scenarios of ACE are selected and then the methods applied and compared with each other to see how each of them perform in comparison. The 4 selected scenarios represent a selection of the challenges posed in ACE: from very abstract ones to very operational ones.

\subsection{Comparison}
Each of the selected scenarios is then implemented using the selected methods where each solution is then compared against the following criteria: 

\begin{enumerate}
\item suitability for scientific computation
\item robustness
\item error-sources
\item testability
\item stability
\item extendability
\item size of code
\item maintainability
\item time taken for development
\item verification \& correctness
\item replications \& parallelism
\item EDSL
\end{enumerate}

This will then allow to compare the different methods against each other and to show under which circumstances functional methods shine and when they should not be used.

%----------------------------------------------------------------------------------------
%	SECTION 2
%----------------------------------------------------------------------------------------

%----------------
%	2 a
%----------------
\section{Literature Research: Agent-based Economics (ACE)}

\subsection{Foundations}
TODO: the reading should pull out the essence of what types of ACE there are and what features each type has (continuous/discrete time, complex agent communication, equilibriua, networks amongst agents,...)

NOTE: I REALLY need to work out what is special in ACE? what is the unique property of ACE AS compared to other ABM/S? Conjecture: equilibrium of dynamics is the central aspect.


\url{http://www2.econ.iastate.edu/tesfatsi/ace.htm}

\cite{mandel_2015} Agent-based modeling and economic theory: where do we stand? - Ballot, Mandel, Vignes \\
\cite{richiardi_2007} Agent-based Computational Economics. A Short Introduction - Richiardi \\
\cite{tesfatsion_2006} Agent-based computational economics: a constructive approach to economic theory - tesfatsion \\
\cite{kleinberg_easley_2015} Introduction to computer science and economic theory - blume, easley, kleinberg \\
\cite{tesfatsion_2002} agent-based computational economics - tesfatsion 

TODO: need to find the major fields of ACE and then select them

\subsubsection{Equilibrium \& Out-Of-Equilibrium Models}
\cite{Botta20114025} \\
\cite{Gintis2006} \\
\cite{gode_sunder_1991} Zero-Intelligence Models - Gode And Sunder \\

\subsubsection{Public Goods Game}
TODO: look for ACE in Public Goods Game

Continuous time and communication in a public-goods experiment - oprea, charness, friedman \cite{friedman_2012}: showing how continuous time can be applied to public-goods game. using continuous and discrete time and free-form communication between persons where the actions which influenced the outcome were very limited and one-way: just sending a message how much they contribute.

\subsubsection{Artifical Economies}
\cite{gintis_dynamics_2007} \\
\cite{emergent_gaffeo_gatti_2008} \\
\cite{adaptive_gaffeo_gatti_2008} \\

\subsubsection{Networks}
TODO: look for ACE in networks
\cite{glasserman_2015} Contagion in Financial Networks - Paul Glasserman and H. Peyton Young\\
\url{http://www2.econ.iastate.edu/tesfatsi/anetwork.htm}
\url{http://www2.econ.iastate.edu/tesfatsi/tnghome.htm}

\subsubsection{Auction Theory}
TODO: look for ACE in auction theory

\subsubsection{Market Microstructure}
TODO: look for ACE in market microstructure

\cite{Budish2015} \\
\cite{aldridge_high_frequency_2009} \\


%----------------
%	2 b
%----------------
\section{Literature Research: Methods in ACE \& ABM}
TODO: the reading should look which of the given methods and how they are used in ACE/ABM. Here a pre-selection of methods was done because the thesis is interested in researching the - novel - functional approach and thus has its special focus on this approach and needs other classical/state-of-the-art approaches for comparison.

\subsection{Object-Oriented Languages}
Java \\

\subsection{Functional Languages}
\subsubsection{Haskell}
\url{https://github.com/ivendrov/frabjous2}
\cite{Schneider_2012}
\url{http://oliverschneider.ca/frabjous/}
\cite{Vendrov_2014}

\cite{Nilsson2002}
\cite{Nilsson_2014}
\cite{Nilsson_2003}
\cite{Hudak2003}
\cite{Courtney2003}

\cite{Sulzmann_2007}

\cite{bauer_adaptation_2011}

\cite{Jankovic_2007}


aivika paper

games in haskell

\subsubsection{Scala}

\subsection{Agent-Models}
\subsubsection{Actor Model}
\cite{Poggi_2015}
\cite{Parker_2011}
\cite{DiStefano_2007}
\cite{Varela_2004}
\cite{DiStefano_2005}

\cite{Hewitt_1973}
 \cite{Greif_1975}
 \cite{Clinger_1981}
 \cite{Agha_1986}
 \cite{Agha_1997}
 \cite{Hewitt_2007}
 \cite{Hewitt_2010}
 \cite{Agha_2004}
\subsection{Simulation Tools}
AnyLogic
Repast
NetLogo

%----------------------------------------------------------------------------------------
%	SECTION 3
%----------------------------------------------------------------------------------------
\section{Scenarios}
\subsection{Scenario 1: Public goods games}
This scenario should be very well suited for functional methods

\subsection{Scenario 2: Decentralized bartering \& trading}
Something in between 1 and 4 but closer to 1

\subsection{Scenario 3: Auctions}
Something in between 1 and 4 but closer to 4

\subsection{Scenario 4: Artificial economies}
This scenario should be NOT well suited for functional methods

Many parallel, concurrently interacting objects with non-continuous, complex behaviour.
explicit time needed
dynamic networks


%----------------------------------------------------------------------------------------
%	SECTION 4
%----------------------------------------------------------------------------------------

\section{Methods}
\subsection{Method 1: Haskell}
This is the main functional method this thesis wants to investigate as it is the purest functional programming language of all the methods.

\subsection{Method 2: Scala \& Actors (Akka)}
This method was selected because Scala is an object-oriented functional programming language and has a powerful library included which implements the actor-model. Because actors and agents are closely related this is an obvious method to follow.

\subsection{Method 3: AnyLogic / NetLogo / Repast}
These tools are state-of-the-art in ABM/S and ACE and are included to show how one can perform scenarios (see below) with these tools.

\subsection{Method 4: Java}
Java is the state-of-the-art programming language in ABM/S and ACE and is thus included as well as a benchmark against such a state-of-the-art.

%----------------------------------------------------------------------------------------
%	SECTION 5
%----------------------------------------------------------------------------------------

\pagebreak

\section{Goals/TODOs 1st Year}
\subsection{Practical}
\begin{itemize}
\item Implement SIRS in AnyLogic - DONE.
\item Implement Wildfire in Akka.
\item Implement discrete SIRS in Akka.
\item Implement decentralized bartering in Akka.
\item Implement SIRS using Monads.
\item Implement discrete and continuous SIRS in Yampa.
\item Implement the ACE model of interest (e.g. H.Gintis decentralized bartering) in Haskell and Akka at the end of the year with all the knowledge acquired so far.
\item Explore data-parallelism in Haskell to speed-up.
\end{itemize}

\subsection{Reading}
\begin{itemize}
\item Understanding Capitalism. \cite{bowles_understanding_2005}
\item Debt . TODO cite
\item Multiagent Systems Wooldridge. TODO cite
\item Multiagent Systems Weiss. TODO cite
\item Actor model Agha. TODO cite
\item A computable universe. TODO cite
\item The nature of computation. TODO cite
\item Economics and Computation by Parkes and Seuken \url{http://economicsandcomputation.org/}
\item Functional Compiler Design and Internals.
\end{itemize}

\subsection{Studying}
\begin{itemize}
\item Get into basics of economics and equilibrium theory. Why: because i need to understand basics to understand the models better and to talk and sell my models better to economists.
\item Get into category theory. Why: deeper understanding of Haskell type theory and computation.
\item Get into theoretical basics of agent-based simulation and get to know more types of agent-based models. Why: To know the requirements my EDSL/framework has to cover.
\item Understand theory of out-of-equilibrium / non-walrasian models: TODO (various Gintis \& Mandel Papers)
\item Understand Market Micro-structure: \cite{LehalleLaruelle2013}, \cite{baker_market_2013} Part II: Chapter 8-12.
\item Do literature research on dynamics of equilibrium: \cite{emergent_2008}
\end{itemize}

\subsection{Research on Papers (see the papers pdf)}
\begin{itemize}
\item The use of Actor-Model in ABM/S.
\item A model for pure functional agents.
\end{itemize}
%----------------------------------------------------------------------------------------
%	BIBLIOGRAPHY
%----------------------------------------------------------------------------------------

\bibliographystyle{apalike}

\bibliography{./bib/researchproposal.bib}

%----------------------------------------------------------------------------------------

\end{document}