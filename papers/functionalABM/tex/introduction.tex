\chapter{Introduction}

\section{Reasoning}
Allowing to reason about a program is one of the most interesting and powerful features of a Haskell-program. Just by looking at the types one can show that there is no randomness in the simulation \textit{after} the random initialization, which is not slightest possible in the case of a Java, Scala, ReLogo or NetLogo solution. Things we can reason about just by looking at types:

\begin{itemize}
\item Concurrency involved?
\item Randomness involved?
\item IO with the system (e.g. user-input, read/write to file(s),...) involved?
\end{itemize}

This all boils down to the question of whether there are \textit{side-effects} included in the simulation or not.

\subsection{Reasoning about termination}
What about reasoning about the termination? Is this possible in Haskell? Is it possible by types alone? My hypothesis is that the types are an important hint but are not able to give a clear hint about termination and thus we we need a closer look at the implementation. In dependently-typed programming languages like Agda this should be then possible and the program is then also a 
proof that the program itself terminates.

\subsection{Debugging}
Because functions compose easier than classes \& objects (TODO: we need hard claims here, look for literature supporting this thesis or proof it by myself) it is also much easier to debug \textit{parts} of the implementation e.g. the rendering of the agents without any changes to the system as a whole - just the main-loop has do be adopted. Then it is very easy to calculate e.g. only one iteration and to freeze the result or to manually create agents instead of randomly create initial ones.

\section{World}
The coordinates calculated by the agents are \textit{virtual} ones ranging between 0.0 and 1.0. This prevents us from knowing the rendering-resolution and polluting code which has nothing to do with rendering with these implementation-details. Also this simulation could run without rendering-output or any rendering-frontend thus sticking to virtual coordinates is also very useful regarding this (but then again: what is the use of this simulation without any visual output=

\begin{itemize}
\item Clipping: \textit{calculated} coordinates are clipped at 0.0 and 1.0 
\item Wraparound: \textit{calculated} coordinates are wrapped around to 0.0 when reaching 1.0. Will lead pursuing friends to change direction apruptly when wrapping around.
\item Remainder: \textit{calculated} coordinates are never clipped but in rendering only the remainder after the comma is taken. Of course this then always requires a visual output to observe the effects. This will result in an effect like the classic asteroid game where agents their friends but wont change direction when they seem to clamp around like in the wraparound version.
\end{itemize}

\section{Lazy Evaluation}
can run a simulation for a given number of steps but 

\section{Performance}
Java outperforms Haskell implementation easily with 100.000 Agents - at first not surprising because of in-place updates of friend and enemies and no massive copy-overhead as in haskell. But look WHERE exactly we loose / where the hotspots are in both solutions. 1000.000 seems to be too much even for the Java-implementation.

\section{Numerical Stability}
The agents in the Java-implementation collapsed after a given number of iterations into a single point as during normalization of the direction-vector the length was calculated to be 0. This could be possible if agents come close enough to each other e.g. in the border-worldtype it was highly probable after some iterations when enough agents have assembled at the borders whereas in the Wrapping-WorldType it didn't occur in any run done so far. \\
In the case of a 0-length vector a division by 0  resulting in NaN which \textit{spread} through the network of neighbourhood as every agent calculated its new position it got \textit{infected} by the NaN of a neighbour at some point. The solution was to simply return a 0-vector instead of the normalized which resulted in no movement at all for the current iteration step of the agent. 

\section{Visualization}
Render all
Render cowards only
Render heroes only


\section{Update-Strategies}
\begin{enumerate}
\item All states are copied/frozen which has the effect that all agents update their positions \textit{simultaneously}
\item Updating one agent after another utilizing aliasing (sharing of references) to allow agents updated \textit{after} agents before to see the agents updated before them. Here we have also two strategies: deterministic- and random-traversal
\end{enumerate}

\subsection{Different Behaviour with different Update-Strategies}
Experienced the emergency of patterns in border-world: heroes form a cross in the center and cowards 4 crosses in each subdividing area.

window-size:	300x300
agent-size:		2pix
heroes: 		20%
random-seed: 	42
agents:			10000


BORDER-WORLD
	consecutive, random-traversal: 					patterns emerge
	consecutive, deterministic-traversal: 			patterns emerge

	simultaneous, random-traversal:					collapse into ecliptic point-cloud
	simultaneous, deterministic-traversal: 			patterns emerge
													in both cases seems to have initial "waves"
WRAPPING-WORLD
	consecutive, random-traversal: 					different patterns: multiple rows of cowards
	consecutive, deterministic-traversal: 			same as above

	simultaneous, random-traversal:					collapse into ecliptic point-cloud
	simultaneous, deterministic-traversal: 			same patterns as in consecutive version
	
The relevant observation here is not the shape of the pattern but that in the case of simultaneous, random updates the whole model collapses into an elliptic point-cloud. This proofs that at least in this simulation the difference in update-strategies is of importance. Why is that so? Can this be generalized to other models/simulations as well? \\
It does not make a sense to me that especially in the simultaneous version this should make a difference - maybe its a bug?

In Java: clearly visible
In Scala&Actors: clearly visible
In Haskell: not as clearly but it is there, problem: performance