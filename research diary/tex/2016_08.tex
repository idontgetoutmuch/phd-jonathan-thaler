\section*{2016 August 8th}
\subsection*{The process so far}

-> Reflect on the past Months since starting working toward a PhD in Nottingham which was around November 2015 
	-> describe process, why to notts, 
	
-> Reflect on the topic search process over the last months
	-> what i learned / read about
		-> erlang, haskell, agda
		-> pi-calculus
		-> actor model

	-> the insight 
		-> erlang is a dead-end as it is not a pure functional language and reasoning is not really possible due to parallel nature and language constructs 	
		-> actor model will not be the direction I will head into because research seemed to move into different direction
			-> my agents are most probably of different nature
			-> but if not then maybe the actor model gives a framework in which to reason about parallel and concurrent objects: agents
		
	-> what i will focus on
		-> category theory 
		-> constructive type theory 
		-> pi-caluclus but I am not so sure about this anymore
		
	-> i want too much


	Preparations so far (since January 2016)
•	Learned Erlang – implemented basic version of masterthesis CDA
•	Learned Haskell (Graham Hutton Book \& Tutorials on wiki.haskell.org on Monadic programming) – also implemented basic version of masterthesis CDA
•	Learned basic Agda 
%(http://learnyouanagda.liamoc.net/toc.html, https://www.youtube.com/watch?v=shXKb2MTkUc&index=1&list=PLB7F836675DCE009C, Wikipedia of agda, Thorsten’s computer aided formal reasoning lectur)
•	Reading and studying book on category theory for the sciences
•	Reading book on Actor Model (Agha Gul’s)
•	Reading book on pi-calculus (Robin Millner’s)
•	Reading papers formal ABM/S
•	Reading papers on pi-calculus
•	Reading papers on ABS in computational economics
•	Reading papers on market designs
Discussions with my Supervisors
So far the listed tasks is a huge amount for 3 years and I doubt it is realistic to do in 3 years. Also it is yet totally unclear whether it makes sense/is possible to map ZI-Agents and the auction-types to category theory AND write specifications in pi-calculus for both. Maybe only one of them is really necessary or maybe only part of each? This is to be discussed with my supervisors.
Does it even make sense to study these auction-types with these types of agents? To look only at the dynamics? Is computational economics interested in these dynamics of the equilibrium processes? I really need to look into theoretical work on the various auction-types AND the computational economics approach to them.
Do we really need the pi-calculus when we have category-theory and vice versa? Does not just one method suffice? Or is there a mapping between both methods / a connection?

	
-> My PhD \& Philosophy: The philosophy of computation

[ ] what connects our view of computation to reality: where does computation occur in reality? is our notion of computation even applicable to reality? what is computation? what is computation in nature? does computation even occur in nature? is reality discrete or continuous?
[ ] i have the feeling that the way we regard computation works well for our very limited computer tools but has absolutely no connection to reality and that we must look at computation from a complete different approach to connect it to reality.
[ ] the way we look at reality is not suitable to touch the most fundamental questions. we look at it separated from it. we must look inward ourselves, there we will find answers. this is contrary to existing scientific approaches. thus a new way of scientific working has to be established. the problem of existing science is that it piles up endless miles of complexity which is tractable only for specialcases and leads to further distance and separation from reality.
[ ] when we have conceived a new way of science and a new way of thinking of computation which has actually a connection to reality this will bring us closer to the concepts of "reality", "the creation", "self-conciousness", "existence", "truth" and "god"
