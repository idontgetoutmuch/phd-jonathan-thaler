\section*{2016 August 8th}
Note that this section was written in retrospective and that the next entry (10th) will give a short overview of the topic as it was at that point (10th).

\subsection*{How to get a PhD study at Nottingham}
The idea to do a PhD at Nottingham came up during the visit of Nottingham for the Leverage-Cycle Project which was 21st - 23rd May 2015. I was part of a small team consisting of Thomas Breuer, Martin Summer, Hans-Joachim Vollbrecht and me and our goal was to demonstrate Simon Gächter our software for running a specific type of economic experiment with real agents. After 3 sessions we had him convinced and he agreed to join in a research-cooperation in which the goal was to get a grant to further study the equilibrium theory of Geanokoplos. The idea was that in the funding of the research-project a doctoral study position should be included which would be taylored towards me so I could come to Nottingham and do a PhD in computer-science but in the context of the Leverage-Cycle Project. \\
The problem of getting such research-project grants is that it is a quite long process and the outcome is very unsure and it is highly probable that one gets rejected. Thus when starting the preparations for a PhD study at Nottingham in October I had to search for a 'Plan B' covering the case of the rejection of the research-grant. Also I tried to make use of our connections to Nottingham so I first contacted Martin Summer who gave me contact details of Uwe Aickelin, the Head of the Computer School which has his research-interest in Agent-Based Modeling/Simulation and Data-Mining. I wrote him an e-mail about a potential supervision and he quickly replied tell me that he will leave Nottingham and move to the chinese campus of the University of Nottingham but referring me to Peer-Olaf Siebers who is also active in the ABM/S field. \\
Peer-Olaf quickly replied and we did an initial Skype Call to get to know each other. We both then agreed that we would like to work with each other, especially my interest to go in a (computational) economics direction was very well received by him as he tries to get foot into this area as well but hasn't managed to do so far. \\
Now I had to think about what I would like to do in my PhD as so far I had only a few vague "Visions" and brainstormed a few vague Ideas. The overall direction was clear: I wanted to do something in the field of ABM/S and economics but all approached from the direction of computer science as this is my very field of expertise and I would get rejected by economists because of complete different wording and approach. Prior to my application I already had the idea to head into the direction of functional programming in the field of ABM/S as the dominant method is Object-Orientation (OO) so I thought it would be a nice idea to try something very new. \\
The idea I came up with was to apply functional programming in Erlang to ABM/S to simulate "something in economics" where something was "trading in virtual economies", "High-frequency trading", "Electronic Trading Platforms"... which could be boiled down to "Market Design". It was an intense and interesting process as I had to read quite a few papers and had to "clean up" and clarify my ideas and my mind. So after about 2 weeks of intense reading, reflecting and meditating I decided to go into the direction of "applying functional programming with special regard of ERLANG to ABM/S with a case study in simulating the influence of different auction- and market types in electronic trading platforms." \\
I presented this to Peer-Olaf in a skype-call on the 4th of january. He told me that he will get a co-supervisor because he is no specialist in functional programming and more important, that I should apply for one of the 10 studentships granted by the school of computer-science. He made it clear that he will accept me as a student only if I will come to Nottingham and only if get this Studentship. \\
So the next goal was to apply for the studentship which was also a quite time-consuming process as you have to REALLY nail it - it's gotta be perfect otherwise one gets rejected really fast as you are competing with lots of other applicants. I had to provide a document containing the following:

\begin{enumerate}
\item Half a page describing reasons for wishing to pursue a PhD at Nottingham.
\item Half a page describing the proposed research area and topic (including a few references).
\item A demonstration of my technical writing skills: my Masterthesis.
\item Contact details of a supervisor at Nottingham who has already agreed to supervise my PhD.
\item Contact details of 2 Academic Referees - I did include Thomas Breuer and Simon Gächter
\item A CV including my degree classes.
\item A scan of my Bachelors and Masters degrees of all courses and the diploma supplement of both.
\end{enumerate}
      
The studentship would start on 1st October 2016 and would grant me 14,507 pounds a year AND tuition fees (which are about 9,000 pounds a year). I had time until the 29th of February but Peer-Olaf told me not to wait until the deadline but to submit maybe one month earlier to be able to react if something is rejected. 
With a little help of my Prof. Hans-Vollbrecht, Thomas Breuer and BIG help of Peer-Olaf I did write and compile all the necessary information and applied for the Studentship on the 18th January. Peer-Olaf also did find a second supervisor - Thorsten Altenkirch from the Functional Programming Group who has his interst in Computer Aided Formal Verification and Type Theory, which makes a perfect match. \\

After about 2 months on the 16th March I got news from the school of computer-science that I have been allocated on of the school PhD studentships. I nearly freaked out so happy was I but deep within me I always had the feeling that things would work out and that I'd go to Nottingham - the one way or the other. 

\subsection*{Studentship received - what now?}
Getting the studentship is one thing but organizing everything is another big thing. What had to be done was:

\begin{enumerate}
\item Officially apply with Admissions Office so a conditional offer can be made. Conditional means that a few things need to be provided in order to receive an unconditional offer which finally guarantees me that I will study there. The things to provide were an IELTS test (see below) and the referee's reports (see below).
\item Apply for the VC's Scholarship for Research Excellence (EU). The school of computer-science will guarantee a fully funded place but wants me to apply for this scholarship. If I receive it (which I did) then they just have to top up to 14.507.
\item Provide an Academic IELTS certificate with overall score not less than 6.5 and in no part less than 6.0. I easily passed getting an overall Band Score of 8.5 / CEFR Level of C2 with having achieved Listening 9.0, Reading 8.5, Writing 7.5, Speaking 8.0.
\item Provide a referee's report of both my academic referees (which they were happy to do).
\item Apply for an accomodation in one of the many student hostels on the campus. I opted for Melton Hall, non self-catered, En Suite (including a bath in my room) for 5,890.5 Pounds a Year. 
\item And most important: prepare myself and refine my topic so that I come well-prepared to Nottingham.
\end{enumerate}

I could provide all the details and got an unconditional offer which made me a future student starting my PhD Programm on 1st October 2016. But now I had to really focus and dig into the topic

\subsection*{Preparing for my PhD}
Since January I did a lot of preparation and learning for my PhD (which was done parallel to 24hrs/Week Job)

\begin{itemize}
\item Learned Erlang – implemented basic version of masterthesis CDA
\item Learned Haskell (Graham Hutton Book \& Tutorials on wiki.haskell.org on Monadic programming) – also implemented basic version of masterthesis CDA
\item Gained basic understanding of what Agda is and what is is capable of (Wikipedia of agda, Thorsten’s computer aided formal reasoning lecture, learnyouanagda, ...)
\item Reading and studying book "Category theory for the sciences"
\item Reading book Actor Model by Agha Gul
\item Reading book PI-calculus by Robin Millner’s
\item Reading papers on formal ABM/S
\item Reading papers on PI-calculus
\item Reading papers on ABS in computational economics
\item Reading papers on market designs
\end{itemize}

Initially I focused too much on the method but couldn't really argue why I was using it and what I want to achieve in the end. Writing a Research Proposal, giving a Presentation of my PhD Topic at FHV and discussing my ideas with others was of a great help to focus more on these missing points. \\

The insights during the preparation-process were the following:

\begin{itemize}
\item Erlang is a dead-end as it is not a pure functional language and reasoning is not really possible due to parallel nature and language constructs. Haskell is the way to go.
\item The Actor Model will most probably not be the direction I will head to because research seemed to move into different direction than that was I need, my agents are most probably of different nature. Actor Model is more about the concurrent and parallel interaction between agents (actors) which is an interesting aspect but most probably not the focus of my research (and would open up a whole new pandoras-box).
\item I want too much, I need to focus on less - this is a big thing to discuss with both my supervisors and I am sure they can help me with narrowing down my research topic.
\end{itemize}

\section*{2016 August 10th}
\subsection*{Where do I stand now}
On 21st June I had the opportunity to give a Phd-Seminar presenting my research ideas within 45mins + 15min discussion. This gave me the opportunity to make up my mind about the ideas I want to follow and which ones I want to reject. Also to better shape my ideas and to give a clear overview of them I started to write down a research proposal about 3 months ago. Of course this proposal is always changing and will continue to do so during the 1st semester as according to my supervisor the 1st semester is reserved to get started and to really carve out the research ideas and most important of all the research questions. So this is a still ongoing process but to give an overview where I stand NOW regarding the research direction I have will cite here the abstract of my research proposal which I have recently reworked (again) and which pretty well sums up the overall ideas and directions I want to follow:

\begin{quote}
Agent-Based Modeling and Simulation (ABM/S) is still a young discipline and the dominant approach to it is object oriented computation. This thesis goes into the opposite direction and asks how ABM/S can be mapped to and implemented using pure functional computation and what one gains from doing so. To the best knowledge of the author, so far no proper treatment of ABM/S in pure functional computation exists but only a few papers which only scratch the surface. The author argues that approaching ABM/S from a pure functional direction  offers a wealth of new powerful tools and methods. The most obvious one is that when using pure functional computation (equational) resoning about the correctness and about total and partial correctness of the simulation becomes possible. The ultimate benefit is that Agda becomes applicable which is both a pure functional programming language and a proof assistant allowing both to compute the dynamics of the simulation and to look at meta-level properties of the simulation - termination, convergence, equilibria, domain specific properties - by constructing proofs utilizing computer aided verification. \\
To map ABM/S to pure functional computation the idea is to apply both Robin Milner's PI-calculus and category theory. The PI-calculus will be used for a formal modelling of the problem and allows already a basic form of algebraic reasoning. Then the agents and the process of the agent-simulation will be mapped to category theory because pure functional programming approaches complex problems from the direction of category theory in the form of monadic programming. \\
The application will be in the field of agent-based computational economics where the approach will be to take an established model/theory and then apply the above mentioned methods to it and to show that using them will lead to the same results. 
\end{quote}

Note that the goal is not to establish new economic theories but to provide methods and tools for deeper insight and verification in the context of agent-based computational economics - this is after all still a thesis purely rooted in computer science.

\subsection*{PhD Roadmap Semesters}
\begin{enumerate}
\item Semester: Literature research \& research questions. Topics: 
	\begin{itemize}
	\item Functional computation \& programming: Agda, Haskell \& monads, category theory, type-systems, computer aided formal verification 
	\item Formal ABM/S: pi-calculus, various formal agent-types
	\item Finding application in agent-based computational economics: auction theory \& auction types, market design
	\end{itemize}
\item Semester: First publication: ? (Mapping auction types and ABM/S to category theory)
\item Semester: Second publication: ? (formalization of ABM/S and auction types in pi-calculus )
\item Semester: Simulation Framework, Finished Simulation framework implementation: basic framework with correctness \& proofs 
\item Semester: Third publication: ABM/S in Agda
\item Semester: Finalizing PhD - Writing final thesis combining all research \& results.
\end{enumerate}

\subsection*{Discussions with my Supervisors}
\begin{itemize}
\item So far the proposed ideas are a huge amount for 3 years and I doubt it is realistic to do in 3 years. Also it is yet totally unclear whether it makes sense/is possible to map ZI-Agents and the auction-types to category theory AND write specifications in pi-calculus for both. Maybe only one of them is really necessary or maybe only part of each? This is to be discussed with my supervisors.
\item Does it even make sense to study these auction-types with these types of agents? To look only at the dynamics? Is computational economics interested in these dynamics of the equilibrium processes? I really need to look into theoretical work on the various auction-types AND the computational economics approach to them.
\item Do we really need the pi-calculus when we have category-theory and vice versa? Does not just one method suffice? Or is there a mapping between both methods / a connection?
\end{itemize}