\section*{2016 September 2nd}
\subsection*{Black Swan}
I've been reading the book "Black Swan" (TODO: cite) since about one week and I'm thrilled by it. \\ 
He does a fundamental critic on economics as a "science" which he claims it is not because it can't predict due to the extreme complexity and uncertainty (and randomness) of the world. The question is then whether computational economics and simulation are also a dead end or not? \\
On position 4289 he writes that "If you hear a "prominent" economist using the word equilibrium, or normal distribution, do not argue with him; just ignore him, or try to put a rat down his shirt.". I thought right away from the beginning that the author would make fun of equilibria (and normal distribution) because the world is just not in an equilibrium and not normal distributed and will never be. What does this imply to my PhD studies? It should encourage me to go away from equilibrium research towards simulation of dynamics and to look force a different approach to computational economics. Fact is: I don't want to do something unrelated to reality and purely academic - but is it really within my reach to make an impact? It is not (one never can plan to make an impact on the world, as shown in the Black Swan Book) but I can select my focus accordingly to increase the possibility of an impact. Thus I feel that I should focus on alternative approaches to computational economics which abandon equilibrium theories altogether and look only on the dynamics instead of equilibria - a thing which I already mentioned in the entry of the 24th of August 2016.

\bigskip

The author also attacks linear regression and r square: I also never understood what one gains from it as it is only a linear fit and can blowup terribly when compared to the real-world. I remember an economic workshop in which I participated where all presenters gave statistics how well their model fits linear regression. I felt it was just dead numbers, giving you absolution and a stamp on your model which says: \textit{accepted by the (conservative) economics community} - but it all seemed so far away from reality.

\bigskip

The author also praises a few authors - I noted 4 from which I should definitely read a bit of.

\begin{itemize}
\item Karl Popper
\item Henri Poincaré
\item J.M. Keynes
\item Friedrich Hayek
\end{itemize}

Hayek is a major proponent of the Austrian School of Economics, which I should definitely look at because it seems to be in opposition to the neo-classical (cambridge school) which emphasises mathematical models of equilibrium where the Austrian School focuses on TODO?

\bigskip

Also I am aware that I know very very little of economics to be in the position to be a competent critic of it. 

\section*{2016 September 5th}
On the ferry from Bergen to Hirtshals on my way home. \\

\subsection*{Black Swan finished}
I've finished the Black-Swan book 2 days ago. The quintessence of it is basically an attack on the established economic theories and the people behind it. It all boils down that to a massive critique of the Gaussian Normal Distribution. Although it has its applications in the distribution of \"normal\" quantities like height, weight, age of persons it fails completely in case of predicting probabilities of unbounded quantities like they are common in economics where those parameters can totally explode. Also the author says that history is totally unpredictable due to it being an extremely complex process thus also implies that prediction for economics is impossible.  What is interesting is that the author does never go explicitly in discussing equilibria and their theories but only mentions them twice from where it becomes imminent that he thinks that equilibria are bullshit as well as gaussian distribution - this is what I've already expected at the beginning of the book. \\
This book has thus quite an implication for my Ph.D. as I have the feeling this author is right I cannot ignore what he says and continue as if nothing happened. Thus I have to question the use of the gaussian distribution and more specific the use and the pursue of equilibria in my applications. But still I do not know enough - but I just stumbled across a book of Mandelbrot (TODO: cite) in which he goes into analyzing markets and price-formation from a fractal point of view (which he invented). \bigskip
But what I think the author of the Black Swan misses or never mentioned is that we all want too much: newer cars, newer phones, newer computers. We want it every year and we want it cheap. In fact the problem is greed, we must return to a more simplistic life-style and not live above what is ours.

\subsection*{Mandelbrot book started}
This book is not about predicting prices or future economic trends but to explain and better understand the overall structure of price formation and volatility. Also a critique of the gaussian distribution and a huge proponent of power laws (see my Masterthesis). \bigskip
On position 160 (kindle) Mandelbrot claims that biologists do research on the healthy body, physicists collide particles, meteorologists look into hurricanes but economists seem to be curious incurious. \\
Thus my implication: economists should rely much much more on simulation and simulate their models using computer power (Mandelbrot says in the introduction that \"financial economics as a discipline is where chemistry was in the 16th century", which pretty says all. Also the author of black swan says that there rarely happens a feedback from the real-world to the models). This could be a more specific direction of my research (as already mentioned in other diary notes): looking into the dynamics of markets but with non-gaussian, fractal models. The result will not be a predictive price model but a system in which dynamics which resemble those on markets can be analysed and better understood to be better prepared in case of crashes. Mandelbrot himself sums it best up: \"basic research into the dynamics of pricing and volatility in a global marketplace gets short shrift." - this is what I want to do in my simulations.

\subsection*{A 1st semester Roadmap}
I still need to read and learn much more about economics, but how much exactly and which kind of direction? I think it is enough to cover the basics and then go deep into the subjects of equilibrium theory, market micro-structure, the fractal models of markets and agent-based computational economics. Thus the very basic approach in the 1st semester is doing literature research and do prototyping of the new formal agent-based simulation methods. During literature research the goal is to find a paper to start with, which allows to apply the methods developed and then to do further research based upon that paper in which I can incorporate my ideas of dynamics instead of equilibria and fractal/creative randomness/complexity from simplicity instead of predictability, Gaussian distribution and complexity from complex models.

\begin{itemize}
\item Literature Research in Economics
	\begin{itemize}
	\item Basic overview by reading the "Understanding Capitalism" Book (TODO: cite)
	\item Basic understanding in equilibrium theory by studying specific chapters of microeconomics book (TODO: cite)
	\item Basic understanding of market micro-structure by reading specific chapters of the Market Micro-structure book (TODO: cite)
	\item Find and read papers in the fields of 
		\begin{itemize}
		\item Fractal Market Models inspired by Mandelbrot
		\item agent-based computational economics
		\item Combinations of both.
		\end{itemize}		
	\end{itemize}
\item Method development and experiments in Computer Science
	\begin{itemize}
	\item Get better understanding of Agda
	\item Get deeper into Haskell and monadic programming
	\item Get basic understanding of Category Theory
	\item Get introduction into Computer Aided Formal Verification
	\end{itemize}
\end{itemize}

\subsection*{TODOs Update}
Note that these TODOs should be done till the 26th of September because I expect to meet my supervisors then - maybe there is time until the 1st of October but better earlier than too late (Research Proposal!) \\ 
I think I will abandon the whole Ethereum stuff as I will need all the time for the preparations and the Ph.D. itself - and because I also want a little bit of free-time without comptuers.

\begin{itemize}
\item Read Mandelbrot Book (TODO: cite)
\item Read Gode \& Sunders paper again (TODO: cite).
\item Read Everything what you wanted to know about continuous double-auction paper again (TODO: cite).
\item Read my Masterthesis.
\item Read the following chapters of "Handbook of Market-Design" (TODO: cite) in the given order: 2,3,12,10,13,16,17
\item Rework my Reseach-Proposal and reformulate it to cover the above idea. 
This will then be the basics of the initial meeting with my supervisors. Also include the thoughts of the entry to the diary of 2016 August 24th. And include essence of entry to diary of 2016 August 30th.
\item Get basic understanding in Agda 
	\begin{itemize}
	\item Work through "Dependent Types at Work"
	\item Look at Thorsten Altenkirchs Lecture "Computer Aided Formal Verification
	\item Read online Lectures about Dependent Types: TODO
	\end{itemize}
\item Haskell Learning: Implement the core of my Masterthesis in Haskell as I know it at the moment and trying to incorporate Monads: the core is the replicated auction-mechanism where every Agent knows all others.
\end{itemize}

\section*{2016 September 6th}
Currently on the Wewelsburg on a stop on my way home. Of course I can't stop thinking about a potential application of my formal methods in agent-based computational economics. \bigskip

My hypothesis is that a finite number of interacting agents following very simple rules in placing bids and asks lead to price dynamics that resemble power law distributions invented by Mandelbrot - price formation is unpredictable but the dynamics follow a rule and emerge out of the interaction of the agents which form a market. I would need to build a model of the agents and interactions among them and then implement the simulation. I should also look at continuous double auction and continuous batch auction and compare their results. A big question is what the role of equilibrium (theory) in this approach is.

\bigskip

I want to abandon the idea of a simulation which will hunt for equilibrium as I became convinced by the books of black swan and Mandelbrot that this is infeasible arcane magic, something I already thought right away from the beginning of my masterthesis and which I also discussed with Prof. Vollbrecht. I MUST NOT blindly follow a scientific road in which I don't believe and I don't believe in equilibrium in economics as the world is totally obviously NOT going towards equilibrium. \\
Of course the process of reaching it must not be confused with equilibrium itself and no process has been given so far thus meaning we could be on the way to reaching it and will at an unknown time in the future because we don't know the process - but that lies no in the field of belief and I don't believe thus abandoning the search for it. Also the conditions of the system in which equilibrium may unfold are always a changing: traders enter and leave at various points thus shuffling all new as one knows regarding system dynamic. \medskip

But i have to be honest: I find the concept of equilibrium beautiful and it is still fascinating to look at the dynamics of how equilibria form in a process: how long does it take, how efficient is it, is it even reachable, ... maybe i can nonetheless incorporate it in SOME way in the new approach: see if and what Mandelbrot says about them and incorporate that then.

\bigskip

So the overall direction is clear and will be one of two ways (or a combination of both, as both topics are extremely interesting):

\begin{enumerate}
\item Research equilibrium processes: look at how and under which conditions equilibrium is established, what properties the agents have, how long it takes, if it is feasible,...
\item Research price dynamics and trading dynamcis: ignore equilibrium and only look at the dynamics of trading processes where tools developed of Mandelbrot should be used instead of neo-classical economics (gaussian distributions, equilibrium theory)
\end{enumerate}

\bigskip

I've written an email to Martin Summer asking him a few questions about the Black Swand and Mandelbrot Books regarding Gaussian distribution and equilibrium theory - I am curious what he will answer. I will give a summary of the questions and his answers when he answered.

\subsection*{Updated TODOs}
I added 2 new items: the search for agent-based computational economics papers and papers in field of Mandelbrot agent-based economic simulation.

\begin{itemize}
\item Read Mandelbrot Book (TODO: cite)
\item Search for agent-based computational economics papers.
\item Search for Mandelbrot agent-based ((computational) economics) papers.
\item Search for agent-based economics equilibrium process papers.
\item Read Gode \& Sunders paper again (TODO: cite).
\item Read Everything what you wanted to know about continuous double-auction paper again (TODO: cite).
\item Read my Masterthesis.
\item Read the following chapters of "Handbook of Market-Design" (TODO: cite) in the given order: 2,3,12,10,13,16,17
\item Rework my Reseach-Proposal and reformulate it to cover the above idea. 
This will then be the basics of the initial meeting with my supervisors. Also include the thoughts of the entry to the diary of 2016 August 24th. And include essence of entry to diary of 2016 August 30th. Read through ALL the research-diary entries and incorporate the essence of them.
\item Get basic understanding in Agda 
	\begin{itemize}
	\item Work through "Dependent Types at Work"
	\item Look at Thorsten Altenkirchs Lecture "Computer Aided Formal Verification
	\item Read online Lectures about Dependent Types: TODO
	\end{itemize}
\item Haskell Learning: Implement the core of my Masterthesis in Haskell as I know it at the moment and trying to incorporate Monads: the core is the replicated auction-mechanism where every Agent knows all others.
\end{itemize}